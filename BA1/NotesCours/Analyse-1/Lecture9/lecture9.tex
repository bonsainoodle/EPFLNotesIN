\documentclass[a4paper]{article}

% Expanded on 2021-10-20 at 10:07:51.

\usepackage{../../style}

\title{Analyse 1}
\author{Joachim Favre}
\date{Mercredi 20 octobre 2021}

\begin{document}
\maketitle

\lecture{9}{2021-10-20}{Constante d'Euler, récurrence et sous-suites}{
}

\subsection{Le nombre $e$}
\parag{Proposition}{
    Soient $\left(x_n\right)$ et $\left(y_n\right)$ deux suites définies telles que:
    \[x_0 = y_0 = 1 \mathspace x_n = \left(1 + \frac{1}{n}\right)^n \mathspace y_n = 1 + \frac{1}{1!} + \frac{1}{2!} + \ldots + \frac{1}{n!} \mathspace \forall n \geq 1\]

    Alors:
    \begin{enumerate}
        \item $x_n \leq y_n$, $\forall n \in \mathbb{N}$
        \item $y_n \leq 3$, $\forall n \in \mathbb{N}$
        \item $\left(y_n\right)\uparrow$, $\forall n \in \mathbb{N}$
        \item $\left(x_n\right)\uparrow$, $\forall n \in \mathbb{N}$
    \end{enumerate}

    De (2) et (3) on peut en déduire que $\left(y_n\right)$ converge et que sa limite est plus petite ou égale à 3. Par là, et (1) et (4) on en déduit que $\left(x_n\right)$ converge, vers une valeur plus petite ou égale à la limite de $\left(y_n\right)$, donc plus petite ou égale à 3.
}

\parag{Définition de $e$}{
    On définit:
    \[\lim_{n \to \infty} \left(1 + \frac{1}{n}\right)^{n} \equiv e\]

    \subparag{Estimation}{
        On a:
        \[e \approx 2.718281828459045\ldots\]
    }
}


\subsection{Suites définies par récurrence}

\parag{Proposition (récurrence linéaire)}{
    Soit $a_0 \in \mathbb{R}$, et avec $a_{n+1} = q a_n + b$, où $q, b \in \mathbb{R}$. Alors:
    \begin{enumerate}
        \item Si $\left|q\right| < 1$, alors $\left(a_n\right)$ converge vers vers
              \[\lim_{n \to \infty} a_n = \frac{b}{1-q}\]
        \item Si $\left|q\right| \geq 1$, alors $\left(a_n\right)$ diverge, sauf si $\left(a_n\right)$ est une suite constante.
    \end{enumerate}
}

\parag{Proposition}{
    Soit $x_0 \in \mathbb{R}$, $x_{n+1} = g\left(x_n\right)$ et $g : E \mapsto E \subset \mathbb{R}$.

    \begin{itemize}
        \item Si $g$ est bornée, c'est-à-dire que $\exists m, M \in \mathbb{R}$ tels que
              \[m \leq g\left(x\right) \leq M \mathspace \forall x \in E\]
              alors la suite est bornée.

        \item Si $g$ est croissante, c'est-à-dire que $\forall x_1, x_2 \in E$:
              \[x_1 \leq x_2 \implies g\left(x_1\right) < g\left(x_2\right)\]
              alors la suite est monotone.
    \end{itemize}

    \subparag{Remarque}{
        Si $g$ est décroissante, alors $\left(x_n\right)$ n'est pas monotone, on en est sûr (mais la suite peut être convergente).
    }
}

\parag{Astuces pour étudier les suites définies par récurrence}{
    \begin{enumerate}[left=0pt]
        \item Trouver les candidats pour la limite, en supposant que cette dernière existe. Si cette équation n'admet pas de solution, alors la suite diverge.
        \item Étudier la convergence.
              \begin{enumerate}
                  \item Récurrence linéaire, c'est à dire que $x_{n + 1} = q x_n + b$. Alors:
                        \begin{itemize}
                            \item Si $\left|q\right| < 1$, alors
                                  \[\lim_{n \to \infty} x_n = \frac{b}{1 - q}\]
                            \item Si $\left|q\right| > 1$, alors $\left(x_n\right)$ diverge (sauf si $x_n$ est une constante, c'est à dire que $x_0 = \frac{b}{1 -q}$).
                            \item Si $\left|q\right| = 1$, alors $\left(x_n\right)$ diverge (sauf si $\left(x_n\right)$ est constante).
                        \end{itemize}

                  \item Si $x_{n+1} = g\left(x_n\right)$ avec $g$ croissante. Alors, on sait que la suite est monotone.
                        \begin{itemize}
                            \item Si $x_0 < x_1$, alors la suite est croissante, et on veut montrer qu'elle est majorée, auquel cas elle converge.
                            \item Si $x_0 > x_1$, alors la suite est décroissante, et on veut montrer qu'elle est minorée, auquel cas elle converge.
                        \end{itemize}

                  \item Si $x_{n+1} = g\left(x_n\right)$, mais que $g$ n'est ni linéaire ni croissante. Alors, on peut faire un graphique pour se donner une idée. On peut essayer d'utiliser la proposition suivante.

                  \item Démontrer que $\left(x_n\right)$ est une suite de Cauchy (qu'on définira plus tard).
              \end{enumerate}
    \end{enumerate}
}

\parag{Proposition}{
Si $\left(x_n\right)$ et $\left(a_n\right)$ deux suites, où
\[0 < a_n < 1 \mathspace \forall n \in \mathbb{N}\]
et que $\exists \ell \in \mathbb{R}$ tel que
\[\left(x_{n+1} - \ell\right) = a_n \left(x_n - \ell\right)\]
alors $\left(x_n\right)$ converge.

De la même manière, si $\left|x_{n+1} - \ell\right| \leq b_n \left|x_n - \ell\right|$ et $0 < b_n < \rho < 1$, alors $x_n$ converge et $\lim_{n \to \infty} x_n = \ell$
}

\subsection{Sous-suites}
\parag{Définition des sous-suites}{
    Une \important{sous-suite} d'une suite $\left(a_n\right)$ est une suite $k \mapsto a_{n_k}$ où $k \mapsto n_k$ est une suite \emph{strictement} croissante de nombres naturels.
}

\parag{Proposition (convergence d'une sous-suite)}{
    Si $\left(a_n\right)$ converge vers une limite $\ell$, alors toute sous suite $\left(a_{n_{k}}\right)$ converge aussi vers $\ell$.

    \subparag{Note personnelle: contraposée}{
        La contraposée de ce théorème est que s'il existe deux sous-suites de $\left(a_n\right)$ qui convergent vers des valeurs différentes, alors $\left(a_n\right)$ est divergente.
    }
}

\parag{Théorème de Bolzano-Weierstrass}{
    Dans toute suite bornée, il existe une sous-suite convergente. En d'autres mots, si on a $\left(a_n\right)$ tel que
    \[\exists m, M \telque m \leq a_n \leq M \implies \exists \left(a_{n_k}\right) \subset \left(a_n\right) \telque \lim_{k \to \infty} a_{n_k} = \ell \in \mathbb{R}\]
}

\end{document}
