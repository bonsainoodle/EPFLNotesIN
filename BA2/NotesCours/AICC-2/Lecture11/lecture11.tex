\documentclass[a4paper]{article}

% Expanded on 2022-03-29 at 15:31:42.

\usepackage{../../style}

\title{AICC-2}
\author{Joachim Favre}
\date{Mardi 29 mars 2022}

\begin{document}
\maketitle

\lecture{11}{2022-03-29}{Differences in Analysis, now divisions in AICC\ldots}{
}

\subsection{Modulo arithmetic}
\parag{Introduction}{
    We notice that in the set of integers, $\mathbb{Z}$, we can add, subtract and multiply, but we cannot divide (in general). We will dig deeper in this question.
}

\parag{Euclid's Division Algorithm}{
    For all integers $a$ and $m \neq 0$, there exist unique integers $q$ (the quotient) and $r$ (the remainder) such that:
    \[a = mq + r\]
    where $0 \leq r < \left|m\right|$
}

\parag{Mod operation}{
    By $r = a \Mod m$, we denote the remainder when dividing $a$ by $m$. Thus:
    \[a = mq + r \implies a \Mod m = r\]
}

\parag{Congruence}{
    Two integers $a$ and $b$ are called \important{congruent} modulo $m$ if $m \divides a - b$, i.e. if $m$ divides $a - b$. This is denoted by:
    \[\congruent{a}{b}{m}\]

    \subparag{Equivalent definitions}{
        In fact, we can see that all the following statements are equivalent:
        \begin{itemize}
            \item $\congruent{a}{b}{m}$
            \item $\left(a - b\right) \Mod m = 0$
            \item $a \Mod m = b \Mod m$
            \item $r_a = r_b$
        \end{itemize}
    }
}

\parag{Definition: Equivalence relation}{
    On a set, a \important{equivalence relation} is a relation (denoted by $\sim$) which satisfies:
    \begin{enumerate}
        \item Reflexivity: $a \sim a$
        \item Symmetry: $a \sim b \implies b \sim a$
        \item Transitivity: $a \sim b \land b \sim c \implies a \sim c$
    \end{enumerate}
}

\parag{Lemma: Equivalence relation}{
    Congruence modulo $m$ is an equivalence relation on the set of integers.
}

\parag{Theorem: Properties}{
    If $\congruent{a}{a'}{m}$ and $\congruent{b}{b'}{m}$, then:
    \begin{enumerate}
        \item $\congruent{a + b}{a' + b'}{m}$
        \item $\congruent{ab}{a'b'}{m}$
        \item $\congruent{a^n}{\left(a'\right)^n}{m}$
    \end{enumerate}

    \vspace{1em}

    An equivalent stating of this theorem is:
    \begin{enumerate}
        \item $\left(a + b\right) \Mod m = \left(a \Mod m + b \Mod m\right) \Mod m$
        \item $\left(ab\right) \Mod m = \left(\left(a \Mod m\right)\left(b \Mod m\right)\right) \Mod m$
        \item $a^n \Mod m = \left(\left(a \Mod m\right)^n\right) \Mod m$
    \end{enumerate}

    We have to be careful to see that the mod does not get distributed into the $n$ of the exponent.
}

\end{document}
