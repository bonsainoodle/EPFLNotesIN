\documentclass[a4paper]{article}

% Expanded on 2021-11-24 at 10:15:50.

\usepackage{../../style}

\title{Analyse 1}
\author{Joachim Favre}
\date{Mercredi 24 novembre 2021}

\begin{document}
\maketitle

\lecture{18}{2021-11-24}{Exponentielles, hyperboles et extrema}{
}

\parag{Proposition (dérivée de la fonction réciproque)}{
    Soient $I$ un intervalle ouvert, et $f : I \mapsto F$ une fonction bijective continue sur $I$ et dérivable en $x_0 \in I$, telle que $f'\left(x_0\right) \neq 0$.

    La fonction réciproque $f^{-1} : F \mapsto I$ est dérivable en $y_0 = f\left(x_0\right)$ et:
    \[\left(f^{-1}\right)'\left(y_0\right) = \frac{1}{f'\left(x_0\right)}, \mathspace \text{où } y_0 = f\left(x_0\right) \iff x_0 = f^{-1}\left(y_0\right)\]
}

\parag{Corollaire}{
    Soient $I$ et $F$, deux intervalles fermés. Si $f : I \mapsto F$ et $f^{-1} : F \mapsto I$ sont deux fonctions réciproques continues sur leurs domaines et dérivables à l'intérieur (partout, sauf aux bornes), alors pour tout $x$ à l'intérieur de $F$, tel que $f'\left(f^{-1}\left(x\right)\right) \neq 0$, on a:
    \[\left(f^{-1}\right)'\left(x\right) = \frac{1}{f'\left(f^{-1}\left(x\right)\right)}\]
}

\parag{Dérivée de l'exponentielle}{
    \[\left(e^x\right)' = e^x, \mathspace \forall x \in \mathbb{R}\]
}

\parag{Dérivée du logarithme}{
    \[\left(\log\left(x\right)\right)' = \frac{1}{x}, \mathspace \forall x > 0\]
}

\parag{Remarque}{
    On peut définir la fonction suivante:
    \[a^x \over{=}{déf} e^{x \log\left(a\right)}, \mathspace \forall x \in \mathbb{R}, a > 0, a \neq 1\]

    On peut aussi trouver la fonction réciproque:
    \[y = e^{x \log\left(a\right)} \iff \log\left(y\right) = x \log\left(a\right) \iff x = \frac{\log\left(y\right)}{\log\left(a\right)}\]

    On sait que $\log\left(a\right) \neq 0$ puisque $a \neq 1$. On définit donc:
    \[\log_a\left(y\right) \over{=}{déf} \frac{\log\left(y\right)}{\log\left(a\right)}, \mathspace \forall y \in \mathbb{R}^*_+, a > 0, a \neq 1\]


    Calculons leur dérivée:
    \[\left(a^x\right)' = \left(e^{x \log\left(a\right)}\right)' = \log\left(a\right) e^{x \log\left(a\right)} = a^x\log\left(a\right)\]
    \[\left(\log\left(a\right)\right)' = \left(\frac{\log\left(x\right)}{\log\left(a\right)}\right)' = \frac{1}{x\log\left(a\right)}\]
}

\parag{Dérivée logarithmique}{
    Soit $f\left(x\right) = f_1\left(x\right)^{f_2\left(x\right)}$.

    De manière générale, on trouve donc que:
    \[\left(f\left(x\right)\right)' = f\left(x\right)\left(\log\left(f\left(x\right)\right)\right)'\]
    si le logarithme est bien défini.

    Ce résultat est indispensable pour calculer les fonctions qui sous la forme $f_1\left(x\right)^{f_2\left(x\right)}$, et est très pratique si nous avons un très gros produit.
}

\subsection{Fonctions hyperboliques}
\parag{Définition}{
    On définit le sinus hyperbolique:
    \[\sh\left(x\right) = \sinh\left(x\right) \over{=}{déf} \frac{e^{x} - e^{-x}}{2}\]

    C'est une fonction impaire, définie pour $\forall x \in \mathbb{R}$. On peut aussi définir le cosinus hyperbolique:
    \[\ch\left(x\right) = \cosh\left(x\right) \over{=}{déf} \frac{e^x + e^{-x}}{2}\]

    C'est une fonction paire, définie pour $\forall x \in \mathbb{R}$.
}

\parag{Dérivée}{
    On remarque que leurs dérivées sont symmétriques:
    \[\left(\sinh\left(x\right)\right)' = \cosh\left(x\right), \mathspace \left(\cosh\left(x\right)\right)' = \sinh\left(x\right)\]
}

\parag{Propriété}{
    On remarque que ces fonctions ont la propriété suivante:
    \[\cosh^2\left(x\right) - \sinh^2\left(x\right) = \frac{e^{2x} + 2 + e^{-2x} - e^{2x} + 2 - e^{-2x}}{4} = \frac{4}{4} = 1\]

}

\parag{Tangente et cotangente hyperbolique}{
    On définit:
    \[\badth\left(x\right) = \tanh\left(x\right) \over{=}{déf} \frac{\sinh\left(x\right)}{\cosh\left(x\right)} = \frac{e^{x} - e^{-x}}{e^x + e^{-x}}, \mathspace \forall x \in \mathbb{R}\]

    On remarque que $-1 < \tanh\left(x\right) < 1$ pour tout $x$. De plus, on définit:
    \[\coth \over{=}{déf} \frac{\cosh\left(x\right)}{\sinh\left(x\right)} = \frac{e^x + e^{-x}}{e^{x} - e^{-x}}, \mathspace \forall x \in \mathbb{R}^*\]
}

\parag{Fonctions réciproques}{
    On parle de ``l'argument du sinus hyperbolique'', donc on note la fonction inverse $\Argsh = \arcsinh : \mathbb{R} \mapsto \mathbb{R}$. On a:
    \[y = \sinh\left(x\right) \iff x = \arcsinh\left(y\right), \mathspace \forall x, y \in \mathbb{R}\]

    On peut calculer la dérivée de cette fonction:
    \[\left(\arcsinh\left(y\right)\right)' = \frac{1}{\sinh'\left(\arcsinh\left(y\right)\right)} = \frac{1}{\cosh\left(\arcsinh\left(y\right)\right)}\]

    On a $\cosh\left(x\right) = \pm \sqrt{1 + \sinh^2\left(x\right)}$. Mais, puisque $\cosh\left(x\right) > 0$ pour tout $x$ on prend la version positive. Donc:
    \[\left(\arcsinh\left(y\right)\right)' = \frac{1}{\sqrt{1 + \sinh^2\left(\arcsinh\left(y\right)\right)}} = \frac{1}{\sqrt{1 + y^2}}, \mathspace \forall y \in \mathbb{R}\]

    On peut aussi trouver des fonctions réciproques pour les autres fonctions hyperboliques.
}

\subsection{Dérivées multiples}

\parag{Définition}{
    On définit la \important{dérivée d'ordre $n$} par:
    \[f^{\left(n\right)}\left(x\right) \over{=}{déf} \left(f^{\left(n-1\right)}\left(x\right)\right)\]

    Il est important de faire la différence avec les puissances. Pour des dérivées d'ordre $n$, on met des parenthèses, pas pour les puissances.
}

\parag{Définition}{
    $f : E \mapsto F$ est \important{$n$ fois dérivable} si elle admet une dérivée d'ordre $n$.
}

\parag{Définition de classe}{
    $f : E \mapsto F$ est \important{de classe $C^n\left(E\right)$} si elle admet une dérivée d'ordre $n$ qui est continue sur $E$. On dit qu'elle est ``$n$ fois continûment dérivable''.
}

\subsection{Théorème des accroissement finis}
\parag{Proposition}{
    Si $f : E \mapsto F$ est une fonction dérivable en $x_0 \in E$ telle que $f$ admet un extremum local en $x_0$, alors $f'\left(x_0\right) = 0$.

    \subparag{Réciproque}{
        La réciproque est fausse.
    }
}

\parag{Définition des points stationnaires}{
    Si $f : E \mapsto F$ est dérivable en $x_0$ et $f'\left(x_0\right) = 0$, on dit que $x_0$ est un \important{point stationnaire} de $f$.
}

\parag{Théorème de Rolle}{
Soient $a < b \in \mathbb{R}$ et $f : \left[a, b\right] \mapsto F$ telle que:
\begin{enumerate}
    \item $f : \left[a, b\right] \mapsto F$ est continue
    \item $f$ est dérivable sur $\left]a, b\right[ $
    \item $f\left(a\right) = f\left(b\right)$
\end{enumerate}

Alors, il existe au moins un point $c \in \left]a, b\right[$ tel que $f'\left(c\right) = 0$.

}

\end{document}
