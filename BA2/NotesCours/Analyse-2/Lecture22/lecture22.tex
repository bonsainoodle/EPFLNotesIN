\documentclass[a4paper]{article}

% Expanded on 2022-05-16 at 10:14:40.

\usepackage{../../style}

\title{Analyse 2}
\author{Joachim Favre}
\date{Lundi 16 mai 2022}

\begin{document}
\maketitle

\lecture{22}{2022-05-16}{Mon intégrale elle est douce}{
}
\section[Calcul intégral]{Calcul intégral des fonctions de plusieurs variables}
\subsection{Intégrale sur un pavé fermé}

\parag{Définition: Pavé}{
Un \important{pavé fermé} est un sous-ensemble de $\mathbb{R}^n$ qui est le produit Cartésien de $n$ intervalles fermés bornés:
\[P = \left[a_i, b_i\right] \times \left[a_2, b_2\right] \times \ldots \times \left[a_n, b_n\right], \mathspace a_i < b_i \ \forall i = 1, \ldots, n\]

Nous notons le \important{pavé ouvert} par:
\[\mathring{P} = \left]a_1, b_1\right[ \times \ldots \times \left]a_n b_n\right[ \]
}

\parag{Définition: Volume}{
    Le \important{volume} d'un pavé fermé est défini par:
    \[\left|P\right| = \left(b_1 - a_1\right)\left(b_2 - a_2\right)\cdots\left(b_n -a_n\right)\]
}

\parag{Définition: Subdivision}{
    Soit $\sigma_j$ une subdivision de $\left[a_j, b_j\right]$ (comme nous l'avions définie en Analyse 1), où $a_j < b_j$. Notez que chaque subdivision n'a pas besoin d'être régulière. Nous avons donc:
    \[\sigma_j = \left\{a_j = x_{j,0} < x_{j,1} < \ldots < x_{j, n_j} < b_j\right\}\]

    Alors, $\sigma = \left(\sigma_1, \ldots, \sigma_n\right)$ est appelée une \important{subdivision} de $P$.

    Nous notons $D\left(\sigma\right)$ la collection des pavés engendrés par la subdivision.
}

\parag{Définition: Sommes de Darboux}{
Soit $P$ un pavé fermé et soit $f: P \mapsto \mathbb{R}$ une fonction bornée sur $P$. Alors, on définit les sommes de Darboux de $f$ sur $P$.

Soit $D\left(\sigma\right)$ une collection de pavés fermés engendrée par la subdivision $\sigma$. Alors:
\[\underline{S}_{\sigma}\left(f\right) \over{=}{déf} \sum_{Q \in D\left(\sigma\right)}^{} m\left(Q\right) \left|Q\right|, \mathspace \text{où } m\left(Q\right) = \inf_{\bvec{x} \in Q}\left(f\left(\bvec{x}\right)\right)\]
\[{\color{red}\overline{S}_{\sigma}\left(f\right)} \over{=}{déf} \sum_{Q \in D\left(\sigma\right)}^{} {\color{red}M\left(Q\right)} \left|Q\right|, \mathspace \text{où } {\color{red}M\left(Q\right)} = {\color{red}\sup_{\bvec{x} \in Q}}\left(f\left(\bvec{x}\right)\right)\]

La \important{somme de Darboux inférieure} est définie par:
\[\underline{S}\left(f\right) \over{=}{déf} \sup\left\{\underline{S}_{\sigma}\left(f\right) : \sigma \text{ est une subdivision de } P\right\}\]

La \important{somme de Darboux supérieure} est définie par:
\[{\color{red}\overline{S}\left(f\right)} \over{=}{déf} {\color{red}\inf}\left\{{\color{red}\overline{S}_{\sigma}\left(f\right)} : \sigma \text{ est une subdivision de } P\right\}\]
}

\parag{Définition: Fonction intégrable}{
    Soit $P \subset \mathbb{R}^n$ un pavé fermé et $f: P \mapsto \mathbb{R}$ une fonction bornée.

    $f$ est \important{intégrable} sur $P$ si et seulement si:
    \[\underline{S}\left(f\right) = \overline{S}\left(f\right)\]

    Dans ce cas, \important{l'intégrale de $f$ sur $P$} est définie par:
    \[\int_P f\left(\bvec{x}\right)d \bvec{x} = \idotsint_P f\left(x_1, \ldots, x_n\right) dx_1 \ldots dx_n \over{=}{déf} \underline{S}\left(f\right) = \overline{S}\left(f\right)\]
}

\parag{Théorème}{
    Toute fonction continue est intégrable sur un pavé fermé.
}

\parag{Propriétés de l'intégrale}{
\subparag{Propriété 1: Additivité}{
Soit $P$ un pavé fermé, et $\left\{P_i\right\}_{i \in I}$ une famille dénombrable de pavés fermés disjoints (l'intersection entre l'intérieur de n'importe quel deux pavé est vide, $\mathring{P}_i \cap \mathring{P}_j = \o$ pour $i \neq j$) telle que $P = \bigcup_{i \in I} P_i$.

Alors, pour toute fonction continue $f: P \mapsto \mathbb{R}$, nous avons:
\[\int_{P} f\left(\bvec{x}\right) d \bvec{x} = \sum_{i \in I}^{} \int_{P_i} f\left(\bvec{x}\right) d \bvec{x}\]
}

\subparag{Propriété 2: Linéarité}{
    Soit $P$ un pavé fermé, et soient $f, g : P \mapsto \mathbb{R}$ deux fonctions continues. Alors, pour tout $\alpha, \beta \in \mathbb{R}$:
    \[\int_{P} \left(\alpha f\left(\bvec{x}\right) + \beta g\left(\bvec{x}\right)\right)d \bvec{x} = \alpha \int_{P} f\left(\bvec{x}\right)d \bvec{x} + \beta \int_{P} g\left(\bvec{x}\right) d \bvec{x}\]
}

\subparag{Propriété 3}{
    Soit $P$ un pavé fermé, et soit $f : P \mapsto\mathbb{R}$ une fonction bornée, intégrable sur $P$, et telle que:
    \[\left|f\left(\bvec{x}\right)\right| \leq K \in \mathbb{R}_{\geq 0} \iff -K \leq f\left(\bvec{x}\right) \leq K, \mathspace \forall \bvec{x} \in P\]

    Alors:
    \[-K \left|P\right| \leq \int_P f\left(\bvec{x}\right)d \bvec{x} \leq K \left|P\right|\]
}
}

\parag{Définition: Volume}{
    Soit $P \subset\mathbb{R}^2$ un pavé fermé de dimension 2, et soit $f : P \mapsto \mathbb{R}_+$ une fonction intégrable. Alors, le \important{volume} de l'ensemble sous la surface $z = f\left(x, y\right) \geq0$ est défini par:
    \[V \over{=}{déf} \iint_P f\left(x, y\right) dx dy\]

    En d'autres mots, $V$ est le volume du sous-ensemble entre $z = 0$ et $z = f\left(x, y\right) \geq 0$ au dessus du pavé fermé $P \subset \mathbb{R}^2$.
}

\parag{Théorème de Fubini}{
Soit $P = \left[a_1, b_1\right] \times \ldots \times \left[a_n, b_n\right]\subset \mathbb{R}^{n}$ un pavé fermé, et soit $f: P \mapsto \mathbb{R}$ une fonction continue.

Alors, $f$ est intégrable sur $P$, et on a:
\[\int_{P} f\left(\bvec{x}\right)d \bvec{x} = \int_{a_n}^{b_n} \left(\int_{a_{n-1}}^{b_{n-1}} \cdots \left(\int_{a_1}^{b_1} f\left(x_1, \ldots, x_n\right)dx_1\right) \cdots dx_{n-1}\right)dx_{n}\]

Ceci marche pour n'importe quel choix de l'ordre d'intégration.

\subparag{Remarque}{
    Ce théorème n'est uniquement valide sur un pavé fermé, nous verrons ensuite une autre version de ce théorème pour d'autres ensembles.
}

}

\end{document}
