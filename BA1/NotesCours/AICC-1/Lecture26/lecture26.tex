\documentclass[a4paper]{article}

% Expanded on 2021-12-15 at 16:23:14.

\usepackage{../../style}

\title{AICC}
\author{Joachim Favre}
\date{Mercredi 15 décembre 2021}

\begin{document}
\maketitle

\lecture{26}{2021-12-15}{Reality vs. expectations}{
}

\parag{Definition: distribution of a random variable}{
    The \important{distribution} of a random variable $X$ on a sample space $S$ is the set of pairs $\left(r, p\left(X = r\right)\right)$ for all $r \in X\left(S\right)$, where $p\left(X = r\right)$ is the probability that $X$ takes the value $r$:
    \[p\left(X = r\right) = \sum_{s \in S, X\left(s\right) = r}^{} p\left(s\right)\]

    \subparag{Probability mass function}{
        If the range of the function $X$ is countable, then $p\left(X = r\right)$ can be interpreted as a function: $p : X\left(S\right) \mapsto \mathbb{R}$. The, this function is called \important{probability mass function} and it is a probability distribution over the sample space $X\left(S\right)$.
    }
}

\subsection{Expected value}
\parag{Definition: Expected value}{
    The \important{expected value} (or \important{expectation} or \important{mean}) of the random variable $X$ on the sample space $S$ is equal to:
    \[E\left(X\right) = \sum_{s \in S}^{} p\left(s\right) X\left(s\right)\]
}

\parag{Theorem}{
    If $X$ is a random variable and $p\left(X = r\right)$ is the probability distribution, where $p\left(X = r\right) = \sum_{s \in S, X\left(s\right) = r}^{} p\left(s\right)$, then:
    \[E\left(X\right) = \sum_{r \in X\left(S\right)}^{} p\left(X = r\right)r\]
}

\parag{Theorem: Expected value of Bernoulli trials}{
    The expected number of successes when $n$ mutually independent Bernoulli trials are performed is $np$, where $p$ is the probability of success of each trial.

    \subparag{Proof}{
        We could prove this theorem my using our theorem above, and the following equality:
        \[k \binom{n}{k} = n \binom{n-1}{k-1}\]

        However, this would take around 5 lines, whereas we will be able to prove this theorem in one line next week (it is on the next page in this document, but for me it is next week (in fact, I am cleaning those notes on Saturday the \nth{18} of December, so it's not exactly in a week, but you get what I mean)).
    }
}

\end{document}
