\documentclass[a4paper]{article}

% Expanded on 2022-04-28 at 14:29:30.

\usepackage{../../style}

\title{AICC-2}
\author{Joachim Favre}
\date{Jeudi 28 avril 2022}

\begin{document}
\maketitle

\lecture{18}{2022-04-28}{Summing everything up}{
}

\parag{AES}{
    As mentioned in the last lecture, AES is an encryption standard that gives symmetric-key encryption.
    This algorithm offers less protection that RSA, but it is much faster and less memory consuming.
}

\subsection{Fun fact and summary}
\parag{Cyclic groups}{
    We proved that, in a group, for all $a \in G$, there exists a $\ell \in \mathbb{N}^*$ such that $a^\ell = e$, the identity element. The smallest one is named the order of this element. We also proved that, necessarily, $a^{\left|G\right|} = e$.

    Thus, we can see that the following definition may be interesting.

    \subparag{Definition}{
        If there exists a $g \in G$  such that $\left\{g, g^2, \ldots, g^{\left|G\right| - 1}, e\right\} = G$, then $\left(G, \star\right)$ is called a \important{cyclic group}. $g$ is named a \important{generator}.

        In other words, the generator has order $\left|G\right|$.
    }

    \subparag{Theorem}{
        All cyclic groups that have the same order (meaning that $G$ has the same cardinality) are isomorphic.

        In other words, there is only one cyclic group of order $n$.
    }
}

\parag{Summary}{
    We have seen three main cryptography algorithms: one-time pad (used by secret agences), Diffie-Hellman and El Gamal, and RSA.

    \subparag{One-time-pad}{
        Alice and Bob have a common private key $k$ (that they need to have shared at some point) which is as long as the text $t$ Alice wants to send. Alice publishes $c = t \oplus k$, and Bob can decipher by doing $c \ominus k = t$. $k$ needs to be completely random, leading to perfect secrecy: nobody can distinguish $c$ from a sequence of coin flips, the cryptogram and the plaintext are independent (the cryptogram holds no information on the plaintext).
    }

    \subparag{Diffie-Hellman and El Gamal}{
        Alice picks a number $a$ and Bob picks a number $b$. They choose publicly a $m$ and a $g$, and they publish $A = g^a$ and $B = g^b$ respectively. This allows both of them to compute $K = g^{ab} = \left(g^a\right)^b = \left(g^b\right)^a$, which gives them a common key. This is Diffie-Hellman.

        El Gamal adds a text $t$, which Alice wants to send. She computes and publishes $K t$, and Bob can compute $K^{-1} K t$ to recover the text, everything mod $m$.

        This is secret, as long as discrete logs stay impossible to compute.
    }

    \subparag{RSA}{
        Bob secretly chooses $m = pq$ and $d$, and publishes $\left(m, e\right)$. Alice can publish $c = \left[t^e\right]_m$, and Bob can recover it by doing $t = \left[c^d\right]_m$.

        This is secret, as long as factoring and discrete logs stay impossible to compute.
    }
}

\end{document}
