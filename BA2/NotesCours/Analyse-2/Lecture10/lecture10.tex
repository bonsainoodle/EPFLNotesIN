\documentclass[a4paper]{article}

% Expanded on 2022-03-21 at 17:23:50.

\usepackage{../../style}

\title{Analyse 2}
\author{Joachim Favre}
\date{Mercredi 23 mars 2022}

\begin{document}
\maketitle

\lecture{10}{2022-03-23}{Le retour des gendarmes}{
}

\parag{Exemple d'utilisation du changement de coordonnées}{
Soit la fonction $f: \mathbb{R}^2 \mapsto \mathbb{R}$ définie par:
\begin{functionbypart}{f\left(x\right)}
    \frac{x^3 + y^3}{x^2 + y^2}, \mathspace \text{si } \left(x, y\right) \neq \left(0, 0\right) \\
    0, \mathspace \text{autrement}
\end{functionbypart}

Utilisons nos deux mêmes limites: $\left\{\bvec{a_k}\right\} = \left(\frac{1}{k}, 0\right)$ et $\left\{\bvec{b}_k\right\} = \left(\frac{1}{k}, \frac{1}{k}\right)$:
\[\lim_{k \to \infty} f\left(\bvec{a_k}\right) = \lim_{k \to \infty} \frac{\frac{1}{k^3}}{\frac{1}{k^2}} = \lim_{k \to \infty} \frac{\frac{1}{k}}{1} = 0\]
\[\lim_{k \to \infty} f\left(\bvec{b_k}\right) = \lim_{k \to \infty} \frac{\frac{1}{k^3} + \frac{1}{k^3}}{\frac{2}{k^2}} = \lim_{k \to \infty} \frac{\frac{2}{k}}{1} = 0\]

Nous décidons donc de formuler l'hypothèse que $\lim_{\left(x, y\right) \to \left(0, 0\right)} = 0$.

\subparag{Preuve par changement de coordonnées}{
Il est possible de démontrer cette limite par la définition. Cependant, une deuxième manière beaucoup plus efficace consiste à utiliser un changement de variable vers les coordonnées polaires.
\imagehere[0.5]{ChangementVariableCoordonneesPolaires.png}

Ce changement de variable nous donne:
\[x = r\cos\left(\phi\right), \mathspace y = r\sin\left(\phi\right)\]
où $r \in \mathbb{R}_{\geq 0}$ et, si $r \neq 0$, $\phi \in \left[0, 2\pi\right[$.

Ainsi, on obtient que notre fonction \textit{est égale} à:
\begin{multiequality}
    f\left(r, \phi\right) =\ & \frac{r^3 \cos^3\left(\phi\right) + r^3 \sin^3\left(\phi\right)}{r^2 \cos^2\left(\phi\right) + r^2 \sin^2\left(\phi\right)}  \\
    =\ & \frac{r^3 \left(\cos^3\left(\phi\right) + \sin^3\left(\phi\right)\right)}{r^2} \\
    =\ & r\left(\cos^3\left(\phi\right) + \sin^3\left(\phi\right)\right)
\end{multiequality}

Or, nous savons que $\left(x, y\right) \to \left(0, 0\right)$ est équivalent à dire que $r = \sqrt{x^2 + y^2} \to 0$ et $\phi$ est une fonction inconnue de $r$. Ceci nous donne que:
\[\lim_{r \to 0} \left|f\left(r, \phi\right)\right| = \lim_{r \to 0} \underbrace{r}_{\to 0}\underbrace{\left|\cos^3\left(\phi\right) + \sin^3\left(\phi\right)\right|}_{\leq 2} = 0\]

On en déduit donc que:
\[\lim_{\left(x, y\right) \to \left(0, 0\right)} f\left(x, y\right) = 0\]

Cette méthode est souvent efficace (mais pas toujours) pour montrer l'existence des limites à l'origine. Elle est à retenir.

\qed
}
}

\parag{Remarque 1}{
    Nous ne pouvons pas calculer la limite d'une fonction de plusieurs variables de la façon suivante:
    \[\lim_{\left(x, y\right) \to \left(x_0, y_0\right)} f\left(x, y\right) \neq \lim_{x \to x_0} \left(\lim_{y \to y_0} f\left(x, y\right)\right)\]
}

\parag{Remarque 2}{
    Si la limite $\lim_{\left(x, y\right) \to \left(a, b\right)} f\left(x, y\right)$ existe, et les limites par rapport à chaque variables existent pour tout $x$ et tout $y$ dans le domaine de $f$, alors on peut échanger l'ordre des limites:
    \[\lim_{x \to a} \left(\lim_{y \to b} f\left(x, y\right)\right) = \lim_{y \to b} \left(\lim_{x \to a} f\left(x, y\right)\right)\]
}

\parag{Remarque 3}{
    L'existence de la limite $\lim_{\left(x, y\right) \to \left(a, b\right)} f\left(x, y\right)$ n'implique pas en général l'existence des limites $\lim_{x \to a} f\left(x, y\right)$ et $\lim_{y \to b} f\left(x, y\right)$.
}

\parag{Théorème des 2 gendarmes}{
    Soient $f, g, h : E \mapsto \mathbb{R}$, où $E \subset \mathbb{R}^n$, telles que:
    \begin{enumerate}
        \item $\displaystyle \lim_{\bvec{x} \to \bvec{x_0}} = \lim_{\bvec{x} \to \bvec{x_0}} g\left(\bvec{x}\right) = \ell$
        \item Il existe un $\alpha > 0$ tel que pour tout $x \in \left\{x \in E : 0 < \left\|\bvec{x} - \bvec{x_0}\right\| \leq \alpha\right\} = \bar{B\left(\bvec{x_0}, \alpha\right)} \setminus \left\{\bvec{x_0}\right\}$, on a:
              \[f\left(\bvec{x}\right) \leq h\left(\bvec{x}\right) \leq g\left(\bvec{x}\right)\]
    \end{enumerate}

    Alors:
    \[\lim_{\bvec{x} \to \bvec{x_0}} h\left(\bvec{x}\right) = \ell\]
}

\parag{Proposition: Continuité d'une fonction composée}{
    Soient 2 sous-ensembles ensembles $A \subset \mathbb{R}^n$, $B \subset \mathbb{R}^p$. De plus, soient deux fonctions $\bvec{g} : A \mapsto B$ et $f: B \mapsto \mathbb{R}$ où:
    \[\bvec{g}\left(\bvec{x}\right) = \left(g_1\left(\bvec{x}\right), \ldots, g_p\left(\bvec{x}\right)\right)\]

    Si $g_1, \ldots, g_p$ sont continues en $\bvec{a} \in A$, et $f$ est continue en $\left(g_1\left(\bvec{a}\right), \ldots, g_p\left(\bvec{a}\right)\right)$, alors $f \circ \bvec{g}\left(\bvec{x}\right)$ est continue en $\bvec{x} = \bvec{a}$.
}

\subsection[Min et max sur un compact]{Maximum et minimum d'une fonction sur un ensemble compact}
\parag{Définition: Maximum}{
    Soit la fonction $f: E \mapsto \mathbb{R}$, où $E \subset \mathbb{R}^n$.

    Si $M \in \mathbb{R}$ satisfait:
    \begin{enumerate}
        \item $\displaystyle f\left(\bvec{x}\right) \leq M$ pour tout $\bvec{x} \in E$
        \item $\displaystyle M \in f\left(E\right)$
    \end{enumerate}

    Alors $M$ est le \important{maximum} de la fonction $f$ sur $E$.
}

\parag{Définition: Minimum}{
    Soit la fonction $f: E \mapsto \mathbb{R}$, où $E \subset \mathbb{R}^n$.

    Si $m \in \mathbb{R}$ satisfait:
    \begin{enumerate}
        \item $\displaystyle f\left(\bvec{x}\right)\ {\color{red}\geq}\ m$ pour tout $\bvec{x} \in E$
        \item $\displaystyle m \in f\left(E\right)$
    \end{enumerate}

    Alors $m$ est le \important{minimum} de la fonction $f$ sur $E$.
}

\parag{Théorème du min et du max sur un compact}{
    Une fonction continue sur un sous-ensemble compact $E \subset \mathbb{R}^2$ atteint son maximum et son minimum, i.e.:
    \[\exists \max_{\bvec{x} \in E} f\left(\bvec{x}\right), \mathspace \exists \min_{\bvec{x} \in E} f\left(\bvec{x}\right)\]

    \subparag{Remarque}{
        Dans $\mathbb{R}^n$, pour que $f$ atteigne aussi toute valeur intermédiaire entre $m$ et $M$, il faut que $E$ soit compact, mais aussi qu'il soit connexe par chemins (la définition arrive juste après).
    }
}

\parag{Définition: Connexité par chemins}{
    Un ensemble $E$ est \important{connexe par chemins} si, pour n'importe quels 2 points, il existe un chemin d'un point à l'autre qui est continu et qui est contenu entièrement dans $E$.
}

\end{document}
