\documentclass[a4paper]{article}

% Expanded on 2022-03-08 at 13:18:32.

\usepackage{../../style}

\title{Analyse 2}
\author{Joachim Favre}
\date{Mercredi 09 mars 2022}

\begin{document}
\maketitle

\lecture{6}{2022-03-09}{Fin des équations différentielles}{
}

\parag{Méthode des coefficients indéterminés}{
    Cette méthode permet de trouver une solution particulière à l'équation $y''\left(x\right) + p y'\left(x\right) + qy\left(x\right) = f\left(x\right), \mathspace p, q \in \mathbb{R}$

    Pour que cette méthode fonctionne, il faut que $f\left(x\right)$ soit une combinaison linéaire de:
    \[e^{cx} R_n\left(x\right) \mathspace \text{ et } \mathspace e^{ax}\left(\cos\left(bx\right) P_n\left(x\right) + \sin\left(bx\right)Q_m\left(x\right)\right)\]
    où $R_n\left(x\right), P_n\left(x\right), Q_m\left(x\right)$ sont des polynômes de degré $n$ et $m$, et $c, a, b \in \mathbb{R}$.

    Ainsi, si $f\left(x\right)$ est une combinaison linéaire de ces deux fonctions, on utilise le principe de la superposition des solutions.

    Si $f\left(x\right) = e^{cx} R_n\left(x\right)$:
    \begin{center}
        \begin{tabularx}{\textwidth}{|>{\hsize=.25\textwidth}C|C|}
            \hline
            $c$ est une racine de $\lambda^2 + p\lambda + q = 0$ & Ansatz                                    \\
            \hhline{|=|=|}
            Non                                                  & $y_{part} = e^{cx} T_n\left(x\right)$     \\
            \hline
            Oui                                                  & $y_{part} = x^r e^{cx} T_n\left(x\right)$ \\
            \hline
        \end{tabularx}
    \end{center}
    où $r$ est la multiplicité de la racine $\lambda = c$, et $T_n\left(x\right)$ est polynôme de degré $n$ à coefficients indéterminé.

    Si $f\left(x\right) = e^{ax}\left(\cos\left(bx\right) P_n\left(x\right) + \sin\left(bx\right) Q_n\left(x\right)\right)$:
    \begin{center}
        \begin{tabularx}{\textwidth}{|>{\hsize=.25\textwidth}C|C|}
            \hline
            $a + ib$ est une racine de $\lambda^2 + p\lambda + q = 0$ & Ansatz                                                                                                          \\
            \hhline{|=|=|}
            Non                                                       & $y_{part} = e^{ax} \left(T_N\left(x\right) \cos\left(bx\right) + S_N\left(x\right) \sin\left(bx\right)\right)$  \\
            \hline
            Oui                                                       & $y_{part} = xe^{ax} \left(T_N\left(x\right) \cos\left(bx\right) + S_N\left(x\right) \sin\left(bx\right)\right)$ \\
            \hline
        \end{tabularx}
    \end{center}
    où $N = \max\left(n, m\right)$, et $T_N\left(x\right)$ et $S_N\left(x\right)$ sont des polynômes de degré $N$ à coefficients indéterminés.
}

\parag{Note personnelle: Résumé}{
\subparag{EDVS}{
    Une équation de la forme $f\left(y\right)y' = g\left(x\right)$ se résout en séparant les variables:
    \[f\left(y\right) \frac{dy}{dx} = g\left(x\right) \implies f\left(y\right) dy = g\left(x\right) dx \implies \int f\left(y\right) dy = \int g\left(x\right)dx\]
}

\subparag{EDL1 homogène}{
    La solution générale d'une EDL1 homogène $y'\left(x\right) + p\left(x\right) y\left(x\right) = 0$ est donnée par:
    \[y\left(x\right) = Ce^{-P\left(x\right)}, \mathspace C \in\mathbb{R}\]
    où $P\left(x\right)$ est une primitive (sans la constante) de $p\left(x\right)$.
}

\subparag{EDL1 complète}{
Pour résoudre une EDL1 complète, $y'\left(x\right) + p\left(x\right)y\left(x\right) = f\left(x\right)$, nous utilisons plusieurs principes. La méthode de la variation de la constante nous donne une solution particulière:
\[v\left(x\right) = c\left(x\right)e^{-P\left(x\right)}, \mathspace \text{avec } c\left(x\right) = \int f\left(x\right) e^{P\left(x\right)}\]
où $P\left(x\right)$ est une primitive (sans la constante) de $p\left(x\right)$.

Ensuite, en trouvant $v_0\left(x\right)$, la solution générale à l'équation homogène associée, nous utilisons le principe de superposition des solutions pour trouver la solution générale à notre équation:
\[y\left(x\right) = v\left(x\right) + v_0\left(x\right)\]
}

\subparag{EDL2 homogène à coefficients constants}{
    Pour résoudre une EDL2 homogène à coefficients constants, $y''\left(x\right) + py\left(x\right) + qy\left(x\right) = 0$ où $p, q \in \mathbb{R}$, nous cherchons les racines $a, b \in \mathbb{C}$ du polynôme caractéristique:
    \[\lambda^2 + p\lambda + q = 0\]

    \begin{functionbypart}{y\left(x\right)}
        C_1 e^{ax} + C_2 e^{bx}, \mathspace \text{ si } a, b \in \mathbb{R}, a \neq b \\
        C_1 e^{ax} + C_2 xe^{ax}, \mathspace \text{ si } a = b \\
        C_1 e^{\alpha x} \cos\left(\beta x\right) + C_2 e^{\alpha x} \sin\left(\beta x\right), \mathspace \text{ si } a = \alpha + i\beta = \bar{b} \not \in \mathbb{R}
    \end{functionbypart}
}

\subparag{EDL2 homogène}{
    Pour résoudre une EDL2 homogène, $y''\left(x\right) + p\left(x\right)y'\left(x\right) + q\left(x\right)y\left(x\right) = 0$, nous avons besoin de deviner une solution particulière $v_1\left(x\right)$. De là, nous pouvons calculer une deuxième solution linéairement indépendante:
    \[v_2\left(x\right) = c\left(x\right) v_1\left(x\right), \mathspace \text{avec } c\left(x\right) = \int \frac{e^{-P\left(x\right)}}{v_1^2\left(x\right)}dx\]
    où $P\left(x\right)$ est une primitive (sans la constante) de $p\left(x\right)$.

    Finalement, la solution générale est donnée par:
    \[y\left(x\right) = C_1 v_1\left(x\right) + C_2 v_2\left(x\right), \mathspace \forall C_1, C_2 \in \mathbb{R}\]
}

\subparag{EDL2 à coefficients constants}{
    Pour résoudre une EDL2 à coefficients constants, $y''\left(x\right) + py'\left(x\right) + qy\left(x\right) = f\left(x\right)$ où $p, q \in \mathbb{R}$, nous pouvons parfois aller plus vite que la méthode de la variation de la constante en utilisant la méthode des coefficients indéterminés, si $f\left(x\right)$ a une forme particulière.
}

\subparag{EDL2 complète}{
    Pour résoudre une EDL2 complète, $y''\left(x\right) + p\left(x\right)y'\left(x\right) + q\left(x\right)y\left(x\right) = f\left(x\right)$, nous cherchons d'abord deux solutions particulières linéairement indépendantes à l'EDL2 homogène associée $v_1\left(x\right)$ et $v_2\left(x\right)$ (qui nous donnent donc aussi la solution générale à cette EDL2 homogène associée). De là, nous obtenons une solution particulière à notre EDL2 complète à l'aide de la méthode de la variation de la constante:
    \[v\left(x\right) = c_1\left(x\right) v_1\left(x\right) + c_2\left(x\right) v_2\left(x\right)\]

    Avec:
    \[c_1\left(x\right) = - \int \frac{f\left(x\right)v_2\left(x\right)}{W\left[v_1, v_2\right]\left(x\right)}dx\]
    \[c_2\left(x\right) = \int \frac{f\left(x\right) v_1\left(x\right)}{W\left[v_1, v_2\right]\left(x\right)}dx\]
    \[W\left[v_1, v_2\right]\left(x\right) = \det\begin{pmatrix} v_1\left(x\right) & v_2\left(x\right) \\ v_1'\left(x\right) & v_2'\left(x\right) \end{pmatrix} \]

    De là, nous pouvons trouver notre solution générale:
    \[y\left(x\right) = C_1 v_1\left(x\right) + C_2 v_2\left(x\right) + v\left(x\right), \mathspace \forall C_1, C_2 \in\mathbb{R}\]
}
}

\end{document}
