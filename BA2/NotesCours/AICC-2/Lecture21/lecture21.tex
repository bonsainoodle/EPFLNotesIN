\documentclass[a4paper]{article}

% Expanded on 2022-05-10 at 15:12:04.

\usepackage{../../style}

\title{AICC 2}
\author{Joachim Favre}
\date{Mardi 10 mai 2022}

\begin{document}
\maketitle

\lecture{21}{2022-05-10}{Let's make crop circles in that field}{
}

\subsection{Finite fields}
\parag{Definition: Field}{
    A \important{field} (in French, ``un corps'') $\left(K, +, \times\right)$ is such that:
    \begin{itemize}
        \item $\left(\mathcal{K}, +\right)$ is a commutative group with identity element $0$.
        \item $\left(\mathcal{K} \setminus \left\{0\right\}, \times\right)$ is a commutative group.
        \item The operations $+$ and $\times$ follow the distributive property:
              \[a \times\left(b + c\right)= a \times b + a \times c, \mathspace \forall a, b, c \in \mathcal{K}\]
    \end{itemize}

    \subparag{Remark}{
        \begin{itemize}[left=0pt]
            \item If $\mathcal{K}$ is finite, then $\left(\mathcal{K}, +, \times\right)$ is a \important{finite field}.
            \item $ab$ is a shorthand notation for $a \times b$.
            \item $a - b$ is a shorthand notation for $a + \left(-b\right)$, where $\left(-b\right)$ is the additive inverse of $b$.
            \item Let $n$ to be an integer, and $b \in \mathcal{K}$. We define $nb = b + b + \ldots + b$, where $b$ is added $n$ times.
        \end{itemize}
    }
}

\parag{Properties}{
    Let $\left(\mathcal{K}, +, \cdot\right)$ be a field with $0$ and $1$ being its additive and multiplicative identity, respectively. Then:
    \begin{enumerate}
        \item $0\cdot x = 0$ for all $x \in \mathcal{K}$.
        \item Zero-product property: $xy = 0 \implies x = 0 \lor y = 0$.
        \item For all $x \in \mathcal{K}, k \in \mathbb{Z}$, we have $x^k = 0 \implies x = 0$.
        \item $\left(-1\right)\cdot x= -x$.
        \item $\left(a + b\right)^2 = a^2 + 2ab + b^2$
    \end{enumerate}
}

\parag{Definition: Characteristic}{
We know that every finite field has a multiplicative identity, let's call it 1.

The order of 1 with respect to $+$, is called the \important{characteristic} of the finite field.

\subparag{Observation}{
Let $p$ be the characteristic of a finite field. Then, we can see that for any element $a$:
\[\underbrace{a + a + \ldots + a}_{\text{$p$ times}} = a\left(1 + 1 + \ldots + 1\right) = a\cdot 0 = 0\]
}
}

\parag{Theorem}{
    The characteristic of a finite field $\left(F, +, \cdot\right)$ is a prime number.
}


\parag{Definition: Isomorphism}{
    An \important{isomorphism} between two finite fields $\mathbb{F} = \left(\mathcal{F}, +, \times\right)$ and $\mathbb{K} = \left(\mathcal{K}, \oplus, \otimes\right)$ is a bijection $\phi : \mathcal{F} \mapsto \mathcal{K}$ such that, for all $a, b \in \mathcal{F}$:
    \[\phi\left(a + b\right) = \phi\left(a\right) \oplus \phi\left(b\right)\]
    \[\phi\left(a \times b\right) = \phi\left(a\right) \otimes \phi\left(b\right)\]

    We say that $\mathbb{F}$ and $\mathbb{K}$ are \important{isomorphic} if there exists an isomorphism between them.

    \subparag{Intuition}{
        Two finite fields are isomorphic if we can make them be the same by modifying the name of the elements.
    }
}

\parag{Theorem}{
    \begin{enumerate}[left=0pt]
        \item The cardinality of a finite field is an integer power of its characteristic (hence all finite fields have cardinality $p^m$ for some prime $p$ and some integer $m$).
        \item All finite fields of the same cardinality are isomorphic.
        \item For every prime number $p$ and positive integer $m$, there is a finite field of cardinality $p^m$.
    \end{enumerate}

    \subparag{Implication}{
        \begin{itemize}[left=0pt]
            \item There is exactly one finite field of cardinality $k$ if $k = p^m$, and none if the cardinality is not an integer power of a prime.
            \item $\left(\mathbb{Z} / k\mathbb{Z}, +, \cdot\right)$ is a finite field if and only if $k = p$ for some prime $p$.
            \item A field that has $p$ elements is isomorphic to $\left(\mathbb{Z} / p\mathbb{Z}, +, \cdot\right)$
        \end{itemize}
    }

    \subparag{Remark}{
        The field with $p^m$ elements is denoted by $\mathbb{F}_{p^m}$.
    }
}

\subsection{Vector spaces}
\parag{Definition: Vector space}{
    $\left(\mathcal{V}, +, \cdot\right)$, where $\mathcal{V}$ is a non-empty set $\mathcal{V}$, is a \important{vector space} over a field $\mathbb{F}$ if $\left(\mathcal{V}, +\right)$ is a commutative group, if the scalar multiplication operator $\cdot$ follows the following properties:
    \begin{enumerate}
        \item Closure: $\alpha \bvec{v} \in \mathcal{V}$ for all $\alpha \in \mathbb{F}$ and $\bvec{v} \in \mathcal{V}$.
        \item Associativity: $\alpha\left(\beta \bvec{v}\right) = \left(\alpha \beta\right)\bvec{v}$
        \item Identity element: there exists $1 \in \mathbb{F}$ such that $1 \bvec{v} = \bvec{v}$
    \end{enumerate}
    and if we have the following distributivity laws:
    \begin{enumerate}
        \item $\alpha\left(\bvec{u} + \bvec{v}\right) = \alpha \bvec{u} + \alpha \bvec{v}$
        \item $\left(\alpha + \beta\right)\bvec{v} = \alpha \bvec{v} + \beta \bvec{v}$
    \end{enumerate}
}

\end{document}
