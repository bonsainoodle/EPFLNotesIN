\documentclass[a4paper]{article}

% Expanded on 2021-11-17 at 10:15:20.

\usepackage{../../style}

\title{Analyse 1}
\author{Joachim Favre}
\date{Mercredi 17 novembre 2021}

\begin{document}
\maketitle

\lecture{16}{2021-11-17}{Continuité}{
}

\subsection{Fonctions continues}
\subsubsection{Continuité}
\parag{Définition de la continuité}{
    Une fonction $f : E \mapsto F$ est continue en un point $x_0 \in E$ si
    \[\lim_{x \to x_0} f\left(x\right) = f\left(x_0\right)\]

    Cela nous donne trois conditions pour qu'une fonction soit continue:
    \begin{enumerate}
        \item $\lim\limits_{x \to x_0} f\left(x\right) = \ell \in \mathbb{R}$ existe
        \item $f$ est bien définie en $x = x_0$
        \item $\lim\limits_{x \to x_0} f\left(x\right) = f\left(x_0\right)$
    \end{enumerate}
}

\parag{Critère de Cauchy pour les fonctions continues}{
    $f : E \mapsto F$ définie au voisinage de $x = x_0$ et en $x_0$. $f$ est continue en $x = x_0$ si et seulement si:
    \[\forall \epsilon > 0 \ \exists \delta > 0 \telque \forall x_1, x_2 \in \left\{x \in E \telque \left|x - x_0\right| \leq \delta\right\} \text{ on a } \left|f\left(x_1\right) - f\left(x_2\right)\right| \leq \epsilon\]

    \subparag{Reformulation}{
        En d'autres mots, pour tout écart vertical $\epsilon$, il existe un écart horizontal $\delta$ tel que pour toute paire de nombre, où les deux nombres sont delta-proches de $x_0$, leur image sont epsilon-proches.
    }
}

\parag{Continuité sur le côté}{
    $f : E \mapsto F$ est dite continue à droite en $x_0 \in E$ si :
    \[\lim_{x \to x_0^+} f\left(x\right) = f\left(x_0\right)\]

    Mêmement, $f : E \mapsto F$ est dite continue à \textcolor{red}{gauche} en $x_0 \in E$ si :
    \[\lim_{{\color{red}x \to x_0^-}} f\left(x\right) = f\left(x_0\right)\]

    \subparag{Remarque}{
        On voit que $f$ est continue en $x = x_0$ si et seulement si elle est continue à gauche et à droite.
    }

}

\parag{Opérations algébriques sur les fonctions continues}{
    Si $f$ et $g$ sont continues en $x = x_0$, alors:
    \begin{enumerate}
        \item $\alpha f + \beta g$ est continue en $x = x_0$ pour tout $\alpha, \beta \in \mathbb{R}$.
        \item $f\cdot g$ est continue en $x = x_0$.
        \item $\frac{f}{g}$ est continue en $x = x_0$ si $g\left(x_0\right) \neq 0$ (puisque $g$ est continue, ça veut donc dire que la limite n'est pas 0 en $x_0$ donc c'est effectivement tout bon).
    \end{enumerate}

    De plus, soient $f : E\mapsto F$ et $g : G \mapsto H$ où $f\left(E\right) \subset G$. Si $f$ est continue en $x_0 \in E$ et $g$ est continue en $f\left(x_0\right) \in G$, alors $\left(g \circ f\right)$ est continue en $x_0$.

    \subparag{Remarque}{
        Si $\left(g \circ f\right)$ est continue en $x = x_0$, alors on ne sait rien sur la continuité de $f$ en $x_0$ et de $g$ en $f\left(x_0\right)$.
    }

}

\subsubsection{Prolongement par continuité d'une fonction en un point}
\parag{Définition}{
    Soit $f : E \mapsto F$ une fonction telle que $c \not\in E$ ($f$ n'est pas définie en $x = c$) et:
    \[\lim_{x \to c} f\left(x\right) \in \mathbb{R} \text{ existe}\]

    Alors, la fonction $\hat{f} : E \cup \left\{c\right\} \mapsto F$:
    \begin{functionbypart}{\hat{f}\left(x\right)}
        f\left(x\right), \mathspace x \in E \\
        \lim\limits_{x \to c} f\left(x\right), \mathspace x = c
    \end{functionbypart}

    Cette fonction est appelée \important{le prolongement par continuité} de $f$ au point $x = c$. S'il existe, un tel prolongement est unique (puisque la limite est unique), et la fonction $\hat{f}$ est continue en $x = c$.
}

\subsubsection{Fonctions continues sur un intervalle}
\parag{Définition}{
Une fonction $f : I \mapsto F$ --- où $I$ est un intervalle ouvert non-vide --- est \important{continue sur $I$} si $f$ est continue en tous point point $x \in I$.

Une fonction $f : \left[a, b\right] \mapsto F$, avec $a < b$, est \important{continue sur $\left[a,b\right]$} si elle est continue sur $\left]a, b\right[ $ et continue à droite en $x = a$ et à gauche en $x = b$.
}

\parag{Théorème (très important)}{
    Soient $a < b \in \mathbb{R}$ et $f : \left[a, b\right] \mapsto F$ une fonction continue sur l'intervalle fermé et borné $\left[a, b\right]$.

    Alors, $f$ atteint son infimum et son supremum sur $\left[a, b\right]$.
}

\parag{Théorème de la valeur intermédiaire (TVI)}{
    Soient $a < b \in \mathbb{R}$ et $f : \left[a,b\right]  \mapsto \mathbb{R}$ une fonction continue.

    Alors, $f$ atteint son suprémum, son infimum, et toutes les valeurs comprises entre les deux.

    \subparag{Remarque 1}{
        Une autre écriture de ce théorème est de dire que
        \[f\left(\left[a, b\right]\right) = \left[\min_{\left[a,b\right]} f\left(x\right), \max_{\left[a,b\right]} f\left(x\right)\right]\]
    }

    \subparag{Remarque 2}{
        Une autre écriture équivalente du théorème est:
        \[\forall c \in \left[\min_{\left[a,b\right]} f\left(x\right), \max_{\left[a,b\right]} f\left(x\right)\right], \mathspace \exists x \telque f\left(x\right) = c\]

        Il existe au moins un tel $x$, il pourrait y en avoir plus.
    }
}

\parag{Corollaire 1}{
Soient $a < b \in \mathbb{R}$ et $f: \left[a,b\right] \mapsto \mathbb{R}$ une fonction continue telle que $f\left(a\right)f\left(b\right) < 0$ ($f\left(a\right)$ et $f\left(b\right)$ sont de signes opposés et sont non-nuls).

Alors, il existe au moins un point $c \in \left]a, b\right[ $ tel que $f\left(c\right) = 0$.

\subparag{Utilité}{
    On peut utiliser ce théorème pour démontrer qu'au moins une solution existe.
}
}

\end{document}
