\documentclass[a4paper]{article}

% Expanded on 2022-03-15 at 13:45:36.

\usepackage{../../style}

\title{Analyse 2}
\author{Joachim Favre}
\date{Mercredi 16 mars 2022}

\begin{document}
\maketitle

\lecture{8}{2022-03-16}{J'ai failli ajouter un d à ce nom de théorème}{
}

\parag{Définition: Adhérence}{
    Soit $E \subset \mathbb{R}^n$ un sous-ensemble non-vide.

    L'intersection de tous les sous-ensembles fermés contenant $E$ est appelée \important{l'adhérence} de $E$. $\bar{E}$ est la notation de l'adhérence de $E$ dans $\mathbb{R}^n$.

    \subparag{Remarque}{
        On voit que si $E \subset \mathbb{R}^n$ est fermé, alors on a $E = \bar{E}$ par définition.
    }
}

\parag{Définition: Frontière}{
    Soit un ensemble $E \subset \mathbb{R}^n$ non vide, où $E \neq \mathbb{R}^n$.

    Un point $\bvec{x} \in \mathbb{R}^n$ est un \important{point de frontière} de $E$ si toute boule ouverte de centre $\bvec{x}$ contient au moins un point de $E$ et au moins un point de $CE$.

    L'ensemble des points de frontière de $E$ s'appelle la \important{frontière} de $E$, notée $\partial E$.

    Nous définissons aussi les deux frontières suivantes:
    \[\partial \o = \o, \mathspace \partial \mathbb{R}^n = \o\]

}

\parag{Propriétés}{
    Soit $E \subset \mathbb{R}^n$ non-vide. Alors, nous avons les propriétés suivantes:
    \begin{enumerate}
        \item $\mathring{E} \cap \partial E = \o$
        \item $\mathring{E} \cup \partial E = \bar{E}$
        \item $\partial E = \bar{E} \setminus \mathring{E}$
    \end{enumerate}
}

\subsection{Suites dans $\mathbb{R}^n$}

\parag{Définition: Suite dans $\mathbb{R}^n$}{
Une suite d'éléments de $\mathbb{R}^n$ est une application $f: \mathbb{N} \mapsto \mathbb{R}^n$:
\[f: k \mapsto \bvec{x_k} = \begin{pmatrix} x_{1k} & \cdots & x_{nk} \end{pmatrix} \in \mathbb{R}^n\]

$\left\{\bvec{x_k}\right\}_{k=0}^{\infty}$ est une suite d'éléments de $\mathbb{R}^n$.
}

\parag{Définition: Convergence}{
Une suite $\left\{x_k\right\}_{k=0}^{\infty}$ est \important{convergente} et admet pour \important{limite} $\bvec{x} \in \mathbb{R}^n$ si pour tout $\epsilon > 0$, il existe $k_0 \in \mathbb{N}$ tel que $\forall k \geq k_0$, nous avons:
\[\left\|\bvec{x_k} - \bvec{x}\right\| \leq \epsilon\]

Nous notons:
\[\lim_{k \to \infty} \bvec{x_k} = \bvec{x}\]

\subparag{Remarque}{
    Notez que $\left\|\bvec{x_k} - \bvec{x}\right\| \leq \epsilon$ est équivalent à $\bvec{x_k} \in \bar{B\left(\bvec{x}, \epsilon\right)}$.
}
}

\parag{Proposition}{
    Soit $\bvec{x} = \begin{pmatrix} x_1 & \cdots & x_n \end{pmatrix} \in \mathbb{R}^n$. Alors:
    \[\lim_{k \to \infty} \bvec{x_k} = \bvec{x} \iff \lim_{k \to \infty} x_{j, k} = x_j, \mathspace \forall j = 1, \ldots, n\]
}


\parag{Définition: Suite bornée}{
    Une suite $\left\{\bvec{x_k}\right\}$ est \important{bornée} s'il existe un $M > 0$ tel que $\left\{\bvec{x_k}\right\}$ est contenue dans la boule fermée $\bar{B\left(\bvec{0}, M\right)}$.
}


\parag{Propriétés}{
    \begin{enumerate}[left=0pt]
        \item La limite d'une suite $\left\{\bvec{x_k}\right\}$, si elle existe, est unique.
        \item Toute suite convergente $\left\{\bvec{x_k}\right\}$ est bornée.
    \end{enumerate}
}

\parag{Théorème de Bolzano-Weierstrass}{
    Nous pouvons extraire une sous-suite convergente de toute suite bornée $\left\{\bvec{x_k}\right\} \subset \mathbb{R}^n$.
}

\parag{Théorème: Lien entre les suites dans $\mathbb{R}^n$ et la topologie}{
    Un sous-ensemble non-vide $E \subset \mathbb{R}^n$ est fermé si et seulement si toute suite $\left\{\bvec{x_k}\right\} \subset E$ d'éléments de $E$ qui converge a pour limite un élément de $E$.
}

\parag{Remarque}{
    Pour construire l'adhérence $\bar{E}$ d'un sous-ensemble non-vide $E \subset \mathbb{R}^n$, il faut et il suffit d'ajouter les limites de toutes suites convergentes d'éléments de $E$.
}

\parag{Définition: Ensemble borné}{
    Un ensemble $E \subset \mathbb{R}^n$ est \important{borné} s'il existe un $M > 0$ tel que:
    \[E \subset \bar{B\left(\bvec{0}, M\right)}\]
}


\parag{Définition: Ensemble compact}{
    Un sous-ensemble non-vide $E \subset \mathbb{R}^n$ est \important{compact} s'il est fermé et borné.
}

\parag{Note} {
    Une boule fermée est compacte.
}

\parag{Définition: Recouvrement}{
    Un \important{recouvrement} d'un ensemble $E$ est défini par:
    \[E \subset \bigcup_{i \in I} A_i, \mathspace A_i \subset \mathbb{R}^n \text{ ouverts}\]

    Notez que $I$ peut être innombrable.
}


\parag{Théorème de Heine-Borel-Lebesgue}{
    Un sous-ensemble non-vide $E \subset \mathbb{R}^n$ est compact si et seulement si de \textit{tout} recouvrement de $E$ par des sous-ensembles ouverts dans $\mathbb{R}^n$ on peut extraire une famille finie d'ensembles qui forment un recouvrement de $E$.
}

\end{document}
