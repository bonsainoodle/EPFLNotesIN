\documentclass[a4paper]{article}

% Expanded on 2022-05-19 at 11:10:14.

\usepackage{../../style}

\title{Analyse 2}
\author{Joachim Favre}
\date{Mercredi 18 mai}

\begin{document}
\maketitle

\lecture{23}{2022-05-18}{Fubini on steroids}{
}

\subsection{Intégrales sur un ensemble borné}
\parag{Définition: Fonction intégrable sur un ensemble borné quelconque}{
    Soit $E \subset \mathbb{R}^n$ un ensemble borné. Clairement, puisqu'il est borné, il existe un pavé fermé tel que $E \subset P \subset\mathbb{R}^n$. Soit aussi $f :E \mapsto \mathbb{R}$ une fonction bornée sur $E$.

    Posons maintenant la fonction suivante:
    \begin{functionbypart}{\hat{f}\left(\bvec{x}\right)}
        f\left(\bvec{x}\right), \mathspace \text{si } \bvec{x} \in E \\
        0, \mathspace \text{si } \bvec{x} \in P \setminus E
    \end{functionbypart}

    La fonction $f$ est \important{intégrable} sur $E$, si $\hat{f}$ est intégrable sur $P$. Dans ce cas, on pose:
    \[\int_{E} f\left(\bvec{x}\right) d \bvec{x} \over{=}{déf} \int_P \hat{f}\left(\bvec{x}\right)d \bvec{x}\]
}

\parag{Définition: Frontière régulière}{
    Une frontière est dite régulière (de mesure nulle) si, pour tout $\epsilon > 0$, il existe un ensemble de pavé fermés $\left\{q_1, q_2, \ldots\right\}$ tels que:
    \[\sum_{i \in I}^{} \left|q_i\right| < \epsilon \mathspace \text{et} \mathspace \partial E \subset \bigcup_{i \in I} q_i\]
}

\parag{Théorème}{
    Si $f : E \mapsto \mathbb{R}$ est bornée sur $E$, continue sur l'intérieur $\mathring{E}$, et la frontière $\partial E$ est assez régulière, alors $f\left(\bvec{x}\right)$ est intégrable sur $E$.
}

\parag{Théorème de Fubini pour les domaines à frontière régulière}{
    \begin{enumerate}[left=0pt]
        \item Soient:
              \begin{itemize}
                  \item $\left[a, b\right] \subset \mathbb{R}$ un intervalle, où $a < b$.
                  \item $\phi_1, \phi_2 : \left[a, b\right]\mapsto \mathbb{R}$ deux fonctions continues telles que $\phi_1\left(x\right) < \phi_2\left(x\right)$ pour tout $x \in \left]a, b\right[ $.
                  \item $D = \left\{\left(x, y\right) \in \mathbb{R}^2 : a < x < b, \phi_1\left(x\right) < y < \phi_2\left(x\right)\right\}$ (appelé le domaine à frontière régulière de type 1).
              \end{itemize}

              Alors, pour tout fonction continue $f: \bar{D} \mapsto \mathbb{R}$, nous avons:
              \[\iint_D f\left(x, y\right) dx dy = \int_{a}^{b} \left(\int_{\phi_1\left(x\right)}^{\phi_2\left(x\right)} f\left(x, y\right)dy\right)dx\]

              Le choix du sens des variables d'intégration \textit{ne peut pas} se faire arbitrairement.
        \item Soient:
              \begin{itemize}
                  \item $\left[c, d\right] \subset \mathbb{R}$ un intervalle, où $c < d$.
                  \item $\psi_1, \psi_2 : \left[c, d\right] \mapsto \mathbb{R}$ deux fonctions continues telles que $\psi_1\left(y\right) < \psi_2\left(y\right)$ pour tout $y \in \left]c, d\right[ $.
                  \item $D = \left\{\left(x, y\right) \in \mathbb{R}^2 : c < y < d, \psi_1\left(y\right) < x < \psi_2\left(y\right)\right\}$ (appelé le domaine à frontière régulière de type 2).
              \end{itemize}

              Alors, pour toute fonction continue $f: \bar{D} \mapsto \mathbb{R}$, nous avons:
              \[\iint_D f\left(x, y\right)dxdy = \int_{c}^{d} \left(\int_{\psi_1\left(y\right)}^{\psi_2\left(y\right)} f\left(x, y\right)dx\right)dy\]
              Le choix du sens des variables d'intégration \textit{ne peut pas} se faire arbitrairement.
    \end{enumerate}
}

\end{document}
