\documentclass[a4paper]{article}

% Expanded on 2022-03-31 at 17:56:02.

\usepackage{../../style}

\title{AICC-2}
\author{Joachim Favre}
\date{Jeudi 31 mars 2022}

\begin{document}
\maketitle

\lecture{12}{2022-03-31}{Prime time}{
}

\parag{Definition: Prime number}{
    A \important{prime number} (or a \important{prime}) is an integer $> 1$ that has no positive divisors other than 1 and itself.

    Numbers that are not prime are called \important{composite.}
}

\parag{Theorem: Prime factorisation}{
    Every integer greater than 1 has a unique prime factorisation (except for order).
}

\parag{Definition: Divide}{
We write $a \divides b$ if $a$ divides $b$, i.e. if $\exists c \in \mathbb{Z}$ such that:
\[b = ac\]

Equivalently in terms of prime factors:
\[a = p_1^{\gamma_1} \cdots p_n^{\gamma_n}\]
\[b = p_1^{\widetilde{\gamma}_1} \cdots p_n^{\widetilde{\gamma}_n} \cdot p_{n+1}^{\widetilde{\gamma}_{n+1}} \cdots p_m^{\widetilde{\gamma}_m}\]
where $\widetilde{\gamma}_i \geq \gamma_i$, for $i = 1, \ldots, n$.
}

\parag{Definition: GCD}{
    Let $a$ and $b$ be integers, not both zero. The largest integer that divides both is called the \important{greatest common divisor} (GCD) of $a$ and $b$. It is denoted by $\gcd\left(a, b\right)$.
}

\parag{Theorem}{
Let $a$ and $b$ be positive integers, not both zero, and let $p_1, \ldots, p_k$ be the sequence of prime numbers that divide $a$ or $b$. We already know we can write:
\[a = p_1^{\alpha_1} \cdots p_k^{\alpha_k}\]
\[b = p_1^{\beta_1} \cdots p_k^{\beta_k}\]
where $\alpha_i \geq 0$ and $\beta_i \geq 0$.

Then, letting $\gamma_i = \min\left(\alpha_i, \beta_i\right)$, we have:
\[\gcd\left(a, b\right) = p_1^{\gamma_1} \cdots p_k^{\gamma_k}\]
}

\parag{Definition: Coprime numbers}{
    Two numbers $a$ and $b$ are called \important{coprime} (or relatively prime, or mutually prime) when:
    \[\gcd\left(a, b\right) = 1\]

    This is equivalent to saying that they have no common prime factor.
}

\parag{Theorem}{
    Let $p$ be a prime number, and let $a$ be an integer such that $0 < a < p$. Then:
    \[\gcd\left(p, a\right) = 1\]

}

\parag{Properties}{
    Let $a$, $b$, and $c$ be integers.
    \begin{enumerate}
        \item If $ab \divides c$, then $a \divides c$ and $b \divides c$.
        \item If $a \divides c$ and $b \divides c$ and $\gcd\left(a, b\right) = 1$, then $ab \divides c$.
    \end{enumerate}
}

\parag{Theorem: Infinite number of primes}{
    There are an infinite number of primes.
}

\end{document}
