\documentclass[a4paper]{article}

% Expanded on 2022-03-24 at 17:37:31.

\usepackage{../../style}

\title{AICC-2}
\author{Joachim Favre}
\date{Jeudi 24 mars 2022}

\begin{document}
\maketitle

\lecture{10}{2022-03-24}{Diagrams which took me way too much time}{
}

\parag{One-time pad}{
    One-time pad is being used in practice, but it's very expensive. The key is very long, and it has to be sent every time. Moreover, the key has to be sent (and we cannot do it over a public channel), this can be a complicated task.

    We reach the minimum length of a key for any binary crpytosystem if the plaintext cannot be compressed and is of length $n$, and we want perfect secrecy. Then, since $H\left(K\right) \geq H\left(T\right)$, we need at least $n$ bits. Thus, the one-time pad is optimal for perfect secrecy.
}

\subsection{Public channel}

\parag{Definition}{
    Let $p$ be a prime number. If $\left\{g^i : i = 0, 1, \ldots, p-2\right\}$ is a permutation of $\left\{1, 2, \ldots, p-1\right\}$, then $g$ is called a \important{generator}.
}

\parag{Diffie-Hellman cryptosystem}{
    First, Alice and Bob agree publicly on a prime number $p$ and a generator $g$. Then, Alice picks a secret $a \in \left\{0, 1, \ldots, p-1\right\} = \mathcal{A}$, and calculates $A = g^a \Mod p$. Similarly, Bob picks a secret $b \in \mathcal{A}$, and calculates $B = g^b \Mod p$. Then, both write $A$ and $B$ in a public directory.

    Now, Alice gets $B$ from the directory, and she computes:
    \[B^a = \left(g^{b}\right)^{a} \Mod p = g^{ab} \Mod p\]

    Similarly, Bob get $A$ from the directory, and computes:
    \[A^b = \left(g^{a}\right)^{b} \Mod p = g^{ab} \Mod p\]

    They thus get the same key, and can use it for sending private messages.
}

\parag{Definition: One-way function}{
    A \important{one-way function} is a function for which one way is easy to compute, but its reciprocal is really hard.

    \subparag{Example}{
        $g^a$, where $g$ is known but $a$ is unknown, is conjectured to be one-way (we can use the square and multiply algorithm to compute it very efficiently but the other way around we face the discrete logarithm problem). This is not completely proven yet, but we think (and we hope) that it is true.
    }
}

\parag{Definition: Trapdoor one-way function}{
    A \important{trapdoor one-way function} is a one-way function, but for which the reciprocal is easy to compute when we have the key $k$ to a trapdoor.

    \subparag{Remark}{
        $g^a$, where $g$ is known but $a$ is unknown, is \textit{not} a trapdoor one-way function.
    }
}

\parag{Multiplicative inverse}{
    If $a \cdot i = 1 \Mod m$, then we say that $a$ is the \important{multiplicative inverse} of $i$.
}

\parag{El Gamal's cryptosystem}{
    There is a ciphering system other than RSA that works with trapdoor one-way functions, named \important{El Gamal's crypto}.

    Alice and Bob choose a $g$ publicly. Then, Alice picks a random number $y$, and Bob a random number $x$. Alice publishes $g^y$ and Bob $g^x$. If Alice wants to send her message, she can compute $c = g^{xy}t\ \Mod\ \left|\mathcal{A}\right|$, where $g^{xy}$ can be easily computed as in the Diffie-Hellman cryptosystem. Then, Bob can decipher the message by using the multiplicative inverse of $g^{xy}$. Indeed, if $a$ is the multiplicative inverse of $g^{xy}$:
    \[ac = a\left(g^{xy} t\right) = a g^{xy} t = t\]
}

\end{document}
