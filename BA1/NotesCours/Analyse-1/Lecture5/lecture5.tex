\documentclass[a4paper]{article}

% Expanded on 2021-10-06 at 10:09:20.
\usepackage{../../style}

\title{Analyse 1}
\author{Joachim Favre}
\date{Mercredi 06 octobre 2021}

\begin{document}
\maketitle

\lecture{5}{2021-10-06}{Fin des nombres complexes}{
}

\parag{Définition de conjugaison}{
    Soit $z = a + ib \in \mathbb{C}$. Son \important{conjugué}, $\bar{z}$, est défini comme :
    \[\bar{z} \over{=}{\text{déf}} a - ib\]

    Si $z \neq 0$, on peut donc noter
    \[z^{-1} = \frac{a - ib}{a^2 + b^2} = \frac{\bar{z}}{\left|z\right|^2}\]

    Donc, puisque $zz^{-1} = 1$, on a que:
    \[\bar{z} z = \left|z\right|^2 \in \mathbb{R}\]
}

\parag{Conjugaison sous forme polaire}{
    Soit $z = \rho\left(\cos\left(\phi\right) + i\sin\left(\phi\right)\right)$. On a que
    \[\bar{z} = \rho\left(\cos\left(\phi\right) - i\sin\left(\phi\right)\right) = \rho\left(\cos\left(-\phi\right) + i\sin\left(-\phi\right)\right) \implies \bar{z} = \rho e^{-i\phi}\]

    Puisque $\sin\left(x\right)$ est impair et $\cos\left(x\right)$ est pair.
}

\parag{Propriétés de la conjugaison}{
    Pour $z, w \in \mathbb{C}$, on a:
    \begin{enumerate}[left=0pt]
        \item $\bar{z \pm w} = \bar{z} \pm \bar{w}$
        \item $\bar{z\cdot w} = \bar{z} \cdot \bar{w}$
        \item $\bar{\frac{z}{w}} = \frac{\bar{z}}{\bar{w}}$, $w \neq 0$
        \item $\left|\bar{z}\right| = \left|z\right|$
    \end{enumerate}

    De plus, puisque $z = a + ib$ et $\bar{z} = a - ib$, alors on a que
    \[\Re\left(z\right) = \frac{z + \bar{z}}{2}, \mathspace \Im\left(z\right) = \frac{z - \bar{z}}{2i}\]

    En particulier, si $\left|z\right| = 1$, on peut écrire $z$ sous la forme:
    \[z = \cos\left(\phi\right) + i\sin\left(\phi\right) = e^{i\phi}\]

    On peut en déduire la formule suivante $\forall \phi = \mathbb{R}$:
    \begin{systemofequations}{}
        &\ \cos\left(\phi\right) = \frac{e^{i\phi} + e^{-i\phi}}{2} \\
        &\ \sin\left(\phi\right) = \frac{e^{i\phi} - e^{-i\phi}}{2i}
    \end{systemofequations}
}

\subsection{Racines de nombres complexes}
\parag{Proposition}{
Si $w = \rho e^{i\phi} \in \mathbb{C}^*$, alors pour tout $n \in \mathbb{N}^*$:
\[\left\{z \in \mathbb{C}^* \telque z^{n} = w\right\} = \left\{\sqrt[n]{\rho}e^{i \frac{\phi + 2k\pi}{n}}, k = 0, 1 , 2, \ldots, n-1\right\}\]
}

\parag{Remarque}{
    En général, les racines n-ièmes de $w \in \mathbb{C}^*$ sont situées sur un cercle de rayon $\sqrt[n]{\left|w\right|}$ aux sommets d'un polygône régulier à $n$ côtés, puisque la différence entre chaque argument est toujours $\frac{2\pi}{n}$.

    L'orientation du polygône dépend de l'argument de $w$.
}

\subsection{Équations polynomiales dans $\mathbb{C}$}

\parag{Équation quadratique}{
    On veut trouver les solutions de $az^2 + bz + c = 0$ dans $\mathbb{C}$ avec $a, b, c \in \mathbb{C}$, et $a \neq 0$. En complétant le carré, on peut démontrer que la solution est toujours
    \[z = \frac{-b \pm \sqrt{b^2 - 4ac}}{2a}\]

    Où la racine carrée est celle d'un nombre complexe, et donc celle qu'on a définit plus haut.

    Si $b^2 - 4ac = 0$, alors il existe une seule solution $z = -\frac{b}{2a}$. Sinon, si $b^2 - 4ac \neq 0$, alors il existe toujours deux solutions complexes.
}

\parag{Théorème fondamental de l'algèbre}{
    Tout polynôme $P\left(z\right) = a_n z^n + a_{n - 1} z^{n-1} + \ldots + a_1 z + a_0$, avec $a_n, \ldots, a_0 \in \mathbb{C}$ et $a_n \neq 0$, s'écrit sous la forme
    \[P\left(z\right) = a_n \left(z - z_1\right)\left(z - z_2\right)\ldots\left(z - z_n\right)\]

    où $z_1, \ldots, z_n \in \mathbb{C}$ (peut-être avec des répétitions). Sans répétition nous avons
    \[P\left(z\right) = a-n \left(z - w_1\right)^{m_1} \left(z - w_2\right)^{m_2}\ldots\left(z - w_p\right)^{m_p}\]

    où $w_1, \ldots, w_p \in \mathbb{C}$ sont distincts et $m_1 + m_2 + \ldots + m_p = n$, avec $m_1, \ldots, m_p \in \mathbb{N}^*$. On dit que $m_i$ est la multiplicité de la racine $w_i$.

    \subparag{Remarque}{
        Ce n'est pas vrai dans $\mathbb{R}$.
    }
}

\subsection{Polynômes à coefficients réels}
\parag{Proposition}{
    Si $z \in \mathbb{C}$ est une racine de $P\left(z\right)$, un polynôme à \important{coefficients réels}, alors $\bar{z}$ l'est aussi.
}

\parag{Corolaire}{
    Tout polynôme non-constant à coefficients réels peut être factorisé en produit de polynômes à coefficients réels de degré 1 ou 2.
}

\subsection{Sous-ensemble du plan complexe}
\parag{Exemple 1}{
    Soit $z_0 \in \mathbb{C}$ et $r > 0$ (donc $r \in \mathbb{R}$, c'est sous-entendu quand on dit que $r > 0$, puisque les nombres complexes ne sont pas ordonnés). Considérons $\left\{z \in \mathbb{C} \telque \left|z - z_0\right| = r\right\}$. Cela trace un cercle sur le plan complexe. En effet:
    \[\left|z - z_0\right| = \left|x + iy - x_0 - iy_0\right| = \left|x - x_0 + i\left(y - y_0\right)\right| = \sqrt{\left(x - x_0\right)^2 + \left(y - y_0\right)^2} = r\]

    Ce qui est équivalent à
    \[\left(x - x_0\right)^2 + \left(y - y_0\right)^2 = r^2\]
    qui est l'équation du cercle de rayon $r$ et centre $\left(x_0, y_0\right)$.

    \imagehere{EnsembleCerclePlanComplexe.png}
}

\parag{Exemple 2}{
    Soit l'ensemble suivant
    \[\left\{z \in \mathbb{C}^* \telque \left(\frac{z}{\left|z\right|}\right)^3 = i\right\}\]

    Prenons $z = \rho e^{i\phi}$ avec $\rho > 0$. Donc,
    \[\left(\frac{\rho e^{i\phi}}{\rho}\right)^3 = i \implies e^{3i\phi} = i = e^{\frac{\pi}{2}i + 2k\pi i} \implies \phi \in \left\{\frac{\pi}{6}, \frac{\pi}{6} + \frac{2\pi}{3}, \frac{\pi}{6} + \frac{4\pi}{3}\right\}\]

    Puisque $\rho$ est arbitraire, on n'a pas de condition dessus, ce sont donc les trois demis droites (sans le point $z = 0$car $\rho > 0$) des trois angles de l'ensemble ci-dessus.

    \imagehere{EnsembleDemiDroitesPlanComplexe.png}
}

\end{document}
