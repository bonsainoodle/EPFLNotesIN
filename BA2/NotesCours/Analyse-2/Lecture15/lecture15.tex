\documentclass[a4paper]{article}

% Expanded on 2022-04-11 at 10:16:28.

\usepackage{../../style}

\title{Analyse 2}
\author{Joachim Favre}
\date{Lundi 11 avril 2022}

\begin{document}
\maketitle

\lecture{15}{2022-04-11}{Masterclass Jacob}{
}
\subsection{Fonctions à valeurs dans $\mathbb{R}^m$}

\parag{Définition: $k$-ème dérivée partielle}{
    Soit $E \subset \mathbb{R}^n$.

    La \important{$k$-ème dérivée partielle} de $\bvec{f}: E \mapsto \mathbb{R}^m$ en $\bvec{a} \in E$, est définie par:
    \[\frac{\partial \bvec{f}}{\partial x_k}\left(\bvec{a}\right) \over{=}{déf} \begin{pmatrix} \frac{\partial f_1}{\partial x_k}\left(\bvec{a}\right) \\ \vdots \\ \frac{\partial f_m}{\partial x_k}\left(\bvec{a}\right) \end{pmatrix}\]
    si chacune des fonctions $f_1, \ldots, f_m$ admet la dérivée partielle $\frac{\partial}{\partial x_k}$ en $\bvec{a}$.
}

\parag{Définition: Dérivée directionnelle}{
    Soit $E \subset \mathbb{R}^n$, et soit $\bvec{v} \in \mathbb{R}^n$ tel que $\bvec{v} \neq \bvec{0}$.

    La \important{dérivée directionnelle} de $\bvec{f}: E \mapsto \mathbb{R}^m$ suivant $\bvec{v}$ en $\bvec{a} \in E$, est:
    \[D \bvec{f}\left(\bvec{a}, \bvec{v}\right) \over{=}{déf} \begin{pmatrix} Df_1\left(\bvec{a}, \bvec{v}\right) \\ \vdots \\ Df_m\left(\bvec{a}, \bvec{v}\right) \end{pmatrix}\]
    si $Df_i\left(\bvec{a}, \bvec{v}\right)$ existent pour tout $i = 1, \ldots, m$.
}

\parag{Définition: Limite}{
    Soit $E \subset \mathbb{R}^{n}$.

    Une fonction $\bvec{f} : E \mapsto \mathbb{R}^m$ admet $\bvec{\ell} \in \mathbb{R}^m$ pour \important{limite} lorsque $\bvec{x}$ tend vers $\bvec{a}$ si $\forall \epsilon > 0$, $\exists \delta > 0$ tel que pour tout $\bvec{x} \in E$, on a:
    \[0 < \left\|\bvec{x} - \bvec{a}\right\| \leq \delta \implies \left\|f\left(\bvec{x}\right) - \bvec{\ell}\right\| \leq \epsilon\]

    \subparag{Remarque}{
        En particulier, nous avons:
        \[\lim_{\bvec{x} \to \bvec{a}} \bvec{f}\left(\bvec{x}\right) = \begin{pmatrix} \lim\limits_{\bvec{x} \to \bvec{a}} f_1\left(\bvec{x}\right) \\ \vdots \\ \lim\limits_{\bvec{x} \to \bvec{a}} f_m\left(\bvec{x}\right) \end{pmatrix} \]
    }
}

\parag{Définition: Dérivabilité}{
Soit $E \subset \mathbb{R}^n$.

$\bvec{f}: E \mapsto \mathbb{R}^m$ est \important{dérivable} au point $\bvec{a} \in E$ s'il existe une transformation linéaire $\bvec{L}_{\bvec{a}}: \mathbb{R}^n \mapsto \mathbb{R}^m$ et une fonction $\bvec{r}: E \mapsto \mathbb{R}^m$ telles que:
\[\bvec{f}\left(\bvec{x}\right) = \bvec{f}\left(\bvec{a}\right) + \bvec{L}_{\bvec{a}}\left(\bvec{x} - \bvec{a}\right) + \bvec{r}\left(\bvec{x}\right)\]

De plus, il faut aussi que:
\[\lim_{\bvec{x} \to \bvec{a}} \frac{\bvec{r}\left(\bvec{x}\right)}{\left\|\bvec{x} - \bvec{a}\right\|} = 0\]

Si $\bvec{f}$ est dérivable, alors $\bvec{L}_{\bvec{a}}: \mathbb{R}^n \mapsto \mathbb{R}^m$ est appelée la \important{différentielle} de $\bvec{f}$ en $\bvec{a}$.
}

\parag{Proposition: Dérivabilité pour chaque composante}{
Soit $E \subset \mathbb{R}^n$.

$\bvec{f} = \left(f_1, \ldots, f_m\right): E \mapsto \mathbb{R}^m$ est dérivable en $\bvec{a} \in E$ \textit{si et seulement si} chaque composante $f_i : E \mapsto \mathbb{R}$ est dérivable en $\bvec{a} \in E$ pour $i = 1, \ldots, m$. De plus, nous pouvons construire:
\[\bvec{L}_{\bvec{a}}\left(\bvec{v}\right) = \begin{pmatrix} L_{1, \bvec{a}}\left(\bvec{v}\right) \\ \vdots \\ L_{m, \bvec{a}}\left(\bvec{v}\right) \end{pmatrix}, \mathspace \bvec{v} \in \mathbb{R}^n, \bvec{v} \neq \bvec{0}\]
où $L_{i, \bvec{a}}\left(\bvec{v}\right)$ est la différentielle de $f_i$ calculée en $\bvec{a}$ et appliquée en $\bvec{v}$. En d'autres mots:
\[L_{i, \bvec{a}}\left(\bvec{v}\right) = D f_i\left(\bvec{a}, \bvec{v}\right) = \left<\nabla f_i\left(\bvec{a}\right), \bvec{v}\right>\]
}

\parag{Définition: Matrice Jacobienne}{
    Soit $E \subset \mathbb{R}^n$. Si $\bvec{f}: E \mapsto \mathbb{R}^m$ possède toutes ses dérivées partielles en $\bvec{a}\in E$, alors sa \important{matrice Jacobienne} (matrice de Jacobi) est définie par:
    \[J_{\bvec{f}}\left(\bvec{a}\right) \over{=}{déf} \begin{pmatrix} \dfrac{\partial f_1}{\partial x_1}\left(\bvec{a}\right) & \dfrac{\partial f_1}{\partial x_2}\left(\bvec{a}\right) & \cdots & \dfrac{\partial f_1}{\partial x_n}\left(\bvec{a}\right) \\ \dfrac{\partial f_2}{\partial x_1}\left(\bvec{a}\right) & \dfrac{\partial f_2}{\partial x_2}\left(\bvec{a}\right) & \cdots & \dfrac{\partial f_2}{\partial x_n}\left(\bvec{a}\right) \\ \vdots & \vdots & \ddots & \vdots \\ \dfrac{\partial f_m}{\partial x_1}\left(\bvec{a}\right) & \dfrac{\partial f_m}{\partial x_2}\left(\bvec{a}\right) & \cdots & \dfrac{\partial f_m}{\partial x_n}\left(\bvec{a}\right) \end{pmatrix} = \begin{pmatrix} \nabla f_1\left(\bvec{a}\right) \\ \nabla f_2\left(\bvec{a}\right) \\ \vdots \\ \nabla f_m\left(\bvec{a}\right) \end{pmatrix}\]

    \subparag{Remarque 1}{
        Si $\bvec{f}$ est dérivable en $\bvec{a} \in E$, alors nous avons:
        \[J_{\bvec{f}}\left(\bvec{a}\right) = \begin{pmatrix} L_{1, \bvec{a}}\left(\bvec{e_1}\right) & \cdots & L_{1, \bvec{a}}\left(\bvec{e_n}\right) \\ \vdots & \ddots & \vdots \\ L_{m, \bvec{a}}\left(\bvec{e_1}\right)  & \cdots & L_{m, \bvec{a}}\left(\bvec{e_n}\right) \end{pmatrix} \]

        Ainsi, si $\bvec{f}$ est dérivable en $\bvec{a}$, la matrice Jacobienne nous donne la matrice de la différentielle de $\bvec{f}$.
    }

    \subparag{Remarque 2}{
        Si $\bvec{f}$ est dérivable, alors:
        \begin{multiequality}
            D \bvec{f}\left(\bvec{a}, \bvec{v}\right) =\ & \begin{pmatrix} D f_1\left(\bvec{a}, \bvec{v}\right) \\ \vdots \\ D f_m\left(\bvec{a}, \bvec{v}\right) \end{pmatrix}  \\
            =\ & \begin{pmatrix} \left<\nabla f_1\left(\bvec{a}\right), \bvec{v}\right> \\ \vdots \\ \left<\nabla f_m\left(\bvec{a}\right), \bvec{v}\right> \end{pmatrix}  \\
            =\ & \underbrace{\begin{pmatrix}  & \nabla f_1\left(\bvec{a}\right) &  \\  & \vdots &  \\  & \nabla f_m\left(\bvec{a}\right) &  \end{pmatrix}}_{m \times n} \underbrace{\begin{pmatrix} v_1 \\ \vdots \\ v_n \end{pmatrix}}_{n \times 1}  \\
            =\ & \left(J_{\bvec{f}}\left(\bvec{a}\right)\right)\cdot \bvec{v}
        \end{multiequality}

        Cela rejoint notre première remarque.
    }
}

\parag{Définition: Jacobien}{
Soit $E \subset \mathbb{R}^n$. Si $\bvec{f}: E \mapsto \mathbb{R}^m$ possède toutes ses dérivées partielles en $\bvec{a}\in E$, et si $m = n$, alors on définit le \important{déterminant de Jacobi}, aussi appelé \important{le Jacobien}, de $\bvec{f}$ en $\bvec{a}$ comme:
\[\left|J_{\bvec{f}}\left(\bvec{a}\right)\right| = \det\left(J_{\bvec{f}}\left(\bvec{a}\right)\right) \over{=}{déf} \det\begin{pmatrix} \dfrac{\partial f_1}{\partial x_1}\left(\bvec{a}\right) & \cdots & \dfrac{\partial f_1}{\partial x_n}\left(\bvec{a}\right) \\ \vdots & \ddots & \vdots \\ \dfrac{\partial f_m}{\partial x_1}\left(\bvec{a}\right) & \cdots & \dfrac{\partial f_m}{\partial x_n}\left(\bvec{a}\right) \end{pmatrix}\]
}

\parag{Remarque}{
    Soit $E \subset \mathbb{R}^n$.

    D'après la définition, nous avons, pour toute fonction $f: E \mapsto \mathbb{R}$ (attention, $\mathbb{R}$ et non pas $\mathbb{R}^n$) de classe $C^2$ sur $E$:
    \[J_{\left(\nabla f\right)^T}\left(\bvec{x}\right) = Hess\left(f\right)\left(\bvec{x}\right)\]
}

\subsection{Application des matrices Jacobiennes}

\parag{Théorème}{
    Soient $A, B$ deux ensembles tels que $A \subset \mathbb{R}^n$ et $\bvec{g}\left(A\right) \subset B \subset \mathbb{R}^p$. Soient $\bvec{g}: A \mapsto \mathbb{R}^p$ et $\bvec{f} : B \mapsto \mathbb{R}^q$. En d'autres mots, nous avons:
    \[\mathbb{R}^n \over{\mapsto}{$\bvec{g}$} \mathbb{R}^p \over{\mapsto}{$\bvec{f}$} \mathbb{R}^q\]

    Soient $\bvec{a} \in A$ et $\bvec{b} = \bvec{g}\left(\bvec{a}\right) \in B$. Supposons que $\bvec{g}$ est dérivable en $\bvec{a}$ avec la différentielle $L_{\bvec{g}, \bvec{a}}$ et $\bvec{f}$ est dérivable en $\bvec{b}$ avec la différentielle $L_{\bvec{f}, \bvec{b}}$.

    \vspace{1em}

    Alors $\bvec{f} \circ \bvec{g}$ est dérivable en $\bvec{a}$, et on a:
    \begin{enumerate}
        \item $\displaystyle \bvec{L}_{\bvec{f} \circ \bvec{g}, \bvec{a}} = \bvec{L}_{\bvec{f}, \bvec{b}} \circ \bvec{L}_{\bvec{g}, \bvec{a}}$
        \item $\displaystyle J_{\bvec{f}\circ \bvec{g}}\left(\bvec{a}\right) = J_{\bvec{f}}\left(\bvec{g}\left(\bvec{a}\right)\right) \cdot J_{\bvec{g}}\left(\bvec{a}\right)$
        \item Si $n = p = q$, alors $\displaystyle \left|J_{\bvec{f} \circ \bvec{g}}\left(\bvec{a}\right)\right| = \left|J_{\bvec{f}}\left(\bvec{g}\left(\bvec{a}\right)\right)\right|\cdot \left|J_{\bvec{g}}\left(\bvec{a}\right)\right|$
    \end{enumerate}
}

\end{document}
