\documentclass[a4paper]{article}

% Expanded on 2021-12-07 at 08:34:20.

\usepackage{../../style}

\title{AICC}
\author{Joachim Favre}
\date{Mardi 07 décembre 2021}

\begin{document}
\maketitle

\lecture{23}{2021-12-07}{Alienating sets, and \textit{désolé pour le dérangement}}{
}

% \parag{Fibonacci using generating function}{
%     As a recall, Fibonacci is given by:
%     \[f_n = f_{n-1} + f_{n-2}, \mathspace f_0 = 1, \mathspace f_1 = 1\]

%     Let's take a look at the generating function:
%     \begin{multiequality}
%         G\left(x\right) =\ & \sum_{k=0}^{\infty} f_k x^k  \\
%         =\ & f_0 + f_1 x + \sum_{k=2}^{\infty}  f_k x^k \\
%         =\ & f_0 + f_1 x + \sum_{k=2}^{\infty} \left(f_{k-1} + f_{k-2}\right)x^k \\
%         =\ & f_0 + f_1 x + \sum_{k=2}^{\infty} f_{k-1}  x^k + \sum_{k=2}^{\infty} f_{k-2}x^k  \\
%         =\ & 1 + x + x \sum_{k=2}^{\infty} f_{k-1} x^{k-1} + x^2 \sum_{k=2}^{\infty} f_{k-2}x^{k-2} \\
%         =\ & 1 + x + x \sum_{k=1}^{\infty} f_k x^k + x^2 x^2 \sum_{k=0}^{\infty} f_k x^k \\
%         =\ & 1 + x + x \left(G\left(x\right) - f_0\right) + x^2 G\left(x\right)  \\
%         =\ & 1 + x + x \left(G\left(x\right) - 1\right) + x^2 G\left(x\right)  \\
%     \end{multiequality}

%     Solving this equation, we find:
%     \[G\left(x\right) = \frac{1}{1 - x - x^2} = \frac{1}{-\left(x - r_1\right)\left(x - r_2\right)}, \mathspace r_1 = \frac{-1 - \sqrt{5}}{2} = -\phi, r_2 = \frac{\sqrt{5} - 1}{2} = -\bar{\phi}\]

%     Let's use partial fractions:
%     \[G\left(x\right) = \frac{1}{1 - x - x^2} = \frac{a}{x - r_1} + \frac{b}{x - r_1} \implies a\left(x - r_2\right) + b\left(x - r_1\right) = 1\]

%     Solving this equation, we get $a = \frac{1}{\sqrt{5}}$ and $b = -\frac{1}{\sqrt{5}}$. So:
%     \[G\left(x\right) = \frac{1}{\sqrt{5}} \frac{1}{x - r_1} - \frac{1}{\sqrt{5}} \frac{1}{x - r_2} = \frac{-1}{\sqrt{5}r_1} \frac{1}{1 - \frac{x}{r_1}} + \frac{1}{r_2 \sqrt{5}} \frac{1}{1 - \frac{x}{r_2}}\]

%     Using remarkable generating functions:
%     \[G\left(x\right) = -\frac{1}{r_1\sqrt{5}} \sum_{k=0}^{\infty} \frac{1}{r_1^k} x^k + \frac{1}{r_2\sqrt{5}} \sum_{k=0}^{\infty} \frac{1}{r_2^k} x^k\]

%     So, we get that:
%     \[f_n = \frac{\frac{-1}{r_1^{k+1}} + \frac{1}{r_2^{k+1}}}{\sqrt{5}} = \frac{\phi^{k+1} - \bar{\phi}^{k+1}}{\sqrt{5}}\]
% }

% \subsection{More on generating functions}
% \parag{Combinations using generating functions}{
%     We want to distribute 2 cookies to 3 children such that no child receives more than 1 cookie, and we wonder in how many ways it can be done.

%     The first way of reasoning is that we want to choose two children to which we give the cookie, so we have $C\left(3, 2\right)$.

%     The second way of reasoning is considering children as the polynomial $\left(1 + x\right)$: the power of the $x$ tells how many cookies this kid has. Considering all three children together, we have to expand the following polynomial:
%     \[\left(1 + x\right)^3 = 1 + 3x + 3x^2 + x^3\]

%     Then, reading off the coefficient of $x^2$, we see that the result is $3$. It basically states that $x^2$ can be formed as:
%     \[x\cdot x\cdot 1 = x^2, \mathspace x\cdot 1\cdot x = x^2, \mathspace 1\cdot x\cdot x = x^2\]
%     where each child is either represented by a 1 or a $x$: if they have a 1 they do not get a cookie, if they have a $x$ they receive a cookie.

%     \subparag{Personal note}{
%         Meanwhile, the poor Cookie Monster has zero cookie:
%         \begin{center}
%         \url{https://www.youtube.com/watch?v=1ZoIP4aoM0g}
%         \end{center}
%     }
% }

% \parag{Combinations}{
%     We want to find the number of $k$-combinations with $n$ elements using generating functions.

%     The number of $k$-combinations is the coefficient of $x^k$ in the following generating function:
%     \[f\left(x\right) = \left(1 + x\right)^n = \sum_{k=0}^{n} \binom{n}{k} x^k\]

%     Thus, the number of $k$-combinations is $\binom{n}{k}$, as expected.
% }

% \parag{Definition (extended binomial coefficients)}{
%     Let $u$ be a real number, and $k$ be a nonnegative integer. The \important{extended binomial coefficient} is defined as:
%     \begin{functionbypart}{\binom{u}{k}}
%         \frac{u\left(u-1\right)\cdots\left(u - k + 1\right)}{k!}, \mathspace \text{if } k > 0 \\
%         1, \mathspace \text{if } k = 0
%     \end{functionbypart}

%     \subparag{Example}{
%         For, example:
%         \[\binom{-2}{3} = \frac{\left(-2\right)\left(-3\right)\left(-4\right)}{3!} = -4\]

%         \[\binom{1 / 2}{2} = \frac{\left(\frac{1}{2}\right)\left(- \frac{1}{2}\right)}{2!} = -\frac{1}{8}\]
%     }
% }

% \parag{Theorem (extended binomial theorem)}{
%     Let $x$ and $u$ be real numbers, with $\left|x\right| < 1$. Then:
%     \[\left(1 + x\right)^u = \sum_{k=0}^{\infty} \binom{u}{k} x^k\]

%     \subparag{Observation}{
%         Let's look at what happens when $u$ is an integer, and $k$ is greater than $u$. For example:
%         \[\binom{n}{n+1} = \frac{n\left(n-1\right) \cdots \left(n-n\right)}{\left(n+1\right)!} = 0\]

%         Thus, we are adding infinitely many zeroes when we only have integers. This is why, in the ``regular'' binomial theorem, we are only summing until $n$.
%     }
% }

% \parag{Combination with repetitions using generating functions}{
%     We want to distribute 2 cookies to 3 children, and we wonder in how many ways it can be done.

%     Now, we can give each child 0, 1 or 2 cookies. This is a combination with repetition, so the number is given by:
%     \[C\left(3 + 2 - 1, 2\right) = C\left(4, 2\right) = 6\]

%     Let's also consider this problem using generating functions. Again, let's consider that each child is a polynomial, $\left(1 + x + x^2\right)$, where the power of $x$ is the number of cookies we give to them. So, again, we need to read off the coefficient of $x^2$ in
%     \[\left(1 + x + x^2\right)^3\]
% }

% \parag{Combinations with repetition}{
%     We want to find the number of $k$-combinations of a set with $n$ elements, where repetition is allowed, using generating functions.

%     The number of $k$-combinations is the coefficient of $x^k$ in the generating function:
%     \[f\left(x\right) = \left(1 + x + x^2 + \ldots\right)^n\]

%     As long as $\left|x\right| < 1$, we get:
%     \[f\left(x\right) = \frac{1}{\left(1 + \left(-x\right)\right)^n} = \left(1 - x\right)^{-n} = \sum_{k=0}^{\infty} \binom{-n}{k} \left(-x\right)^k = \sum_{k=0}^{\infty} \binom{-n}{k} \left(-1\right)^k \left(x\right)^k \]

%     So, the coefficient of $x^k$ is $\binom{-n}{k} \left(-1\right)^k$, let's simplify it:
%     \begin{multiequality}
%         \left(-1\right)^k \binom{-n}{k} =\ & \left(-1\right)^k \frac{-n\left(-n-1\right) \cdots \left(-n - k + 1\right)}{k!}  \\
%     =\ & \left(-1\right)^k \frac{\left(-1\right)^k n\left(n+1\right)\cdots \left(n + k -1\right)}{k!}  \\
%     =\ & \left(-1\right)^{2k} \binom{n + k -1}{k} \\
%     =\ & \binom{n + k -1}{k}
%     \end{multiequality}
%     as expected.
% }

% \parag{Example of generalisation 1}{
%     Let's now say that we want to distribute 8 cookies to 3 children, where each children receives 2, 3 or 4 cookies, and we wonder in how many ways it can be done.

%     We can translate it to a generating function, and we would only need to read off the coefficient of $x^8$ in:
%     \[\left(x^2 + x^3 + x^4\right)^3\]
% }

% \parag{Example of generalisation 2}{
%     Let's say we want to find the number of solutions of $e_1 + e_2 + e_3 = 17$, where $e_1, e_2, e_3$ are nonnegative integers with $2 \leq e_1 \leq 5$, $3 \leq e_2 \leq 6$ and $4 \leq e_3 \leq 7$. Translating this to a generating function would help us:
%     \[\left(x^2 + x^3 + x^4 + x^5\right)\left(x^3 + x^4 + x^5 + x^6\right)\left(x^4 + x^5 + x^6 + x^7\right)\]


%     Note that this still is not very efficient since, computationally we are not far from counting every possible ways. We don't need to use generating functions; thinking a little bit about this problem is sufficient (we can for example see that each child must have at least 2 cookies, so we can diminish all our constraints by 2, and have $e_1 + e_2 + e_3 = 17 - 6 = 11$).
% }


\subsection{Principle of Inclusion-Exclusion}

\parag{Theorem: Principle of Inclusion-Exclusion}{
    Let $A_1, \ldots, A_n$ be fine sets. Then $\left|A_1 \cup \ldots \cup A_n\right|$ is given by:
    \[\sum_{1 \leq i \leq n}^{} \left|A_i\right| - \sum_{i \leq i < k \leq n}^{} \left|A_i \cap A_k\right| + \sum_{1 \leq i < j < k \leq n}^{} \left|A_i \cap A_j \cap A_k\right| - \ldots + \left(-1\right)^{n+1} \left|\bigcap_{1 \leq i \leq n} A_i\right|\]
}

\parag{Definition: derangement}{
    A \important{derangement} is a permutation of objects that leaves no object in the original position.
}

\parag{Theorem}{
    The number of derangements of a set with $n$ elements is given by:
    \[D_n = n! \left(1 - \frac{1}{1!} + \frac{1}{2!} - \frac{1}{3!} + \ldots + \left(-1\right)^n \frac{1}{n!}\right)\]
}

\end{document}
