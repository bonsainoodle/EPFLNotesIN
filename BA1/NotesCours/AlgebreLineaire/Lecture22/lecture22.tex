\documentclass[a4paper]{article}

% Expanded on 2021-12-06 at 13:14:43.

\usepackage{../../style}

\title{Algèbre linéaire}
\author{Joachim Favre}
\date{Lundi 06 décembre 2021}

\begin{document}
\maketitle

\lecture{22}{2021-12-06}{La la la Schtroumpf la la}{
}

\subsection{Orthonormalisation}

\parag{Gram-Schmidt et factorisation QR}{
    Soit $A$ une matrice de taille $m \times n$ dont les colonnes sont linéairement indépendants. En d'autres mots, $\left(\bvec{x}_1, \ldots, \bvec{x}_n\right)$ est une base de $W = \vect\left\{\bvec{x}_1, \ldots, \bvec{x}_n\right\} = \im A$.

    \important{L'algorithme de Gram-Schmidt} (Grand Schtroumpf) revient à calculer: 
    \[\bvec{v}_n = \bvec{x}_n - \proj_{W_{n-1}}\left(\bvec{x}_n\right) = \bvec{x}_n - \left(\bvec{x}_n \dotprod \bvec{u}_1\right) \bvec{u}_1 - \ldots - \left(\bvec{x}_n \dotprod \bvec{u}_{n-1}\right)\bvec{u}_{n-1}\]
    \[\bvec{u}_n = \frac{1}{\left\|\bvec{v}_n\right\|} \bvec{v}_n\]
    où $W_{n-1} = \vect\left\{\bvec{x}_1, \ldots, \bvec{x}_{n-1}\right\}$.

    Ces calculs nous donnent $\left(\bvec{u}_1, \ldots, \bvec{u}_n\right)$, une \important{base orthonormée} de $W = \im A$. De plus, ils révèlent la factorisation $QR$ de $A$: 
    \[A = QR, \mathspace Q = \begin{bmatrix}  &  &  \\ \bvec{u}_1 & \ldots & \bvec{u}_n \\  &  &  \end{bmatrix}, \mathspace R = \begin{bmatrix} \left\|\bvec{x}_1\right\| & \bvec{x}_2 \dotprod \bvec{u}_1 & \ldots & \bvec{x}_n \dotprod \bvec{u}_1 \\ 0 & \left\|\bvec{v}_2\right\| & \ldots & \bvec{x}_n \dotprod \bvec{u}_2 \\ \vdots & \vdots & \ddots & \vdots \\ 0 & 0 & \ldots & \left\|\bvec{v}_n\right\| \end{bmatrix} \]
}

\parag{Matrice carrée}{
    Étudions la factorisation $QR$ d'une matrice $A \in \mathbb{R}^{n \times n}$ carrée. 

    Appliquons le Grand Schtroumpf à ses colonnes. Il y a deux cas possibles. Si cet algorithme échoue (il y a une division par zéro), alors $A$ n'est pas inversible. S'il aboutit, alors on trouve $Q \in \mathbb{R}^{n \times n}$ et $R \in \mathbb{R}^{n \times n}$ telles que: 
    \[A = QR, \mathspace Q^T Q = Q Q^T = I_n \implies Q^{-1} = Q^T\]
    et où $R$ est triangulaire est inversible (elle n'a que des nombres strictement positifs sur sa diagonale, donc son déterminant n'est pas nul).

    On peut utiliser ce résultat pour résoudre $A \bvec{x} = \bvec{b}$. En effet, on remarque que:
    \[A \bvec{x} = \bvec{b} \iff QR \bvec{x} = \bvec{b} \iff Q^T Q R \bvec{x} = Q^T \bvec{b} \iff R \bvec{x} = Q^T \bvec{b}\]
    
    De plus, on peut aussi utiliser ce résultat pour calculer (et mieux comprendre) $\det\left(A\right)$: 
    \[1 = \det\left(I\right) = \det\left(Q^T Q\right) = \det\left(Q^T\right) \det\left(Q\right) = \det\left(Q\right)^2 \implies \det\left(Q\right) = \pm 1\]
    
    Ainsi, on peut utiliser ceci pour calculer $\det\left(A\right)$: 
    \[\det\left(A\right) = \det\left(QR\right) = \det\left(Q\right)\det\left(R\right) = \pm\det\left(R\right) \implies \left|\det\left(A\right)\right| = \left|\det\left(R\right)\right|\]
    
    Ainsi, on en déduit que: 
    \[\left|\det\left(A\right)\right| = \left\|\bvec{x}_1\right\| \left\|\bvec{v}_2\right\| \left\|\bvec{v}_3\right\|\cdots \left\|\bvec{v}_n\right\|\]
    

    Donc $\left|\det\left(A\right)\right|$ est l'hyper-volume de l'hyper-parallélépipède de dimension $n$

}

\end{document}
