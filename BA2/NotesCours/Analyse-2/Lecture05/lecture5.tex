\documentclass[a4paper]{article}

% Expanded on 2022-03-07 at 10:18:06.

\usepackage{../../style}

\title{Analyse 1}
\author{Joachim Favre}
\date{Lundi 07 mars 2022}

\begin{document}
\maketitle

\lecture{5}{2022-03-07}{Wronskien}{
}

\subsection{Wronskien}
\parag{Définition: Wronskien}{
    Soient $v_1, v_2 : I \mapsto \mathbb{R}$ deux fonctions dérivables sur $I \subset \mathbb{R}$. Nous appelons la fonction $W\left[v_1, v_2\right] : I \mapsto \mathbb{R}$ définie par:
    \[W\left[v_1, v_2\right] = \det\begin{pmatrix} v_1\left(x\right) & v_2\left(x\right) \\ v_1'\left(x\right) & v_2'\left(x\right) \end{pmatrix} = v_1\left(x\right) v_2'\left(x\right) - v_2\left(x\right) v_1'\left(x\right)\]
    le \important{Wronskien} de $v_1$ et $v_2$.
}

\parag{Proposition pour le Wronskien}{
    Soient $v_1, v_2 : I \mapsto \mathbb{R}$ deux solutions de l'équation $y''\left(x\right) + p\left(x\right)y'\left(x\right) + q\left(x\right)y\left(x\right) = 0$ (EDL2 homogène).

    $v_1\left(x\right)$ et $v_2\left(x\right)$ sont linéairement indépendants si et seulement si $W\left[v_1, v_2\right]\left(x\right) \neq 0$ pour tout $x \in I$.
}
\parag{Théorème: Forme des solutions aux EDL2 homogènes}{
    Soient $v_1, v_2 : I \mapsto \mathbb{R}$ deux solutions linéairement indépendantes de l'équation $y''\left(x\right) + p\left(x\right) y'\left(x\right) + q\left(x\right) y\left(x\right) = 0$.

    Alors, la solution générale de cette équation est de la forme:
    \[v\left(x\right) = C_1 v_1\left(x\right) + C_2 v_2\left(x\right), \mathspace C_1, C_2 \in \mathbb{R}, x \in I\]
}

\subsection[EDL2 complètes]{Équations différentielles linéaires d'ordre 2 complètes}

\parag{Superposition des solutions}{
    Si $v\left(x\right)$ est une solution de l'équation $y''\left(x\right) + p\left(x\right) y'\left(x\right) + q\left(x\right) y\left(x\right) = f\left(x\right)$ et $u\left(x\right)$ est une solution de l'équation homogène associée, alors $v\left(x\right) + u\left(x\right)$ est une solution de cette équation.
}

\parag{Méthode de la variation de la constante}{
Nous cherchons une solution particulière de l'équation complète, supposant que nous connaissons deux solutions linéairement indépendantes de l'équation homogène associée: $v_1, v_2 : I \mapsto \mathbb{R}$.

Prenons l'\textit{Ansatz} $v_0\left(x\right) = c_1\left(x\right) v_1\left(x\right) + c_2\left(x\right) v_2\left(x\right)$, où $c_1\left(x\right)$ et $c_2\left(x\right)$ sont deux fonctions inconnues de classe $C^1$ sur $I$.

Nous pouvons dériver notre solution présumée:
\[v_0'\left(x\right) = {\color{red}c_1'\left(x\right) v_1\left(x\right) + c_2'\left(x\right) v_2\left(x\right)} + c_1\left(x\right) v_1'\left(x\right) + c_2\left(x\right) v_2'\left(x\right)\]

Supposons aussi que le bout en rouge est égal à 0, car cela permet non seulement de diminuer grandement la taille de notre solution, mais aussi de ne pas avoir de $c''\left(x\right)$ (puisque nous les avons prises de classe $C^1$).

Mettons ce que nous avons trouvé dans notre équation $v_0''\left(x\right) + p\left(x\right)v_0'\left(x\right) + q\left(x\right) v_0\left(x\right) = f\left(x\right)$, de manière à trouver des conditions sur $c_1\left(x\right)$ et $c_2\left(x\right)$:
\begin{multiequality}
    & c_1'\left(x\right) v_1'\left(x\right) + c_2'\left(x\right) v_2'\left(x\right) + {\color{blue}c_1\left(x\right)v_1''\left(x\right)} + {\color{red}c_2\left(x\right)v_2''\left(x\right)}  \\
    & + {\color{blue} p\left(x\right) c_1\left(x\right)v_1'\left(x\right)} + {\color{red} p\left(x\right) c_1\left(x\right) v_2'\left(x\right)}  \\
    & + {\color{blue} q\left(x\right) c_1\left(x\right) v_1\left(x\right)} + {\color{red}q\left(x\right) c_2\left(x\right)v_2\left(x\right)} \\
    & = f'\left(x\right)
\end{multiequality}

En tout, nous avons deux conditions (celle que nous venons de trouver, et celle que nous avions prise de manière arbitraire):
\[\begin{systemofequations} c_1'\left(x\right) v_1\left(x\right) + c_2'\left(x\right) v_2\left(x\right) = 0 \\ c_1'\left(x\right) v_1'\left(x\right) + c_2'\left(x\right) v_2'\left(x\right) = f\left(x\right) \end{systemofequations}\]

Or nous pouvons écrire notre système sous la forme:
\[\begin{pmatrix} v_1\left(x\right) & v_2\left(x\right) \\ v_1'\left(x\right) & v_2'\left(x\right) \end{pmatrix} \begin{pmatrix} c_1'\left(x\right) \\ c_2'\left(x\right) \end{pmatrix} = \begin{pmatrix} 0 \\ f\left(x\right) \end{pmatrix} \]

Cependant, c'est un système d'équations sur $c_1'\left(x\right)$ et $c_2'\left(x\right)$. Puisqu'on sait que $v_1\left(x\right)$ et $v_2\left(x\right)$ sont linéairement indépendantes, nous avons $W\left[v_1, v_2\right]\left(x\right) = \det\begin{psmallmatrix} v_1\left(x\right) & v_2\left(x\right) \\ v_1'\left(x\right) & v_2'\left(x\right) \end{psmallmatrix} \neq 0$ pour tout $x \in I$. Ceci nous dit donc que notre matrice est inversible, et que nous pouvons résoudre notre système de cette manière:
\[\begin{pmatrix} c_1'\left(x\right) \\ c_2'\left(x\right) \end{pmatrix} = \frac{1}{W\left[v_1, v_2\right]\left(x\right)} \begin{pmatrix} v_2'\left(x\right) & -v_2\left(x\right) \\ -v_1'\left(x\right) & v_1\left(x\right) \end{pmatrix} \begin{pmatrix} 0 \\ f\left(x\right) \end{pmatrix} = \frac{1}{W\left[v_1, v_2\right]\left(x\right)} \begin{pmatrix} -v_2\left(x\right) f\left(x\right) \\ v_1\left(x\right) f\left(x\right) \end{pmatrix} \]

Nous pouvons maintenant intégrer:
\[c_1\left(x\right) = -\int \frac{f\left(x\right) v_2\left(x\right)}{W\left[v_1, v_2\right]\left(x\right)}dx, \mathspace c_2\left(x\right) = \int \frac{f\left(x\right) v_1\left(x\right)}{W\left[v_1, v_2\right]\left(x\right)}dx\]
où on supprime les constantes.

Nous obtenons donc $v_0\left(x\right) = c_1\left(x\right) v_1\left(x\right) + c_2\left(x\right) v_2\left(x\right)$ est une solution de l'équation complète. Nous pouvons même obtenir la solution générale à cette équation:
\[v\left(x\right) = C_1 v_1\left(x\right) + C_2 v_2\left(x\right) + v_0\left(x\right), \mathspace \text{où } C_1, C_2 \in \mathbb{R}, x \in I\]
}

\end{document}
