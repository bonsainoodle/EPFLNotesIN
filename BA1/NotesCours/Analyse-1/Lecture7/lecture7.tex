\documentclass[a4paper]{article}

% Expanded on 2021-10-13 at 10:13:44.

\usepackage{../../style}

\title{Analyse I}
\author{Joachim Favre}
\date{Mercredi 13 octobre 2021}

\begin{document}
\maketitle

\lecture{7}{2021-10-13}{Suite des limites de suites}{
}

\parag{Proposition (Unicité de la limite)}{
Soit $\left(a_n\right)_{n\in\mathbb{N}}$ une suite de nombres réels et supposons que $a \in \mathbb{R}$ et $b \in \mathbb{R}$ sont des limites de $\left(a_n\right)$. Alors, $a = b$.
}

\parag{Inégalité triangulaire}{
    Pour tout $x, y \in \mathbb{R}$, on a que:
    \[\left|x + y\right| \leq \left|x\right| + \left|y\right|\]
}

\parag{Proposition}{
    Toute suite convergente est bornée.

    \subparag{Réciproque}{
        La réciproque de la proposition est fausse. En effet, on a démontré un contre-exemple ci-dessus. $a_n = \left(-1\right)^n$ est bornée par $-1$ et $1$, cependant elle est divergente.
    }
}

\subsection{Opération algébriques sur les limites}
\parag{Proposition}{
    Soient $\left(a_n\right)$ et $\left(b_n\right)$ deux suites convergentes. En d'autres mots:
    \[\lim_{n \to \infty} a_n = a \mathspace \text{ et } \lim_{n \to \infty} b_n = b\]

    Alors:
    \begin{enumerate}
        \item $\lim\limits_{n \to \infty} \left(a_n \pm b_n\right) = a \pm b$
        \item $\lim\limits_{n \to \infty} pa_n = pa$ pour tout $p \in \mathbb{R}$.
        \item $\lim\limits_{n \to \infty} \left(a_n\cdot b_n\right) = a\cdot b$
        \item $\lim\limits_{n \to \infty} \left(\frac{a_n}{b_n}\right) = \frac{a}{b}$, si $b \neq 0$.
    \end{enumerate}
}

\parag{Convergence des opérations}{
    Soit $\left(a_n\right)$ et $\left(b_n\right)$ des suites. On peut étudier la convergence de leurs opérations:
    \begin{enumerate}
        \item\label{enum:convegenceSommeSuites} Si $\left(a_n + b_n\right)$ converge, alors soit $\left(a_n\right)$ et $\left(b_n\right)$ convergent, soit elles divergent toutes les deux.
        \item Si $\left(a_n b_n\right)$ converge, on ne sait rien de la convergence de $\left(a_n\right)$ et $\left(b_n\right)$.
        \item Si $\lim_{n \to \infty} a_n = 0$, alors la suite $\left(\frac{1}{a_n}\right)$ est divergente, si elle existe (s'il n'y a pas de division par zéro).
    \end{enumerate}

    De plus:
    \begin{enumerate}
        \setcounter{enumi}{3}
        \item Si $\left(a_n + b_n\right)$ converge et que $\left(b_n\right)$ converge, alors $\left(a_n\right)$ converge.
        \item Si $\left(a_n b_n\right)$ converge et que $\left(b_n\right)$ converge vers quelque chose d'autre que 0, alors $\left(a_n\right)$ est convergente.
    \end{enumerate}
}

\parag{Proposition (quotient de deux suites polynomiales)}{
    Pour $p, q \in \mathbb{N}^*$, si on a:
    \[x_n = a_p n^p + \ldots + a_1 n + a_0 \mathspace a_i \in \mathbb{R},\ a_p \neq 0\]
    et:
    \[y_n = b_q n^q + \ldots + b_1 n + b_0 \mathspace b_i \in \mathbb{R},\ b_q \neq 0\]

    Alors,
    \begin{functionbypart}{\lim_{n \to \infty} \left(\frac{x_n}{y_n}\right)}
        0, \mathspace \text{si } p < q \\
        \frac{a_p}{b_q}, \mathspace \text{si } p = q \\
        \text{diverge}, \mathspace \text{si } p < q
    \end{functionbypart}
}

\subsection{Relation d'ordre}
\parag{Proposition}{
    Soient $\left(a_n\right)$ et $\left(b_n\right)$ deux suites convergentes. En d'autres mots:
    \[\lim_{n \to \infty} a_n = a \mathspace \text{ et } \mathspace \lim_{n \to \infty} b_n = b\]

    Supposons que $\exists m_0 \in \mathbb{N}$ tel que $\forall n \geq m_0$ on a que
    \[a_n \geq b_n\]

    Alors, on sait que
    \[a \geq b\]
}

\parag{Théorème des deux gendarmes pour les suites}{
    Soient $\left(a_n\right)$, $\left(b_n\right)$ et $\left(c_n\right)$ telles que
    \begin{enumerate}
        \item $\lim\limits_{n \to \infty} a_n = \lim\limits_{n \to \infty} c_n = \ell$
        \item $\exists k \in \mathbb{N}$ tel que $\forall n \geq k$ on a $a_n \leq b_n \leq c_n$
    \end{enumerate}

    Alors
    \[\lim_{n \to \infty} b_n = \ell\]
}

\end{document}
