\documentclass{article}

% Expanded on 2021-09-22 at 22:30:01.

\usepackage{../../style}

\title{Algèbre linéaire}
\author{Joachim Favre}
\date{Jeudi 23 septembre 2021}

\begin{document}
\maketitle

\lecture{1}{2021-09-23}{Début des systèmes d'équations linéaires}{
}

\section{Équations linéaires}
\subsection{Le cas d'une unique équation}
\parag{Définition des équations linéaires}{
    Pour des variables (= inconnues) $x_1, \ldots, x_n$, une \important{équation linéaire} est une équation de la forme
    \[a_1 x_1 + a_2 x_2 + \ldots + a_n x_n = b\]
    pour des nombres $a_1, a_2, \ldots, a_n$ et $b$ indépendants des variables.
}

\subsection{Système (d'équations) linéaires}
\parag{Définition des systèmes d'équations linéaires, des solutions et des ensembles solutions}{
    Un \important{système d'équation linéaires} est une collection de une ou plusieurs équations linéaire en des variables $x_1, \ldots, x_n$. S'il y a $m$ équations en $n$ variables, on écrit
    \begin{systemofequations}{}
        &\ a_{11} x_1 + \ldots + a_{1n} x_n = b_1 \\
        &\ \vdots \\
        &\ a_{m1} x_1 + \ldots + a_{mn} x_n = b_m
    \end{systemofequations}
    avec $a_{ij}$, le coefficient de $x_j$ dans la ième équation.

    Étant donnée une liste de nombres $\left(s_1, \ldots, s_n\right)$, on dit que cette liste est une \important{solution} (du système) si les $m$ équations sont vraies quand on remplace $x_1$ par $s_1$, \ldots, et $x_n$ par $s_n$.

    L'ensemble des solutions d'un système est son \important{ensemble solution.} Deux systèmes sont \important{équivalents} s'ils ont le même ensemble solutions.
}

\parag{Théorème}{
    Un système linéaire a zéro (aucune solution n'existe), une (la solution existe et est unique) ou une infinité (des solutions existent, mais elles ne sont pas uniques) de solutions.
}

\subsection{Système triangulaire}
\parag{Remarque}{
    Certains systèmes sont plus faciles à résoudre que d'autres. Les diagonaux sont plus simples que le triangulaires, qui sont plus simples que les ``généraux''.  La stratégie va être de transformer un système général en système triangulaire (ou aussi triangulaire que possible).
}

\subsection{Opérations élémentaires}
\parag{Théorème des opérations élémentaires}{
    Les opérations suivantes ne changent pas l'ensemble solution:
    \begin{enumerate}
        \item Permuter deux équations.
        \item Multiplier une équation par un nombre non nul.
        \item Additionner à une équation un multiple d'une autre équation.
    \end{enumerate}

    On les appelle des \important{opérations élémentaires}.
}

\subsection{Matrices}
\parag{Notation matricielle}{
    Les équations sont de la forme $a_1 x_1 + \ldots a_n x_n = b_n$. Vu qu'elles ont toutes la même forme, seuls les coefficients $a_1, \ldots, a_n, b_n$ sont importants. Si on a un système
    \begin{systemofequations}{}
        &\ a_{11} x_1 + \ldots + a_{1n} x_n = b_1 \\
        &\ \vdots \\
        &\ a_{m1} x_1 + \ldots + a_{mn} x_n = b_m
    \end{systemofequations}

    On collecte les nombres utiles dans deux tableaux appelés matrices:
    \[\begin{bmatrix} a_{1,1} & \ldots & a_{1n} \\ \vdots &  & \vdots \\ a_{m1} & \ldots & \cdot a_{mn} \end{bmatrix} \mathspace \begin{bmatrix} a_{1,1} & \cdots & a_{1n} & b_1 \\ \vdots &   & \vdots & \vdots \\ a_{m1} & \ldots & a_{mn} & b_m \end{bmatrix}\]

    Sur la gauche, la \important{matrice des coefficients} $a_{ij}$ est de taille $m \times n$. Sur la droite, c'est la \important{matrice augmentée}, qui est de taille $m \times \left(n + 1\right)$ (on peut mettre des traitillés juste avant la dernière colonne de cette matrice augmentée, mais on n'est pas obligé).

    Chaque système linéaire correspond à une matrice augmentée, et vice versa.
}

\parag{Les opérations élémentaires}{
    Nous retrouvons ces mêmes opérations élémentaires pour une matrice augmenté:
    \begin{enumerate}
        \item Permuter deux équations. / Permuter deux lignes.
        \item Multiplier une équation par un nombre non nul. / Multiplier une ligne par un nombre non nul.
        \item Additionner à une équation un multiple d'une autre équation. / Additionner à une ligne un multiple d'une autre ligne.
    \end{enumerate}
}

\end{document}
