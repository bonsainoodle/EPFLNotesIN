\documentclass[a4paper]{article}

% Expanded on 2021-12-16 at 08:19:01.

\usepackage{../../style}

\title{Algèbre}
\author{Joachim Favre}
\date{Jeudi 16 décembre 2021}

\begin{document}
\maketitle

\lecture{25}{2021-12-16}{James Bond Spectre}{
}

\parag{Théorème spectral des matrices symétriques}{
    Toute matrice \textit{symétrique} $A \in \mathbb{R}^{n \times n}$ vérifie les propriétés suivantes:
    \begin{enumerate}
        \item $A$ admet $n$ valeurs propres réelles (pas forcément distinctes).
        \item Pour chaque valeur propre $\lambda$, sa multiplicité géométrique est égale à sa multiplicité algébrique.
        \item Les sous-espaces propres sont deux à deux orthogonaux, ce qui signifie que deux vecteurs propres associés à des valeurs propres distinctes sont orthogonaux.
        \item $A$ est diagonalisable en base orthonormée.
    \end{enumerate}

    Ce dernier fait est une condition nécessaire et suffisante: $A = A^T$ si et seulement si $A$ est diagonalisable en base orthonormée.
}

\parag{Remarque}{
    Si les valeurs propres ne sont pas distinctes, alors il est nécessaire d'appliquer l'algorithme du Grand-Schtroumpf à chacun des sous-espaces propres (de dimension supérieure à 1 évidemment) pour que tous les vecteurs propres soient orthogonaux deux à deux.
    On applique l'algorithme du Grand-Schtroumpf sur chacun des sous-espace propre individuellement car sinon on ne resterait en général pas dans le même sous-espace propre.

    Pour les vecteurs propres associés à une valeur propre unique, il ne faut pas oublier de les normaliser.
}

\parag{Théorème}{
    Si $A \in \mathbb{R}^{n\times n}$ est symétrique alors:
    \begin{enumerate}
        \item $A$ a $n$ valeurs propres réelles $\lambda_1, \ldots, \lambda_n$ (multiplicités comptées, c'est-à-dire qu'elles ne sont pas forcément distinctes).
        \item Elle admet une base de vecteurs propres orthonormés $\left(\bvec{u}_1, \ldots, \bvec{u}_n\right)$.
        \item On peut trouver la décomposition suivante, appelée \important{décomposition spectrale}:
              \[A = \lambda_1 \bvec{u}_1 \bvec{u}_1^T + \ldots + \lambda_n \bvec{u}_n \bvec{u}_n^T\]
    \end{enumerate}
}

\subsection{Décomposition en valeurs singulières}

\parag{Définition: matrice diagonale}{
    Une matrice $n \times m$ est diagonale si:
    \[a_{ij} \neq 0 \implies i = j\]

    Voici quelques exemples de matrices diagonales:
    \[\begin{bmatrix} 3 & 0 & 0 \\ 0 & 1 & 0 \\ 0 & 0 & 4 \end{bmatrix}, \mathspace \begin{bmatrix} 1 & 0 & 0 & 0 & 0 \\ 0 & 5 & 0 & 0 & 0 \\ 0 & 0 & 9 & 0 & 0 \end{bmatrix}, \mathspace \begin{bmatrix} 2 & 0 & 0 \\ 0 & 6 & 0 \\ 0 & 0 & 5 \\ 0 & 0 & 0 \\ 0 & 0 & 0 \end{bmatrix}  \]
}

\end{document}
