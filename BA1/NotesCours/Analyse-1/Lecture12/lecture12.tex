\documentclass[a4paper]{article}

% Expanded on 2021-11-01 at 10:11:24.

\usepackage{../../style}

\title{Analyse 1}
\author{Joachim Favre}
\date{Lundi 01 novembre 2021}

\begin{document}
\maketitle

\lecture{12}{2021-11-01}{Le cours le plus simple, selon la Professeure}{
}

\section{Fonctions réelles}
\subsection{Définitions et propriétés de bases}
\parag{Définition des fonctions}{
    Une \important{fonction} $f: E \mapsto F$ où $E, F \subset \mathbb{R}$ est une application (une règle) qui donne pour tout élément $x \in D\left(f\right) = E$, un élément $y = f\left(x\right) \in F$. La notion réellement importante c'est que, à un élément de $D\left(f\right)$ il ne peut être associé qu'une seule valeur de $f\left(D\right)$.


    On dit que $D\left(f\right)$ est le \important{domaine de définition} de $f$ (qui est égal à $E$). De plus, on note $f\left(D\right)$ \important{l'ensemble image} (qui est un sous-ensemble $F$).

    On note $x \mapsto f\left(x\right)$.
}

\parag{Fonctions données par formule}{
    Si la fonction est donnée par une formule, alors $D\left(f\right)$ est le plus grand sous-ensemble de $\mathbb{R}$ où l'expression $f\left(x\right)$ est bien définie.
}

\parag{Croissance}{
    $f\left(x\right)$ est \important{croissante} sur $D\left(f\right)$ si $\forall x_1 < x_2 \in D\left(f\right)$, alors on a:
    \[f\left(x_1\right) \leq f\left(x_2\right)\]

    $f\left(x\right)$ est \important{strictement croissante} sur $D\left(f\right)$ si, pour les mêmes hypothèses, alors:
    \[f\left(x_1\right) < f\left(x_2\right)\]
}

\parag{Décroissance}{
    $f\left(x\right)$ est \important{décroissante} sur $D\left(f\right)$ si $\forall x_1 < x_2 \in D\left(f\right)$, alors on a:
    \[f\left(x_1\right) \geq f\left(x_2\right)\]

    $f\left(x\right)$ est \important{strictement décroissante} sur $D\left(f\right)$ si, pour les mêmes hypothèses, alors:
    \[f\left(x_1\right) > f\left(x_2\right)\]
}

\parag{Monotonicité}{
    Si $f$ est croissante ou décroissante sur $D\left(f\right)$, alors elle est \important{monotone}.

    De la même manière, $f$ est \important{strictement monotone} si elle est strictement croissante ou strictement décroissante sur son domaine de définition.
}

\parag{Parité}{
    On dit qu'un domaine de définition est symétrique si:
    \[x \in D\left(f\right) \implies -x \in D\left(f\right)\]


    $f$ est \important{paire} si $D\left(f\right)$ est symétrique, et si $f$ est telle que
    \[f\left(-x\right) = f\left(x\right), \mathspace \forall x \in D\left(f\right)\]


    $f$ est \important{impaire} si $D\left(f\right)$ est symétrique et si elle est telle que:
    \[f\left(-x\right) = -f\left(x\right), \mathspace \forall x \in D\left(f\right)\]
}

\parag{Périodicité}{
    $f: E \mapsto F$ est \important{périodique} s'il existe $P \in \mathbb{R}^*$ tel que pour tout $x \in E$, alors $x + P \in E$ et:
    \[f\left(x + P\right) = f\left(x\right), \mathspace \forall x \in E\]

    On appelle $P$ une \important{période} de $f$. On remarque que si $P$ est une période, alors $nP$ avec $n \in \mathbb{N}$ est aussi une période. Elle n'est donc pas unique.

    Puisque, si $f$ est périodique alors $x \in E \implies x + nP \in E$, on en déduit que $E$ n'est pas borné.
}

\parag{Bornes}{
On dit que $f : E\mapsto F$ est \important{majorée} sur $A \subset E$ si l'ensemble $f\left(A\right) \subset \mathbb{R}$ est majoré.

On dit que $f$ est \important{minorée} sur $A \subset E$ si l'ensemble $f\left(A\right) \subset \mathbb{R}$ est minorée.

Si $f\left(x\right)$ est à la fois minorée et majorée sur $A$, alors elle est \important{bornée} sur $A$. C'est équivalent à la propriété suivante:
\[\exists M \in \mathbb{R}_+ \telque \left|f\left(x\right)\right|_{x \in A} \leq M\]

On définit \important{la borne supérieure} de $f$:
\[\sup_{x \in A} f\left(x\right) \over{=}{déf} \sup\left\{f\left(x\right), x \in A\right\}\]

De la même manière, la \important{borne inférieure} est donnée par:
\[\inf_{x \in A} f\left(x\right) \over{=}{déf}  \inf\left\{f\left(x\right), x \in A\right\}\]
}

\parag{Maximum et minimum locaux}{
    Soit $f: E \mapsto F$, avec $x_{0} \in E$. On dit que $f$ admet un \important{maximum local} au point $x_0$ s'il existe $\delta > 0$ tel que pour tout $x \in D\left(f\right)$, alors:
    \[\left|x - x_0\right| \leq \delta \implies f\left(x\right) \leq f\left(x_0\right)\]

    En d'autre mots, il existe un voisinage autour de $x_0$, dans lequel toutes les images sont plus petites ou égales à celle de $f\left(x_0\right)$.

    De la même manière, on dit que $f$ admet un \important{minimum local} au point $x_0$ s'il existe $\delta > 0$ tel que pour tout $x \in D\left(f\right)$, alors:
    \[\left|x - x_0\right| \leq \delta \implies f\left(x\right) \geq f\left(x_0\right)\]
}

\parag{Maximum et minimum globaux}{
    Soit $f: E \mapsto F$ et $M \in \left\{f\left(x\right), x \in E\right\} = f\left(E\right)$ (c'est juste une manière \textit{fancy} de dire que $M$ est une valeur de la fonction), tels que pour tout $x \in E$, alors on a
    \[f\left(x\right) \leq M\]

    On appelle $M$ le \important{maximum global} de $f$, et on le note:
    \[\max_{x \in E} f\left(x\right) = M\]


    De la même manière, si on a $m \in f\left(E\right)$, tel que pour tout $x \in E$, alors on a
    \[f\left(x\right) \geq m\]

    On appelle $m$ le \important{minimum global} de $f$, et on le note:
    \[\min_{x \in E} f\left(x\right) = m\]

    Si $f\left(x_0\right) = M$ ou $f\left(x_0\right) = m$, on dit que la fonction $f$ atteint son maximum global, ou son minimum global, respectivement, sur $E$ au point $x_0$.
}

\parag{Remarque 1}{
    Si $\max_{x \in E} f\left(x\right)$ existe, alors $f$ est majorée sur $E$, et
    \[\sup_{x \in E} f\left(x\right) = \max_{x \in E} f\left(x\right)\]

    De la même manière, si $\min_{x \in E} f\left(x\right)$ existe, alors $f$ est minorée sur $E$, et
    \[\inf_{x \in E} f\left(x\right) = \min_{x \in E} f\left(x\right)\]
}

\parag{Remarque 2}{
Une fonction bornée sur $E$ n'atteint pas forcément son min ou max sur $E$.

\subparag{Exemple}{
Par exemple, $f\left(x\right) = x^2 + 3$ sur $E = \left]0,1\right[ $ est bornée, mais elle n'atteint ni son minimum, ni son maximum sur $E$.
}
}

\parag{Surjectivité}{
    Une fonction $f: E \mapsto F$ est \important{surjective} si pour tout $y \in F$, il existe au moins un $x \in E$ tel que $f\left(x\right) = y$.

    \subparag{Remarque}{
        Si $f$ n'est pas surjective, on peut réduire l'ensemble d'arrivée $F$ pour que cela devienne le cas.
    }
}

\parag{Injectivité}{
    Une fonction $f: E \mapsto F$ est \important{injective} si pour tout $y \in F$, il existe au plus un $x \in E$ tel que $f\left(x\right) = y$.

    La définition suivante est équivalente:
    \[f\left(x_1\right) = f\left(x_2\right) \implies x_1 = x_2\]
    avec $x_1, x_2 \in E$.

    \subparag{Remarque}{
        Si $f$ n'est pas injective, alors on peut réduire l'ensemble de départ $E$ pour que cela devienne le cas.
    }

    \subparag{Test sur un graphique}{
        Si on peut tracer une droite horizontale qui croise plus d'une fois la fonction, alors elle n'est pas injective.
    }
}

\parag{Bijectivité}{
    Si une fonction $f: E \mapsto F$ est à la fois injective et surjective, alors elle est aussi \important{bijective}.
}

\parag{Réciprocité}{
    Si $f: E \mapsto F$ est bijective, on peut définir la fonction réciproque par la formules suivante:
    \[y = f\left(x\right), x \in E \implies x = f^{-1}\left(y\right), y \in F\]
}

\parag{Remarque}{
    Par convention, on choisit les domaines suivants pour que les fonctions trigonométriques soient bijectives:
    \begin{itemize}
        \item $\sin x : \left[-\frac{\pi}{2}, \frac{\pi}{2}\right] \mapsto \left[-1, 1\right] $
        \item $\cos x : \left[0, \pi\right] \mapsto \left[-1, 1\right] $
        \item $\tan x : \left]-\frac{\pi}{2}, \frac{\pi}{2}\right[ \mapsto \mathbb{R}$
        \item $\cot x : \left]0, \pi\right[ \mapsto \mathbb{R}$
    \end{itemize}

    On peut donc définir les fonctions réciproques:
    \begin{itemize}
        \item $\arcsin x : \left[-1, 1\right] \mapsto \left[-\frac{\pi}{2}, \frac{\pi}{2}\right]$
        \item $\arccos x : \left[-1, 1\right] \mapsto \left[0, \pi\right]$
        \item $\arctan x : \mathbb{R} \mapsto \left]-\frac{\pi}{2}, \frac{\pi}{2}\right[$
        \item $\arccot x : \mathbb{R} \mapsto \left]0, \pi\right[$
    \end{itemize}
}

\parag{Remarque}{
    Les graphiques des fonctions réciproques sont symétriques par rapport à la droite $y = x$.
}

\end{document}
