\documentclass[a4paper]{article}

% Expanded on 2022-04-04 at 10:17:44.

\usepackage{../../style}

\title{Analyse 2}
\author{Joachim Favre}
\date{Lundi 04 avril 2022}

\begin{document}
\maketitle

\lecture{13}{2022-04-04}{On monte en ordres}{
}

\parag{Rappel}{
    En général, les deux réciproques sont fausses, l'existence des dérivées directionnelles n'implique pas la dérivabilité de la fonction, et l'existence des dérivées partielles n'implique pas l'existence des dérivées directionnelles.

    Cependant, nous avons le théorème qui suit.
}

\parag{Théorème 2 sur la dérivabilité}{
    Soit $E \subset \mathbb{R}^n$ un ensemble ouvert, $f : E \mapsto \mathbb{R}$ et un point $\bvec{a} \in E$.

    Supposons qu'il existe $\delta > 0$ tel que toutes les dérivées partielles $\frac{\partial f}{\partial x_k}\left(\bvec{a}\right)$ existent sur $B\left(\bvec{a}, \delta\right)$ et sont continues en $\bvec{a}$. Alors, $f$ est dérivable en $\bvec{a}\in E$.
}

\subsection{Dérivées partielles d'ordre supérieur}
\parag{Définition: Fonction dérivée partielle}{
    Soit $f: E \mapsto \mathbb{R}$, où $E \subset \mathbb{R}^n$ est ouvert, une fonction telle que $\frac{\partial f}{\partial x_k}$ existe pour un $k$, avec $1 \leq k \leq n$, en tout point de $E$. Alors $\frac{\partial f}{\partial x_k}\left(\bvec{x}\right)$ où $\bvec{x} \in E$, est la \important{fonction $k$-ème dérivée partielle}.
}

\parag{Définition: Dérivée partielle d'ordre supérieur}{
    Soit $f: E \mapsto \mathbb{R}$ une fonction telle que $\frac{\partial f}{\partial x_k}$ existe en tout $\bvec{x} \in E$. Si la fonction $\frac{\partial f}{\partial x_k}$ admet à son tour une dérivée partielle par rapport à $x_i$ (potentiellement une autre variable), on pose:
    \[\frac{\partial }{\partial x_i} \left(\frac{\partial f}{\partial x_k}\right) \over{=}{déf} \frac{\partial^2 f}{\partial x_i \partial x_k}\]

    Nous appelons ceci la \important{dérivée partielle seconde}. Nous pouvons définir ainsi, lorsqu'elles existent, les dérivées partielles d'ordre $p$. Par exemple:
    \[\frac{\partial}{\partial x_j} \left(\frac{\partial}{\partial x_i} \left(\frac{\partial f}{\partial x_k}\right)\right) = \frac{\partial^3 f}{\partial x_j \partial x_i \partial x_k}\]
}

\parag{Définition: Classe}{
    Soit $E \subset \mathbb{R}^n$ un ensemble ouvert.

    Une fonction $f: E \mapsto \mathbb{R}$ est dite de \important{classe} $C^p$ sur $E$, si toutes les dérivées partielles d'ordre $\leq p$ existent et sont continues sur $E$.

    \subparag{Remarque}{
        Le théorème 2 sur la dérivabilité nous dit que, si $f$ est de classe $C^1$ sur $E$, alors $f$ est dérivable sur $E$.
    }
}


\parag{Théorème de Schwarz}{
    Soit $f : E \mapsto \mathbb{R}$ et $\bvec{a} \in E$ tel que les dérivées partielles secondes $\frac{\partial^2 f}{\partial x_i \partial x_j}$ et $\frac{\partial^2 f}{\partial x_j \partial x_i}$ existent dans un voisinage de $\bvec{a}$ et sont continues en $\bvec{a}$ (en d'autres mots, $f$ est de classe $C^2$ sur un ensemble ouvert contenant $\bvec{a}$).

    Alors, nous avons:
    \[\frac{\partial^2 f}{\partial x_i \partial x_j}\left(\bvec{a}\right) = \frac{\partial^2 f}{\partial x_j \partial x_i}\left(\bvec{a}\right)\]
}

\parag{Définition: Matrice Hessienne}{
    La \important{matrice Hessienne} est la matrice des dérivées partielles d'ordre 2 pour une fonction $E \mapsto \mathbb{R}^n$, où $E \subset \mathbb{R}^n$ est un sous-ensemble ouvert, notée:
    \[Hess\left(f\right)\left(\bvec{a}\right) = \begin{pmatrix} \dfrac{\partial^2 f}{\partial x_1^2}\left(\bvec{a}\right) & \dfrac{\partial^2 f}{\partial x_2 x_1}\left(\bvec{a}\right) & \cdots & \dfrac{\partial^2 f}{\partial x_n \partial x_1}\left(\bvec{a}\right) \\ \dfrac{\partial^2 f}{\partial x_1 \partial x_2}\left(\bvec{a}\right) & \dfrac{\partial^2 f}{\partial x_2^2}\left(\bvec{a}\right) & \cdots & \dfrac{\partial^2 f}{\partial x_n \partial x_2}\left(\bvec{a}\right) \\ \vdots & \vdots & \ddots  & \vdots \\ \dfrac{\partial^2 f}{\partial x_1 \partial x_n}\left(\bvec{a}\right) & \dfrac{\partial^2 f}{\partial x_2 \partial x_n}\left(\bvec{a}\right) & \cdots & \dfrac{\partial^2 f}{\partial x_n^2}\left(\bvec{a}\right) \end{pmatrix} \]

    \subparag{Remarque}{
        Si $f$ est de classe $C^2$ sur $E$, alors la matrice Hessienne est symétrique, c'est-à-dire:
        \[Hess\left(f\right)\left(\bvec{a}\right) = Hess\left(f\right)\left(\bvec{a}\right)^T\]
    }
}

\parag{Remarque}{
    Nous avions trouvé que si $f\left(x, y\right)$ est dérivable en $\left(x_0, y_0\right)$, alors le plan tangent à la surface $z = f\left(x, y\right)$ au point $\left(x_0, y_0, f\left(x_0, y_0\right)\right)$ est défini par l'équation:
    \[z = f\left(x_0, y_0\right) + \left<\nabla f\left(x_0, y_0\right), \left(x - x_0, y - y_0\right)\right>\]

    Si $f$ n'est pas dérivable en ce point, alors ce plan n'est pas un plan tangent, même si le gradient $\nabla f\left(x_0, y_0\right)$ existe. Le plan tangent n'est simplement pas défini dans ce cas.
}

\end{document}
