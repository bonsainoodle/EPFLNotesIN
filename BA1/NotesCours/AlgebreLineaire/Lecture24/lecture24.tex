\documentclass[a4paper]{article}

% Expanded on 2021-12-13 at 13:21:02.

\usepackage{../../style}

\title{Algèbre linéaire}
\author{Joachim Favre}
\date{Lundi 13 décembre 2021}

\begin{document}
\maketitle

\lecture{24}{2021-12-13}{Diagonalisation en base orthonormée}{
}

\parag{Méthode}{
    Utilisons notre approche de moindres carrées. Les solutions $\bvec{\hat{\beta}}$ de $A \bvec{\beta} = \bvec{y}$ au sens des moindres carrés sont les solutions des équations normales:
    \[A^T A \bvec{\hat{\beta}} = A^T \bvec{y}\]

    On sait que la solution $\bvec{\hat{\beta}}$ existe toujours. La solution est unique, si et seulement si les colonnes de $A$ sont linéairement indépendants, donc si et seulement si les $x_i$ ne sont pas tous égaux (si et seulement si au moins deux sont différents).
}

\parag{Justification du moindre carré}{
    Ces formules nous permettent de calculer $\hat{b}_0, \hat{b}_1$ tels que $\left\|A \bvec{\hat{\beta}} - \bvec{y}\right\|^2$ est aussi petit que possible. On minimise donc la somme des carrés des erreurs verticales.
}

\section{Matrices symétriques}

\subsection{Diagonalisation des matrices symétriques}

\parag{Rappels}{
    Une matrice $A$ est symétrique si $A = A^T$. On remarque donc que $A$ doit nécessairement être carrée.
}

\parag{Théorème}{
    Les valeurs propres d'une matrice symétrique sont toujours toutes réelles.
}

\parag{Théorème}{
    Soit $A$ une matrice symétrique, i.e. $A^T = A$.

    Si $\bvec{v}_1$ et $\bvec{v}_2$ sont deux vecteurs propres de $A$ associés à des valeurs propres $\lambda_1$ et $\lambda_2$ \textit{distinctes}, alors $\bvec{v}_1$ et $\bvec{v}_2$ sont orthogonaux. En d'autres mots:
    \[\lambda_1 \neq \lambda_2 \implies \bvec{v}_1 \dotprod \bvec{v}_2 = 0\]
}

\end{document}
