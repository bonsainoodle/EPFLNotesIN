\documentclass[a4paper]{article}

% Expanded on 2022-03-28 at 10:16:21.

\usepackage{../../style}

\title{Analyse 2}
\author{Joachim Favre}
\date{Lundi 28 mars 2022}

\begin{document}
\maketitle

\lecture{11}{2022-03-28}{Et le retour des différences maintenant! \smiley}{
}

\section[Calcul différentiel]{Calcul différentiel des fonctions de plusieurs variables}
\subsection{Dérivée partielles et le gradient}
\parag{Définition: dérivée partielle}{
    Soit $f: E \mapsto \mathbb{R}^n$ une fonction, où $E \subset \mathbb{R}^n$ est un ensemble ouvert, et soit la fonction d'une seule variable suivante:
    \[g\left(s\right) = f\left(a_1, a_2, \ldots, a_{k-1}, s, a_{k+1}, \ldots, a_n\right), \mathspace \text{où } \bvec{a} = \left(a_1, \ldots, a_n\right) \in E\]

    Le domaine de définition de $g$ est:
    \[D_g = \left\{s \in \mathbb{R} \telque \left(a_1, a_2, \ldots, a_{k - 1}, s, a_{k+1}, \ldots, a_n\right) \in E\right\}\]

    Alors, si $g$ est dérivable en $a_k \in D_g$, on dit que la \important{$k$-ème dérivée partielle} de $f$ en $\bvec{a} \in E$ existe est égale à $g'\left(a_k\right)$, notée:
    \[g'\left(a_k\right) \over{=}{déf} \frac{\partial f}{\partial x_k}\left(\bvec{a}\right) = D_k f\left(\bvec{a}\right)\]

    On remarque que nous avons:
    \[\frac{\partial f}{\partial x_k}\left(\bvec{a}\right) = \lim_{t \to 0} \frac{g\left(a_k + t\right) - g\left(a_k\right)}{t} = \lim_{t \to 0} \frac{f\left(\bvec{a} + t \bvec{e_k}\right) - f\left(\bvec{a}\right)}{t}\]
}

\parag{Définition: Gradient}{
    Si toutes les dérivées partielles existent en $\bvec{a} \in E$:
    \[\frac{\partial f}{\partial x_1}\left(\bvec{a}\right), \mathspace \ldots, \mathspace \frac{\partial f}{\partial x_n}\left(\bvec{a}\right)\]

    Alors, on définit le \important{gradient} de $f$ en $\bvec{a}$ comme:
    \[\nabla f\left(\bvec{a}\right) = \left(\frac{\partial f}{\partial x_1}\left(\bvec{a}\right), \ldots, \frac{\partial f}{\partial x_n}\left(\bvec{a}\right)\right)\]

    Notez que $\nabla$ s'appelle le \important{nabla}.
}

\parag{Définition: Dérivée directionnelle}{
    Soit $E \subset \mathbb{R}^n$ un sous-ensemble ouvert. Soient aussi $\bvec{a} \in E$, et $\bvec{v} \in\mathbb{R}^n$ où $\bvec{v} \neq \bvec{0}$.

    Nous savons que la droite passant par $\bvec{a}$ en direction de $\bvec{v}$ admet la paramétrisation suivante:
    \[\bvec{\ell}\left(t\right) = \bvec{a} + t \bvec{v}, \mathspace \forall t \in \mathbb{R}\]

    Considérons une fonction $f :E \mapsto \mathbb{R}$, et soit la fonction d'une variable $t$ suivante:
    \[g\left(t\right) \over{=}{déf} f\left(\bvec{\ell}\left(t\right)\right) = f\left(\bvec{a} + t \bvec{v}\right), \mathspace \forall t \in \left\{t \in \mathbb{R} \telque \bvec{a} + t \bvec{v} \in E\right\}\]

    Si $g$ est dérivable en $t = 0$, on dit qu'il existe la \important{dérivée directionnelle} de $f$ en $\bvec{a}$ suivant le vecteur $\bvec{v}$ (dans la direction de $\bvec{v}$). La dérivée directionnelle de $f$ en $\bvec{a}$ en direction de $\bvec{v}$ est donnée par:
    \[\lim_{t \to 0} \frac{g\left(t\right) - g\left(0\right)}{t} = \lim_{t \to 0} \frac{f\left(\bvec{a} + t \bvec{v}\right) - f\left(\bvec{a}\right)}{t} \over{=}{déf}  Df\left(\bvec{a}, \bvec{v}\right) = \frac{\partial f}{\partial \bvec{v}}\left(\bvec{a}\right)\]

    \subparag{Remarque}{
        \begin{enumerate}[left=0pt]
            \item La dérivée partielle est un cas particulier de la dérivée directionnelle.

                  Ainsi, si toutes les dérivées directionnelles existent en $\bvec{a}$ (pour tout $\bvec{v} \neq \bvec{0}$), alors toutes les dérivées partielles existent aussi en ce point. Cependant, la réciproque est fausse en générale.
            \item Si la dérivée directionnelle de $f$ en $\bvec{a}$ suivant $\bvec{v}$ existe, alors la dérivée directionnelle de $f$ en $\bvec{a}$ suivant $\lambda \bvec{v}$ existe pour tout $\lambda \in \mathbb{R}^*$. Il suffit donc de calculer les dérivées directionnelles suivant les vecteurs unitaires $\left\|\bvec{v}\right\| = 1$.
        \end{enumerate}
    }
}

\subsection{Dérivabilité et différentielle}
\parag{Définition: dérivabilité}{
    Soit $f: E \mapsto \mathbb{R}$, où $E \subset \mathbb{R}^n$ est ouvert. De plus, soit $\bvec{a} \in E$.

    On dit que $f$ est \important{dérivable} (ou \important{différentiable}) au point $\bvec{a}$ s'il existe une transformation linéaire $L_{\bvec{a}} : \mathbb{R}^n \mapsto \mathbb{R}$ et une fonction $r: E \mapsto \mathbb{R}$ telles que:
    \[f\left(\bvec{x}\right) = f\left(\bvec{a}\right) + L_{\bvec{a}}\left(\bvec{x} - \bvec{a}\right) + r\left(\bvec{x}\right), \mathspace\forall x \in E\]
    et aussi $r$ doit être telle que:
    \[\lim_{\bvec{x} \to \bvec{a}} \frac{r\left(\bvec{x}\right)}{\left\|\bvec{x} - \bvec{a}\right\|} = 0\]

    $L_{\bvec{a}}$ s'appelle la \important{différentielle} de $f$ au point $\bvec{a} \in E$, et est aussi parfois notée:
    \[L_{\bvec{a}} = df\left(\bvec{a}\right)\]
}

\end{document}
