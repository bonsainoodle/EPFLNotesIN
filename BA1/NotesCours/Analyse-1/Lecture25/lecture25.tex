\documentclass[a4paper]{article}

% Expanded on 2021-12-20 at 10:17:20.

\usepackage{../../style}

\title{Analyse}
\author{Joachim Favre}
\date{Lundi 20 décembre 2021}

\begin{document}
\maketitle

\lecture{25}{2021-12-20}{Intégrales généralisées}{
}

\subsection{Intégrales généralisées}

\subsubsection{Intégrales généralisées sur un intervalle borné}

\parag{Définition}{
Soient $a < b$  et $f: \left[a, b\right[  \mapsto \mathbb{R}$ une fonction continue. Alors, on définit l'intégrale généralisée par la limite:
\[\int_{a}^{b^-} f\left(t\right)dt \over{=}{déf} \lim_{x \to b^-} \int_{a}^{x} f\left(t\right)dt\]
si cette limite existe. Si elle n'existe pas, alors $\int_{a}^{b^-} f\left(t\right)dt$ est divergente.

De la même manière soient $a < b$  et $f: {\color{red}\left]a, b\right]}  \mapsto \mathbb{R}$ une fonction continue. Alors, on définit l'intégrale généralisée par la limite:
\[\int_{{\color{red}a^+}}^{b} f\left(t\right)dt \over{=}{déf} \lim_{x \to {\color{red}a^+}} \int_{x}^{b} f\left(t\right)dt\]
si cette limite existe. Si elle n'existe pas, alors $\int_{a^+}^{b} f\left(t\right)dt$ est divergente.
}

\parag{Proposition: Critère de comparaison}{
Soient $f, g : \left[a, b\right[ $ deux fonctions continues telles qu'il existe $c \in \left[a, b\right[ $ tel que:
\[0 \leq f\left(x\right) \leq g\left(x\right), \mathspace \forall x \in \left[c, b\right[ \]

        Alors:
        \[\int_{a}^{b^-} g\left(x\right)dx \text{ converge} \implies \int_{a}^{b^-} f\left(x\right)dx \text{ converge} \]
        \[\int_{a}^{b^-} f\left(x\right)dx \text{ diverge} \implies \int_{a}^{b^-} g\left(x\right)dx \text{ diverge} \]

        Il existe un critère similaire pour $f, g : \left]a, b\right] \mapsto \mathbb{R}$ continues.
    }

    \parag{Formule à retenir pour le critère de comparaison}{
        \begin{functionbypart}{\int_{a}^{b^-} \frac{dt}{\left(b - t\right)^{\alpha}}}
            \frac{1}{1 - \alpha} \left(b - a\right)^{1 - \alpha}, \mathspace \alpha < 1  \\
            \text{divergente}, \mathspace \alpha \geq 1
        \end{functionbypart}
    }

    \parag{Corollaire}{
    Soit $f : \left[a, b\right[ \mapsto \mathbb{R}$ une fonction continue. Supposons qu'il existe $\alpha \in \mathbb{R}$ tel que:
    \[\lim_{x \to b^-} f\left(x\right)\left(b - x\right)^{\alpha} = \ell \in \mathbb{R}^* \mathspace \text{(donc $\ell \neq 0$)}\]

    Alors, l'intégrale généralisée:
    \begin{functionbypart}{\int_{a}^{b^-} f\left(t\right)dt}
        \text{converge }, \mathspace \alpha < 1 \\
        \text{diverge }, \mathspace \alpha \geq 1
    \end{functionbypart}

    D'une façon similaire, soit $f : {\color{red}\left]a, b\right]} \mapsto \mathbb{R}$ une fonction continue. Supposons qu'il existe $\alpha \in \mathbb{R}$ tel que:
    \[\lim_{{\color{red}x \to a^+}} f\left(x\right){\color{red}\left(x - a\right)^{\alpha}} = \ell \in \mathbb{R}^*\]

    Alors, l'intégrale généralisée:
    \begin{functionbypart}{{\color{red}\int_{a^+}^{b}} f\left(t\right)dt}
        \text{converge }, \mathspace \alpha < 1 \\
        \text{diverge }, \mathspace \alpha \geq 1
    \end{functionbypart}
    }

    \parag{Définition}{
    Soient $a < b$, $f : \left]a, b\right[  \mapsto \mathbb{R}$ continue, et $c \in \left]a ,b\right[ $ arbitraire.

    Alors, l'intégrale généralisée:
    \[\int_{a^+}^{b^-} f\left(t\right)dt \over{=}{déf} \int_{a^+}^{c} f\left(t\right)dt + \int_{c}^{b^-} f\left(t\right)dt\]

    Elle converge si et seulement si les deux intégrales généralisées convergent.
    }

    \subsubsection{Intégrales généralisées sur un intervalle non-borné}

    \parag{Définition}{
    Soit $f : \left[a, +\infty\right[ \mapsto \mathbb{R}$ une fonction continue. Alors, l'intégrale généralisée
    \[\int_{a}^{\infty} f\left(t\right)dt \over{=}{déf} \lim_{x \to +\infty} \int_{a}^{x} f\left(t\right)dt\]
    si la limite existe. Si la limite n'existe pas, alors l'intégrale généralisée $\int_{a}^{\infty} f\left(t\right)dt$ est divergente.

    De la même manière, soit $f : {\color{red}\left]-\infty, b\right]} \mapsto \mathbb{R}$ une fonction continue. Alors, l'intégrale généralisée
    \[{\color{red}\int_{-\infty}^{b}} f\left(t\right)dt \over{=}{déf} \lim_{{\color{red}x \to -\infty}} {\color{red}\int_{x}^{b}} f\left(t\right)dt\]
    si la limite existe. Si la limite n'existe pas, alors l'intégrale généralisée $\int_{-\infty}^{b} f\left(t\right)dt$ est divergente.
    }

    \parag{Critère de comparaison}{
        Si $0 \leq f\left(x\right) \leq g\left(x\right)$ pour tout $x > c$ pour un certain $c > a$, alors:
        \[\int_{a}^{\infty} g\left(x\right)dx \text{ converge} \implies \int_{a}^{\infty} f\left(x\right)dx \text{ converge}\]
        \[\int_{a}^{\infty} f\left(x\right)dx \text{ diverge} \implies \int_{a}^{\infty} g\left(x\right)dx \text{ diverge}\]
    }

    \parag{Formule à retenir pour le critère de comparaison}{
        \begin{functionbypart}{\int_{1}^{\infty} \frac{dx}{x^{\beta}}}
            \frac{1}{\beta - 1}, \mathspace \beta > 1 \\
            \text{diverge }, \mathspace \beta \leq 1
        \end{functionbypart}
    }

    \parag{Corollaire}{
    Soient $f : \left[a, +\infty\right[ \mapsto \mathbb{R}$ une fonction continue et $\beta \in \mathbb{R}$ tel que
    \[\lim_{x \to \infty} f\left(x\right) x^{\beta} = \ell \in \mathbb{R}^* \text{(donc $\ell \neq 0$)}\]


    Alors, $\int_{a}^{\infty} f\left(t\right)dt$ converge si et seulement si $\beta > 1$, et diverge si et seulement si $\beta \leq 1$.
Alors, l'intégrale généralisée:
\begin{functionbypart}{\int_{a}^{+\infty} f\left(t\right)dt}
    \text{converge }, \mathspace \beta > 1 \\
    \text{diverge }, \mathspace \beta \leq 1
\end{functionbypart}

\subparag{Note personnelle: mnémotechnie}{
    Pour retrouver si notre constante doit être plus grande que 1 ou plus petite que 1 pour que notre intégrale converge, il nous suffit de regarder une fonction simple, telle que $\frac{1}{x^2}$. En effet:
    \[\int_{0^+}^{1} \frac{1}{x^2} \text{ diverge} \implies \alpha < 1 \text{ pour la convergence}\]
    \[\int_{1}^{+\infty} \frac{1}{x^2} \text{ converge} \implies \beta > 1 \text{ pour la convergence}\]

    Il nous suffit de se souvenir que le point important est 1, qui ne converge jamais.
}
}

\parag{Définition}{
    Soit $f$ une fonction continue sur $\left]a, +\infty\right] $. Alors l'intégrale généralisée
    \[\int_{a^+}^{\infty} f\left(t\right)dt \over{=}{déf} \int_{a^+}^{c} f\left(t\right)dt + \int_{c}^{\infty} f\left(t\right)dt, \mathspace \text{pour un } c \in \left]a, \infty\right] \]
    converge si et seulement si les deux intégrales généralisées convergent.

    La définition ne dépend pas du choix du $c$.
}

\parag{Définition}{
    Soit $f : \mathbb{R}\mapsto  \mathbb{R}$ une fonction continue. Alors, on peut aussi considérer l'intégrale généralisée:
    \[\int_{-\infty}^{\infty} \over{=}{déf} \int_{-\infty}^{c} f\left(t\right)dt + \int_{c}^{\infty} f\left(t\right)dt, \mathspace c \in \mathbb{R}\]
    qui est convergente si et seulement si les deux intégrales convergent.

    La définition ne dépend pas du choix du $c$.
}

\end{document}
