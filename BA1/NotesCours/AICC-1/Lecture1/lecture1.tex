\documentclass{article}

% Expanded on 2021-09-20 at 21:32:34.

\usepackage{../../style}

\title{Advanced information, computation, communication I}
\author{Joachim Favre}
\date{Mardi 21 septembre 2021}

\begin{document}
\maketitle

\lecture{1}{2021-09-21}{Introduction to propositional logic}{
}

\section{Propositional logic}
\subsection{Introduction}

\parag{Propositions}{
    We define a \important{proposition} to be a declarative sentence that is either \important{true or false} (this is very important).
}

\parag{Atomic propositions}{
    \important{Atomic propositions} are propositions that cannot be expressed in terms of simpler propositions.
}


\subsection{Logical connectives}
\parag{Compound propositions}{
    \important{Compound propositions} are constructed from \important{logical connectives} and other propositions.
}

\parag{Truth table}{
    A \important{truth table} lists all possible truth values of the propositional variable occurring in a compound propositions, and corresponding truth values of the compound propositions.

    We will see examples hereinafter.
}

\parag{Negation}{
    Let $p$ be a proposition. The \important{negation} of $p$, denoted $\lnot p$ (or $\bar{p}$) is the statement ``it is not the case that $p$''. We read $\lnot p$ ``not $p$''.

    Here is its truth table:
    \begin{center}
        \begin{tabular}{c|c}
            $p$ & $\lnot p$ \\
            \hline
            T   & F         \\
            F   & T
        \end{tabular}
    \end{center}
}

\parag{Conjunction}{
    Let $p$ and $q$ be propositions. The \important{conjunction} of $p$ and $q$, denoted $p \land q$, is the proposition ``$p$ and $q$''.

    Here is its truth table:
    \begin{center}
        \begin{tabular}{c|c|c}
            $p$ & $q$ & $p \land q$ \\
            \hline
            T   & T   & T           \\
            F   & T   & F           \\
            T   & F   & F           \\
            F   & F   & F
        \end{tabular}
    \end{center}
}

\parag{Disjunction}{
    Let $p$ and $q$ be propositions. The \important{disjunction} of $p$ and $q$, denoted $p \lor q$, is the proposition ``$p$ or $q$''.

    Here is its truth table:
    \begin{center}
        \begin{tabular}{c|c|c}
            $p$ & $q$ & $p \lor q$ \\
            \hline
            T   & T   & T          \\
            F   & T   & T          \\
            T   & F   & T          \\
            F   & F   & F
        \end{tabular}
    \end{center}
}

\parag{Exclusive or}{
    Let $p$ and $q$ be propositions. The \important{exclusive or}, also called xor, of $p$ and $q$, denoted $p \oplus q$, is the proposition ``$p$ or $q$, but not both''.

    Here is its truth table:
    \begin{center}
        \begin{tabular}{cc|c}
            $p$ & $q$ & $p \oplus q$ \\
            \hline
            T   & T   & F            \\
            F   & T   & T            \\
            T   & F   & T            \\
            F   & F   & F
        \end{tabular}
    \end{center}
}

\parag{Implications}{
    Let $p$ and $q$ be propositions. The \important{conditional statement} $p \to q$ is the proposition ``if $p$, then $q$''. We call $p$ the \important{premise} (or hypothesis, or antecedent), and we call $q$ the \important{conclusion} (or consequence).

    Here is its truth table:
    \begin{center}
        \begin{tabular}{cc|c}
            $p$ & $q$ & $p \to q$ \\
            \hline
            T   & T   & T         \\
            T   & F   & F         \\
            F   & T   & T         \\
            F   & F   & T
        \end{tabular}
    \end{center}
}

\parag{Biconditional}{
    Let $p$ and $q$ be propositions. The \important{biconditional statement} $p \leftrightarrow q$ is the propositions ``$p$ if and only $q$'' ($p$ iff $q$).

    Here is its truth table:
    \begin{center}
        \begin{tabular}{c|c|c}
            $p$ & $q$ & $p \leftrightarrow q$ \\
            \hline
            T   & T   & T                     \\
            T   & F   & F                     \\
            F   & T   & F                     \\
            F   & F   & T
        \end{tabular}
    \end{center}
}

\parag{Precedence}{
    To avoid having to use thousands of parenthesis, we use the following rule of precedence (priority of operations):
    \begin{center}
        \begin{tabular}{|c|c|}
            \hline
            \textbf{Operator} & \textbf{Precedence} \\
            \hline
            $\lnot$           & 1                   \\
            \hline
            $\land$           & 2                   \\
            \hline
            $\lor$            & 3                   \\
            \hline
            $\to$             & 4                   \\
            \hline
            $\leftrightarrow$ & 5                   \\
            \hline
        \end{tabular}
    \end{center}
}

\end{document}
