\documentclass[a4paper]{article}

% Expanded on 2021-11-18 at 08:22:38.

\usepackage{../../style}

\title{Algèbre linéaire}
\author{Joachim Favre}
\date{Jeudi 18 novembre 2021}

\begin{document}
\maketitle

\lecture{17}{2021-11-18}{Diagonalisation}{
}

\parag{Propriété de la diagonalisation}{
Si $A = PDP^{-1}$, où:
\[D = \begin{bmatrix} \lambda_1 & & \\ & \ddots & \\ &  & \lambda_n \end{bmatrix} \]

Alors, on a:
\[A^2 = A\cdot A = PDP^{-1} \cdot PDP^{-1} = PDDP^{-1}= PD^2 P^{-1} \]

Calculer $D^2$ est très facile:
\[D^2 = \begin{bmatrix} \lambda_1^2 &  &  \\  & \ddots &  \\  &  & \lambda_n^2 \end{bmatrix} \]

De manière très générale:
\[A^k = A\cdot A^{k-1} = P D P^{-1} \cdot PD^{k-1}P^{-1} = PD^{k}P^{-1}\]

Ce n'est pas très important, mais on voit que le comportement de $A^k$ pour $k$ grand est déterminé par ses valeurs propres (si leur valeur absolue est strictement plus petite que 0, alors $\lambda^k$ tend vers 0).
}

\parag{Théorème}{
    $A \in \mathbb{R}^{n \times n}$ est diagonalisable si et seulement si elle a $n$ vecteurs propres $\bvec{v}_1, \ldots, \bvec{v}_n$ linéairement indépendants.

    Dans ce cas, on peut choisir:
    \[P = \begin{bmatrix} & &  \\ \bvec{v}_1 & \ldots & \bvec{v}_n \\  &  &  \end{bmatrix}, \mathspace D = \begin{bmatrix} \lambda_1 &  &  \\  & \ddots &  \\  &  & \lambda_n \end{bmatrix} \]
    où $A \bvec{v}_i = \lambda_i \bvec{v}_i$.

    On appelle $\left(\bvec{v}_1, \ldots, \bvec{v}_n\right)$ une \important{base de vecteurs propres de $A$}.
}

\parag{Théorème}{
    Tout matrice $n \times n$ admettant $n$ valeurs propres \textit{distinctes} est diagonalisable.

    \subparag{Remarque}{
        Ceci est une condition suffisante, mais non pas nécessaire.
    }
}

\parag{Généralisation}{
    On se demande maintenant ce qu'il faut pour qu'une matrice $A \in \mathbb{R}^{n \times n}$ soit diagonalisable. Par notre théorème, on sait qu'il nous faut $n$ vecteurs propres linéairement indépendants.

    Si $\lambda$ est une valeur propre de $A$, on se demande donc combien de vecteurs propres associés à $\lambda$ qui sont linéairement indépendants on peut trouver. Tous les vecteurs propres associés à $\lambda$ sont dans l'espace associé à $\lambda$, qui est donné par $\ker\left(A - \lambda I_n\right)$. Or, on sait qu'on peut choisir au maximum
    \[\dim\ker\left(A - \lambda I_n\right)\]
    vecteurs dans ce sous espace. On appelle ceci la \important{multiplicité géométrique} de $\lambda$ (à ne pas confondre avec la multiplicité algébrique).
}

\parag{Théorème (hors de la matière du cours)}{
    La multiplicité géométrique d'une valeur propre est toujours plus petite ou égale à sa multiplicité algébrique.
}


\end{document}
