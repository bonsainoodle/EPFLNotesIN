\documentclass[a4paper]{article}

% Expanded on 2021-11-03 at 10:13:00.

\usepackage{../../style}

\title{Analyse}
\author{Joachim Favre}
\date{Mercredi 03 novembre 2021}

\begin{document}
\maketitle

\lecture{13}{2021-11-03}{Définition epsilon-delta des limites}{
}

\parag{Composition de fonctions}{
    Soient $f: E \mapsto F$ et $g : G \mapsto H$, où $E, F, G, H \subset \mathbb{R}$.

    Si $f\left(E\right) \subset G$, alors on peut définir la fonction composée:
    \[g \circ f\left(x\right) = g\left(f\left(x\right)\right) : E \mapsto H\]

    Si $f\left(G\right) \subset E$, alors on peut définir la fonction composée, avec l'autre ordre de composition:
    \[f \circ g\left(x\right) = f\left(g\left(x\right)\right) : G \mapsto F\]

    L'ordre de composition est important.
}

\parag{Remarque}{
    Si $f : E\mapsto F$ est bijective, alors il existe une fonction réciproque $f^{-1} : F \mapsto E$. Les conditions sont remplies pour avoir une fonction composée des deux côtés. Pour rappel, la fonction réciproque a comme définition:
    \[f^{-1}\left(y\right) = x \iff f\left(x\right) = y\]

    On a donc:
    \[f^{-1} \circ f\left(x\right) = f^{-1}\left(f\left(x\right)\right) = f^{-1}\left(y\right) = x \implies f^{-1} \circ f\left(x\right) = x \mathspace \forall x \in E\]

    De plus:
    \[f \circ f^{-1}\left(y\right) = f\left(f^{-1}\left(y\right)\right) = f\left(x\right) = y \implies f \circ f^{-1}\left(y\right) = y \mathspace \forall y \in F\]
}

\subsection{Limite d'une fonction}
\parag{Définition du voisinage}{
    Une fonction $f : E \mapsto F$ est définie au \important{voisinage} de $x_0 \in \mathbb{R}$ s'il existe $\delta > 0$ tel que
    \[\left\{x \in \mathbb{R} : 0 < \left|x - x_0\right| < \delta\right\} \subset E\]
}

\parag{Définition de la limite}{
    Une fonction $f: E \mapsto F$ définie au voisinage de $x_0$ admet pour \important{limite} le nombre réel $\ell$ lorsque $x$ tend vers $x_0$, si pour tout $\epsilon > 0$ il existe $\delta > 0$ tel que pour tout $x \in E$ tels que $0 < \left|x - x_0\right| \leq \delta$, on a :
    \[\left|f\left(x\right) - \ell\right| \leq \epsilon\]

    On dit que $x$ et $x_0$ sont $\delta$-proches, et que $f\left(x\right)$ et $\ell$ sont $\epsilon$-proches.

    \subparag{Notation}{
        On écrit:
        \[\lim_{x \to x_0} f\left(x\right) = \ell\]
    }
}

\parag{Proposition (Caractérisation de la limite d'une fonction à partir des suites)}{
    Soit $f : E \mapsto F$, et $A$ l'ensemble des suites $\left(a_n\right) \subset E \setminus \left\{x_0\right\}$ telles que $\lim_{n \to \infty} a_n = x_0$. On a :

    \[\lim_{x \to x_0} f\left(x\right) = \ell \iff \forall \left(a_n\right) \in A, \lim_{n \to \infty} f\left(a_n\right) = \ell\]
}

\parag{Remarque}{
    ``Toute suite'' dans la définition est très important.
}

\parag{Corollaire}{
    Soit $f  : E \mapsto F$ définie au voisinage de $x_0$. Supposons que pour toute suite $\left(a_n\right) \subset E \setminus \left\{x_0\right\}$ telle que $\lim_{n \to \infty} a_n = x_0$, la suite $\left(f\left(a_n\right)\right)$ converge.

    Alors, $\lim_{x \to x_0} f\left(x\right)$ existe.
}

\parag{Proposition (unicité de la limite)}{
    Si $\lim_{x \to x_0} f\left(x\right) = \ell_1$ et $\lim_{x \to x_0} f\left(x\right) = \ell_2$, alors:
    \[\ell_1 = \ell_2\]
}

\parag{Proposition (Critère de Cauchy pour les fonctions)}{
    $\exists \lim_{x \to x_0} f\left(x\right)$ si et seulement si pour tout $\epsilon > 0$, il existe $\delta > 0$ tels que
    \[\forall x_1, x_2 \in \left\{x \in E \telque 0 < \left|x - x_0\right| \leq \delta\right\} \implies \left|f\left(x_1\right) - f\left(x_2\right)\right| \leq \epsilon\]

    \subparag{Autre formulation}{
        En d'autres mots, pour tout écart vertical arbitraire à la limite, il existe un écart horizontal à $x_0$, tel que pour toute paire de nombres $x_1$ et $x_2$ $\delta$-proche de $x_0$, alors elles leur image sont $\epsilon$-proches.
    }
}

\parag{Proposition (opérations algébriques sur les limites)}{
    Soient $f: E\mapsto \mathbb{R}$ et $g: E \mapsto \mathbb{R}$ telles que:
    \[\lim_{x \to x_0} f\left(x\right) = \ell_1 \in \mathbb{R} \mathspace \text{ et } \mathspace \lim_{x \to x_0} g\left(x\right) = \ell_2 \in \mathbb{R}\]

    Alors:
    \begin{itemize}
        \item $\lim\limits_{x \to x_0} \left(\alpha f\left(x\right) + \beta g\left(x\right)\right) = \alpha \ell_1 + \beta \ell_2$
        \item $\lim\limits_{x \to x_0} \left(f\left(x\right)g\left(x\right)\right) = \ell_1 \ell_2$
        \item $\lim\limits_{x \to x_0} \left(\frac{f\left(x\right)}{g\left(x\right)}\right) = \frac{\ell_1}{\ell_2}$ si $\ell_2 \neq 0$ et $g\left(x\right) \neq 0$ au voisinage de $x_0$.
    \end{itemize}
}

\parag{Limite de polynômes}{
    Pour tout polynôme et fonction rationnelle,
    \[\lim_{x \to x_0} f\left(x\right) = f\left(x_0\right) \]
    pour tout $x_0 \in \mathbb{R}$ sauf les zéros du dénominateur de $f\left(x\right)$.

    \subparag{Continuité}{
        Ce fait implique que ces fonctions sont continues partout sauf aux zéros de leur dénominateur.
    }
}

\parag{Théorème des deux gendarmes pour les fonctions}{
    Soient $f,g,h : E \mapsto F$ telles que:
    \begin{enumerate}
        \item $\lim\limits_{x \to x_0} f\left(x\right) = \lim\limits_{x \to x_0} g\left(x\right) = \ell$
        \item $\exists \alpha > 0$ tel que $\forall x \in \left\{x \in E \telque 0 < \left|x - x_0\right| \leq \alpha\right\}$ on a:
              \[f\left(x\right) \leq h\left(x\right) \leq g\left(x\right)\]

              En d'autres mots, il existe un $\alpha$ tels que pour tout $x$ $\alpha$-proche de $x_0$, la relation d'ordre ci-dessus est tenue.
    \end{enumerate}

    Alors, si ces deux propriétés sont tenues,
    \[\lim_{x \to x_0} h\left(x\right) = \ell\]
}

\end{document}
