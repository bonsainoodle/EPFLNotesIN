\documentclass[a4paper]{article}

% Expanded on 2021-12-08 at 10:16:59.

\usepackage{../../style}

\title{Analyse}
\author{Joachim Favre}
\date{Mercredi 08 décembre 2021}

\begin{document}
\maketitle

\lecture{22}{2021-12-08}{Séries de Taylor et primitives de séries entières}{
}

\subsection{Série de Taylor}
\parag{Définition}{
    Soit $f : I \mapsto \mathbb{R}$ (où $I$ est un intervalle ouvert, comme d'habitude) une fonction de classe $C^{\infty}\left(I\right)$ (elle est indéfiniment dérivable sur cet intervalle), et $x_0 \in I$.

    Alors, la \important{série de Taylor} au point $x_0$ est:
    \[\sum_{k=0}^{\infty} \frac{f^{\left(k\right)}\left(x_0\right)}{k!}\left(x - x_0\right)^{k}\]

    \subparag{Terminologie}{
        Si $x_0 = 0$, alors cette série est appelée la \important{série de Maclaurin}:
        \[\sum_{k=0}^{\infty} \frac{f^{\left(k\right)}\left(0\right)}{k!}x^k\]

    }
}

\parag{Convergence}{
    Nous voulons maintenant aussi trouver l'ensemble $E \subset D$ dans lequel la série converge vers sa fonction, i.e:
    \[f\left(x\right) = \sum_{k=0}^{\infty} \frac{f^{\left(k\right)}\left(x_0\right)}{k!}\left(x - x_0\right)^k, \mathspace \forall x \in E\]

    On se souvient que la formule de Taylor nous donne:
    \[f\left(x\right) = \sum_{k=0}^{n} \frac{f^{\left(k\right)}\left(x_0\right)}{k!}\left(x - x_0\right)^k + \frac{f^{\left(n+1\right)}\left(u\right)}{\left(n+1\right)!}\left(x - x_0\right)^{n+1}\]
    où $u$ est entre $x$ et $x_0$.

    Ainsi, $\sum_{k=0}^{\infty} \frac{f\left(k\right)\left(x_0\right)}{k!}\left(x - x_0\right)^k$ converge vers $f\left(x\right)$ si et seulement si:
    \[\lim_{n \to \infty} R_n\left(f\right)\left(x\right) = 0\]
}

\parag{Séries de Taylor remarquables}{
    Les séries de Taylor suivantes convergent vers leur fonction pour tout $x \in \mathbb{R}$:
    \begin{center}
        \begin{tabular}{|c|c|}
            \hline
            $\displaystyle \sin\left(x\right)$  & $\displaystyle \sum_{k=0}^{\infty} \frac{\left(-1\right)^k}{\left(2k + 1\right)!} x^{2k + 1}$ \\
            \hline
            $\displaystyle \cos\left(x\right)$  & $\displaystyle \sum_{k=0}^{\infty} \frac{\left(-1\right)^k}{\left(2k\right)!} x^{2k}$         \\
            \hline
            $\displaystyle e^x$                 & $\displaystyle \sum_{k=0}^{\infty} \frac{x^k}{k!}$                                            \\
            \hhline{|=|=|}
            $\displaystyle \sinh\left(x\right)$ & $\displaystyle \sum_{k=0}^{\infty} \frac{1}{\left(2k + 1\right)!} x^{2k + 1}$                 \\
            \hline
            $\displaystyle \cosh\left(x\right)$ & $\displaystyle \sum_{k=0}^{\infty} \frac{1}{\left(2k\right)!} x^{2k}$                         \\
            \hline
        \end{tabular}
    \end{center}

    Il est important de se souvenir des trois premières, mais $\sinh$ et $\cosh$ peuvent être retrouvées à partir de $e^x$ (surtout si on se souvient que ce sont les fonctions paires et impaires qui décomposent $e^x$, et que le développement limité d'une fonction impaire n'a que des coefficients devant des $x^k$ où $k$ est impair, et de manière similaire pour les fonctions paires).
}

\subsection[Primitive et dérivée]{Primitive et dérivée d'une fonction définie par une série entière}
\parag{Définition}{
Soit $f : \left[a, b\right] \mapsto \mathbb{R}$ une fonction continue sur $\left[a, b\right] $.

La fonction $F : \left[a, b\right] \mapsto \mathbb{R}$ est une \important{primitive} de $f$ sur $\left[a, b\right] $ si:
\[F'\left(x\right) = f\left(x\right), \mathspace \forall x \in \left]a, b\right[ \]
}

\parag{Exemples}{
    Nous pouvons déjà trouver les primitives suivantes:
    \begin{center}
        \begin{tabular}{|c|c|c|}
            \hline
            \fullbf{$\displaystyle f\left(x\right)$} & \fullbf{$\displaystyle F\left(x\right)$}  & \fullbf{$\displaystyle D$}               \\
            \hline
            $\displaystyle \sin\left(x\right)$       & $\displaystyle -\cos\left(x\right) + C$   & $\displaystyle \forall x \in \mathbb{R}$ \\
            \hline
            $\displaystyle \cos\left(x\right)$       & $\displaystyle \sin\left(x\right) + C$    & $\displaystyle \forall x \in \mathbb{R}$ \\
            \hline
            $\displaystyle \frac{1}{x}$              & $\displaystyle \log\left(x\right) + C$    & $\displaystyle x > 0$                    \\
            \hline
            $\displaystyle x^k$                      & $\displaystyle \frac{1}{k+1} x^{k+1} + C$ & $\displaystyle k \neq 1, x > 0$          \\
            \hline
            $\displaystyle e^x$                      & $\displaystyle e^x + C$                   & $\displaystyle \forall x \in \mathbb{R}$ \\
            \hline
        \end{tabular}
    \end{center}

}

\parag{Théorème}{
    \begin{enumerate}[left=0pt]
        \item Les deux séries entières suivantes ont le même rayon de convergence, $r$:
              \[f\left(x\right) = \sum_{k=0}^{\infty} b_k \left(x - x_0\right)^{k}, \mathspace F\left(x\right) = \sum_{k=0}^{\infty} \frac{b_k}{k+1} \left(x - x_0\right)^{k+1}\]
        \item Si $r > 0$, alors $f\left(x\right)$ est continue sur $\left]x_0 - r, x_0 + r\right[ $.
        \item Si $r > 0$, alors $F\left(x\right)$ est la primitive de $f\left(x\right)$ sur $\left]x_0 - r, x_0 + r\right[ $ telle que $F\left(x_0\right) = 0$.
    \end{enumerate}
}

\parag{Corollaire}{
Les deux séries entières suivantes ont le même rayon de convergence $r$.
\[f\left(x\right) = \sum_{k=0}^{\infty} a_k \left(x - x_0\right)^k, \mathspace g\left(x\right) = \sum_{{\color{red}k=1}}^{\infty} k a_k \left(x - x_0\right)^{k-1}\]

De plus, si $r > 0$, alors $f\left(x\right)$ est continûment dérivable sur $\left]x_0 - r, x_0 + r\right[ $, et $f'\left(x\right) = g\left(x\right)$.

\subparag{Remarque}{
    Le rayon de convergence est le même, mais le domaine de convergence ne l'est pas forcément : la convergence aux bornes peut varier.
}
}

\end{document}
