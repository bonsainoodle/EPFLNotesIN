\documentclass[a4paper]{article}

% Expanded on 2021-12-13 at 10:44:23.

\usepackage{../../style}

\title{Analyse 1}
\author{Joachim Favre}
\date{Lundi 13 décembre 2021}

\begin{document}
\maketitle

\lecture{23}{2021-12-13}{Maintenant on fait des additions, facile aussi!}{
}

\section{Calcul intégral}
\subsection{Intégrale d'une fonction continue}

\parag{Subdivision}{
    Soit $\left[a, b\right]$ un intervalle. Une \important{subdivision} de $\left[a, b\right]$ est un ensemble tel que:
    \[\sigma = \left\{x_0, = a < x_1 < x_2 < \ldots x_n = b\right\}\]

    On définit aussi le \important{pas de subdivision} tel que:
    \[\mathcal{P}\left(\sigma\right) = \max\left\{x_i - x_{i-1}\right\}\]

    \subparag{Subdivision régulière}{
        Une subdivision régulière d'ordre $n$ est donnée par:
        \[\sigma = \left\{a, a + \frac{b - a}{n}, \ldots, a + k \frac{b - a}{n}, \ldots, b\right\} \implies \mathcal{P}\left(\sigma\right) = \frac{b-a}{n}\]
    }
}

\parag{Définition: Sommes de Darboux}{
Soient $f : \left[a, b\right] \mapsto \mathbb{R}$ une fonction continue, et $\sigma$, une subdivision de $\left[a, b\right]$.

Alors, la \important{somme de Darboux supérieure} de $f$ relativement à $\sigma$ est donnée par:
\[\bar{S_{\sigma}}\left(f\right) \over{=}{déf} \sum_{k=1}^{n} M_k\left(x_k - x_{k - 1}\right), \mathspace \text{où } M_k = \max_{x_{k-1}, x_k}f\left(x\right)\]

De plus, la \important{somme de Darboux inférieure} de $f$ relativement à $\sigma$ est donnée par:
\[\underline{S_{\sigma}}\left(f\right) \over{=}{déf} \sum_{k=1}^{n} {\color{red}m_k}\left(x_k - x_{k - 1}\right), \mathspace \text{où } {\color{red}m_k} = {\color{red}\min_{{\color{black}x_{k-1}, x_k}}}f\left(x\right)\]

On sait que ces minimums et maximums existent, les $M_k$ et les $m_k$, puisque la fonction est continue et on considère un intervalle fermé borné.
}

\parag{Proposition}{
    Soient:
    \[\bar{S}\left(f\right) = \inf\left\{\bar{S_{\sigma}}\left(f\right), \sigma \text{ subdivisions de } \left[a, b\right]\right\}\]
    \[\underline{S}\left(f\right) = \sup\left\{\underline{S_{\sigma}}\left(f\right), \sigma \text{ subdivisions de } \left[a, b\right]\right\}\]

    Alors, si $f$ est continue sur $\left[a, b\right]$, on a que $\bar{S}\left(f\right) = \underline{S}\left(f\right)$.
}

\parag{Définition: intégrale de Riemann}{
    Soit $f : \left[a, b\right] : \mathbb{R}$ une fonction continue, avec $a < b$. Alors, on définit \important{l'intégrale de Riemann} de la fonction $f$ sur $\left[a, b\right]$:
    \[\int_{a}^{b} f\left(x\right)dx \over{=}{déf} \bar{S}\left(f\right) = \underline{S}\left(f\right)\]
}

\parag{Définition: échange de borne}{
    Si $b < a$, alors on définit:
    \[\int_{a}^{b} f\left(x\right)dx \over{=}{déf} -\int_{b}^{a} f\left(x\right)dx\]

    De plus, si $b = a$:
    \[\int_{a}^{a} f\left(x\right)dx = 0\]
}

\parag{Calcul d'intégrale}{
    On remarque qu'on peut calculer nos intégrales de la manière suivante:
    \[\int_{a}^{b} f\left(x\right)dx = \lim_{n \to \infty} \bar{S_{\sigma_n}}\left(f\right) = \lim_{n \to \infty} \underline{S_{\sigma_n}}\]
    où $\left\{\sigma_n\right\}$ est une suite de subdivisions de $\left[a, b\right]$ telle que:
    \[\lim_{n \to \infty} \mathcal{P}\left(\sigma_n\right) = 0\]
}

\parag{Propriété 1}{
    Soit $f\left(x\right)$ continue sur $\left[a,b\right]$, et $c \in \left[a, b\right]$. Alors:
    \[\int_{a}^{b} f\left(x\right)dx = \int_{a}^{c} f\left(x\right)dx + \int_{c}^{b} f\left(x\right)dx\]
}

\parag{Propriété 2}{
    Comme mentionné plus tôt:
    \[m\left(b - a\right) < \underline{S_{\sigma}}\left(f\right) \leq \bar{S_{\sigma}}\left(f\right) \leq M\left(b - a\right)\]

    Où:
    \[m = \min_{\left[a, b\right]} f\left(x\right), \mathspace M = \max_{\left[a, b\right]}f\left(x\right)\]

    Ainsi, on en déduit que:
    \[m\left(b - a\right) \leq \int_{a}^{b} f\left(x\right) dx \leq M\left(b - a\right)\]
}

\parag{Théorème de la moyenne}{
    Soit $f\left(x\right)$ une fonction continue sur $\left[a, b\right]$, avec $a < b$.

    Alors, il existe un point $c \in \left[a, b\right]$ tel que:
    \[\int_{a}^{b} f\left(x\right) dx = f\left(c\right)\left(b - a\right)\]
}

\subsection{Relation entre l'intégrale et la primitive}

\parag{Théorème fondamental du calcul intégral partie 1}{
    Soient $a < b$ et $f$ une fonction continue sur $\left[a,b\right]$.

    Alors, la fonction suivante est la primitive de $f\left(x\right)$ sur $\left[a, b\right]$ telle que $F\left(a\right) = 0$:
    \[F\left(x\right) = \int_{a}^{x} f\left(t\right)dt\]

    \subparag{Note}{
        Notez que, puisqu'on utilise $x$ dans les bornes de l'intégrale, alors on ne peut pas l'utiliser comme variable d'intégration.
    }
}

\parag{Corollaire: Théorème fondamental du calcul intégral partie 2}{
    Soient $a < b$ et $f\left(x\right)$ continue sur $\left[a,b\right]$.

    Si $G\left(x\right)$ est une primitive de $f\left(x\right)$ sur $\left[a, b\right]$, alors:
    \[\int_{a}^{b} f\left(x\right) dx = G\left(b\right) - G\left(a\right)\]
}

\parag{Primitives remarquables}{
    On peut dessiner les tableaux suivants:

    \begin{center}
        \begin{tabular}{|c|c|}
            \hline
            \fullbf{$f\left(x\right)$}           & \fullbf{$F\left(x\right)$}               \\
            \hline
            $\displaystyle e^x$                  & $\displaystyle e^x + C$                  \\
            $\displaystyle \sin\left(x\right)$   & $\displaystyle -\cos\left(x\right) + C$  \\
            $\displaystyle \cos\left(x\right)$   & $\displaystyle \sin\left(x\right) + C$   \\
            $\displaystyle \sinh\left(x\right)$  & $\displaystyle \cosh\left(x\right) + C$  \\
            $\displaystyle \cosh\left(x\right)$  & $\displaystyle \sinh\left(x\right) + C$  \\
            $\displaystyle a^x, a > 0, a \neq 1$ & $\displaystyle \frac{1}{\log a} a^x + C$ \\
            \hline
        \end{tabular}
        \hspace{1em}
        \begin{tabular}{|c|c|}
            \hline
            \fullbf{$f\left(x\right)$}                     & \fullbf{$F\left(x\right)$}                    \\
            \hline
            $\displaystyle \frac{1}{x}$                    & $\displaystyle \log\left|x\right| + C$        \\
            $\displaystyle x^r, r \neq -1$                 & $\displaystyle \frac{1}{r + 1} x^{r + 1} + C$ \\
            $\displaystyle \frac{1}{\cos^2\left(x\right)}$ & $\displaystyle \tan\left(x\right) + C$        \\
            $\displaystyle \frac{1}{\sin^2\left(x\right)}$ & $\displaystyle -\cot\left(x\right) + C$       \\
            $\displaystyle \frac{1}{1 + x^2}$              & $\displaystyle \arctan\left(x\right) + C$     \\
            $\displaystyle \frac{1}{\sqrt{1 - x^2}}$       & $\displaystyle \arcsin\left(x\right) + C$     \\
            \hline
        \end{tabular}
    \end{center}
}

\parag{Linéarité}{
    Les intégrales sont linéaires:
    \[\int_{a}^{b} \left(\alpha f\left(x\right) + \beta g\left(x\right)\right)dx = \alpha \int_{a}^{b} f\left(x\right) + \beta \int_{a}^{b} g\left(x\right)dx\]
}

\parag{Relation d'ordre}{
    Si $f\left(x\right) \leq g\left(x\right)$ pour tout $x \in \left[a, b\right] $, alors:
    \[\int_{a}^{b} f\left(x\right)dx \leq \int_{a}^{b} g\left(x\right) dx\]

    \subparag{Corollaire}{
        Si $f\left(x\right) \geq 0$, alors
        \[\int_{a}^{b} f\left(x\right)dx = 0 \iff f\left(x\right) = 0, \mathspace \forall x \in \left[a, b\right] \]
    }
}

\parag{Dérivée d'une intégrale avec des fonctions comme bornes}{
    Soit $f: \left[a, b\right] \mapsto \mathbb{R}$ une fonction continue, et $g, h : I \mapsto \left[a, b\right] $ dérivables sur $I$. Alors:
    \[\frac{d}{dx} \left(\int_{h\left(x\right)}^{g\left(x\right)} f\left(t\right)dt\right) = f\left(g\left(x\right)\right)g'\left(x\right) - f\left(h\left(x\right)\right) h'\left(x\right)\]
}

\subsection{Techniques d'intégration}

\parag{Proposition: Formule de changement de variable}{
    Soit $f : \left[a, b\right] \mapsto \mathbb{R}$ une fonction continue, et $\phi : \left[\alpha, \beta\right] \mapsto \left[a, b\right]$ continûment dérivable sur $I$, où $\left[\alpha, \beta\right] \subset I$.

    Alors:
    \[\int_{\phi\left(\alpha\right)}^{\phi\left(\beta\right)} f\left(x\right)dx = \int_{\alpha}^{\beta} f\left(\phi\left(t\right)\right)\phi'\left(t\right)dt, \mathspace \text{où } x = \phi\left(t\right)\]
}

\end{document}
