\documentclass[a4paper]{article}

% Expanded on 2021-12-01 at 10:17:21.

\usepackage{../../style}

\title{Analyse}
\author{Joachim Favre}
\date{Mercredi 01 décembre 2021}

\begin{document}
\maketitle

\lecture{20}{2021-12-01}{DL, extrema, points d'inflexion et concavité}{
}

\subsection{Développements limités}

\parag{Théorème (formule de Taylor Lagrange)}{
    Soit $f : I \mapsto F$ une fonction $\left(n + 1\right)$ fois dérivables sur $I$, avec $a \in I$. Alors, $\forall x \in I$, il existe $u$ entre $a$ et $x$ (on n'utilise pas de notation avec des intervalles puisqu'on sait pas si $a$ est à gauche ou à droite) tel que:
    \[f\left(x\right) = \underbrace{f\left(a\right) + f'\left(a\right)\left(x-a\right) + \frac{f''\left(a\right)}{2} \left(x-a\right)^2 + \ldots + \frac{f^{\left(n\right)}\left(a\right)}{n!} \left(x - a\right)^{n}}_{P_n\left(f\right) \text{: Polynôme de Taylor}} + \underbrace{\frac{f^{\left(n+1\right)}\left(u\right)}{\left(n + 1\right)!} \left(x - a\right)^{n+1}}_{R_n\left(f\right) \text{: Reste}}\]

    Le tout est la formule de Taylor.

    \subparag{Terminologie}{
        La formule de Taylor s'appelle la formule de Maclaurin si $a = 0$.
    }
}

\parag{Définition (développement limité)}{
    Soit $f : E\mapsto F$ une fonction définie au voisinage de $x = a$.

    S'il existe des nombres $a_0, \ldots, a_n$ et une fonction $\epsilon\left(x\right)$, tels que $\forall x \in E, x \neq a$, on a:
    \[f\left(x\right) = a_0 + a_1\left(x - a\right) + a_2\left(x - a\right)^2 + \ldots + a_n\left(x - a\right)^n + \left(x - a\right)^{n} \epsilon\left(x\right), \mathspace \lim_{x \to a} \epsilon\left(x\right) = 0\]
    on dit que $f$ admet un \important{développement limité (DL)} d'ordre $n$ autour de $x = a$.

    On appelle les $n$ premiers termes (donc la partie sous forme de polynôme) \important{la partie principale du développement limité}, $P_n\left(x\right)$. On appelle le $n+1$-ème terme \important{le reste du DL}, $R_n\left(x\right) = \left(x-a\right)^{n} \epsilon\left(x\right)$.
}

\parag{Proposition (unicité)}{
    Si $f : E \mapsto F$ admet un développement limité d'ordre $n$ autour de $x = a$, alors celui-ci est unique.
}

\parag{Corollaire}{
    Soit $a \in I$, $f : I \mapsto \mathbb{R}$ une fonction $\left(n+1\right)$ continûment dérivable sur $I$. Alors, la formule de Taylor nous fournit le DL d'ordre $n$ de la fonction $f$ autour de $x = a$.

    \subparag{Remarque 1}{
        Il suffit d'avoir $f$ $n$ fois continûment dérivable sur $I$ pour avoir
        \[f\left(x\right) = P_n\left(x\right) + \left(x - a\right)^{n} \epsilon\left(x\right), \mathspace \lim_{x \to a} \epsilon\left(x\right) = 0\]

        Dans ce cas on ne peut pas définir $\epsilon\left(x\right)$ comme la $n+1$ dérivée, mais on peut aussi trouver une expression (plus compliquée).
    }

    \subparag{Remarque 2}{
        $f : E \mapsto F$ peut avoir un DL sans que la formule de Taylor lui soit applicable.

        Cependant, dans notre cours ce genre de cas sont rares.
    }
}

\parag{Résumé}{
    Soit $f : I \mapsto F$ $n$ fois dérivable sur $I$, $a, x \in I$, $x \neq a$. Alors:
    \[f\left(x\right) = f\left(a\right) + f'\left(a\right)\left(x-a\right) + \ldots + \frac{f^{\left(n\right)} \left(a\right)}{n!} \left(x - a\right)^n + \left(x-a\right)^n \epsilon\left(x\right), \mathspace \lim_{x \to a} \epsilon\left(x\right) = 0\]

    Si $f$ est $\left(n+1\right)$ fois dérivable sur $I$, alors:
    \[f\left(x\right) = f\left(a\right) + f'\left(a\right)\left(x-a\right) + \ldots + \frac{f^{\left(n\right)} \left(a\right)}{n!} \left(x - a\right)^n + \frac{f^{\left(n+1\right)} \left(u\right)}{\left(n+1\right)!} \left(x - a\right)^{n+1}\]

    Ainsi, si $f : E \mapsto F$ est $(n + 1)$ fois continûment dérivable sur un intervalle ouvert contenant $x = a$, alors la formule de Taylor nous fournit le DL d'ordre $n$ de $f$ autour de $x = a$.
}

\subsection{Étude de fonctions}

\parag{Proposition (condition suffisante pour un extremum local)}{
    Soit $f : I \mapsto F$ une fonction $n$ fois continûment dérivable sur $I$, où \important{$n \in \mathbb{N}^*$ est pair}.

    Si cette fonction est telle que
    \[f'\left(c\right) = f''\left(c\right) = \ldots = f^{\left(n-1\right)}\left(c\right) = 0, \mathspace \text{mais} \mathspace f^{\left(n\right)}\left(c\right) \neq 0\]

    Alors, si $f^{\left(n\right)}\left(c\right) > 0$, $f$ admet un minimum local en $x = c$, et si $f^{\left(n\right)}\left(c\right) < 0$, alors $f$ admet un maximum local en $x = c$.

    \subparag{Mnémotechnie}{
        On peut utiliser la fonction $x^2$ pour se souvenir de ce théorème.
    }
}

\parag{Définition (tangente)}{
    Soit $f : E \mapsto F$ une fonction dérivable en $a \in E$.

    $\ell\left(x\right) = f\left(a\right) + f'\left(a\right)\left(x - a\right)$ est la \important{tangente} à la courbe $y = f\left(x\right)$ en $\left<a, f\left(a\right)\right>$.
}

\parag{Définition (point d'inflexion)}{
    Prenons la fonction suivante:
    \[\psi\left(x\right) \over{=}{déf} f\left(x\right) - \ell\left(x\right) = f\left(x\right) - f\left(a\right) - f'\left(a\right)\left(x - a\right)\]

    Si $\psi\left(x\right)$ change de signe en $x=a$, alors $\left<a, f\left(a\right)\right>$ est un \important{point d'inflexion de $f$}.
}

\parag{Proposition}{
    Soit $f : I \mapsto F$ une fonction $n$ fois continûment dérivable sur $I$, où \important{$n \in \mathbb{N}$ est impair et $n > 1$}.

    Si la fonction est telle que:
    \[f''\left(a\right) = f'''\left(a\right) = \ldots = f^{\left(n-1\right)}\left(a\right) = 0, \mathspace \text{mais} \mathspace f^{\left(n\right)}\left(a\right) \neq 0\]

    Alors, le point $\left<a, f\left(a\right)\right>$ est un point d'inflexion de $f$.
}

\parag{Définition (convexité et concavité)}{
    $f : I \mapsto F$ est \important{convexe} sur $I$ si pour tout couple $a < b \in I$, le graphique de $f\left(x\right)$ se trouve au dessous de la droite passant par $\left<a, f\left(a\right)\right>$ et $\left<b, f\left(b\right)\right>$.

    De manière similaire, $f : I \mapsto F$ est \important{concave} sur $I$ si pour tout couple $a < b \in I$, le graphique de $f\left(x\right)$ se trouve \textcolor{red}{au dessus} de la droite passant par $\left<a, f\left(a\right)\right>$ et $\left<b, f\left(b\right)\right>$.

    \subparag{Mnémotechnie}{
        $f\left(x\right) = -x^2$ est concave et elle ``descend à la cave''.
    }
}

\parag{Proposition}{
Soit $f : I \mapsto F$ deux fois dérivable sur $I$.

$f$ est convexe sur $I$ si et seulement si $f''\left(x\right) \geq 0$ sur $I$ (ce qui est équivalent à dire que $f'\left(x\right)$ est croissante sur $I$).

$f$ est \textcolor{red}{concave} sur $I$ si et seulement si $f''\left(x\right)\ {\color{red}\leq}\ 0$ sur $I$ (ce qui est équivalent à dire que $f'\left(x\right)$ est \textcolor{red}{décroissante} sur $I$).
}

\end{document}
