\documentclass[a4paper]{article}

% Expanded on 2021-11-22 at 13:16:54.

\usepackage{../../style}

\title{Algèbre linéaire}
\author{Joachim Favre}
\date{Lundi 22 novembre 2021}

\begin{document}
\maketitle

\lecture{18}{2021-11-22}{Fin de la propreté et début des produits scalaires}{
}

\parag{Théorème}{
    Soit $A \in \mathbb{R}^{n\times n}$ une matrice quelconque et soient $\lambda_1, \ldots \lambda_p$ les valeurs propres distinctes de $A$.

    On a toujours que $\dim\ker\left(A - \lambda_i I_n\right)$ (la dimension géométrique de $\lambda_i$) est inférieure ou égale à la multiplicité algébrique de $\lambda_i$ pour chaque $i$.

    $A$ est diagonalisable si et seulement si $\dim\ker\left(A - \lambda_i I_n\right)$ est égale à la multiplicité algébrique de $\lambda_i$ pour tout $i \in \left\{1, 2, \ldots, p\right\}$

    \subparag{Remarque}{
        Si $A$ est diagonalisable, alors pour chaque $\lambda_i$, on peut trouver une base $\mathcal{B}_i$ de $\ker\left(A - \lambda_i I_n\right)$. Mises toutes ensemble, les bases $\mathcal{B}_1, \ldots, \mathcal{B}_p$ forment une base de vecteurs propres de $A$ pour $\mathbb{R}^{n}$.
    }
}

\parag{Trace d'une matrice}{
    Soit $A \in \mathbb{R}^{n\times n}$ une matrice carrée. La \important{trace} de $A$ est la somme des éléments de la diagonale principale de $A$:
    \[tr\left(A\right) = \sum_{i=1}^{n} a_{ii} = a_{11} + a_{22} + \ldots + a_{nn}\]

    \subparag{Remarque}{
        La trace d'une matrice est égale à la somme de ses valeurs propres et le déterminant d'une matrice est égal au produit de ses valeurs propres.
    }
}

\section{Orthogonalité et méthode des moindres carrés}

\parag{Définition du produit scalaire}{
    Soient les deux vecteurs de $\mathbb{R}^n$ suivants:
    \[\bvec{u} = \begin{bmatrix} u_1 \\ \vdots \\ u_n \end{bmatrix}, \mathspace \bvec{v} = \begin{bmatrix} v_1 \\ \vdots \\ v_n \end{bmatrix} \]

    Le \important{produit scalaire} de $\bvec{u}$ et $\bvec{v}$ est un scalaire noté $\bvec{u} \dotprod \bvec{v}$ défini comme suit:
    \[\bvec{u} \dotprod \bvec{v} = \bvec{u}^T \bvec{v} = \begin{bmatrix} u_1 & \cdots & u_n \end{bmatrix} \begin{bmatrix} v_1 \\ \vdots \\ v_n \end{bmatrix} = u_1 v_1 + \ldots + u_n v_n \]
}

\end{document}
