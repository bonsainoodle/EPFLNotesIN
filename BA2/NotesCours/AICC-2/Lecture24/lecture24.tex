\documentclass[a4paper]{article}

% Expanded on 2022-05-20 at 08:58:42.

\usepackage{../../style}

\title{AICC 2}
\author{Joachim Favre}
\date{Jeudi 19 mai 2022}

\begin{document}
\maketitle

\lecture{24}{2022-05-19}{Hamming code}{
}

\parag{Definition: Parity-check matrix}{
    A \important{parity-check matrix} $H$ for a linear $\left(n, k\right)$ block code $\mathcal{C}$ is an $\left(n-k\right)\times n$ matrix that contains the coefficients of the $n-k$ linear homogenous equations, the solutions to which construct $\mathcal{C}$.

    This tells us $H \in \mathbb{F}^{\left(n-k\right) \times n}$ is such that:
    \[\bvec{y} H^T = \bvec{0} \iff \bvec{y} \in \mathcal{C}\]
}

\parag{Lemma}{
    Let $G$ and $H$ be full rank matrices (their rank is the highest it can be) and the right dimension.

    $\left(G, H\right)$ are generator/parity-check matrices if and only if:
    \[GH^T = 0\]
}


\parag{Theorem: Systematic code}{
    Let's say we have the following systematic generator matrix:
    \[G = \begin{pmatrix}  &  &  &  &  &  \\  & I_k &  &  & P &  \\  &  &  &  &  &  \end{pmatrix} \]

    Then, the parity-check matrix is given by:
    \[H = \begin{pmatrix}  &  &  &  &  &  \\  & -P^T &  &  & I_{n-k} &  \\  &  &  &  &  &  \end{pmatrix} \]
}

\parag{Definition: Syndrome}{
    Let $\mathcal{C} \subseteq \mathbb{F}^n$  be an $\left(n, k\right)$ linear block code with parity check matrix $H$.

    The vector $\bvec{s} = \bvec{y} H^T \in \mathbb{F}^{n-k}$ is the \important{syndrome} of $\bvec{y}$.
}

\parag{Definition: Hamming code}{
    Let $n = 2^m - 1 \in \left\{1, 3, 7, 15, 31, \ldots\right\}$ and $k = n - m$ for some $m \in \mathbb{Z}_+$. A $\left(n, k\right)$ linear code $\mathcal{C} \subseteq \mathbb{F}_2^n$ is named a \important{Hamming code} if the binary representation of every number between $1$ and $2^m$ are represented in the columns exactly once (their order does not matter).
}

\parag{Minimum distance}{
    Any Hamming code has $d_{min} = 3$.
}

\parag{Singleton bound}{
    Hamming codes are not MDS codes.
}

\parag{Theorem: Minimum distance via parity check matrix}{
    Let $\mathcal{C} \subseteq \mathbb{F}_q^n$ be a linear $\left(n, k\right)$ code and $H$ be a parity check matrix for $\mathcal{C}$.

    Then, $d_{min}\left(\mathcal{C}\right)$ is the smallest number of linearly dependent columns of $H$.

    \subparag{Remarks}{
        We could have used this theorem for showing that Hamming codes have $d_{min} = 3$; and the proof is similar.

        Also, this theorem still has a cost, since we have to look for every combination of linearly dependant vectors.
    }
}

\end{document}
