\documentclass[a4paper]{article}

% Expanded on 2021-10-26 at 08:30:21.

\usepackage{../../style}

\title{AICC 1}
\author{Joachim Favre}
\date{Mardi 26 octobre 2021}

\begin{document}
\maketitle

\lecture{11}{2021-10-26}{Introduction to algorithms}{
}

\section{Algorithms}
\subsection{Introduction}
\parag{Definition}{
    An algorithm is a finite set of well-defined instructions to perform a specified task.
}

\parag{Specifying algorithms}{
    We have multiple ways to describe algorithms: natural language, pseudo-code and programming languages.
}

\parag{Typical problems}{
    \begin{description}[left=0pt]
        \item[Searching problems:] Find an element in a list.
        \item[Sorting problems:] Put elements in a list in order.
        \item[Optimisation problems:] Determine the optimal value of a particular quantity over all possible inputs.
    \end{description}

    The two first ones are really important for managing data.

}

\subsection{Searching algorithms}
\parag{Searching problem}{
    Given a list $S = a_1, \ldots, a_n$ of distinct elements, and some $x$. If $x \in S$, then it must return $i$ such that $a_i = x$, else it must return 0.
}

\begin{filecontents*}[overwrite]{linearsearch.code}
    procedure linear_search(x: integer, a: array of n distinct integers):
    i := 1
    while(i <= n and x != a[i])
    i := i + 1
    if i == n + 1 then location := 0 else location := i
    return location
\end{filecontents*}

\parag{Linear search algorithm}{
    The most simple one is to traverse the list, considering elements one by one, starting at the beginning. If we reach the element, then we have found it, else if we get out of the list without finding it, then it is not in it. Here is the pseudocode:
    \importcode{linearsearch.code}{pseudo}
}

\begin{filecontents*}[overwrite]{binarysearch.code}
    procedure binary_search(x: integer, a: array of n increasing integers)
    i := 1  // left-most index
    j := n  // right-most index
    while i < j
    m := floor((i + j)/2)
    if x > a[m] then i := m+1 else j := m
    if x == a[i] then location := i else location := 0
    return location
\end{filecontents*}


\parag{Binary Search}{
    Let's assume that, this time, the input is a list of items in increasing order (they have been sorted previously).

    \begin{enumerate}
        \item We begin by comparing the element to be found with the middle element.
              \begin{enumerate}
                  \item If the middle element is lower, the search proceeds with the upper half of the list.
                  \item If it is not lower, the search with the lower part of the list (including the middle position).
              \end{enumerate}
        \item Repeat this process until we have a list of size $1$.
              \begin{enumerate}
                  \item If the element we are looking for is equal to the element in the list, the position is returned.
                  \item Otherwise, 0 is returned to indicate that the element was not found.
              \end{enumerate}
    \end{enumerate}

    \importcode{binarysearch.code}{pseudo}
}

\subsection{Sorting algorithms}
\parag{Sorting problems}{
    Given a list $S = a_1, \ldots, a_n$, we want to return a list where the elements are put in increasing order.
}

\begin{filecontents*}[overwrite]{bubblesort.code}
    procedure bubble_sort(a : array of n >= 2 real numbers):
    for i := 1 to n - 1
    for j := 1 to n - i
    if a[j] > a[j+1] then interchange(a[j], a[j+1])
\end{filecontents*}


\parag{Bubble sort}{
    We will do different passes through the list:
    \begin{enumerate}
        \item In one pass, every pair of elements that are found to be out of order are interchanged.
        \item Since the last element is guaranteed to be the largest after the first pass, in the second pass it does not need anymore to be inspected.
        \item We can repeat this process, and every pass we have one less element to inspect.
    \end{enumerate}

    \importcode{bubblesort.code}{pseudo}

}

\begin{filecontents*}[overwrite]{insertionsort.code}
    procedure insertion_sort(a: array of n >= 2 real numbers):
    for j := 2 to n
    i := 1
    while a[j] > a[i] and i < j  // move a[j] to the right position
    i := i + 1
    m := a[j]
    for k := 0 to j - i - 1  // shift elements to make space for m = a[j]
    a[j-k] := a[j-k-1]
    a[i] := m
\end{filecontents*}

\parag{Insertion sort}{
    \begin{enumerate}[left=0pt]
        \item Compare the second element with the first and puts it before the first if it is not larger.
        \item Then the third element is put in corret position among the three first using linear search.
        \item We continue this way every pass, increasing the size of our sorted subarray.
    \end{enumerate}

    \importcode{insertionsort.code}{pseudo}

}

\end{document}
