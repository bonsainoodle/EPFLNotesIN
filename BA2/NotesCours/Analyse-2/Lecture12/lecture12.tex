\documentclass[a4paper]{article}

% Expanded on 2022-03-28 at 23:52:33.

\usepackage{../../style}

\title{Analyse 2}
\author{Joachim Favre}
\date{Mercredi 30 mars 2022}

\begin{document}
\maketitle

\lecture{12}{2022-03-30}{Gros théorème très très utile}{
}

\parag{Théorème 1 sur la dérivabilité}{
    Soit $f: E \mapsto \mathbb{R}$, où $E \subset \mathbb{R}^n$, une fonction dérivable en $\bvec{a} \in E$ et de différentielle $L_{\bvec{a}} : \mathbb{R}^n \mapsto \mathbb{R}$. Alors:
    \begin{enumerate}
        \item $f$ est continue en $\bvec{a} \in E$.
        \item Pour tout $\bvec{v} \in \mathbb{R}^n$, où $\bvec{v} \neq \bvec{0}$, la dérivée directionnelle $Df\left(\bvec{a}, \bvec{v}\right)$ existe et:
              \[Df\left(\bvec{a}, \bvec{v}\right)= L_{\bvec{a}}\left(\bvec{v}\right)\]
        \item Toutes les dérivées partielles de $f$ en $\bvec{a}$ existent, et:
              \[\frac{\partial f}{\partial x_{k}}\left(\bvec{a}\right) = L_{\bvec{a}}\left(\bvec{e_k}\right)\]
              où $\bvec{e_k}$ est le $k$-ème vecteur de la base canonique.

              Le gradient de $f$ existe en $\bvec{a}$ et:
              \[\nabla f\left(\bvec{a}\right) = \left(L_{\bvec{a}}\left(\bvec{e_1}\right), \ldots, L_{\bvec{a}}\left(\bvec{e_n}\right)\right)\]

        \item Pour tout $\bvec{v} \in \mathbb{R}^n$, où $\bvec{v} \neq \bvec{0}$, alors:
              \[L_{\bvec{a}}\left(\bvec{v}\right) = Df\left(\bvec{a}, \bvec{v}\right) = \left<\nabla f\left(\bvec{a}\right), \bvec{v}\right>\]
        \item Pour tout $\bvec{v} \in \mathbb{R}^n$ tel que $\left\|\bvec{v}\right\| = 1$, nous avons:
              \[Df\left(\bvec{a}, \bvec{v}\right) \leq \left\|\nabla f\left(\bvec{a}\right)\right\|\]

              De plus:
              \[Df\left(\bvec{a}, \frac{\nabla f\left(\bvec{a}\right)}{\left\|\nabla f\left(\bvec{a}\right)\right\|}\right) = \left\|\nabla f\left(\bvec{a}\right)\right\|\]

              En d'autres mots, le gradient donne la direction et la valeur de la plus grande pente de $f$ en $\bvec{a}$ (si $\nabla f\left(\bvec{a}\right) \neq \bvec{0}$, sinon la fonction est simplement plate à cet endroit, comme nous le verrons plus tard).
    \end{enumerate}
}

\parag{Application: Plan tangent à une surface}{
    Soit $f: E \mapsto \mathbb{R}$, où $E \subset \mathbb{R}^2$, une fonction dérivable sur $E$ (c'est-à-dire qu'elle est dérivable en tout point de $E$). Soit aussi $\bvec{a} = \left(x_0, y_0, f\left(x_0, y_0\right)\right) \in \mathbb{R}^3$ un point du graphique de $f$. On cherche une équation du plant tangent à $z = f\left(x, y\right)$ à ce point.

    Soit $F\left(x, y, z\right) = z - f\left(x, y\right)$, qui est définie sur $D \mapsto \mathbb{R}$, où $D \subset \mathbb{R}^3$, et qui est dérivable puisque $f\left(x, y\right)$ est dérivable. Par définition de $F\left(x, y, z\right)$, nous avons paramétrisé la surface de notre graphique avec l'équation $F\left(x, y, z\right) = 0$ (puisqu'on sait que c'est équivalent à $z = f\left(x, y\right)$, qui est la définition de notre graphique).

    Le gradient de $F$ est donné par:
    \[\nabla F\left(x, y, z\right) = \left(- \frac{\partial f}{\partial x}\left(\bvec{a}\right), - \frac{\partial f}{\partial y}\left(\bvec{a}\right), 1\right) \neq \bvec{0}\]

    Par un argument similaire à ce que nous venons de trouver avec la courbe de niveau, nous savons que le gradient est orthogonal au plan tangent à la courbe. En effet, la courbe de niveau de $F$ telle que $F\left(x, y, z\right) = 0$ est exactement notre graphique. Ainsi, tout vecteur de notre plan $\bvec{v}$ est tel que:
    \[\left<\nabla F\left(\bvec{a}\right), \bvec{v}\right> = 0\]

    Prenons $z_0 = f\left(x_0, y_0\right)$. Puisque nous savons que le point $\left(x_0, y_0, z_0\right)$ appartient au plan, nous savons que $\bvec{v}$, allant de $\left(x_0, y_0, z_0\right)$ à un vecteur quelconque, est de la forme $\left(x - x_0, y- y_0, z - z_0\right)$. Ainsi, cela nous donne:
    \begin{multiequation}
        & \left<\nabla F\left(\bvec{a}\right), \left(x - x_0, y - y_0, z - z_0\right)\right> = 0  \\
        \iff & - \frac{\partial f}{\partial x}\left(\bvec{a}\right) \left(x - x_0\right) - \frac{\partial f}{\partial y}\left(\bvec{a}\right) \left(y-  y_0\right) + 1\left(z - f\left(x_0, y_0\right)\right) = 0 \\
        \iff & z = f\left(x_0, y_0\right) + \left<\nabla f\left(x_0, y_0\right), \left(x - x_0, y - y_0\right)\right>
    \end{multiequation}

    Il faut connaitre ce résultat.

    \subparag{Remarque}{
        Nous pouvons comparer ce résultat avec la tangente à une fonction d'une seule variable en $a \in E$:
        \[T_a\left(x\right) = f\left(a\right) + f'\left(a\right)\left(x - a\right)\]
    }

}

\end{document}
