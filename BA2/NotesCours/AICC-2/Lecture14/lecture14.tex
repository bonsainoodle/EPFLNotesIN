\documentclass[a4paper]{article}

% Expanded on 2022-04-07 at 13:44:22.

\usepackage{../../style}

\title{AICC-2}
\author{Joachim Favre}
\date{Jeudi 07 avril 2022}

\begin{document}
\maketitle

\lecture{14}{2022-04-07}{Groups}{
}

\parag{Lemma}{
    The following identity holds:
    \[\gcd\left(a, b\right) = \gcd\left(a - kb, b\right)\]
}

\parag{Euclid algorithm}{
    We know we can write $a = bq + r$, where $0 \leq r < b$. Thus, from our lemma:
    \[\gcd\left(a, b\right) = \gcd\left(b, r\right) = \gcd\left(b, a \Mod b\right)\]

    We can continue that way recursively.
}

\parag{Theorem: Bézout's identity}{
    Let $a$ and $b$ be integers, not both zero.

    Then, there exist integers $u$ and $v$ (possibly negative), such that:
    \[\gcd\left(a, b\right) = au + bv\]

    \subparag{Example}{
        We can draw the following table:
        \begin{center}
            \begin{tabular}{|c||c|c|c|c|c|c|c}
                \hline
                $\gcd\left(a, b\right)$     & $a = bq + r$      & $q$ & $u = \widetilde{v}$ & $v = \widetilde{u} - q\widetilde{v}$ \\
                \hhline{=#=|=|=|=|}
                $\gcd\left(150, 33\right)$  & $33 \cdot 4 + 18$ & 4   & $2$                 & $-1 -8 = -9$                         \\
                \hline
                $\gcd\left(33, 18\right)$   & $18\cdot 1 + 15$  & 1   & $-1$                & $1 - \left(-1\right)1 = 2$           \\
                \hline
                $\gcd\left(18, 15\right)$   & $15\cdot 1 + 3$   & $1$ & $1$                 & $0 - 1 = -1$                         \\
                \hline
                $\gcd\left(15, 3\right)$    & $3\cdot 5 + 0$    & 5   & 0                   & 1                                    \\
                \hline
                $\gcd\left(3, 0\right) = 3$ &                   &     & 1                   & 0                                    \\
                \hline
            \end{tabular}
        \end{center}

        We complete the three first columns first, and then complete the last two columns by going from bottom to top.
    }
}

\parag{Theorem}{
    $\left[a\right]_m$ has a multiplicative inverse \textit{if and only if} $a$ and $m$ are coprime (meaning that $\gcd\left(a, m\right) = 1$).
}

\parag{Corollary 1}{
    $\gcd\left(a, m\right) = 1$ if and only if there exist integers $u$ and $v$ such that $1 = au + mv$.
}

\parag{Corollary 2}{
    If $p$ is a prime, all elements of $\mathbb{Z} / p\mathbb{Z}$, except for $\left[0\right]_p$, have a multiplicative inverse.
}

\parag{Procedure mod 97--10}{
    This algorithm is a variant of the previous one.

    We start with a number $n$. We multiply our number by 100, getting $100n$, compute $r = 100n \Mod 98$, then add a check $c = 98 - r$ to our number multiplied by 100. In other words, we compute:
    \[m = 100n + \left(98 - 100n \Mod 97\right)\]

    Then, to check if our number was modified, we can verify that the result modulo 97 is 1. In other words, the test passes if:
    \[\left[\text{number with check digits}\right]_{97} = \left[1\right]_{97}\]
}

\subsection{Commutative groups}

\parag{Definition: Commutative group}{
    Let $G$ be a finite set, and $\star$ be an operation on pairs of elements of $G$, such that:
    \begin{enumerate}
        \item Closure: $\forall a, b \in G$, $a \star b \in G$
        \item Associativity: $a \star \left(b \star c\right) = \left(a \star b\right) \star c$
        \item Commutativity: $a\star b = b \star a$
        \item Existence of identity: $\exists e \in G$ such that $e \star a = a\star e = a$
        \item Existence of inverse: $\forall a \in G, \exists b \in G$ such that $a \star b = b\star a = e$
    \end{enumerate}

    Then, a \important{commutative group} (or Abelian group) is $\left(G, \star\right)$.
}

\parag{Definition}{
    We define $\mathbb{Z} / m\mathbb{Z}^*$ to be the set of elements of $\mathbb{Z} / m\mathbb{Z}$ that have a multiplicative inverse.
}

\parag{Theorem}{
    For every integer $m > 1$, $\left(\mathbb{Z} / m\mathbb{Z}^*, \cdot\right)$ is a commutative group.
}

\parag{Definition: Euler's totient function}{
    We define \important{Euler's totient function}, $\phi\left(n\right)$, to be the number of $i \in\left\{1, 2, \ldots, n\right\}$ such that $i$ and $n$ are coprime (meaning $\gcd\left(i, n\right) = 1$).

    \subparag{Remark}{
        Note that, when we work modulo $n$:
        \[\left\{1, 2, \ldots, n\right\} = \left\{0, 1, \ldots, n-1\right\}\]

        Thus, our function is exactly what we have seen in our last paragraph.
    }

    \subparag{Properties}{
        As we have seen, $\phi\left(m\right)$ is the cardinality of $\mathbb{Z} / m\mathbb{Z}^*$. Also, we can see that, if $p$ is prime, then $\phi\left(p\right) = p-1$.
    }
}

\parag{Theorem}{
    Let $p$ be a prime number, and $k \in \mathbb{Z}_+$. Then:
    \[\phi\left(p^k\right) = p^k - p^{k-1}\]
}

\parag{Theorem}{
    Let $p$ and $q$ be two distinct primes. Then:
    \[\phi\left(pq\right) = \left(p-1\right)\left(q -1\right)\]
}

\parag{Definition: Regular Cartesian product}{
    Let $\mathcal{A}_1$ and $\mathcal{A}_2$ be two sets. The \important{Cartesian product} $\mathcal{A}_1 \times \mathcal{A}_2$ is the set:
    \[\mathcal{A} = \mathcal{A}_1 \times \mathcal{A}_2 = \left\{\left(a_1, a_2\right) \text{ such that } a_1 \in \mathcal{A}_1, a_2 \in \mathcal{A}_2\right\}\]
}

\parag{Definition: Cartesian product for groups}{
    Similarly, the \important{Cartesian product of two groups}, shorten the \important{product group}, $\left(G, \star\right) = \left(G_1, \star_1\right)\times \left(G_2, \star_2\right)$ is the set $G_1 \times G_2$ endowed with the binary operation $\star$, defined by:
    \[\left(a_1, a_2\right) \star \left(b_1, b_2\right) = \left(a_1 \star_1 b_1, a_2 \star_2 b_2\right)\]
}

\end{document}
