\documentclass[a4paper]{article}

% Expanded on 2021-10-13 at 15:14:34.

\usepackage{../../style}

\title{AICC-1}
\author{Joachim Favre}
\date{Mercredi 13 octobre 2021}

\begin{document}
\maketitle

\lecture{8}{2021-10-13}{End of functions and beginning of relations}{
}

\parag{Proofs}{
    Let $f : A \mapsto B$ be a function.

    To prove that $f$ is injective, we select arbitrary $x, y \in A$, we show that if $f\left(x\right) = f\left(y\right)$ then $x = y$. To show that $f$ is not injective, we find $x, y \in A$ such that $x \neq y$ and $f\left(x\right) = f\left(y\right).$

    To prove that $f$ is surjective, we select an arbitrary $y \in B$, and we show the existence of an element $x \in A$ such that $f\left(x\right) = y$. To show that $f$ is not surjective, we must find a $y \in B$ such that $f\left(x\right) \neq y$ for all $x \in A$.
}

\parag{Inverse functions}{
    Let $f$ be a bijection from $A$ to $B$. The \important{inverse} of $f$, denoted $f^{-1}$, is the function from $B$ to $A$ defined as:
    \[f^{-1}\left(y\right) = x \iff f\left(x\right) = y\]

    \subparag{Importance of bijectivity}{
        If $f$ was not injective, then $f^{-1}$ would have multiple values for some $x$.

        If $f$ was not surjective, the $f^{-1}$ would not be defined for some $x$.
    }
}

\parag{Composition of functions}{
    Let $f: B \mapsto C$ and $g: A \mapsto B$ be functions. The \important{composition} of $f$ with $g$, denoted $f \circ g$ is the function from $A$ to $C$ defined by:
    \[\left(f \circ g\right)\left(x\right) = f\left(g\left(x\right)\right)\]
}

\parag{Partial functions}{
    A \important{partial function} $f$ from a set $A$ to a set $B$ is an assignment to each element $a$ in a \textit{subset} of $A$, called the \important{domain of definition} of $f$, of a unique element $b$ in $B$.

    The sets $A$ and $B$ are called the \important{domain} and \important{codomain} of $f$ respectively. We say that $f$ is \important{undefined} for elements in $A$ that are not in the domain of definition of $f$. When the domain of definition of $f$ equals $A$, we say that $f$ is a \important{total function}.
}

\parag{Proposition}{
    Let $f, g$ be functions. If $g$ and $g \circ f$ are both injective, then $f$ is injective.
}

\parag{Properties}{
    Let $f: A \mapsto B$. Then for any sets $S, T \subseteq A$:
    \begin{enumerate}
        \item $f\left(S \cup T\right) = f\left(S\right) \cup f\left(T\right)$
        \item $f\left(S \cap T\right) \subseteq f\left(S\right) \cap f\left(T\right)$
    \end{enumerate}
}

\section{Relations, sequences and summations}
\subsection{Relations}
\parag{Definition: binary relations}{
    A \important{binary relation} $R$ from a set $A$ to a set $B$ is a subset of their Cartesian product:
    \[R \subseteq A \times B\]

    We notice that $\o$ is a relation from any two sets $A$ and $B$.
}


\parag{Combining relations}{
    Let $R_1$ and $R_2$ be two relations. We can combine them using basic set operations to form new relations, such as:
    \[R_1 \cup R_2, \mathspace R_1 \cap R_2, \mathspace R_1 \setminus R_2, \mathspace R_2 \setminus R_1\]
}

\parag{Compositions}{
    Let $A, B, C$ be sets, $R$ be a relation from $A$ to $B$, and $S$ be a relation from $B$ to $C$. The \important{composite} of $R$ and $S$, denoted $S \circ R$, is the relation consisting of ordered pairs $\left(a, c\right)$, where $a \in A$, $c \in C$ and for which there exists an element $b \in B$ such that $\left(a, b\right) \in R$ and $\left(b, c\right) \in S$.

    Note that the order is important, $S \circ R \neq R \circ S$ in general (one may even be defined, whereas the other not).
}

\parag{Definition: $N$-ary relations}{
    Let $A_1, \ldots, A_n$ be sets. An \important{$n$-ary relation} on these sets is a subset of the Cartesian product $A_1 \times \ldots \times A_n$. The sets are called the \important{domains} of the relation, and $n$ is called its \important{degree}.
}

\subsection{Relations on a set}
\parag{Definition: Relation on a set}{
    A \important{binary relation $R$ on a set $A$} is a subset of the Cartesian product $A \times A$. In other words, it is a relation from $A$ to $A$.
}

\parag{Summary}{
    We can draw the following table:
    \begin{center}
        \begin{tabular}{|ll|}
            \hline
            \fullbf{Name}          & \fullbf{Definition}                                                                                                            \\
            \hline
            \textit{Reflexive}     & $\forall x \left(x \in A \to \left(x, x\right) \in R\right)$                                                                   \\
            \textit{Symmetric}     & $\forall x \forall y \left(\left(x, y\right) \in R \to \left(y, x\right) \in R\right)$                                         \\
            \textit{Antisymmetric} & $\forall x \forall y \left(\left(x, y\right) \in R \land \left(y, x\right) \in R \to x = y\right)$                             \\
            \textit{Transitive}    & $\forall x \forall y \forall z \left(\left(x, y\right) \in R \land \left(y, z\right) \in R \to \left(x, z\right) \in R\right)$ \\
            \hline
        \end{tabular}
    \end{center}
}


\parag{Number of relations on a set}{
    We wonder how many relations can be made on a set $A$. We know that:
    \[\left|A \times A\right| = \left|A\right|\cdot \left|A\right| = \left|A\right|^2\]

    Then, every subset of $A \times A$ can be a relation, so we are looking at the power set:
    \[\left|\mathcal{P}\left(A \times A\right)\right| = 2^{\left|A \times A\right|} = 2^{\left|A\right|^2}\]

    So, there are often many possible relations, but it is finite on a finite set.
}

\end{document}
