\documentclass[a4paper]{article}

% Expanded on 2022-04-04 at 17:19:53.

\usepackage{../../style}

\title{Analyse 2}
\author{Joachim Favre}
\date{Mercredi 06 avril 2022}

\begin{document}
\maketitle

\lecture{14}{2022-04-06}{Toujours plus d'exemples}{
}

\parag{Démonstration de la dérivabilité}{
    Voici deux méthodes pour démontrer qu'une fonction est dérivable.

    \subparag{Méthode 1}{
        Nous savons que si toutes les dérivées partielles d'ordre 1 sont continues au point donné, alors nous savons que cela implique que $f$ est dérivable.

        Il est important de voir que le fait qu'une ou plusieurs dérivées partielles $\frac{\partial f}{\partial x_i}$ ne soient pas continues en $\bvec{a}$ n'implique pas nécessairement que $f$ n'est pas dérivable en $\bvec{a}$, comme nous le verrons dans l'exemple 5, et comme nous pouvons voir sur le schéma de résumé.
    }

    \subparag{Méthode 2}{
        Si le gradient $\nabla f\left(\bvec{a}\right)$ n'existe pas, alors nous savons que $f$ n'est pas dérivable en $\bvec{a}$. S'il existe, nous pouvons poser:
        \[r\left(\bvec{x}\right) = f\left(\bvec{x}\right) - f\left(\bvec{a}\right) - \left<\nabla f\left(\bvec{a}\right), \bvec{x} - \bvec{a}\right>\]

        Alors, si $\lim_{\bvec{x} \to \bvec{a}} \frac{r\left(\bvec{x}\right)}{\left\|\bvec{x} - \bvec{a}\right\|} = 0$, nous savons que $f$ est dérivable en $\bvec{a}$ par définition.

        De manière similaire, si $\lim_{\bvec{x} \to \bvec{a}} \frac{r\left(\bvec{x}\right)}{\left\|\bvec{x} - \bvec{a}\right\|} \neq 0$, alors $f$ n'est pas dérivable par notre premier théorème. En effet, si une fonction est dérivable, alors $L_{\bvec{a}}\cdot \bvec{v} = \left<\nabla f\left(\bvec{a}\right), \bvec{v}\right>$, et donc $r\left(\bvec{x}\right)$ et telle que donnée ci-dessus. Ainsi, si notre limite ne donne pas 0, c'est une contradiction avec le fait que la fonction soit dérivable.
    }
}

\parag{Résumé}{
    Nous pouvons à nouveau voir notre résumé. Soit $f: E \mapsto \mathbb{R}$, où $E$ est ouvert, alors:
    \svghere[0.5]{DiagrammeDerivabilite.svg}
}

\end{document}
