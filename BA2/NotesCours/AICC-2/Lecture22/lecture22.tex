\documentclass[a4paper]{article}

% Expanded on 2022-05-15 at 16:30:06.

\usepackage{../../style}

\title{AICC 2}
\author{Joachim Favre}
\date{Jeudi 12 mai 2022}

\begin{document}
\maketitle

\lecture{22}{2022-05-12}{Matrices!}{
}

\parag{Definition: Subspace}{
    Let $V$ be a vector space.

    A set $S \subseteq V$ is called a \important{subspace} if it is non-empty and closed under the vector space operations, meaning that:
    \[\forall \alpha \in \mathbb{F}\ \left(\bvec{s} \in S \implies \alpha \bvec{s} \in S\right)\]
    \[\bvec{s_1}, \bvec{s_2} \in S \implies \bvec{s_1} + \bvec{s_2} \in S\]
}

\parag{Definition: Linear combination}{
    A \important{linear combination} of a list of vectors $\left\{\bvec{v_1}, \ldots, \bvec{v_r}\right\}$ is any vector of the form:
    \[\sum_{i=1}^{r} \lambda_i \bvec{v}_i, \mathspace \lambda_i \in \mathbb{F}\]

}

\parag{Definition: Span}{
    The \important{span} $S$ of a list of vectors is the set of all linear combinations we can make from the list. We say that the list spans $S$.
}

\parag{Definition: Finite-dimensional vector space}{
    A vector space $V$ that is the span of a list of vectors is a \important{finite-dimensional vector space}.
}

\parag{Definition: Linear independence}{
    The vectors $\left\{\bvec{v_1}, \ldots, \bvec{v_n}\right\}$ are said to be \important{linearly independent}, if:
    \[\sum_{i=1}^{n} \lambda_i \bvec{v_i} = \bvec{0} \implies \lambda_i = 0 \text{ for all } i\]

    In other words, no vector can be written as a linear combination of the others.
}

\parag{Definition: Basis}{
    A \important{basis} for a finite-dimensional vector space is a list of linearly independent vectors that spans $V$.
}

\parag{Theorem}{
    A list $\left\{\bvec{v_1}, \ldots, \bvec{v_n}\right\}$ is a basis \textit{if and only if} every $\bvec{v} \in V$ can be written \textit{uniquely} as a linear combination of this list $\sum_{i=1}^{n} \lambda_i \bvec{v_i}$.
}

\parag{Theorem}{
    Every spanning list in a vector space can be reduced to a basis of the vector space.

    \subparag{Algorithm}{
        \begin{enumerate}[left=0pt]
            \item Remove all zero-elements of the list.
            \item Of the new list, remove the second element if it is in the linear span of the first. We repeat this step until we have a list where the second element is not in the linear span of the first.
            \item We do similarly with the third element: we remove it if it is in the linear span of the first two.
            \item We continue until we have considered the last vector of the list. This yields a list of vector that spans the vector space and are linearly independent (else one vector could be written as the linear combination of vectors with smaller index).
        \end{enumerate}
    }
}

\parag{Theorem}{
    Any two bases of a finite-dimensional vector space have the same length.
}

\parag{Definition: Dimension}{
    The \important{dimension} of a finite-dimensional vector space $V$, denoted by $\dim\left(V\right)$, is defined to be the length of any basis of $V$.

}

\parag{Dimension: Properties}{
    Let $V$ be a vector space with $\dim\left(V\right) = n$. Then, we have the following properties.

    \begin{itemize}
        \item A list of $n$ linearly independent vectors in $V$, is a basis of $V$.
        \item A list of $n$ vectors in $V$ spanning $V$, is a basis of $V$.
        \item Any list of $m > n$ vectors in $V$ cannot be linearly independent.
        \item Any list of $m < n$ vectors cannot span $V$.
    \end{itemize}
}

\parag{Canonical basis}{
    Let $\mathbb{F}$ be a finite field with $0$ being its additive neutral element and $1$ being its multiplicative identity element. Then, the \important{canonical basis} of $\mathbb{F}^n$, is:
    \[\left(\left(1, 0, 0, \ldots, 0, 0\right), \left(0, 1, 0, \ldots, 0, 0\right), \ldots, \left(0, 0, 0, \ldots, 0, 1\right)\right)\]
}

\subsection{Vector spaces over finite fields}
\parag{Lemma}{
    Let $\mathbb{F}$ be a finite field with cardinality $\left|\mathbb{F}\right|$. Then, the vector space $V = \mathbb{F}^n$ is finite and has cardinality:
    \[\left|V\right| = \left|\mathbb{F}\right|^n\]

    \subparag{Implication}{
        This implies that $\mathbb{F}_p^n$ has cardinality $p^n$.
    }
}

\parag{Theorem}{
    A $k$-dimensional vector space $V$ over a finite field $\mathbb{F}$ is finite and has cardinality:
    \[\left|V\right| = \left|\mathbb{F}\right|^k\]
}

\parag{Definition: Rank}{
    The number of linearly independent rows of a matrix is called the \important{rank}.

    \subparag{Remark}{
        Note that this is equal to the number of linearly independent columns of the matrix.
    }
}

\parag{Definition}{
    A vector $\bvec{v} = \left(v_1, \ldots, v_n\right) \in \mathbb{F}^n$ \important{fulfills a system of linear homogeneous equations} with coefficients in $\mathbb{F}$ if it satisfies $\bvec{v} A^T = \bvec{0}$ where $A$ is the matrix of equations.
}

\parag{Theorem}{
    \begin{itemize}[left=0pt]
        \item The set of solutions in $\mathbb{F}^n$ of a set of $m$ homogeneous linear equations in $n$ variables and coefficients in $\mathbb{F}$ is a subspace $S$ of $\mathbb{F}^n$.
        \item Let $A$ be the coefficient matrix, and $r$ be the rank of $A$. Then, $\dim\left(S\right) = n - r$.
        \item Conversely, if $S$ is a subspace of dimension $\dim\left(S\right) = k$, then there exists a system of $m = n - k$ linear homogeneous equations with coefficients in $\mathbb{F}$, the solution of which is exactly $S$.
    \end{itemize}
}

\end{document}
