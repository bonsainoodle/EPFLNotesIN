\documentclass[a4paper]{article}

% Expanded on 2022-03-28 at 10:22:21.

\usepackage{../../style}

\title{Méthodes de démonstration}
\author{Joachim Favre}
\date{Lundi 28 mars 2022}

\begin{document}
\maketitle

\lecture{11}{2022-03-28}{Récurrence}{
}

\parag{Méthode 7: Récurrence}{
    Soit $P\left(n\right)$ une proposition qui dépend de $n \in \mathbb{N}$, où $n \geq n_0$. Supposons que:
    \begin{enumerate}
        \item $P\left(n_0\right)$ est vraie.
        \item $P\left(n\right)$ implique que $P\left(n+1\right)$ pour tous $n \geq n_0$ naturels.
    \end{enumerate}

    Alors $P\left(n\right)$ est vraie pour tous $n \geq n_0$.
}

\parag{Récurrence généralisée}{
    Soit $P\left(n\right)$ une proposition qui dépend de $n \in \mathbb{N}$, où $n \geq n_0$. Supposons qu'il existe un $k \in \mathbb{N}$ tel que:
    \begin{enumerate}
        \item $P\left(n_0\right), \ldots, P\left(n_0 + k\right)$ sont vraies.
        \item $\left\{P\left(n\right), \ldots, P\left(n + k\right)\right\}$ impliquent $P\left(n + k + 1\right)$ pour tout $n \geq n_0$.
    \end{enumerate}

    Alors, $P\left(n\right)$ est vraie pour tout $n \geq n_0$, où $n \in \mathbb{N}$.
}

\end{document}
