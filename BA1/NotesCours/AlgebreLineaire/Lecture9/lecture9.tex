\documentclass[a4paper]{article}

% Expanded on 2021-10-21 at 08:13:36.

\usepackage{../../style}

\title{Algèbre}
\author{Joachim Favre}
\date{Jeudi 21 octobre 2021}

\begin{document}
\maketitle

\lecture{9}{2021-10-21}{Détermination(.) du déterminant}{
}


\parag{Théorème}{
    Soit $A$ une matrice carrée.
    \begin{enumerate}
        \item Si l'on \important{ajoute à une ligne} de $A$ un multiple d'une autre ligne, alors la matrice $B$ obtenue vérifie
              \[\det B = \det A\]

        \item Si l'on \important{échange deux lignes} de $A$, alors la matrice $B$ obtenue vérifie
              \[\det B = -\det A\]

        \item Si l'on \important{multiplie une ligne} de $A$ par $k$, alors la matrice $B$ obtenue vérifie
              \[\det B = k\det A\]
    \end{enumerate}

    \subparag{Remarque}{
        Ce théorème est très pratique pour calculer un déterminant d'une matrice. On peut transformer notre matrice $A$ en matrice diagonale, puis il est très simple de retrouver le déterminant si on a pris notes des opérations qu'on a utilisée.
    }
}

\parag{Théorème (multiplication de matrice élémentaire)}{
    Si $E$ est une matrice élémentaire, et $A$ est une matrice quelconque, alors
    \[\det\left(EA\right) = \det\left(E\right)\det\left(A\right)\]

    \subparag{Généralisation}{
        On se rendra compte que c'est vrai pour n'importe quelle matrice $A$ et $B$.
    }
}

\parag{Théorème}{
    $A \in \mathbb{R}^{n \times n}$ est inversible si et seulement si $\det\left(A\right) \neq 0$.
}

\parag{Théorème (produit de déterminant)}{
    Pour $A, B \in \mathbb{R}^{n\times n}$ (matrices carrées de même taille), alors on a:
    \[\det\left(AB\right) = \det\left(A\right)\det\left(B\right)\]
}

\parag{Déterminant de l'inverse}{
    Si $A \in \mathbb{R}^{n\times n}$ est inversible, alors on a que:
    \[1 = \det\left(I_n\right) = \det\left(A A^{-1}\right) = \det\left(A\right)\det\left(A^{-1}\right) \]

    Donc:
    \[\det\left(A^{-1}\right) = \frac{1}{\det\left(A\right)}\]
}

\parag{Déterminant d'une puissance}{
Si on a $A \in \mathbb{R}^{n \times n}$ et $k \in \mathbb{N}^*$, alors:
\[\det\left(A^{k}\right) = \det\left(A \cdots A\right) = \underbrace{\det\left(A\right)\cdots \det\left(A\right)}_{k \text{ fois}} = \det\left(A\right)^k\]

\subparag{Inversibilité de la puissance}{
    Il est donc intéressant de voir que si $A$ n'est pas inversible, alors $A^{k}$ ne le sera jamais. De la même manière, si $A$ est inversible, alors $A^{k}$ l'est forcément.
}

}

\parag{Déterminant du produit avec un scalaire}{
    Si on a $A \in \mathbb{R}^{n\times n}$ et $k \in \mathbb{R}$, alors:
    \[\det\left(kA\right) = \det\left(kI_n A\right) = \det\left(k I_n\right) \det\left(A\right) = k^{n}\det\left(A\right)\]

}

\parag{Théorème (interprétation géométrique)}{
    Si $A$ est une matrice $2 \times 2$, l'aire du parallélogramme défini par les colonnes de $A$ est égale à $\left|\det A\right|$. Si $A$ est une matrice $3\times 3$, le volume du parallélépipède défini par les colonnes de $A$ est égal à $\left|\det A\right|$.

    \subparag{Notation}{
        Il arrive que $\det A$ soit noté $\left|A\right|$. Il ne faut pas confondre cette notation avec la valeur absolue d'un scalaire, comme $\left|\det A\right|$.
    }
}
\parag{Le cas des transformations linéaires}{
    On peut voir qu'une transformation linéaire $T : \mathbb{R}^2 \mapsto \mathbb{R}^2$ définie comme $T\left(\bvec{x}\right) = A \bvec{x}$ a pour effet de multiplier la surface par $\left|\det A\right|$, dans le cas d'un carré.

    De manière générale, si $S$ est une région de $\mathbb{R}^2$ de surface finie, alors
    \[\text{surface}\left(T\left(S\right)\right) = \left|\det A\right|\text{surface}\left(S\right)\]

    On a le résultat analogue dans $\mathbb{R}^3$:
    \[\text{volume}\left(T\left(S\right)\right) = \left|\det A\right| \text{volume}\left(S\right)\]
}

\end{document}
