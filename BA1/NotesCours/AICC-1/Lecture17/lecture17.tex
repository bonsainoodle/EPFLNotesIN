\documentclass[a4paper]{article}

% Expanded on 2021-11-16 at 08:39:14.

\usepackage{../../style}

\title{AICC 1}
\author{Joachim Favre}
\date{Mardi 16 novembre 2021}

\begin{document}
\maketitle

\lecture{17}{2021-11-16}{I have a proof too large to fit in the margin}{
}

\subsection{Merge sort}
\parag{Recursive Sorting}{
    We saw sorting algorithms that had $\Theta\left(n^2\right)$ complexity.

    \important{Merge sort} is a sorting algorithm that performs significantly better. It works by iteratively splitting a list into two sublists of equal length, until each sublist has one element. At each step, a pair of sublists is successively merged into a list with the elements in increasing order. The process ends when all sublists have been merged.

    \imagehere[0.6]{mergeSortRepresentation.png}

}

\begin{filecontents*}[overwrite]{mergefrommergesort.code}
    procedure merge(a1, a2: sorted array)
    a := empty array
    while a1 and a2 are both nonempty
    remove the smaller amongst the first element of a1 and the one of a2 from its list
    put it at the end of a
    if one list is empty then remove all elements from the other list and append them to L.
    return a
\end{filecontents*}


\parag{Merge function}{
    We use the following algorithm to merge two sorted lists into a sorted list:
    \importcode{mergefrommergesort.code}{pseudo}

    The idea behind this function is that we can always consider the first element of both list, and pick the one that is the smallest, since they are both already sorted.

    The cost of merging two lists is at most \texttt{len(a1) + len(a2) - 1}.
}

\begin{filecontents*}[overwrite]{mergesort.code}
    procedure mergesort(a: array of n elements)
    if n > 1 then:
    m := floor(n/2)
    a1 := a[1], a[2], ..., a[m]
    a2 := a[m+1], a[m+2], ..., a[n]
    return merge(mergesort(a1), mergesort(a2))
    else:
    return a
\end{filecontents*}

\parag{Pseudo code}{
    Here is an example on how to write merge sort:
    \importcode{mergesort.code}{pseudo}
}


\parag{Complexity}{
    This algorithm has complexity $\Theta\left(n\log\left(n\right)\right)$, i.e. linearithmic complexity. This is the best complexity that can yet be obtained for sorting, it is much better than $\Theta\left(n^2\right)$.
}

\section{Number theory}
\subsection{Division}
\parag{History}{
    Euclid wrote the most successful mathematics book ever ``the elements'' (it has over 1000 editions); Gauss said ``mathematics is the queen of sciences, number theory is the queen of mathematics''; Wiles proved Fermat's last theorem 358 years after it was conjectures (someone please give bigger margins to Fermat) in 1958; Tao proved a famous theorem on arbitrarily long arithmetic processions of primes in 1975.

    This is a very old and important subject.
}

\parag{Number theory and computer science}{
    We use number theory in the representation and computation of integers, cryptography, coding, pseudo-random number generation, and so on.

    Computing is also used to explore open problems in number theory, such as in the Collatz conjecture.
}

\parag{Definition of division}{
    If $a$ and $b$ are integers, with $a \neq 0$, then \important{$a$ divides $b$} if there exists an integer $c$ such that $b = ac$. When $a$ divides $b$, we say that $a$ is a \important{factor} or \important{divisor} of $b$, and that $b$ is a \important{multiple} of $a$.

    \subparag{Notation}{
        The notation $a \divides b$ denotes $a$ divides $b$. If $a$ does not divide $b$, then we write $a \ndivides b$.
    }

    \subparag{Observation}{
        We see that, if $a \divides b$, then:
        \[a \divides b \iff b = ak \iff \frac{b}{a} = k \in \mathbb{Z}\]

        So, $\frac{b}{a}$ is an integer if and only if $a \divides b$.
    }

}

\parag{Properties of divisibility}{
    \begin{enumerate}[left=0pt]
        \item If $a \divides b$ and $a \divides c$, then $a \divides \left(b + c\right)$
        \item If $a \divides b$, then $a \divides bc$ for all integers $c$
        \item If $a\divides b$ and $b\divides c$, then $a\divides c$
    \end{enumerate}

    \subparag{Proof of property (1)}{
        We can write $b = ak_1$ and $c = ak_2$, where $k_1$ and $k_2$ are integers. Therefore:
        \[b + c = ak_1 + bk_2 = a\left(k_1 + k_2\right)\]

        Since $k_1 + k_2$ is an integer, then $a \divides b + c$.

    }
}

\parag{Corollary}{
    If $a,b,c$ are integers, where $a \neq 0$ such that $a \divides b$ and $a \divides c$, then:
    \[a | mb + nc, \mathspace m,n \in \mathbb{Z}\]
}

\parag{Division algorithm}{
    When an integer is divided by a positive integer, there is a quotient and a remainder. This is traditionally called the ``division algorithm'', but it is in fact a theorem.

    \subparag{Theorem}{
        If $a$ is an integer and $d$ is a positive integer, then there are unique integers $q$ and $r$, with $0 \leq r < d$, such that:
        \[a = dq + r\]

        We can use the well-ordering of the natural numbers (and thus induction) to prove this theorem.
    }

    \subparag{Terminology}{
        If we have $a = dq + r$, then:
        \begin{itemize}
            \item $d$ is called the \important{divisor}.
            \item $a$ is called the \important{dividend}.
            \item $q$ is called the \important{quotient}.
            \item $r$ is called the \important{remainder}.
        \end{itemize}

        We write:
        \[q = a \Div d, \mathspace r = a \Mod d\]

        $\Div$ and $\Mod$ are functions:
        \[\Div : \mathbb{Z} \times \mathbb{Z}_+ \mapsto \mathbb{Z} \mathspace \text{ and } \mathspace \Mod : \mathbb{Z} \times \mathbb{Z}_+ \mapsto \mathbb{N}\]
    }
}

\parag{Example 1}{
    We want to find the quotient and remainder when $101$ is divided by $11$. We see that $101$ can be written as:
    \[101 = 9\cdot 11 + 2\]

    Therefore, $q = 101 \Div 11 = 9$ and $r = 101 \Mod 11 = 2$.
}

\parag{Example 2}{
    Now, we want to divide $-11$ by 3. We see that:
    \[-11 = -4\cdot 3 + 1 \implies q = -11 \Div 3 = -4 \text{ and } r = -11 \Mod 3 = 1\]

    We have to be careful since dividing (positive) 11 by 3 is completely different:
    \[11 \Div 3 = 3, \mathspace 11 \Mod 3 = 2\]
}

\subsection{Congruence}
\parag{Definition of congruence}{
    Let $a$ and $b$ are integers and $m$ be a positive integer. We say that $a$ is \important{congruent} to $b$ modulo $m$ if:
    \[m \divides a - b\]

    We note $\congruent{a}{b}{m}$ to say that $a$ is congruent to $b$ modulo $m$. We say that $\congruent{a}{b}{m}$ is a congruence, and that $m$ is its modulus. If $a$ is not congruent to $b$ modulo $m$, we write $\notcongruent{a}{b}{m}$.

    We say that $\congruent{a}{b}{m}$
}

\parag{Example}{
    We wonder if $17$ is congruent to $5$ modulo 6:
    \[17 -5 = 12, \mathspace 6 \divides 12 \implies \congruent{17}{5}{6}\]

    We can now see if $24$ and $14$ are congruent modulo 6:
    \[24 - 14 = 10, \mathspace 6 \ndivides 10 \implies \notcongruent{24}{14}{6}\]
}

\parag{Theorem}{
    $\congruent{a}{b}{m}$ is an equivalence relation.
}


\parag{Notations}{
    We saw that the notations $\congruent{a}{b}{m}$ and $a \Mod m = b$ are different. The first one is a relation, the second one is a function.
}

\parag{Theorem 1}{
    Let $a$ and $b$ be integers, and let $m$ be a positive integer. We have:
    \[\congruent{a}{b}{m} \iff a \Mod m = b \Mod m\]

    \subparag{Corollary}{
        Two integers are congruent $\Mod m$ if and only if they have the same remainder when divided by $m$.
    }


    \imagehere{CongruenceModuloLinkRepresentation.png}
}

\parag{Theorem 2}{
    Let $m$ be a positive integer, and $a,b$ be integers. We have:
    \[\congruent{a}{b}{m} \iff \exists k \in \mathbb{Z} \text{ such that } a = b + km\]
}

\parag{Theorem 3}{
    Let $m$ be a positive integer. If $\congruent{a}{b}{m}$ and $\congruent{c}{d}{m}$ then:
    \[\congruent{a+c}{b+d}{m} \mathspace \text{ and } \mathspace \congruent{ac}{bd}{m}\]

    \subparag{Warning}{
        We have to beware of the fact that it does not work for powers. Indeed, we can have the following counter-example:
        \[\congruent{2}{-1}{3} \text{ and } \congruent{3}{0}{3}\]

        However:
        \[3 \ndivides 7 \implies 3 \ndivides 2^3 - \left(-1\right)^{0} \implies \notcongruent{2^3}{\left(-1\right)^0}{3}\]
    }

    \subparag{Proof of multiplicativity}{
        We have
        \[a = b + k_1m \mathspace \text{and} \mathspace c = d + k_2 m\]

        So:
        \[ac = \left(b + k_1m\right)\left(d + k_2m\right) = bd + k_1md + k_2mb + k_1k_2m = bd + \bar{k}m\]

        Which allows us to conclude:
        \[\congruent{ac}{bd}{m}\]

    }

    So, this operation is very robust under elementary operations.
}

\parag{Example}{
    If we know that $\congruent{7}{2}{5}$ and $\congruent{11}{1}{5}$, then we know that:
    \[\congruent{77}{2}{5}\]
}

\parag{Corollary}{
    Adding an integer to both sides of a valid congruence preserves validity:
    \[\congruent{a}{b}{m} \implies \congruent{a+c}{b+c}{m}, \mathspace \forall c \in \mathbb{Z}\]

    Multiplying both sides of a valid congruence by an integer preserves validity:
    \[\congruent{a}{b}{m} \implies \congruent{c\cdot a}{c\cdot b}{m}, \mathspace \forall c \in \mathbb{Z}\]

    Putting both sides of a valid congruence to an integer power preserves validity:
    \[\congruent{a}{b}{m} \implies \congruent{a^c}{b^c}{m}, \mathspace \forall c \in \mathbb{Z}\]

    \subparag{Proof}{
        We can use the last theorem for the two first points. Then, for the third one, we can see that:
        \[\congruent{a}{b}{m} \implies \congruent{aa}{bb}{m} \implies \congruent{aaa}{bbb}{m}\]

        Repeating this idea $n$ times:
        \[\congruent{a^n}{b^n}{m}\]
    }

    \subparag{Warning}{
        We have to beware of the fact that we can neither divide by an integer, nor take a nth-root:
        \[\congruent{14}{8}{6} \mathspace \text{but} \mathspace \notcongruent{7}{4}{6}\]
        \[\congruent{16}{9}{7} \mathspace \text{but} \mathspace \notcongruent{4}{3}{7}\]

        Dividing by an integer is a bit more special, because in some specific cases it will work ($\congruent{ac}{bc}{m} \implies \congruent{a}{b}{m}$ if $c$ and $m$ are coprimes).
    }
}

\parag{Property of congruences}{
    We wonder if:
    \[\congruent{a}{b}{pq} \implies \congruent{a}{b}{p}\]

    We have:
    \[\congruent{a}{b}{pq} \implies a = b + kpq \implies a = b + \left(kq\right)p \implies \congruent{a}{b}{p}\]

    So yes. Now we may wonder if the reciprocal holds, but we can find a counterexample:
    \[\congruent{1}{5}{2}, \mathspace \congruent{1}{5}{4} \mathspace \text{ but } \mathspace \notcongruent{1}{5}{8}\]

}


\subsection{Modular arithmetic}
\parag{Corollary}{
    Let $m$ be a positive integer, and let $a and b$ be integers. Then:
    \[\left(a + b\right) \Mod m = \left(\left(a \Mod m\right) + \left(b \Mod m\right)\right) \Mod m\]
    \[\left(a \cdot b\right) \Mod m = \left(\left(a \Mod m\right) \cdot \left(b \Mod m\right)\right) \Mod m\]

    \subparag{Proof}{
        We are using the fact that:
        \[\congruent{a}{a \Mod m}{m}\]
    }
}

\parag{Example}{
    This corollary allows us to simplify some computations:
    \[280 \Mod 6 = \left(28 \Mod 6 \cdot 10 \Mod 6\right) \Mod 6 = \left(4 \cdot 4\right) \Mod 6 = 16 \Mod 6 = 4\]
}


\parag{Definitions}{
    Let \important{$\mathbb{Z}_m$} be the set of nonnegative integers less than $m$:
    \[\mathbb{Z}_m = \left\{0, 1, \ldots, m - 1\right\}\]

    The \important{addition modulo m} $+_m$ is defined as:
    \[a +_m b = \left(a + b\right) \Mod m\]

    The \important{multiplication modulo $m$} $\cdot_m$ is defined as:
    \[a \cdot_m b = \left(a\cdot b\right) \Mod m\]
}

\parag{Example}{
For example:
\[7 +_{11} 9 = 5, \mathspace 7 \cdot_{11} 9 = 8\]
}

\parag{Tables}{
    We can draw tables for our operations. For example, in $\mathbb{Z}_3$:
    \begin{center}
        \begin{tabular}{c||c|c|c}
            + & 0 & 1 & 2 \\
            \hhline{=#===}
            0 & 0 & 1 & 2 \\
            \hhline{-||---}
            1 & 1 & 2 & 0 \\
            \hhline{-||---}
            2 & 2 & 0 & 1
        \end{tabular}
        \hspace{3em}
        \begin{tabular}{c||c|c|c}
            $\cdot$ & 0 & 1 & 2 \\
            \hhline{=#===}
            0       & 0 & 0 & 0 \\
            \hhline{-||---}
            1       & 0 & 1 & 2 \\
            \hhline{-||---}
            2       & 0 & 2 & 1
        \end{tabular}
    \end{center}

    We notice that we get $1\cdot_3 1 = 1$ and $2\cdot_3 2 = 1$. So, for $n \neq 0$, numbers have a multiplicative inverse on this set.
}

\end{document}

