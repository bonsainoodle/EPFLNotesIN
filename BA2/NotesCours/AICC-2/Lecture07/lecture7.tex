\documentclass[a4paper]{article}

% Expanded on 2022-03-15 at 15:17:21.

\usepackage{../../style}

\title{AICC 2}
\author{Joachim Favre}
\date{Mardi 15 mars 2022}

\begin{document}
\maketitle

\lecture{7}{2022-03-15}{Almost to the fundamental theorem}{
}

\parag{Proposition}{
    Let $X, Y, Z$ be three random variables. Then:
    \[p\left(X, Y, Z\right) = p\left(X\right)p\left(Y|X\right)p\left(Z|X, Y\right)\]
}

\parag{Definition}{
    We define the \important{entropy of a symbol} as:
    \[H\left(\mathcal{S}\right) = \lim_{n \to \infty} H\left(S_n\right)\]

    We also define the \important{entropy rate of the source} $\mathcal{S}$:
    \[H^*\left(\mathcal{S}\right) = \lim_{n \to \infty} H\left(S_n | S_1, \ldots, S_{n-1}\right)\]

    The source is said to be \important{regular} if both exist and are finite.
}

\parag{Stationary sources}{
    Let us say that $\left(S_1, \ldots, S_n\right)$ are distributed according to $p_{S_1, \ldots, S_n}\left(x_1, \ldots, x_n\right)$, which we can write as:
    \[\left(S_1, \ldots, S_n\right) \sim p_{S_1, \ldots, S_n}\left(x_1, \ldots, x_n\right)\]

    We can also see have a shifted version:
    \[\left(S_{k+1}, \ldots, S_{k+n}\right) \sim p_{S_{k+1}, \ldots, S_{k+n}}\left(x_1, \ldots, x_n\right)\]

    We call the source to be \important{stationary} if for all $n$ and $k$:
    \[p_{S_1, \ldots, S_n}\left(x_1, \ldots, x_n\right) = p_{S_{k+1}, \ldots, S_{k+n}}\left(x_1, \ldots, x_n\right), \mathspace \forall x_1, \ldots, x_n\]

    \subparag{Implication}{
        A stationary source has the following properties:
        \[p_{S_1} = p_{S_m}, \mathspace \forall m\]
        \[p_{S_1, S_2} = p_{S_m, S_{m+1}}, \mathspace \forall m\]
        \[p_{S_1, S_2, S_3} = p_{S_m, S_{m+1}, S_{m+2}}, \mathspace \forall m\]

        A more complicated property is:
        \[p_{S_m, S_n} = p_{S_{m+k}, S_{n+k}}\]

        Thus, for any subset $\mathcal{I}$ of indices (even indices that do not follow each other ($\mathcal{I} = \left\{0, 3, 7\right\}$ for instance)):
        \[p_{S_\mathcal{I}} = p_{S_{\mathcal{I} + k}}\]

    }
}

\parag{Theorem: Conditioning reduces entropy 2}{
    For any three discrete random variables $X, Y$ and $Z$:
    \[H_D\left(X|Y, Z\right) \leq H_D\left(X | Z\right)\]
    with equality if and only if $X$ and $Y$ are conditionally independent random variables given $Z$ (meaning $p\left(x, y|z\right) = p\left(x|z\right)p\left(y|z\right)$ for all $x, y, z$).
}

\parag{Theorem}{
    A stationary source is regular.

    \subparag{Converse}{
        Note that the converse is not true: a regular source is not necessarily stationary.
    }
}

\parag{Theorem}{
    For a stationary source, we have:
    \[H^*\left(\mathcal{S}\right) \leq H\left(\mathcal{S}\right)\]
    with equality if and only if the symbols are independent.
}

\parag{Theorem: Cesàro means}{
    Let $a_1, a_2, \ldots$ be a real-valued sequence and let $c_1, c_2, \ldots$ be the sequence of running average defined by:
    \[c_n = \frac{a_1 + \ldots + a_n}{n}\]

    If $\lim_{n \to \infty} a_n$ exists, then $c_n$ converges as well, and:
    \[\lim_{n \to \infty} c_n = \lim_{n \to \infty} a_n\]
}

\parag{Theorem}{
    For a stationary source, we have:
    \[\lim_{n \to \infty} \frac{H_D\left(S_1, \ldots, S_n\right)}{n} = H^*\left(\mathcal{S}\right)\]

    Moreover, $a_n = \frac{H_D\left(S_1, \ldots, S_n\right)}{n}$ is a non-increasing sequence.
}

\end{document}
