\documentclass[a4paper]{article}

% Expanded on 2022-04-05 at 15:18:05.

\usepackage{../../style}

\title{AICC-2}
\author{Joachim Favre}
\date{Mardi 05 avril 2022}

\begin{document}
\maketitle

\lecture{13}{2022-04-05}{Modulo world (less nice than Juice WRLD)}{
}

\subsection{$\mathbb{Z}$ modulo $m$}

\parag{Definition: Congruence class}{
    Let $m \in\mathbb{Z}$, where $m > 1$. Also, let $a \in \mathbb{Z}$. We define the \important{congruence class} of $a$ mod $m$ to be:
    \[\left[a\right]_m \over{=}{def} \left\{i \in\mathbb{Z} \text{ such that } \congruent{i}{a}{m}\right\}\]
}

\parag{Equivalences}{
    The following propositions are equivalent:
    \begin{enumerate}
        \item $\left[a\right]_m = \left[b\right]_m$
        \item $\congruent{a}{b}{m}$
        \item $a \Mod m = b \Mod m$
        \item $\left(a - b\right) \Mod m = 0$
    \end{enumerate}
}

\parag{Definition: $\mathbb{Z}$ mod $m$}{
    The set of all congruence classes modulo $m$, $\mathbb{Z}$ mod $m$, is denoted by $\mathbb{Z} / m\mathbb{Z}$ or $\mathbb{Z}_m$
}

\parag{Operations in $\mathbb{Z} / m\mathbb{Z}$}{
    We define the \important{sum} the following way:
    \[\left[a\right]_m + \left[b\right]_m = \left[a + b\right]_m\]

    Similarly, we can define the product as:
    \[\left[a\right]_m \left[b\right]_m = \left[ab\right]_m\]
}

\parag{Proprieties of addition}{
    \begin{itemize}[left=0pt]
        \item Associativity: $\displaystyle \left(\left[a\right]_m + \left[b\right]_m\right) + \left[c\right]_m = \left[a\right]_m + \left(\left[b\right]_m + \left[c\right]_m\right) = \left[a\right]_m + \left[b\right]_m + \left[c\right]_m$

        \item Commutativity: $\left[a\right]_m + \left[b\right]_m = \left[b\right]_m + \left[a\right]_m$

        \item Existence of additive identity: $\left[a\right]_m + \left[0\right]_m = \left[0\right]_m + \left[a\right]_m = \left[a\right]_m$.

        \item Existence of additive inverse: $\left[-a\right]_m + \left[a\right]_m = \left[0\right]_m$
    \end{itemize}
}

\parag{Properties of multiplication}{
    \begin{itemize}[left=0pt]
        \item Associativity: $\displaystyle \left(\left[a\right]_m \cdot \left[b\right]_m\right) \cdot \left[c\right]_m = \left[a\right]_m \cdot \left(\left[b\right]_m \cdot \left[c\right]_m\right) = \left[a\right]_m \cdot \left[b\right]_m \cdot \left[c\right]_m$

        \item Commutativity: $\left[a\right]_m \cdot \left[b\right]_m = \left[b\right]_m \cdot \left[a\right]_m$

        \item Existence of multiplicative identity: $\left[a\right]_m \cdot \left[1\right]_m = \left[1\right]_m \cdot \left[a\right]_m = \left[a\right]_m$.
    \end{itemize}

    However, we can realise that there is not always a multiplicative inverse.
}

\parag{Distributivity}{
    We also have the following property, that links both our operations:
    \[\left[a\right]_m \left(\left[b\right]_m + \left[c\right]_m\right) = \left[a\right]_m \left[b\right]_m + \left[a\right]_m \left[c\right]_m\]
}

\parag{Definition}{
    Let $k \in \mathbb{N}$, where $k \geq 1$. We define the following notation:
    \[k\left[a\right]_m = \underbrace{\left[a\right]_m + \ldots + \left[a\right]_m}_{\text{$k$ times}}\]

    \subparag{Remark}{
        We can prove that:
        \[k\left[a\right]_m = \left[k\right]_m \left[a\right]_m = \left[ka\right]_m\]
    }
}

\parag{Definition: Multiplicative inverse}{
    If it exists, the multiplicative inverse of $\left[a\right]_m$ is an element $\left[b_m\right]$ such that:
    \[\left[a\right]_m \left[b\right]_m = \left[b\right]_m \left[a\right]_m = \left[1\right]_m\]
}

\parag{Theorem: Unicity of the multiplicative inverse}{
    If it exists, the multiplicative inverse is unique.
}

\parag{Definition: Power}{
    Let $k \in \mathbb{N}^*$. We define:
    \[\left(\left[a\right]_m\right)^k = \underbrace{\left[a\right]_m \cdots \left[a\right]_m}_{\text{$k$ times}}\]

    When $k = 0$, we define:
    \[\left(\left[a\right]_m\right)^{0} = \left[1\right]_m\]
}

\parag{Proposition}{
    Let $\left[a\right]_m$ be a congruence class which has an inverse. Then, there exist no number $k \in \mathbb{N}$ such that:
    \[\left(\left[a\right]_m\right)^k = 0\]
}

\parag{Function terminology}{
    Let a function $f: \mathcal{E} \mapsto \mathcal{F}$. We call $\mathcal{E}$ to be the \important{domain}, $\mathcal{F}$ to be the \important{codomain}, $f\left(\mathcal{E}\right)$ to be the image.

    We can draw the following diagrams to state if a function if is injective, surjective, or bijective:
    \imagehere[0.7]{FunctionInjectivitySurjectivityBijectivity.png}

    The pigeonhole principle tells us that:

    \begin{itemize}
        \item $f$ is injective implies that $\left|\mathcal{E}\right| \leq \left|\mathcal{F}\right|$.
        \item $f$ is surjective implies that $\left|\mathcal{E}\right| \geq \left|\mathcal{F}\right|$.
        \item $f$ is bijective implies that $\left|\mathcal{E}\right| = \left|\mathcal{F}\right|$.
    \end{itemize}
}

\parag{Theorem}{
    In $\mathbb{Z} / m\mathbb{Z}$, the following three propositions are equivalent:
    \begin{enumerate}
        \item $\left[a\right]_m$ has an inverse.
        \item For all $\left[b\right]_m$, $\left[a\right]_m x = \left[b\right]_m$ has a unique solution.
        \item There exists a $\left[b\right]_m$ such that $\left[a\right]_m x = \left[b\right]_m$ has a unique solution
    \end{enumerate}
}

\end{document}
