\documentclass[a4paper]{article}

% Expanded on 2022-05-03 at 15:24:43.

\usepackage{../../style}

\title{AICC 2}
\author{Joachim Favre}
\date{Mardi 03 mai 2022}

\begin{document}
\maketitle

\lecture{19}{2022-05-03}{Error detection and correction}{
}

\section{Error detection and error correction codes}
\subsection{Introduction}

\parag{Definition: Erasure weight}{
    The number of erasures done in an erasure channel is the \important{erasure weight}, $p$.
}

\parag{Definition: Error weight}{
    The number of errors done in an error channel is the \important{error weight}, $p$.
}

\subsection{Channel coding}

\parag{Terminology (general case)}{
    \begin{itemize}[left=0pt]
        \item The \important{code} is the set of all codewords:
              \[\mathcal{C} \subseteq {\color{red}\mathcal{A}}^n\]
        \item $n$ is called the \important{block length}.
        \item $k$ is the \important{number of information symbol}, the number of symbols we are really sending. \textcolor{red}{This is defined to be}:
              \[k \over{=}{def}  {\color{red}\log_{\left|\mathcal{A}\right|} \left|\mathcal{C}\right|}\]
        \item The \important{rate} of the code is:
              \[R = \frac{k}{n}\]
              This represents the amount of redundancy we have.
    \end{itemize}
}

\parag{Definition: Hamming distance}{
    Let $x$ and $y$ be sequences of the same length.

    Then, the \important{hamming distance} between $x$ and $y$, $d\left(x, y\right)$, is the number of places where $x$ and $y$ differ.
}

\parag{Remark: Distance}{
    In math, a function of two variables $x$ and $y$ is a \important{distance}, if it satisfies:
    \begin{enumerate}
        \item $d\left(x, y\right) \geq 0, \forall x, y$
        \item $d\left(x, y\right) = 0 \iff x = y$
        \item Symmetry: $d\left(x, y\right) = d\left(y, x\right)$
        \item Triangle inequality (taking a detour cannot be faster than the direct path):
              \[d\left(x, y\right) \leq d\left(x, z\right) + d\left(z, y\right)\]
    \end{enumerate}
}

\parag{Theorem}{
    The Hamming distance is a distance.
}

\end{document}
