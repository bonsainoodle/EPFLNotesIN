\documentclass[a4paper]{article}

% Expanded on 2021-12-23 at 08:17:51.

\usepackage{../../style}

\title{Algèbre linéaire}
\author{Joachim Favre}
\date{Jeudi 23 décembre 2021}

\begin{document}
\maketitle

\lecture{27}{2021-12-23}{La puissance de la SVD}{
}

\parag{Théorème: base de l'image}{
    Les vecteurs $\left\{\bvec{u}_1, \ldots, \bvec{u}_r\right\}$ forment une base orthonormée de $\im A$.
}

\parag{Corollaire 1}{
    Puisqu'on a $r$ vecteurs qui nous font une base, on trouve:
    \[\rang A = \dim\im A = r\]

    Ce qui est le nombre de valeurs singulières strictement positives de $A$.
}

\parag{Corollaire 2}{
    Les vecteurs $\left\{\bvec{u}_{r+1}, \ldots, \bvec{u}_{m}\right\}$ forment une base de $\ker\left(A^T\right)$ (car \[\left(\im A\right)^\perp = \ker\left(A^T\right)\]).

    En particulier, cela nous dit que:
    \[\dim\left(\ker A^T\right) = m - r\]
}

\parag{Théorème}{
    Les vecteurs $\left\{\bvec{v}_1, \ldots, \bvec{v}_r\right\}$ forment une base orthonormée pour:
    \[\im\left(A^T\right)\]
}

\parag{Corollaire}{
    Les vecteurs $\left\{\bvec{v}_{r+1}, \ldots, \bvec{v}_n\right\}$ forment une base orthonormée pour:
    \[\left(\im\left(A^T\right)\right)^\perp = \ker\left(A\right)\]
}

\parag{Résumé}{
    Nous avons trouvé des base orthonormées pour les 4 sous-espaces intéressants de $A$:
    \begin{center}
        \begin{tabular}{ll}
            \textbf{Sous-espace}     & \textbf{Base orthonormée}                                         \\
            $\displaystyle \im A$    & $\displaystyle \left\{\bvec{u}_1, \ldots, \bvec{u}_r\right\}$     \\
            $\displaystyle \ker A^T$ & $\displaystyle \left\{\bvec{u}_{r+1}, \ldots, \bvec{u}_m\right\}$ \\
            $\displaystyle \im A^T$  & $\displaystyle \left\{\bvec{v}_1, \ldots, \bvec{v}_r\right\}$     \\
            $\displaystyle \ker A$   & $\displaystyle \left\{\bvec{v}_{r+1}, \ldots, \bvec{v}_n\right\}$ \\
        \end{tabular}
    \end{center}
}

\end{document}
