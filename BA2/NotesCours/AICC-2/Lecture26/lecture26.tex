\documentclass[a4paper]{article}

% Expanded on 2022-05-31 at 18:12:32.

\usepackage{../../style}

\title{AICC 2}
\author{Joachim Favre}
\date{Mardi 31 mai 2022}

\begin{document}
\maketitle

\lecture{26}{2022-05-31}{Reed-Solomon codes}{
}

\subsection{Polynomials over $\mathbb{F}_q$}
\parag{Definition}{
    Let $\bvec{u} = \left(u_1, \ldots, u_m\right) \in \mathbb{F}_q^m$ be a vector.

    We define the following polynomial:
    \[p_{\bvec{u}}\left(x\right) = u_1 + u_2 x + \ldots + u_m x^{m-1}, \mathspace \forall x \in \mathbb{F}_q\]
}

\parag{Definition: Degree of a polynomial}{
    The \important{degree of a polynomial} is the highest exponent $i$ for which $x^i$ has a non-zero coefficient.

    By convention, we define the degree of the polynomial $0$ to be $-\infty$:
    \[\deg\left(0\right) = -\infty\]
}

\parag{Polynomial interpolation}{
    Given $\left\{\left(a_1, y_1\right), \ldots, \left(a_k, y_k\right)\right\}$, where $\left(a_i, y_i\right) \in \mathbb{F}_q^2$, we want to find a polynomial such that it goes through all those points. We do not care what it does between two points.

    In other words, we want to find a polynomial $p_{\bvec{u}}\left(x\right)$ of degree $\deg\left(p_{\bvec{u}}\left(x\right)\right) \leq k - 1$ such that:
    \[p_{\bvec{u}}\left(a_i\right) = y_i, \mathspace i = 1, \ldots, k\]

    Finding such a polynomial is equivalent to finding its $\bvec{u}$.

    \subparag{Lagrange Interpolation}{
        The solution to this problem is Lagrange interpolation:
        \begin{enumerate}
            \item First, we make a polynomial $Q_i\left(x\right)$ with coefficients in $\mathbb{F}_5$ and degree less than or equal to $k-1$ such that:
                  \begin{functionbypart}{Q_i\left(x\right)}
                      1, \mathspace \text{when } x = a_i \\
                      0, \mathspace \text{when } x = a_j \text{ for } j \neq i
                  \end{functionbypart}

                  To make such a polynomial, we notice that the following almost has the right property, it is equal to 0 when $x = a_j$ for $j\neq i$, but it is not always equal to 1 when $x = a_i$:
                  \[\prod_{j \neq i}^{} \left(x - a_j\right)\]

                  So, to get 1 when $x = a_i$, we only need to normalise it:
                  \[Q_i\left(x\right) = \frac{\prod_{j \neq i}^{} \left(x - a_j\right)}{\prod_{j\neq i}^{} \left(a_i - a_j\right)}\]

                  As required, this is of degree less than or equal to $k-1$.

            \item Second, we can pick:
                  \[p\left(x\right) = \sum_{i=1}^{k} y_i Q_i\left(x\right)\]

                  This does indeed give us that:
                  \[p\left(a_j\right) = \sum_{i=1}^{k} y_i Q_i\left(x\right) = 0 + \ldots + 0 + y_j \cdot 1 + 0 + \ldots + 0 = y_j\]

                  Also, adding polynomials does not change their degrees, so we do get that $\deg\left(p\right) \leq k -1$.

        \end{enumerate}

        This technique always works and gives us a polynomial of degree at most $k - 1$. This is the best polynomial if we want to minimise the degree.
    }
}

\parag{Definition: Roots of a polynomial}{
    The \important{roots} of a polynomial $p\left(x\right)$ over $\mathbb{F}_q$ are numbers $b \in \mathbb{F}_q$ such that:
    \[p\left(b\right) = 0\]
}

\parag{Fundamental theorem of algebra}{
    Let $p\left(x\right)$ be a polynomial over $\mathbb{F}$ of degree at most $m-1$, meaning $\deg\left(p\left(x\right)\right) \leq m -1$.

    If this is not the all-zero polynomial ($\exists x \in \mathbb{F} : p\left(x\right) \neq 0$), then the number of its distinct roots is at most $m - 1$.
}

\subsection{Reed-Solomon codes}

\parag{Reed-Solomon codes}{
    The Reed-Solomon codes (RS codes) are constructed using the following algorithm:
    \begin{enumerate}
        \item We choose a finite field $\mathbb{F} = \mathbb{F}_q$ (where $q$ is a prime power).
        \item We choose $n$ and $k$ such that:
              \[0 < k \leq n \leq q\]
        \item We choose $n$ \textit{distinct} elements:
              \[a_1, \ldots, a_n \in \mathbb{F}\]
        \item The codeword associated to the information word $\bvec{u} \in \mathbb{F}^k$ is given by the following mapping:
              \[\bvec{u} \mapsto \bvec{c} = \left(p_{\bvec{u}}\left(a_1\right), \ldots, p_{\bvec{u}}\left(a_n\right)\right) \in \mathbb{F}\]
    \end{enumerate}

    We notice that $p_{\bvec{u}}\left(x\right)$ is a polynomial of degree $\deg\left(p_{\bvec{u}}\left(x\right)\right) \leq k - 1$.
}

\parag{Properties}{
    \begin{enumerate}[left=0pt]
        \item RS codes are linear.
        \item RS codes have dimension $k$, meaning they have $q^k$ different codewords.
        \item RS codes are MDS, meaning $d_{min}\left(\mathcal{C}\right) = n - k + 1$.
    \end{enumerate}
}

\end{document}
