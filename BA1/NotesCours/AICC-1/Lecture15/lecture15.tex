\documentclass[a4paper]{article}

% Expanded on 2021-11-09 at 08:35:26.

\usepackage{../../style}

\title{AICC 1}
\author{Joachim Favre}
\date{Mardi 09 novembre 2021}

\begin{document}
\maketitle

\lecture{15}{2021-11-09}{P = NP and induction}{
}

\parag{Summary}{
    We have seen the following types of algorithm:

    \begin{description}[left=0pt]
        \item[Tractable problem:] There exists a polynomial time algorithm to solve this problem. These problems are said to belong to the \fullbf{Class P}.
        \item[Class NP:] The solution can be checked in polynomial time, and any algorithm can be solved using exponential time. We wonder if it is possible to solve all NP problems using polynomial time ($P = NP$ problem).
        \item[NP Complete Class:] It is a subset of the NP class. If someone finds a polynomial-complexity algorithm for a member of the class, then it can be used to solve all the problems in the NP class.
        \item[Intractable Problems:] There does not exist a polynomial time algorithm to solve this problem.
        \item[Unsolvable Problem] There does not exist any algorithm to solve the problem (for example, the Halting Problem).
    \end{description}
}

\section{Induction and recursion}
\subsection{Mathematical induction}

\parag{Principle of mathematical induction}{
    Mathematical induction is a tool that allows us to prove that a proposition is true for all integers.

    To do that, we begin with a \important{basis step} by proving that $P\left(1\right)$ is true.

    Then, we make an \important{inductive step}; we assume that $P\left(k\right)$ holds for an arbitrary integer $k$ (\important{inductive hypothesis}), and we prove the proposition for $P\left(k+1\right)$.

    After having done both those steps, we can conclude that the property is true for all $n$. What we have done is proving that all dominos fall, by proving that the first one fall, and that if one falls then the next one falls as well.
}

\parag{Rule of inference}{
    Mathematical induction can be expressed as the following rule of inference:
    \[\left(P\left(1\right) \land \forall k\left(P\left(k\right) \to P\left(k+1\right)\right)\right) \to \forall n P\left(n\right)\]
    where the domain is the set of positive integers.
}

\parag{Important points}{
    \begin{itemize}[left=0pt]
        \item  In a proof by mathematical induction, we do not assume that $P\left(k\right)$ is true for all positive integers, only for some $k$. We then want to prove that $P\left(k+1\right)$ is true.
    \end{itemize}
}

\parag{Validity}{
    Mathematical induction is equivalent to the well ordering property.
}

\parag{Strong induction}{
    To prove that $P\left(n\right)$ is true for all positive integers $n$, where $P\left(n\right)$ is a propositional function, we complete two steps. First, we show that $P\left(1\right)$ is true (nothing changed here). Then, we show that the following implication is true for all positive integers $k$:
    \[P\left(1\right) \land P\left(2\right) \land \ldots \land P\left(k\right) \to P\left(k+1\right)\]

    In other words, we prove that all dominos fall by proving that the first one falls, and that if all the ones behind a domino fall, then the latter falls as well.

    Note that we can always use strong induction instead of mathematical induction, but there is no reason to use it if it is simpler to use mathematical induction. In fact, the principles of mathematical induction, strong induction, and the well-ordering property are equivalent.
}

\subsection{Recursively defined functions}
\parag{Definition}{
    A \important{recursive} or \important{inductive} definition of a function $f$, which has a domain of non-negative integers, consists in two step. First, we define the \important{basis step}: we specify the value of the function at zero. Then, we define the \important{recursive step}: we give a rule for finding the function's value at an integer from its values at smaller integers.

    Note that a function $f\left(n\right)$ over natural numbers is a sequence $a_0, a_1, \ldots$ where $f\left(i\right) = a_i$.
}

\end{document}
