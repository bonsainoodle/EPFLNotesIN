\documentclass[a4paper]{article}

% Expanded on 2022-05-23 at 10:23:49.

\usepackage{../../style}

\title{Analyse 2}
\author{Joachim Favre}
\date{Lundi 23 mai 2022}

\begin{document}
\maketitle

\lecture{24}{2022-05-23}{Changements de variables}{
}

\parag{Théorème de Fubini pour les intégrales triples}{
    Soient:
    \begin{itemize}
        \item Un intervalle $\left[a, b\right]$, où $a < b$.
        \item Deux fonctions $\phi_1, \phi_2 : \left[a, b\right] \mapsto \mathbb{R}$ continues telles que $\phi_1\left(x\right) < \phi_2\left(x\right)$ pour tout $x \in \left]a, b\right[ $
        \item L'ensemble défini par:
              \[D = \left\{\left(x, y\right) \in \mathbb{R}^2 : a < x < b, \phi_1\left(x\right) < y < \phi_2\left(x\right)\right\}\]

        \item Deux fonctions $G, H : \bar{D} \mapsto \mathbb{R}$ continues telles que $G\left(x, y\right) < H\left(x, y\right)$ pour tout $\left(x, y\right) \in D$.
        \item L'ensemble défini par:
              \[E = \left\{\left(x, y, z\right) \in \mathbb{R}^3 : \left(x, y\right) \in D : G\left(x, y\right) < z < H\left(x, y\right)\right\}\]
        \item Une fonction $f: \bar{E} \mapsto \mathbb{R}$.
    \end{itemize}

    Alors, $f$ est intégrable sur $E$, et on a:
    \[\int_{E} f\left(x, y, z\right)dxdydz = \int_{a}^{b} \left(\int_{\phi_1\left(x\right)}^{\phi_2\left(x\right)} \left(\int_{G\left(x, y\right)}^{H\left(x, y\right)} f\left(x, y, z\right)dz\right)dy\right)dx\]

    Nous ne pouvons pas choisir l'ordre d'intégration.

    \subparag{Notation}{
        Pour simplifier la notation d'intégrales multiples, nous pouvons écrire:
        \begin{multiequality}
            & \int_{a}^{b} \left(\int_{\phi_1\left(x\right)}^{\phi_2\left(x\right)} \left(\int_{G\left(x, y\right)}^{H\left(x, y\right)} f\left(x, y, z\right)dz\right)dy\right)dx  \\
            \over{=}{not.}\ & \int_{a}^{b} dx \int_{\phi_1\left(x\right)}^{\phi_2\left(x\right)} dy \int_{G\left(x, y\right)}^{H\left(x, y\right)} f\left(x, y, z\right)dz
        \end{multiequality}
    }
}

\subsection[Changement de variables]{Changement de variables dans une intégrale multiple}
\parag{Théorème}{
    Soit $E \subset \mathbb{R}^n$ un sous-ensemble tel que $\bar{E}$ est compact. Soit aussi $\psi : E \mapsto \mathbb{R}^n$ telle que $\psi \in C^1\left(E\right)$ et $\psi: E \mapsto \psi\left(E\right)$ est bijective (ce qui est équivalent à $J_{\psi}\left(\bvec{u}\right)$ est inversible pour tout $\bvec{u} \in E$, comme vu précédemment). Soit finalement $f: \bar{D} \mapsto \bar{\psi\left(E\right)} \mapsto \mathbb{R}$ une fonction continue.

    Alors:
    \[\int_{D} f\left(\bvec{x}\right)d \bvec{x} = \int_{E} f\left(\psi\left(\bvec{u}\right)\right) \left|\det\left(J_{\psi}\left(\bvec{u}\right)\right)\right|d \bvec{u}\]

    \imagehere[0.5]{TheoremeChangementDeVariablesIntegralesMultiples.png}
}


\parag{Application: Changement de variables polaire}{
Par définition, le changement de variable polaire nous donne:
\[\psi : \mathbb{R}_+ \times \left[0, 2\pi\right[ \mapsto \mathbb{R}^2 \setminus \left\{0\right\}, \mathspace \psi\left(r, \phi\right) = \left(r\cos\left(\phi\right), r\sin\left(\phi\right)\right)\]

Calculons son déterminant Jacobien:
\[J_{\psi}\left(r, \phi\right) = \begin{pmatrix} \cos\left(\phi\right) & -r\sin\left(\phi\right) \\ \sin\left(\phi\right) & r\cos\left(\phi\right) \end{pmatrix} \implies \det\left(J_{\psi}\left(r, \phi\right)\right) = r\left(\cos^2\left(\phi\right) + \sin^2\left(\phi\right)\right) = r\]

Elle est donc bien bijective lorsque $r \neq 0$.

Si nous avons un domaine circulaire dans $\mathbb{R}^2$, il est souvent une bonne idée de faire un changement de variable polaire.
}


\end{document}
