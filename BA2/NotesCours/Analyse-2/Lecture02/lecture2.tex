\documentclass[a4paper]{article}

% Expanded on 2022-02-23 at 12:51:27.

\usepackage{../../style}

\title{Analyse 2}
\author{Joachim Favre}
\date{Mercredi 23 février 2022}

\begin{document}
\maketitle

\lecture{2}{2022-02-23}{D'autres équations avec des différences, facile!}{
}

\parag{Résumé pour les EDVS}{
    Pour résoudre une EDVS, donc une équation sous la forme $f\left(y\right) y' = g\left(x\right)$ où $f: I \mapsto \mathbb{R}$ et $g: J \mapsto \mathbb{R}$ sont continues, on pose l'équation:
    \[\int f\left(y\right) dy = \int g\left(x\right) dx\]

    Puisqu'on a une constante des deux côtés, il nous suffit de prendre une seule constante.
}

\parag{Définition: Solution maximale}{
    Une \important{solution maximale} d'une EDVS avec la condition initiale $y\left(x_0\right) = b_0$ où $x_0 \in J$ et $b_0 \in I$ est une fonction $y\left(x\right)$ de classe $C^1$ satisfaisant l'équation, la condition initiale, et qui est définie sur le plus grand intervalle possible.

    Le théorème des EDVS nous dit que si $f\left(y\right) \neq 0$ sur $I$, alors il existe une unique solution maximale. Toute solution avec la même condition initiale est une restriction de la solution maximale.
}

\subsection[EDL1]{Équations différentielles linéaires du premier ordre}

\parag{Définition: EDL1}{
    Soit $I \subset \mathbb{R}$ un intervalle ouvert. Nous appelons \important{équation différentielle linéaire du premier ordre} (EDL1) une équation de la forme suivante:
    \[y'\left(x\right) + p\left(x\right) y\left(x\right) = f\left(x\right)\]
    où $p, f : I \mapsto \mathbb{R}$ sont continues.


    Une \important{solution} est une fonction $y : I \mapsto \mathbb{R}$ de classe $C^1$ satisfaisant l'équation.
}

\parag{Équation homogène associée}{
    Commençons par considérer l'équation suivante (qui est plus facile):
    \[y'\left(x\right) + p\left(x\right) y\left(x\right) = 0\]

    Cette équation s'appelle \important{l'équation homogène associée} à l'EDL1 $y'\left(x\right) + p\left(x\right)y\left(x\right) = f\left(x\right)$.

    Nous avons deux cas. Soit $y\left(x\right) = 0\ \forall x \in \mathbb{R}$, soit $\frac{y'\left(x\right)}{y\left(x\right)} = -p\left(x\right)$ (qui est une EDVS).

    Continuous à travailler sur le deuxième cas:
    \[\int \frac{dy}{y} = -\int p\left(x\right)dx \implies \log\left|y\right| = -P\left(x\right) + C_1, \mathspace C_1 \in \mathbb{R}\]
    où $P\left(x\right)$ est une primitive de $p\left(x\right)$.

    Ainsi:
    \[\left|y\right| = e^{-P\left(x\right) + C_1} = \underbrace{e^{C_1}}_{C_2 > 0} e^{-P\left(x\right)} \implies y\left(x\right) = C_2 e^{-P\left(x\right)}, \mathspace C_2 \in \mathbb{R}^* \]

    Cependant, puisque $y\left(x\right) = 0$ est aussi une solution, on obtient que la solution générale de l'équation homogène associée sur $I \subset \mathbb{R}$ est:
    \[y\left(x\right) = Ce^{-P\left(x\right)}, \mathspace \forall x \in I, \forall C \in \mathbb{R}\]
}

\parag{Principe de superposition de solutions}{
    Soit $I \subset \mathbb{R}$ un intervalle ouvert, et $p, f_1, f_2 : I \mapsto \mathbb{R}$ des fonction continues.

    Supposons que $v_1 : I \mapsto \mathbb{R}$ de classe $C^1$ est une solution de l'équation:
    \[y' + p\left(x\right) y = f_1\left(x\right)\]

    Supposons aussi que $v_2 : I \mapsto \mathbb{R}$ de classe $C^1$ est une solution de l'équation:
    \[y' + p\left(x\right) y = f_2\left(x\right)\]

    Alors, pour tout couple $C_1, C_2 \in\mathbb{R}$, la fonction $v\left(x\right) =  C_1 v_1\left(x\right) + C_2 v_2\left(x\right)$ est une solution de l'équation:
    \[y' + p\left(x\right)y = C_1 f_1\left(x\right) + C_2 f_2\left(x\right)\]
}

\parag{Méthode de la variation de constante}{
Nous cherchons une solution particulière de $y'\left(x\right) + p\left(x\right)y\left(x\right) = f\left(x\right)$ où $p, f : I\mapsto \mathbb{R}$ sont des fonctions continues, sous la forme suivante:
\[v\left(x\right) = C\left(x\right)e^{-P\left(x\right)}\]
où $P\left(x\right)$ est une primitive de $p\left(x\right)$ sur $I$ et $C\left(x\right)$ est une fonction inconnue de $x$.

Nous appelons ceci un \textit{Ansatz}. On suppose que notre solution est d'une certaine forme et on espère que ça nous amène à une solution (en l'occurrence, on prend une solution similaire à celle qu'on avait trouvée pour les équations homogènes associées).

Nous avons donc trouvé une solution particulière de l'équation, qui est:
\[v\left(x\right) = \left(\int f\left(x\right) e^{P\left(x\right)} dx\right) e^{-P\left(x\right)}\]
où $P\left(x\right)$ est une primitive de $p\left(x\right)$ sur $I$.
}

\parag{Proposition pour les EDL1}{
    Soient $p, f : I \mapsto \mathbb{R}$ des fonctions continues. Supposons que $v_0 : I \mapsto \mathbb{R}$ est une solution particulière de l'équation suivante:
    \[y'\left(x\right) + p\left(x\right) y\left(x\right) = f\left(x\right)\]

    Alors, la solution générale de cette équation est:
    \[v\left(x\right) = v_0\left(x\right) + Ce^{-P\left(x\right)}, \mathspace \forall C \in \mathbb{R}\]
    où $P\left(x\right)$ est une primitive de $p\left(x\right)$ sur $I$.
}

\parag{Résumé}{
    En mettant tout en commun, on obtient que la solution générale d'une EDL1 est:
    \[y\left(x\right) = Ce^{-P\left(x\right)} + \left(\int f\left(x\right) e^{P\left(x\right)} dx\right) e^{-P\left(x\right)}, \mathspace \forall C \in \mathbb{R}, x \in I\]
}

\end{document}
