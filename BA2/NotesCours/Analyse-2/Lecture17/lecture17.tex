\documentclass[a4paper]{article}

% Expanded on 2022-04-25 at 10:46:46.

\usepackage{../../style}

\title{Analyse 2}
\author{Joachim Favre}
\date{Lundi 25 avril 2022}

\begin{document}
\maketitle

\lecture{17}{2022-04-25}{Méthode de physicien}{
}

\subsection[Dérivées en coordonnées polaires]{Application du gradient et du Laplacien en coordonnées polaires}
\parag{Définition: Laplacien}{
Soit $E \subset \mathbb{R}^n$ un ensemble, et soit $f : E \mapsto \mathbb{R}$ de classe $C^2$ sur $E$.

La fonction $\Delta f : E \mapsto \mathbb{R}$ suivante est le \important{Laplacien} de $f$:
\[\Delta f\left(x_1, \ldots, x_n\right) = \frac{\partial^{2} f}{\partial x_1^{2}} + \ldots + \frac{\partial^{2} f}{\partial x_n^{2}}\]

\subparag{L'opérateur Nabla}{
    Définissons la Nabla de la manière suivante:
    \[\nabla = \left(\frac{\partial}{\partial x_1}, \ldots, \frac{\partial}{\partial x_n}\right)\]

    Notez que c'est un vecteur d'opérateurs, le Nabla ne contient pas des valeurs ou des fonctions, mais l'opérateur dérivée.
}
}

\parag{Proposition}{
    Soit $f : \mathbb{R}^2 \mapsto\mathbb{R}$ une fonction de classe $C^2$, et soit $\widetilde{f} = f \circ g\left(r, \phi\right)$, où $g$ est la fonction changement de variable vers les coordonnées polaires. Alors:
    \[\nabla f\left(x, y\right) = \left(\cos\left(\phi\right) \frac{\partial \widetilde{f}}{\partial r} - \frac{1}{r} \sin\left(\phi\right) \frac{\partial \widetilde{f}}{\partial \phi}, \sin\left(\phi\right) \frac{\partial \widetilde{f}}{\partial r} + \frac{1}{r} \cos\left(\phi\right) \frac{\partial \widetilde{f}}{\partial \phi}\right)\]

    \[\Delta f\left(x, y\right) = \frac{\partial^{2} \widetilde{f}}{\partial r^{2}} + \frac{1}{r^2} \frac{\partial^{2} \widetilde{f}}{\partial \phi^{2}} + \frac{1}{r} \frac{\partial \widetilde{f}}{\partial r}\]

    Ce qui est exactement ce qui est exactement le résultat attendu. Il est naturellement possible de faire exactement le même raisonnement pour la dérivée partielle selon $y$ et pour les dérivées secondes.
}

\parag{Définition: Fonctions harmoniques}{
    Une fonction telle que $\Delta f = 0$ sur $E \subset \mathbb{R}^2$ s'appelle \important{harmonique}.
}

\parag{Proposition}{
    Une fonction harmonique sur un domaine compact (borné et fermé) atteint son minimum et son maximum sur la frontière du domaine.

    \subparag{Remarque}{
        Ceci implique qu'une fonction harmonique sur un domaine compact est telle que, pour n'importe quel sous-ensemble compact, elle atteint son minimum et son maximum sur la frontière.
    }
}

\subsection{Formule de Taylor}
\parag{Théorème}{
Soit $E \subset \mathbb{R}^n$, et soit $f : E \mapsto \mathbb{R}$ une fonction de classe $C^{p+1}$ au voisinage de $\bvec{a} \in E$.

Alors, il existe $\delta > 0$ tel que, pour tout $\bvec{x} \in B\left(\bvec{a}, \delta\right) \cap E$, il existe $0 < \theta < 1$ tel que:
\[f\left(\bvec{x}\right) = F\left(0\right) + F'\left(0\right) + \ldots + \frac{1}{p!} F^{\left(p\right)}\left(0\right) + \frac{1}{\left(p+1\right)!}F^{\left(p+1\right)}\left(\theta\right)\]
où $F: I \mapsto \mathbb{R}$, avec $I \subset \left[0, 1\right]$, est définie par $F\left(t\right) = f\left(\bvec{a} + t\left(\bvec{x} - \bvec{a}\right)\right)$.

\subparag{Terminologie}{
    Ce théorème nous dit que nous pouvons écrire:
    \[f\left(\bvec{x}\right) = F\left(0\right) + F'\left(0\right) + \ldots + \frac{1}{p!} F^{\left(p\right)}\left(0\right) + \text{ le reste}\]

    Nous l'appelons le \important{polynôme de Taylor} de $f$ d'ordre $p$ au point $\bvec{a}$.
}
}

\parag{Cas où $n = 2$}{
    Nous pouvons démontrer par récurrence que:
    \[F^{\left(p\right)}\left(0\right) = \sum_{k=0}^{p} \frac{\partial^p f}{\partial x^k \partial y^{p - k}} C_p^k \left(x - a\right)^{k} \left(y - b\right)^{p - k}\]

    Souvent, on utilise l'approximation de Taylor d'ordre 2, donnée par:
    \begin{multiequality}
        f\left(x, y\right) =\ & f\left(a, b\right) + \frac{\partial f}{\partial x}\left(a, b\right)\left(x - a\right) + \frac{\partial f}{\partial y}\left(a, b\right)\left(y - b\right)  \\
        & + \frac{1}{2}\left(\frac{\partial^2 f}{\partial x^2}\left(a, b\right)\left(x - a\right)^2 + 2 \frac{\partial^2 f}{\partial x \partial y}\left(a, b\right)\left(x - a\right)\left(y- b\right) + \frac{\partial^2 f}{\partial y^2}\left(a, b\right)\left(y - b\right)^2\right) \\
        & + \epsilon\left(\left\|\left(x, y\right) - \left(a, b\right)\right\|^2\right)
    \end{multiequality}
    où les deux premières lignes sont $P_{2,f,\left(a, b\right)}$, le polynôme de Taylor de $f$ d'ordre 2 autour de $\left(a, b\right)$, et la troisième ligne est le reste.

    \subparag{Remarque personnelle}{
        En manipulant les opérateurs comme de physiciens, il est tentant d'écrire:
        \[F^{\left(p\right)}\left(0\right) = \left(\left(x - a\right)\frac{\partial}{\partial x} + \left(y - b\right) \frac{\partial}{\partial y}\right)^p f\]
    }
}

\end{document}
