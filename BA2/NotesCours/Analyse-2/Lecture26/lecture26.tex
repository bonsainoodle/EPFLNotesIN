\documentclass[a4paper]{article}

% Expanded on 2022-05-24 at 14:18:22.

\usepackage{../../style}

\title{Analyse 2}
\author{Joachim Favre}
\date{Lundi 30  mai 2022}

\begin{document}
\maketitle

\lecture{26}{2022-05-30}{Toutes les bonnes choses ont une fin}{
}

\parag{Coordonnées cylindriques}{
Nous définissons le changement de variables en coordonnées cylindriques par:
\begin{functionbypart}{G\left(r, \phi, z\right)}
    x = r\cos\left(\phi\right) \\
    y = r\sin\left(\phi\right) \\
    z = z
\end{functionbypart}
où $G : \left[0, +\infty\right[ \times \left[0, 2\pi\right[ \times \mathbb{R} \mapsto \mathbb{R}^3$.

Nous pouvons faire le schéma suivant:
\imagehere[0.4]{ChangementDeVariableCylindrique.png}

Calculons maintenant la matrice Jacobienne:
\begin{multiequation}
    & J_{G}\left(r, \phi, z\right) = \begin{pmatrix} \cos\left(\phi\right) & -r\sin\left(\phi\right) & 0 \\ \sin\left(\phi\right) & r\cos\left(\phi\right) & 0 \\ 0 & 0 & 1 \end{pmatrix}  \\
    \implies & \det J_{G}\left(r, \phi, z\right) = 1 \begin{vmatrix} \cos\left(\phi\right) & -r\sin\left(\phi\right) \\ \sin\left(\phi\right) & r\cos\left(\phi\right) \end{vmatrix} = r\left(\cos^2\left(\phi\right) + \sin^2\left(\phi\right)\right) = r
\end{multiequation}
}

\parag{Remarque}{
Dans les changements de variables sphériques et cylindriques, la coordonnée $z$ joue un rôle spéciale (son expression est plus simple que les deux autres). Cependant, selon la géométrie du domaine et selon la fonction donnée, nous pouvons choisir une autre coordonnée cartésienne pour avoir cette forme spéciale. Par exemple, pour un changement de variable sphérique avec $y$ ayant la forme spéciale, nous aurions:
\begin{functionbypart}{G_{sph}}
    x = r\sin\left(\theta\right) \cos\left(\phi\right) \\
    z = r\sin\left(\theta\right) \sin\left(\phi\right) \\
    y = r \cos\left(\theta\right)
\end{functionbypart}

où $G_{sph} : \left[0, +\infty\right[ \times \left[0, \pi\right] \times \left[0, 2\pi\right]  \mapsto \mathbb{R}^3$ comme d'habitude, et:
\[\left|J_{G_{sph}}\right| = \left|- J_{\widetilde{G}}\right| = r^2 \sin\left(\theta\right)\]
puisqu'échanger deux lignes d'une matrice correspond à multiplier son déterminant par $-1$, mais nous ne considérons que la valeur absolue de notre déterminant.

Nous pouvons bien sûr faire la même chose avec un changement de variable cylindrique. Par exemple:
\begin{functionbypart}{G_{cyl}}
    x = r\cos\left(\phi\right) \\
    z = r \sin\left(\phi\right) \\
    y = y
\end{functionbypart}
où $G_{cyl} = \left[0, +\infty\right[ \times \left[0, 2\pi\right[ \times \mathbb{R} \mapsto\mathbb{R}^3$, et:
\[\left|J_{G_{cyl}}\right| = \left|-r\right| = r\]
pour la même raison.

\imagehere[0.8]{ChangementDeVariableSpheriqueCylindriqueRotation.png}

\subparag{Observation}{
    Une autre manière de voir ceci, est que nous pouvons faire un changement de variable afin de réordrer nos variables:
    \begin{functionbypart}{G\left(\widetilde{x}, \widetilde{y}, \widetilde{z}\right)}
        x = \widetilde{x}  \\
        y = \widetilde{z}  \\
        z = \widetilde{y}
    \end{functionbypart}

    Clairement, $\det J_G = -1$, donc nous avons bien $\left|\det J_G\right| = 1$, ce qui nous permet en effet de réordrer nos variables sans conséquence (sans avoir à multiplier par quoi que ce soit).
}
}

\parag{Résumé des changements de variables}{
    Nous avons vu les changements de variables remarquables suivants:
    \[G_{polaire}\left(r, \phi\right) = \begin{systemofequations} x = r\cos\left(\phi\right) \\ y = r\sin\left(\phi\right) \end{systemofequations} \implies \left|\det\left(J_{G}\right)\right| = r\]
    \[G_{spherique}\left(r, \theta, \phi\right) = \begin{systemofequations} x = r\sin\left(\theta\right)\cos\left(\phi\right) \\ y = r\sin\left(\theta\right) \sin\left(\phi\right) \\ z = r\cos\left(\theta\right) \end{systemofequations} \implies \left|\det\left(J_{G}\right)\right| = r^2 \sin\left(\theta\right)\]
    \[G_{cylindrique}\left(r, \phi, z\right) = \begin{systemofequations} x = r\cos\left(\phi\right) \\y = r\sin\left(\phi\right) \\ z = z \end{systemofequations} \implies \left|\det\left(J_{G}\right)\right| = r\]

    Avec:
    \[r > 0, \mathspace 0 \leq \theta \leq \pi, \mathspace 0 \leq \phi < 2\pi\]
}


\end{document}
