\documentclass[a4paper]{article}

% Expanded on 2022-05-05 at 10:33:54.

\usepackage{../../style}

\title{Analyse 2}
\author{Joachim Favre}
\date{Mercredi 04 mai 2022}

\begin{document}
\maketitle

\lecture{20}{2022-05-04}{Fonctions implicites}{
}

\parag{Définition: Surface de niveau}{
    Une \important{surface de niveau} d'une fonction $F\left(x, y, z\right)$ est la surface définie par l'équation $F\left(x, y, z\right) = C \in \mathbb{R}$. $C$ s'appelle le \important{niveau}.

    De manière similaire, une \textcolor{red}{ligne} de niveau d'une fonction ${\color{red}F\left(x, y\right)}$ est la \textcolor{red}{ligne} définie par l'équation ${\color{red}F\left(x, y\right)} = C \in \mathbb{R}$. $C$ s'appelle le niveau.
}

\parag{Théorème des fonctions implicites (TFI)}{
    Soit $n \geq 2$ et $E \subset \mathbb{R}^n$. Soit aussi $F: E \mapsto \mathbb{R}$ une fonction de classe $C^1$ au voisinage de $\bvec{a} = \left(a_1, \ldots, a_n\right) \in E$ telle que:
    \begin{enumerate}
        \item $F\left(\bvec{a}\right) = 0$
        \item $\frac{\partial F}{\partial x_n}\left(\bvec{a}\right)\neq 0$
    \end{enumerate}

    Alors, il existe un voisinage $B\left(\bcheck{a}, \delta\right)$ de $\bcheck{a} = \left(a_1, \ldots, a_{n-1}\right) \in \mathbb{R}^{n-1}$ (remarquez que $\bcheck{a}$ a $n-1$ composante et non pas $n$) et une fonction $f: B\left(\bcheck{a}, \delta\right) \mapsto\mathbb{R}$ telle que:
    \begin{enumerate}
        \item $a_n = f\left(a_1, \ldots, a_{n-1}\right)$
        \item $F\left(x_1, \ldots, x_{n-1}, f\left(x_1, \ldots, x_{n-1}\right)\right) = 0, \mathspace \forall \left(x_1, \ldots, x_{n-1}\right) \in B\left(\bcheck{a}, \delta\right)$
        \item $f$ est de classe $C^1$ dans un voisinage de $\bcheck{a}$, et on a:
              \[\frac{\partial f}{\partial x_p}\left(x_1, \ldots, x_{n-1}\right) = - \frac{\frac{\partial F}{\partial x_p}\left(x_1, \ldots, x_{n-1}, f\left(x_1, \ldots, x_{n-1}\right)\right)}{\frac{\partial F}{\partial x_n}\left(x_1, \ldots, x_{n-1}, f\left(x_1, \ldots, x_{n-1}\right)\right)}, \mathspace\forall p = 1, \ldots, n-1\]

    \end{enumerate}
}

\parag{Application: Équation de l'hyperplan tangent}{
    Reconstruisons la formule pour trouver un hyperplan tangent.

    Soit $F\left(x_1, \ldots, x_n\right)$ une fonction de classe $C^1$ sur $E \subset \mathbb{R}^n$ telle qu'il existe un $i$, où $1 \leq i \leq n$, tel que, pour un $\bvec{a} \in E$, nous avons $F\left(\bvec{a}\right) = 0$ et:
    \[\frac{\partial F}{\partial x_i}\left(\bvec{a}\right) \neq 0\]

    Par le TFI, nous savons que l'équation $F\left(x_1, \ldots, x_n\right) = 0$ définit une hypersurface $x_i = f\left(a_1, \ldots, a_{i-1}, a_{i+1}, \ldots, a_n\right)$ ($f$ ne contient pas $a_i$ dans son paramètre) qui est de classe $C^1$.

    Or, nous savons que $\exists i$ pour lequel $\frac{\partial F}{\partial x_i}\left(\bvec{a}\right) \neq 0$, est équivalent à $\nabla F\left(\bvec{a}\right) \neq 0$. Ceci implique que $DF\left(\bvec{a}, \bvec{v}\right) = \left<\nabla F\left(\bvec{a}\right), \bvec{v}\right> = 0$ si et seulement si $\bvec{v}$ est tangent à l'hypersurface de niveau. En d'autres mots, pour tout vecteur $\bvec{v}$ dans l'hyperplan tangent à $F\left(\bvec{x}\right) = 0$ au point $\bvec{x} = \bvec{a}$, nous devons avoir:
    \[DF\left(\bvec{a}, \bvec{v}\right) = \left<\underbrace{\nabla F\left(\bvec{a}\right)}_{\neq \bvec{0}}, \underbrace{\bvec{v}}_{\bvec{x} - \bvec{a}}\right> = 0 \]

    L'équation de l'hyperplan tangent à $F\left(\bvec{x}\right) = 0$ au point $\bvec{a}$ tel que $F\left(\bvec{a}\right) = 0$ est donc:
    \[\left<\nabla F\left(\bvec{a}\right), \bvec{x} - \bvec{a}\right> = 0\]
}

\parag{Remarque}{
    Considérons le cas $n = 3$, et faisons le lien avec l'équation du plan tangent au graphique de $z = f\left(x, y\right)$.

    Si $F\left(x, y, z\right) = z - f\left(x, y\right)$, où $f$ est une fonction de classe $C^1$, alors nous avons $\frac{\partial f}{\partial z} = 1$, et donc $\nabla F\left(x, y, z\right) \neq 0$ pour tout $\left(x, y, z\right)$ où $z = f\left(x, y\right)$ est bien définie.

    Ainsi, si $c = f\left(a, b\right)$, c'est à dire si $a, b, c$ appartient à la surface de niveau $F\left(x, y, z\right) = 0$, on trouve par le TFI que l'équation du plan tangent au point $\left(a, b, c\right)$ est:
    \begin{multiequation}
        & \left<\nabla F\left(a, b, c\right), \left(x - a, y - b, z- c\right)\right> = 0 \\
        \iff & \underbrace{\frac{\partial F}{\partial x}\left(a, b, c\right)\left(x - a\right)}_{-\frac{\partial f}{\partial x}\left(a, b\right)} + \underbrace{\frac{\partial F}{\partial y}\left(a, b, c\right)}_{- \frac{\partial f}{\partial y}\left(a, b\right)} + \underbrace{\frac{\partial F}{\partial z}\left(a, b, c\right)}_{1}\left(z - \underbrace{c}_{f\left(a, b\right)}\right) = 0
    \end{multiequation}

    Nous retrouvons donc l'équation:
    \begin{multiequation}
        & -\frac{\partial f}{\partial x}\left(a, b\right)\left(x - a\right) - \frac{\partial f}{\partial y}\left(a, b\right)\left(y - b\right) + \left(z - f\left(a, b\right)\right) = 0  \\
        \iff & z = f\left(a, b\right) + \left<\nabla f\left(a, b\right), \left(x - a, y- b\right)\right>
    \end{multiequation}
}

\end{document}
