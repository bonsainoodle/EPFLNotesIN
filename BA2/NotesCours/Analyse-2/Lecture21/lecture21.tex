\documentclass[a4paper]{article}

% Expanded on 2022-05-11 at 12:14:47.

\usepackage{../../style}

\title{Analyse 2}
\author{Joachim Favre}
\date{Mercredi 11 mai 2022}

\begin{document}
\maketitle

\lecture{21}{2022-05-11}{Lagrange à Laferme avec Lescochons}{
}

\subsection[Multiplicateurs de Lagrange]{Extrema liés --- Méthode des multiplicateurs de Lagrange}

\parag{Théorème: Condition nécessaire pour un extremum sous contrainte quand $n=2$}{
    Soit l'ensemble $E \subset \mathbb{R}^2$ et soient les fonctions $f, g : E \mapsto \mathbb{R}$ de classe $C^1$. Supposons que $f\left(x, y\right)$ admette un extremum en $\left(a, b\right) \in E$ sous la contrainte $g\left(x, y\right) = 0$, et que $\nabla g\left(a, b\right) \neq \bvec{0}$.

    Alors, il existe $\lambda \in \mathbb{R}$, appelé le \important{multiplicateur de Lagrange}, tel que:
    \[\nabla f\left(a, b\right) = \lambda \nabla g\left(a, b\right)\]
}

\parag{Théorème: Condition nécessaire pour un extremum sous contrainte}{
    Soit $E \subset \mathbb{R}^n$ et soient $f, g_1, \ldots, g_m E \mapsto \mathbb{R}$ des fonctions de classe $C^1$, où $m \leq n-1$. Soit $\bvec{a} \in E$ un extremum de $f\left(\bvec{x}\right)$ sous les contraintes $g_1\left(\bvec{x}\right) = \ldots = g_m\left(\bvec{x}\right) = 0$.

    Supposons que les vecteurs $\nabla g_1\left(\bvec{a}\right), \ldots, \nabla g_m\left(\bvec{a}\right)$ sont linéairement indépendants. Alors, il existe un vecteur $\bvec{\lambda} = \left(\lambda_1, \ldots, \lambda_m\right) \in \mathbb{R}^m$ tel que:
    \[\nabla f\left(\bvec{a}\right) = \sum_{k=1}^{m} \lambda_k \nabla g_k\left(\bvec{a}\right) = \lambda_1 \nabla g_1\left(\bvec{a}\right) + \ldots + \lambda_m \nabla g_m\left(\bvec{a}\right)\]

    En particulier, si on cherche un extremum de $f\left(\bvec{x}\right)$ sous une seule contrainte $g\left(\bvec{x}\right) = 0$, on obtient les équations:
    \[\begin{systemofequations} \nabla f\left(\bvec{x}\right) = \lambda \nabla g\left(\bvec{x}\right) \\ g\left(\bvec{x}\right) = 0 \end{systemofequations} \mathspace \text{si } \nabla g\left(\bvec{x}\right) \neq 0\]
}

\end{document}
