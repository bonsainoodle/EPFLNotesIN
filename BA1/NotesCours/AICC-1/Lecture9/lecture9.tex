\documentclass[a4paper]{article}

% Expanded on 2021-10-19 at 08:39:04.

\usepackage{../../style}

\title{AICC-1}
\author{Joachim Favre}
\date{Mardi 19 octobre 2021}

\begin{document}
\maketitle

\lecture{9}{2021-10-19}{Equivalence relations and partial orderings}{
}

\subsection{Equivalence relations}

\parag{Definition: Equivalence}{
    A relation on a set $A$ is called an \important{equivalence relation} if it is reflexive, symmetric, and transitive.

    Two elements $a$ and $b$ are related by an equivalence relation are called \important{equivalent}, often written $a \sim b$.
}

\parag{Definition: Equivalence class}{
    Let $R$ be an equivalence relation on a set $A$. The set of all elements that are related to an element $a$ of $A$ is called the \important{equivalence class} of $a$, denoted $\left[a\right]_{R}$, or $\left[a\right]$.

    In other words:
    \[\left[a\right]_R = \left\{s \in A |\ \left(a, s\right) \in R\right\}\]

    If $b \in \left[a\right]_R$, then $b$ is called a \important{representative} of this equivalence class.
}

\parag{Theorem}{
    Let $R$ be an equivalence relation on a set $A$. These statements for elements $a$ and $b$ of $A$ are equivalent:
    \begin{enumerate}
        \item $R\left(a, b\right)$
        \item $\left[a\right] = \left[b\right]$
        \item $\left[a\right] \cap \left[b\right] \neq \o$
    \end{enumerate}
}

\parag{Definition: Partition of a set}{
    A \important{partition} of a set is a collection of disjoint nonempty subsets of $S$ that have $S$ as their union. More formally, the collection of subsets $A_i \subset S$ form a partition of $S$ if and only if:
    \[A_i \neq \o, \mathspace i \neq j \implies A_i \cap A_j = \o, \mathspace \bigcup_{i} A_i = S\]
}

\parag{Theorem}{
    Let $R$ be an equivalence relation on a set $S$.

    Then, the equivalence classes of $R$ form a partition of $S$. Conversely, given a partition $\left\{A_i\right\}$ of the set $S$, there is an equivalence relation $R$ that has the sets $A_i$ as its equivalence classes.

    \subparag{Other formulation}{
        Set partitions and equivalence classes are equivalent: we can construct one from another.
    }
}

\subsection{Partial orderings}
\parag{Definition}{
    A relation $R$ on a set $S$ is called a \important{partial ordering}, or \important{partial order}, if it is reflexive, antisymmetric, and transitive

    A set together with a partial ordering $R$ is called a \important{partially ordered set}, or \important{poset}, and is denoted by $\left(S, R\right)$.
}

\parag{Definition: Comparability}{
    The elements $a$ and $b$ of a poset $\left(S, \preccurlyeq\right)$ are \important{comparable} if either $a \preccurlyeq b$ or $b \preccurlyeq a$. When $a$ and $b$ are elements of $S$ so that neither $a \preccurlyeq b$ nor $b \preccurlyeq a$, then $a$ and $b$ are called \important{incomparable}.
}

\parag{Hasse diagrams}{
    We know that all partial orderings are reflexive, transitive and anti-symmetric.

    We can draw a directed graph (a). Then, we can omit self-loops (since they are all reflexive) (b). Finally, we can remove transitive edges, and assume that arrows point upwards (no arrow point in both direction since relations are anti-symmetric). The last diagram is a Hasse diagram.

    \imagehere[0.4]{HasseDiagramConstruction.png}
}

\parag{Definition}{
    If $\left(S, \preccurlyeq\right)$ is a poset and every two elements of $S$ are comparable, then $S$ is called a \important{totally ordered} or \important{linearly ordered set}, and $\preccurlyeq$ is called a \important{total order} or a \important{linear order}.

    We say that $\left(S, \preccurlyeq\right)$ is a \important{well-ordered} set if it is a poset such that $\preccurlyeq$ is a total ordering, and every nonempty subset of $S$ has a least element.
}

\parag{Definition}{
    Let $\left(S, \preccurlyeq\right)$ be a partially ordered set.

    An \important{upper bound} $u \in S$ of $A \subseteq S$ is an element such that:
    \[a \preccurlyeq u, \mathspace \forall a \in A\]

    A \important{lower bound} $u \in S$ of $A \subseteq S$ is an element such that:
    \[u \preccurlyeq a, \mathspace \forall a \in A\]

    A \important{least upper bound} $u \in S$ of $A \subseteq S$ is an upper bound of $A$ such that it is less than every other upper bound of $A$.

    A \important{greatest  lower bound} $u \in S$ of $A \subseteq S$ is a lower bound of $A$ such that it is greater than every other bound of $A$.
}

\end{document}
