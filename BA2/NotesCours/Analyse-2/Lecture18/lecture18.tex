\documentclass[a4paper]{article}

% Expanded on 2022-04-27 at 10:01:16.

\usepackage{../../style}

\title{Analyse 2}
\author{Joachim Favre}
\date{Mercredi 27 avril 2022}

\begin{document}
\maketitle

\lecture{18}{2022-04-27}{Vous savez toujours calculer des valeurs propres?}{
}

\parag{Remarque}{
    Il existe une autre méthode pour calculer les polynômes de Taylor, en utilisant les developments limités d'une seule variable.
}

\parag{Conjecture de Taylor en n dimensions}{
    Je vois un pattern avec la formule trouvée dans le cours précédent. Ainsi, je ne sais pas du tout si c'est vrai, mais je conjecture que nous avons, pour $n$ variables:
    \[F^{\left(p\right)}\left(0\right) = \left(\left(x_1 - a_1\right)\frac{\partial}{\partial x_1} + \ldots + \left(x_n - a_n\right) \frac{\partial}{\partial x_n}\right)^p f\left(x_1, \ldots, x_n\right)\]
}

\parag{Méthodes}{
    Nous avons maintenant deux méthodes pour calculer une formule de Taylor:
    \begin{enumerate}
        \item Utiliser la formule de Taylor en plusieurs variables.
        \item Utiliser les développements limités d'une seule variable.
    \end{enumerate}
}

\subsection{Extrema d'une fonction de plusieurs variables}
\parag{Définition: Point stationnaire}{
    Soit $E \subset \mathbb{R}^n$ et $f : E \mapsto \mathbb{R}$.

    $\bvec{a} \in E$ est un \important{point stationnaire} de $f$ si et seulement si:
    \[\nabla f\left(\bvec{a}\right) = \left(\frac{\partial f}{\partial x_1}\left(\bvec{a}\right), \ldots, \frac{\partial f}{\partial x_n}\left(\bvec{a}\right)\right) = \bvec{0}\]
}

\parag{Définition: Maximum local}{
    $f: E \mapsto \mathbb{R}$ admet un \important{maximum local} au point $\bvec{a} \in E$ s'il existe $\delta > 0$ tel que $f\left(\bvec{x}\right) \leq f\left(\bvec{a}\right)$ pour tout $\bvec{x} \in E \cap B\left(\bvec{a}, \delta\right)$.
}

\parag{Définition: Minimum local}{
    $f: E \mapsto \mathbb{R}$ admet un \important{minimum local} au point $\bvec{a} \in E$ s'il existe $\delta > 0$ tel que ${\color{red}f\left(\bvec{x}\right) \geq f\left(\bvec{a}\right)}$ pour tout $\bvec{x} \in E \cap B\left(\bvec{a}, \delta\right)$.
}

\parag{Proposition: Condition nécessaire pour un extremum local}{
    Soit $E \subset \mathbb{R}^n$ et $f: E \mapsto \mathbb{R}$ une fonction admettant un extremum local au point $\bvec{a} \in E$ et telle que $\frac{\partial f}{\partial x_i}\left(\bvec{a}\right)$ existent $\forall i = 1, \ldots, n$.

    Alors, $\bvec{a}$ est un point stationnaire de $f$, i.e. $\nabla f\left(\bvec{a}\right) = \bvec{0}$.

    \subparag{Remarque 1}{
        La réciproque est fausse:
        \[f\left(x, y\right) = x^3\]
    }
}

\parag{Définition: Point critique}{
    $\bvec{a} \in E$ est un \important{point critique} de $f: E \mapsto \mathbb{R}$ si $\bvec{a}$ est un point stationnaire, ou si au moins une des dérivées partielles de $f$ n'existe pas en $\bvec{x} = \bvec{a}$.

    \subparag{Remarque}{
        En utilisant notre théorème, nous avons la proposition suivante:

        Si $\bvec{a}$ est un point d'extremum local, alors $\bvec{a}$ est un point critique.
    }
}

\parag{Théorème: Condition suffisante pour un extremum local}{
    Soit $f: E \mapsto \mathbb{R}$ une fonction de classe $C^2$ sur $E$, et soit $\bvec{a} \in E$ un point stationnaire ($\nabla f\left(\bvec{a}\right) = \bvec{0}$).

    Si toutes les valeurs propres de la matrice Hessienne de $f$ en $\bvec{a}$ sont strictement positives, alors $f$ possède un minimum local en $\bvec{a}$.

    Si toutes les valeurs propres de la matrice Hessienne de $f$ en $\bvec{a}$ sont strictement négatives, alors $f$ possède un maximum local en $\bvec{a}$.

    S'il y a au moins une valeur propre strictement négative et au moins une strictement positive, alors $\bvec{a}$ n'est pas un point d'extremum local.
}

\end{document}
