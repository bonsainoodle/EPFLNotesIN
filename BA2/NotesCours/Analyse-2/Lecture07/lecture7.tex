\documentclass[a4paper]{article}

% Expanded on 2022-03-14 at 10:16:46.

\usepackage{../../style}

\title{Analyse 2}
\author{Joachim Favre}
\date{Lundi 14 mars 2022}

\begin{document}
\maketitle

\lecture{7}{2022-03-14}{On introduit pleins de symboles marrants}{
}

\section{Espace $\mathbb{R}^n$}

\subsection{$\mathbb{R}^n$ est un espace vectoriel normé}

\parag{Définition}{
    $\mathbb{R}^n$ est un ensemble de tous les $n$-tuples ordonnés de nombres réels.
    \[\bvec{x} = \begin{pmatrix} x_1 & \cdots & x_n \end{pmatrix} = \begin{pmatrix} x_1 \\ \vdots \\ x_n \end{pmatrix} \in \mathbb{R}^{n}\]

    On dit parfois que $\bvec{x}$ est un point (élément) de $\mathbb{R}^n$.
}

\parag{Opérations de $\mathbb{R}^n$}{
    $\mathbb{R}^n$ est muni de deux opérations. La première est l'addition $+$:
    \[\bvec{x} = \begin{pmatrix} x_1 & \cdots & x_n \end{pmatrix}, \bvec{y} = \begin{pmatrix} y_1 & \cdots & y_n \end{pmatrix} \implies \bvec{x} + \bvec{y} \over{=}{déf} \begin{pmatrix} x_1 + y_1 & \cdots & x_n + y_n \end{pmatrix}\]

    La deuxième est la multiplication par un nombre réel $\lambda \in \mathbb{R}$:
    \[\bvec{x} = \begin{pmatrix} x_1 & \cdots & x_n \end{pmatrix} \implies \lambda \bvec{x} \over{=}{déf} \begin{pmatrix} \lambda x_1 & \cdots & \lambda x_n \end{pmatrix}\]

    \subparag{Propriétés}{
        On remarque que les opérations satisfont les propriétés suivantes, pour tout $\bvec{x}, \bvec{y} \in \mathbb{R}^n$, $\lambda, \lambda_1, \lambda_2 \in \mathbb{R}$.
        \begin{itemize}
            \item $\left(\lambda_1 \lambda_2\right) \bvec{x} = \lambda_1 \left(\lambda_2 \bvec{x}\right) = \lambda_2 \left(\lambda_1 \bvec{x}\right)$
            \item $0 \bvec{x} = \begin{pmatrix} 0 & \cdots & 0 \end{pmatrix} = \bvec{0}$
            \item $1 \bvec{x}= \bvec{x}$
            \item $\left(\lambda_1 + \lambda_2\right)\bvec{x} = \lambda_1 \bvec{x} + \lambda_2 \bvec{x}$
            \item $\lambda\left(\bvec{x} + \bvec{y}\right) = \lambda \bvec{x} + \lambda \bvec{y}$
        \end{itemize}
    }

}

\parag{Base}{
Nous avons défini $\mathbb{R}^n$ comme des $n$-tuples ordonnés de nombres réels (et non de manière géométrique). Nous pouvons donc prendre la base:
\[\left\{\bvec{e}_i = \begin{pmatrix} 0 & \cdots & 0 & 1 & 0 & \cdots & 0 \end{pmatrix} \right\}_{i=1^n} \implies \bvec{e}_i \in \mathbb{R}^{n}, \mathspace i = 1, \ldots, n\]
où $\bvec{e}_i$ a uniquement un 1 à la $i$-ème position.

On remarque que n'importe quel élément de $\mathbb{R}^n$ peut s'écrire comme combinaison linéaire de cette base:
\[\bvec{x} = \sum_{i=1}^{n} x_i \bvec{e}_i = \begin{pmatrix} x_1 & \cdots & x_n \end{pmatrix} \]
}

\parag{Définition: Produit scalaire}{
    Nous définissons \important{le produit scalaire} comme:
    \[\left<\bvec{x}, \bvec{y}\right> \over{=}{déf} \sum_{i=1}^{n}  x_i y_i = x_1 y_1 + \ldots + x_n y_n \]
}

\parag{Définition: Norme Euclidienne}{
    Nous définissons la norme Euclidienne comme:
    \[\left\|\bvec{x}\right\| \over{=}{déf} \sqrt{\left<\bvec{x}, \bvec{x}\right>} = \sqrt{\sum_{i=1}^{n} x_i^2}\]

}

\parag{Proposition: Inégalité de Cauchy-Schwarz}{
    Pour tout $\bvec{x}, \bvec{y} \in \mathbb{R}^n$, nous avons:
    \[\left|\left<\bvec{x}, \bvec{y}\right>\right|\leq \left\|\bvec{x}\right\| \cdot \left\|\bvec{y}\right\|\]
}

\parag{Propriétés norme Euclidienne}{
    \begin{enumerate}[left=0pt]
        \item La norme est toujours positive:
              \[\left\|\bvec{x}\right\| \geq 0 \mathspace \forall \bvec{x} \in \mathbb{R}^n\]

        \item Si $\left\|\bvec{x}\right\| = 0$, alors:
              \[\bvec{x} = \bvec{0}\]

        \item Linéarité:
              \[\left\|\lambda \bvec{x}\right\| = \left|\lambda\right| \left\|\bvec{x}\right\|, \mathspace \bvec{x} \in \mathbb{R}^n, \lambda \in \mathbb{R}\]

        \item Inégalité triangulaire 1:
              \[\left\|\bvec{x} + \bvec{y}\right\| \leq \left\|\bvec{x}\right\| + \left\|\bvec{y}\right\|, \mathspace \forall \bvec{x}, \bvec{y} \in \mathbb{R}^n\]

        \item Inégalité triangulaire 2:
              \[\left\|\bvec{x} - \bvec{y}\right\| \geq \left|\left\|\bvec{x}\right\| - \left\|\bvec{y}\right\|\right|\]
    \end{enumerate}
}


\parag{Définition: Distance}{
    L'expression $\left\|\bvec{x} - \bvec{y}\right\| = d\left(\bvec{x}, \bvec{y}\right)$ est appelée \important{la distance} entre $\bvec{x}$ et $\bvec{y}$ dans $\mathbb{R}^n$.

    \subparag{Propriétés}{
        \begin{enumerate}[left=0pt]
            \item $d\left(\bvec{x}, \bvec{y}\right) = d\left(\bvec{y}, \bvec{x}\right)$
            \item $d\left(\bvec{x}, \bvec{y}\right) = 0 \iff \bvec{x} = \bvec{y}$
            \item $d\left(\bvec{x}, \bvec{y}\right) \leq d\left(\bvec{x}, \bvec{z}\right) + d\left(\bvec{z}, \bvec{y}\right)$
        \end{enumerate}
    }
}

\subsection{Topologie dans $\mathbb{R}^n$}

\parag{Définition: Boule ouverte}{
    Pour tout $\bvec{x} \in \mathbb{R}^n$ et nombre réel $\delta > 0$, soit:
    \[B\left(\bvec{x}, \delta\right) = \left\{\bvec{y} \in\mathbb{R}^n \telque \left\|\bvec{x} - \bvec{y}\right\| < \delta\right\}\]

    $B\left(\bvec{x}, \delta\right) \subset \mathbb{R}^n$ est appelé la \important{boule ouverte} de centre $\bvec{x}$ et rayon $\delta$.
}


\parag{Définition: Boule Fermée} {
    Pour tout $\bvec{x} \in \mathbb{R}^n$ et nombre réel $\delta > 0$, soit:
    \[B\left[\bvec{x}, \delta\right] = \left\{\bvec{y} \in\mathbb{R}^n \telque \left\|\bvec{x} - \bvec{y}\right\| \leq \delta\right\}\]

    $B\left[\bvec{x}, \delta\right] \subset \mathbb{R}^n$ est appelé la \important{boule fermée} de centre $\bvec{x}$ et rayon $\delta$.
}

\parag{Définition: Ensemble ouvert}{
    Nous définissons que $E \subset \mathbb{R}^n$ est \important{ouvert} si et seulement si:
    \[\forall \bvec{x}\in E\ \exists \delta > 0 \telque B\left(\bvec{x}, \delta\right) \subset E\]

    \subparag{Remarque}{
        Notez que, selon cette définition, $\o$ est un ensemble ouvert. En effet, $\forall x \in \o\ P\left(x\right)$ est une tautologie, peu importe $P\left(x\right)$.
    }
}

\parag{Définition: Point intérieur}{
    Soit $E \subset \mathbb{R}^n$ non-vide. Alors, $\bvec{x} \in E$ est un \important{point intérieur} de $E$ s'il existe $\delta > 0$ tel que $B\left(\bvec{x}, \delta\right) \subset E$.

    L'ensemble des points intérieurs de $E$ est appelé \important{l'intérieur} de $E$, noté $\mathring{E}$.

    Clairement:
    \[\mathring{E} \subset E\]

    \subparag{Observation}{
        Soit $E \subset\mathbb{R}^n$ non-vide.

        On remarque que $E \subset\mathbb{R}^n$ est ouvert si et seulement si $\mathring{E} = E$.
    }
}

\parag{Proposition}{
    La boule ouverte $B\left(\bvec{x}, \delta\right)$ est un ensemble ouvert.
}

\parag{Propriétés}{
    \begin{enumerate}[left=0pt]
        \item Toute réunion (même infinie) $\bigcup_{i \in I} E_i$ de sous-ensembles ouverts est un sous-ensemble ouvert.
        \item Toute intersection \textit{finie} $\bigcap_{i=1}^n E_i$ de sous-ensembles ouverts est un sous-ensemble ouvert.
    \end{enumerate}
}

\parag{Définition: Complémentaire d'un ensemble}{
    Soit $E \subset \mathbb{R}^n$. Son \important{complémentaire}, noté $CE$, est défini par:
    \[CE \over{=}{déf} \left\{\bvec{x} \in \mathbb{R}^n \telque \bvec{x} \not\in E\right\} = \mathbb{R}^n \setminus E\]
}


\parag{Définition: Ensemble fermé}{
    Soit $E \subset \mathbb{R}^n$ un sous-ensemble. $E$ est \important{fermé} dans $\mathbb{R}^n$ si son complémentaire $CE$ est ouvert.
}

\parag{Propriétés}{
    \begin{enumerate}[left=0pt]
        \item Toute intersection (même infinie) de sous-ensembles fermés est un sous-ensemble fermé.
        \item Toute réunion finie de sous-ensembles fermés est un sous-ensemble fermé.
    \end{enumerate}
}

\end{document}
