\documentclass{article}

% Expanded on 2021-10-04 at 10:05:37.

\usepackage{../../style}

\title{Analyse 1}
\author{Joachim Favre}
\date{Lundi 04 octobre 2021}

\begin{document}
\maketitle

\lecture{4}{2021-10-04}{Début des nombres complexes}{
}

\section{Nombres complexes}

\parag{Nombres complexes}{
    On considère les expressions de la forme $\left\{z = a + ib\right\} = \mathbb{C}$, où $a, b \in \mathbb{R}$, avec les opérations suivantes:
    \begin{itemize}
        \item[\important{+} :] $\left(a + ib\right) + \left(c + id\right) = \left(a + c\right) + i\left(b + d\right)$, qui ressemble énormément à la manière dont on additionne des vecteurs de $\mathbb{R}^2$. Par les axiomes de $\mathbb{R}$, on en déduit qu'elle est associative et commutative. De plus, il existe un élément neutre: $0 + 0i = 0$, il existe un opposé par rapport à l'addition: $\left(-a + i\left(-b\right)\right) + \left(a + ib\right) = 0 + i0 = 0 \in \mathbb{C}$.
        \item[\important{$\cdot$ :}] $\left(a + ib\right) \cdot \left(c + id\right) = ac - bd + i\left(ad - bc\right)$ qui est aussi associative et commutative. De plus, il existe aussi un élément neutre: $\left(a + ib\right)\left(1 + i0\right) = a + ib$ et il existe un réciproque si $z \in \mathbb{C}$, $z \neq 0$, qu'on appelle $z^{-1}$ et qui est tel que $z\cdot z^{-1} = z^{-1} \cdot z = 1$.
    \end{itemize}
}

\parag{Proposition pour le nombre réciproque}{
    Soit $z = a + ib \neq 0$. Alors,
    \[z^{-1} = \frac{a - ib}{a^2 + b^2}\]
}

\parag{Distributivité}{
    Le produit est distributif sur l'addition:
    \[z_1\left(z_2 + z_3\right) = z_1 z_2 + z_1 z_3 \mathspace \forall z_1, z_2, z_3 \in \mathbb{C}\]

    Grâce à toutes ces propriétés, on sait que $\mathbb{C}$ est un corps commutatif.
}


\parag{L'ordre de $\mathbb{C}$}{
    On remarque que $\mathbb{C}$ n'est pas ordonné.
}

\subsection{Les trois formes des nombres complexes}

\parag{Formule d'Euler}{
    Une autre formule qui porte le nom d'Euler est que
    \[e^{i\pi} = \cos\pi + i\sin\pi = -1 \implies e^{i\pi} + 1 = 0\]

    Qui est une formule très intéressante, puisqu'elle relie trois constantes fondamentales, $e$, $i$ et $\pi$; le 0 et le 1; ainsi que l'addition, la multiplication et la puissance.
}

\parag{Les trois formes}{
On a donc
\[z = \overbrace{\Re\left(z\right) + i\Im\left(z\right)}^{\text{cartésienne}} = \overbrace{\left|z\right|\left(\cos\left(\arg\left(z\right)\right) + i\sin\left(\arg\left(z\right)\right)\right)}^{\text{polaire trigonométrique}} = \overbrace{\left|z\right|e^{i\arg\left(z\right)}}^{\text{polaire exponentielle}}\]
où, pour passer entre les deux formes polaires c'est très facile, mais il faut faire des calculs pour passer à la forme cartésienne.

De plus, on a le module de $z$:
\[\left|z\right| = \sqrt{\Re\left(z\right)^2 + \Im\left(z\right)^2}\]

Et pour l'argument de $z = a + bi$:
\begin{functionbypart}{\arg\left(a + bi\right)}
    \arctan\left(\frac{b}{a}\right), \mathspace \text{si } a > 0 \\
    \arctan\left(\frac{b}{a}\right) + \pi, \mathspace \text{si } a < 0 \\
    \frac{\pi}{2}, \mathspace \text{si } a = 0 \text{ et } b > 0 \\
    \frac{3\pi}{2}, \mathspace \text{si } a = 0 \text{ et } b < 0
\end{functionbypart}

\subparag{Remarque}{
    La forme polaire peut être notée sous la forme:
    \[z = \rho\left(\cos\left(\phi\right) + i\sin\left(\phi\right)\right) = \rho e^{i\phi}\]
    où $\rho > 0$ et $\phi \in \mathbb{R}$.
}
}

\subsection{Multiplication en forme polaire}

\parag{Proposition}{
    Soient $z_1 = \rho_1 e^{i \phi_1}$ et $z_2 = \rho_2 e^{i \phi_2}$ deux nombres complexes non-nuls. Alors,
    \[z_1 z_2 = \rho_1 \rho_2 e^{i\left(\phi_1 + \phi_2\right)}\]

    Donc, multiplier deux nombres complexes revient à multiplier leur ``longueur'' et additionner leur angle.
}

\parag{Division en forme polaire}{
    Si on a $z = \left|z\right| e^{i\arg z} = \rho e^{i \phi}$, avec $z \neq 0$. Alors:
    \[z^{-1} = \frac{1}{\rho} e^{-i \phi}\]

    De plus, de manière générale, si $z_1 = \left|z_1\right| e^{i \arg z_1}$ et $z_2 = \left|z_2\right| e^{i \arg z_2}$, avec $z_1 \neq 0$ et $z_2 \neq 0$:
    \[\frac{z_1}{z_2} = \frac{\left|z_1\right|}{\left|z_2\right|} e^{i\left(\arg z_1 - \arg z_2\right)}\]
}

\parag{Proposition (Formule de De Moivre)}{
    Pour tout $\rho > 0$, $\phi \in \mathbb{R}$ et $n \in \mathbb{N}^*$, on a:
    \[\left(\rho\left(\cos\left(\phi\right) + i\sin\left(\phi\right)\right)\right)^n = \rho^n \left(\cos\left(n \phi\right) + i\sin\left(n \phi\right)\right)\]

    En forme exponentielle, c'est équivalent à dire que:
    \[\left(\rho e^{i \phi}\right)^n = \rho^n e^{in\phi}\]

    Attention, ce théorème ne marche que pour $n \in \mathbb{Z}$.
}

\end{document}
