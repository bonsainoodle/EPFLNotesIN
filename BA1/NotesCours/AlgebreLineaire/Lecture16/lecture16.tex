\documentclass[a4paper]{article}

% Expanded on 2021-11-15 at 13:18:58.

\usepackage{../../style}

\title{Algèbre linéaire}
\author{Joachim Favre}
\date{Lundi 15 novembre 2021}

\begin{document}
\maketitle

\lecture{16}{2021-11-15}{Équation caractéristique et matrices semblables}{
}

\parag{Théorème}{
    Soit $A$ une matrice $n\times n$ et $\bvec{v}_1, \ldots, \bvec{v}_r$ des vecteurs propres de $A$ associés à des valeurs propres \textit{distinctes} $\lambda_1$, \ldots, $\lambda_r$. Ces vecteurs propres sont linéairement indépendants.
}

\parag{Multiplication des valeurs propres}{
    Si les vecteurs propres de $A$ sont $\lambda_1, \ldots, \lambda_p$ et qu'elles sont associées à des vecteurs propres $\bvec{v}_1, \ldots, \bvec{v}_p$. On se demande ce qu'on peut dire des valeurs propres et des vecteurs propres $B = 2A$: les vecteurs propres restent les mêmes, et les valeurs propres doublent.
}

\subsection{Équation caractéristique}

\parag{Théorème}{
    Les valeurs propres de $A \in \mathbb{R}^{n \times n}$ sont les racines du polynôme caractéristique:
    \[p_A\left(\lambda\right) = \det\left(A - \lambda I_n\right)\]

    Ce polynôme est exactement de degré $n$.

    \subparag{Implication}{
        On sait qu'un polynôme de degré $n$ a exactement $n$ racines, si on compte leur multiplicités et qu'on admet les racines complexes.

        Donc, toute matrice $A \in \mathbb{R}^{n \times n}$ a exactement $n$ valeurs propres $\lambda_1, \ldots, \lambda_n \in \mathbb{C}$, en répétant chaque racine selon sa multiplicité (on l'appelle la \important{multiplicité algébrique} de la valeur propre; on verra un autre type de multiplicité plus tard). On a donc (car quand le degré du polynôme est impair, le coefficient dominant est négatif):
        \[p_A\left(\lambda\right) = \left(\lambda_1 - \lambda\right)\left(\lambda_2 - \lambda\right)\cdots\left(\lambda_n - \lambda\right)\]
    }

    \subparag{Remarque}{
        Le terme indépendant (la constante) est toujours le déterminant de la matrice.
    }
}

\subsection{Matrices semblables}


\parag{Définition des matrices semblables}{
    Soient $A$ et $B$ deux matrices de taille $n \times n$.

    On dit que $A$ est \important{semblable} à $B$ s'il existe $P$ inversible de taille $n \times n$ telle que:
    \[B = P^{-1} A P\]

    \subparag{Observation}{
        $B = P^{-1} A P$ est équivalent à:
        \[A = PBP^{-1}\]
    }

}

\parag{Théorème}{
    Si $A$ et $B$ sont semblables, alors $p_A = p_B$.

    En particulier, $A$ et $B$ ont donc les mêmes valeurs propres, avec les mêmes multiplicités algébriques.

    \subparag{Remarque}{
        La réciproque de ce théorème n'est pas vraie.
    }
}

\subsection{Diagonalisation}
\parag{Définition}{
    Soit $A \in \mathbb{R}^{n\times n}$. On dit que $A$ est \important{diagonalisable} si elle est semblable à une matrice diagonale. C'est-à-dire s'il existe $D$ diagonale et $P$ inversible telles que:
    \[A = PDP^{-1} \iff P^{-1} A P = D \iff AP = PD\]
}

\end{document}
