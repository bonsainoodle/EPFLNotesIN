\documentclass[a4paper]{article}

% Expanded on 2022-04-26 at 15:16:32.

\usepackage{../../style}

\title{AICC 2}
\author{Joachim Favre}
\date{Mardi 26 avril 2022}

\begin{document}
\maketitle

\lecture{17}{2022-04-26}{Digital signatures}{
}

\parag{Decoding exponent}{
    Such $d$ are not unique.
}

\parag{Fast exponentiation}{
    RSA is based on the efficiency of our algorithms to compute $a^k \Mod m$.
    Fast exponentiation is $\Theta\left(\log\left(m\right)\right)$.
}

\parag{RSA summary}{
    \svghere{RSACryptosystem.svg}
}

\subsection{Authenticity}

\parag{Privacy and digital signatures}{
    If Alice wants to have privacy and prove Bob she sent it, she can do a digital signature, and use Bob's key to cipher both the message and the ciphered text, combining both privacy and digital signatures. In other words, she can send $\left(f_B\left(t\right), f_B\left(f_A^{-1}\left(t\right)\right)\right)$ to Bob.
    \svghere[0.75]{PrivacySigningTrapdoorOneWayFunction.svg}

}

\parag{Hash function}{
    When we send a signed document, we are sending twice our text and this is very inefficient. Thus engineers imagined other methods, hash functions are one of them.

    The idea behind a hash function is a ``many-to-one'' function that converts all binary sequence to a fixed-length binary sequence (of 200 bits for instance). Also, given $h\left(x\right)$, it must be ``very hard'' to find $y$ such that $h\left(y\right) = h\left(x\right)$.

    Then, instead of ciphering her whole text to prove that she has the key, Alice can only sign the hash of her plaintext, sending the ciphered text and $s = f_B\left(f_A^{-1}\left(h\left(t\right)\right)\right)$ to Bob. Then, he can verify that:
    \[h\left(t\right) \over{=}{\text{?}} f_A\left(f_B^{-1}\left(s\right)\right)\]
}

\parag{Standards}{
    There are some of the following standards:
    \begin{itemize}
        \item SHA-1 throuh SHA-3 (Secure Hash Algorithm) family: cryptographic hash functions
        \item DSA (Digital Signature Algorithm), ECDSA (Elliptic Curve DSA): standards for digital signature
        \item DES (Data Encryption Standard), AES (Advanced Encryption Standard): symmetric-key encryption standards. They are faster than RSA and require less memory, but are weaker.
    \end{itemize}
}

\end{document}
