\documentclass[a4paper]{article}

% Expanded on 2021-10-20 at 15:04:30.

\usepackage{../../style}

\title{AICC 1}
\author{Joachim Favre}
\date{Mercredi 20 octobre 2021}

\begin{document}
\maketitle

\lecture{10}{2021-10-20}{Sequences and countable infinities}{
}

\parag{Definition: Lattices}{
    A partially ordered set in which every pair of elements has both a least upper bound and a greatest lower bound is called a lattice.
}

\parag{Definition: Lexicographic ordering}{
    Given two posets $\left(A_1, \preccurlyeq_1\right)$ and $\left(A_2, \preccurlyeq_2\right)$, the \important{lexicographic ordering} on $A_1 \times A_2$ is defined by saying that $\left(a_1, a_2\right)\preccurlyeq \left(b_1, b_2\right)$ if:
    \[a_1 \preccurlyeq_1 b_1 \lor \left(a_1 = b_1 \land a_2 \preccurlyeq_2 b_2\right)\]

    This definition can easily be extended to a lexicographic ordering on a $n$-ary Cartesian product.
}

\subsection{Sequences}
\parag{Definition: Sequences}{
    A \important{sequence} is a function from a subset of the integers --- usually $\mathbb{Z}_+$ or $\mathbb{N}$ --- to a set $S$.

    Let $f : \mathbb{Z}_+ \mapsto S$ be the function that defines a sequence. We write $a_n$ to denote the image $f\left(n\right)$ of integer $n$. We call $a_n$ a \important{term} of the sequence.
}

\parag{Types of sequences}{
    We can give name to the following sequences:
    \begin{description}
        \item[Arithmetic Progression:] $a_n = a + n\cdot d$, where $a$ and $d$ are given constants.
        \item[Geometric Progression:] $a_n = ar^{n}$.
        \item[Recurrence Relations:] $a_n = g\left(a_{n-1}, a_{n-2}, \ldots, a_{n-k}\right)$ with $k$ initial conditions.
    \end{description}
}

\parag{Important sums}{
    \[s_n = \sum_{j=1}^{n} d = nd\]
    \[s_n = \sum_{j=1}^{n} j = \frac{n\left(n+1\right)}{2} \]
}

\parag{Important product}{
    \[p_n = \prod_{j=1}^{n} r = r^{n}\]
    \[p_n = \prod_{j=1}^{n} j = n\left(n-1\right)\ldots1 = n!\]
}

\parag{Telescoping series}{
    Given $a_0, \ldots, a_n$, then:
    \[\sum_{j=0}^{n} \left(a_j - a_{j-1}\right) = a_n - a_0\]
}

\parag{Definition: String}{
    A \important{string} is a finite sequence of characters from a finite set $A$ (an alphabet).

    \subparag{Remark}{
        The \important{empty string} is represented by $\lambda$.
    }
}

\parag{Lexicographic ordering on strings}{
    A lexicographic ordering of strings can be defined using the ordering letters in the alphabet.

    We can note that strings with lexicographic ordering are well-ordered sets.
}

\subsection{Countable sets}
\parag{Cardinality comparison}{
    The cardinality of a set $A$ is equal to the cardinality of a set $B$, denoted by $\left|A\right| = \left|B\right|$ if and only if there is a bijection from $A$ to $B$ (we could draw a line going from each element of $a$ to each element of $b$, exactly one).

    If there is an injection of from $A$ to $B$, the cardinality of $A$ is less than or equal to the cardinality of $B$, denoted $\left|A\right| \leq \left|B\right|$.
}

\parag{Definition}{
    A set that is either finite or has the same cardinality as the set of positive integers, $\mathbb{Z}_+$, is called \important{countable}. Else, it is \important{uncountable}.

    When a set is finitely countable, then its cardinality is its number of of elements. If it is countably infinite, its cardinality is $\aleph_0$ (``aleph 0'').
}

\parag{Theorem}{
    Any subset of a countable set is countable.
}

\parag{Theorem}{
    Let $S$ be a set.

    Then there exists no surjective function $f : S \mapsto \mathcal{P}\left(S\right)$. In other words, $\left|S\right| < \left|\mathcal{P}\left(S\right)\right|$.
}

\parag{Theorem: Showing countability}{
    An infinite set $S$ is countable if and only if it is possible to list the elements of the set in a sequence indexed by positive integers.
}

\parag{Cardinality of rational numbers}{
    We want to show that rational numbers, $\mathbb{Q}$, are countable.

    We can arrange them in a table, as $\frac{x}{y}$ in the $x$\Th row and $y$\Th column. We can follow a path that goes through all of them, and skip the ones we get multiple times ($\frac{1}{2}$ and $\frac{2}{4}$, for example).
}

\parag{Theorem}{
    The union of a countable number of countable sets is countable.
}

\parag{Theorem: Real numbers}{
    The set of real numbers $\mathbb{R}$ is uncountable.
}

\end{document}
