\documentclass[a4paper]{article}

% Expanded on 2021-10-25 at 10:13:30.

\usepackage{../../style}

\title{Analyse 1}
\author{Joachim Favre}
\date{Lundi 25 octobre 2021}

\begin{document}
\maketitle

\lecture{10}{2021-10-25}{Cauchy, liminf, limsup et début de Netflix}{
}

\subsection{Suites de Cauchy}

\parag{Définition des suites de Cauchy}{
    La suite $\left(a_n\right)$ est une suite de Cauchy si, $\forall \epsilon > 0$, il existe $n_0 \in \mathbb{N}$ tel que $\forall n \geq n_0$ et $m \geq n_0$, alors:
    \[\left|a_n - a_m\right| < \epsilon\]
}

\parag{Proposition}{
    Une suite $\left(a_n\right)$ est une suite de Cauchy \important{si et seulement si} $\left(a_n\right)$ est convergente.
}

\parag{Remarque}{
    Si on a une limite telle que
    \[\lim_{n \to \infty} \left(a_{n+k} - a_n\right) = 0, \mathspace \forall k \in \mathbb{N}\]
    alors, cela ne veut pas nécessairement dire que c'est une suite de Cauchy.
}

\subsection[Limpsup et liminf]{Limite supérieure et limite inférieure d'une suite bornée}
\parag{Définition}{
    Soit $\left(x_n\right)$ une suite bornée. On sait donc par définition qu'il existe $m, M \in \mathbb{R}$ tels que
    \[m \leq x_n \leq M \mathspace \forall n \in \mathbb{N}\]

    On définit les suites:
    \[y_n = \sup\left\{x_k, k \geq n\right\} \mathspace z_n = \inf\left\{x_k, k \geq n\right\}\]

    On remarque que $y_n$ est décroissante (puisqu'on regarde le suprémum d'un ensemble toujours plus petit ; on jette à chaque fois un terme de plus, puis on regarde le suprémum de ce qui reste) et minorée par $y_n \geq x_n \geq m$ pour tout $n \in \mathbb{N}$.

    De la même manière, on voit que $z_n$ est croissante, et majorée par $z_n \leq x_n \leq M$ pour tout $n \in \mathbb{N}$.

    On définit donc
    \[\lim_{n \to \infty} y_n \over{=}{déf} \limsup_{n \to \infty} x_n \mathspace \text{ et } \mathspace \lim_{n \to \infty} z_n \over{=}{déf} \liminf_{n \to \infty} x_n\]

    Elles existent toujours, et:
    \[z_n \leq x_n \leq y_n \mathspace \forall n \in \mathbb{N}\]
}

\parag{Proposition}{
    Une suite bornée $\left(x_n\right)$ converge vers $\ell \in \mathbb{R}$ si et seulement si:
    \[\liminf_{n \to \infty} x_n = \limsup_{n \to \infty}x_n = \ell\]
}

\section{Séries numériques}
\subsection{Définitions et exemples}
\parag{Définition}{
    La \important{série} de terme général $a_n$ est un couple:
    \begin{enumerate}
        \item La suite $\left(a_n\right)$.
        \item La suite des sommes partielles :
              \[S_n \over{=}{déf} \sum_{k=0}^{n} a_k = a_0 + a_1 + \ldots + a_n\]
    \end{enumerate}
}

\parag{Somme partielle}{
    On note la $n$-ème somme partielle :
    \[S_n = \sum_{k=0}^{n} a_k\]
}

\parag{Série}{
    On note la série de terme général $a_k$:
    \[\sum_{k=0}^{\infty} a_k\]

    On appelle $a_k$ le $k$-ème terme.

    On dit que la série est convergente si et seulement si la suite $\left(S_n\right)$ des sommes partielles est convergente.

    La limite $\lim_{n \to \infty} S_n$ s'appelle la somme de la série $\sum_{k=0}^{\infty} a_k$. Dans le cas où elle existe, on dit que $\sum_{k=0}^{\infty} a_k$ converge vers $\ell$ et on note:
    \[\sum_{k=0}^{\infty} a_k = \ell\]

}

\parag{Divergence}{
    Si $\left(S_n\right)$ est divergente, alors on dit que la série $\sum_{k=0}^{\infty} a_k$ est divergente. En particulier, si $\lim_{n \to \infty} S_n = \pm\infty$, alors on écrit
    \[\sum_{k=0}^{\infty} a_k = \pm\infty\]
}

\parag{Proposition (convergence des séries géométriques)}{
    Nous avons l'égalité suivante:
    \[\sum_{k=0}^{\infty} r^{k} = \frac{1}{1 - r} \mathspace \left|r\right| < 1\]
}

\parag{Proposition (divergence des séries géométriques)}{
    La série
    \[\sum_{k=0}^{\infty} r^{k} \]
    est divergente si $\left|r\right| \geq 1$
}

\end{document}
