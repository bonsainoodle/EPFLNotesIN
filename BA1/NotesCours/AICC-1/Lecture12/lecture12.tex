\documentclass[a4paper]{article}

% Expanded on 2021-10-27 at 15:21:26.

\usepackage{../../style}

\title{AICC 1}
\author{Joachim Favre}
\date{Mercredi 27 octobre 2021}

\begin{document}
\maketitle

\lecture{12}{2021-10-27}{Greed, coins and same-sex marriages}{
}

\subsection{Optimisation algorithms}
\parag{Goal}{
    The goal is to minimise or maximise some parameter over all possible inputs.
}

\parag{Greedy algorithms}{
    They are very short-sighted, and the only take the ``best'' choice at every step.

    This does not always gives us the optimal result.
}

\begin{filecontents*}[overwrite]{cashieralgorithm.code}
    procedure change(c : array of r > 0 values of coins, where c[1] > c[2] > ... > c[r])
    for i := 1 to r
    d[i] := 0  // number of coins of value c[i]
    while n >= c[i]
    d[i] := d[i] + 1
    n := n - c[i]  // modify amount left to give
    return d
\end{filecontents*}

\parag{Cashier's algorithm}{
    Let's say we have coins of different values. Coins can be for example ``quarters'' (25 cents), ``dimes'' (10 cents), ``nickels'' (5 cents) and ``pennies'' (1 cent). We want to be able to give a certain value using the least possible number of coins.

    At each step, we choose the coin with the highest possible value that does not exceed the amount left.

    \importcode{cashieralgorithm.code}{pseudo}
}

\parag{Theorem (optimality)}{
    The greedy change-making algorithm for US coins produces change using the fewest coins possible.
}

\parag{Available coins}{
    Optimality depends on the denominations available. If we have ``quarters'' (25 cents), ``dimes'' (10 cents) and ``pennies'' (1 cent), then the algorithm would not produce the minimum number of coins.
}

\parag{Dropping coins}{
    For denominations, $c_1, \ldots c_n$ where $c_i | c_{i+1}$. The greedy algorithm is optimal (it is a sufficient condition, but not necessary).

    \subparag{Example for Europe}{
        We see that it is not a necessary condition, since $20$ does not divide $50$.
    }
}

\subsection{Stable matching}
\parag{Goal}{
    The task is to pair elements from equally sized two groups considering their preferences for members of the other group so that there are no ways to improve the preferences.
}

\parag{Definition of matching}{
    Given a finite set $A$, a \important{matching} of $A$ is a set of (unordered) pairs of distinct elements of $A$ where any elements occurs in at most one pair (such pairs are called independent)

    A \important{maximum matching} is a matching that contains the largest possible number of independent pairs.
}

\parag{Definition of preferences}{
    A \important{preference list} $L_x$ defines for every element $x \in A$ the order in which the element prefers to be paired with. $x \in A$ prefers $y$ to $z$ if $y$ precedes $z$ on $L_x$.
}

\parag{Stability of matching}{
    A matching is \important{unstable} if there are two pairs $\left(x, y\right),\left(v, w\right)$ in the matching such that $x$ prefers $v$ to $y$ and $v$ prefers $x$ to $w$. In other words, each want to leave their partner.

    A \important{stable} matching is a matching that is not unstable.
}

\parag{Marriage problem}{
    To guarantee the existence of a stable maximum matching, irrespective of the preference lists, it suffices to use a more stringent (harsher) pairing rule.

    Given a set with even cardinality, we partition $A$ into two disjoint subsets $A_1$ and $A_2$, where $A_1 \cup A_2 = A$ and $\left|A_1\right| = \left|A_2\right|$. A matching is a bijection from the elements of one set to the elements of the other set.
}

\begin{filecontents*}[overwrite]{stablematching.code}
    let M be the set of pairs under construction
    Initially M = empty set
    while |M| < |A1|:
    select an unpaired element x in A1
    let x propose to the first element y of A2 on Lx
    if y is unpaired then add the pair (x, y) to M
    else  // y is paired already
    let x' in A1 be the element that y is paired to
    if x' precedes x on Ly then remove y from Lx
    else  // x precedes x' on Ly
    replace (x', y) by (x, y) from M, and remove y from Lx'
\end{filecontents*}

\parag{Existence of stable maximum matching}{
    A greedy algorithm to construct a stable maximum matching for the marriage problem can be used to prove existence of stable matching. It is very efficient, and it gave a Nobel price to its creator. It is named the Gale-Shapley algorithm.

    \importcode{stablematching.code}{pseudo}
}

\end{document}
