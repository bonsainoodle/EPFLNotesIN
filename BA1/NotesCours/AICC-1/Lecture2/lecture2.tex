\documentclass{article}

% Expanded on 2021-09-21 at 22:59:41.

\usepackage{../../style}

\title{Advanced information, computation, communication I}
\author{Joachim Favre}
\date{Mercredi 22 septembre 2021}

\begin{document}
\maketitle

\lecture{2}{2021-09-22}{Logical equivalences and normal forms}{
}

\parag{Satisfiability}{
    Depending on the values the compound expression can take, we define the concept of \important{tautology}, \important{contingency}, \important{contradiction}, \important{satisfiable} and \important{unsatisfiable}:

    \begin{center}
        \begin{tabular}{|c|c|c|c|}
            \hline
            \textbf{Compound expression}        & \textbf{Called} & \textbf{Example}  & \textbf{Satisfiability}      \\
            \hline
            All true                            & Tautology       & $p \lor \lnot p$  & \multirow{2}{*}{Satisfiable} \\
            \cline{1-3}
            True or false, depending on the row & Contingency     & $p$               &                              \\
            \hline
            All false                           & Contradiction   & $p \land \lnot p$ & Unsatisfiable                \\
            \hline
        \end{tabular}
    \end{center}
}

\subsection{Logical equivalences}
\parag{Definition of logical equivalence}{
    Two compound propositions $p$ and $q$ are \important{logically equivalent} if $p \leftrightarrow q$ is a tautology. We write $p \equiv q$.
}

\parag{De Morgan's Laws}{
    \important{De Morgan's Laws} state that:
    \[\lnot\left(p \land q\right) \equiv \lnot p \lor \lnot q\]
    \[\lnot\left(p \lor q\right) \equiv \lnot p \land \lnot q\]
}

\parag{Equivalences with basic connectives}{
    The following equivalences using only $\land$ and $\lor$ are really important and useful:
    \begin{center}
        \begin{tabular}{|c|c|}
            \hline
            \textbf{Equivalence}                                                                      & \textbf{Name}                      \\
            \hline
            \hline
            $p \land T \equiv p$                                                                      & \multirow{2}{*}{Identity laws}     \\
            $p \lor F \equiv p$                                                                       &                                    \\
            \hline
            $p \lor T \equiv T$                                                                       & \multirow{2}{*}{Domination laws}   \\
            $p \land F \equiv F$                                                                      &                                    \\
            \hline
            $p \lor p \equiv p$                                                                       & \multirow{2}{*}{Idempotent laws}   \\
            $p \land p \equiv p$                                                                      &                                    \\
            \hline
            $\lnot\left(\lnot p\right)$                                                               & Double negation law                \\
            \hline
            \hline
            $p \lor \left(p \land q\right) \equiv p$                                                  & \multirow{2}{*}{Absorption laws}   \\
            $p \land \left(p \lor q\right) \equiv p$                                                  &                                    \\
            \hline
            $p \lor \lnot p \equiv T$                                                                 & \multirow{2}{*}{Negation laws}     \\
            $p \land \lnot p \equiv F$                                                                &                                    \\
            \hline
            \hline
            $p \lor q \equiv q \lor p$                                                                & \multirow{2}{*}{Commutative laws}  \\
            $p \land q \equiv q \land p$                                                              &                                    \\
            \hline
            $\left(p \lor q\right) \lor r \equiv p \lor \left(q \lor r\right)$                        & \multirow{2}{*}{Associative laws}  \\
            $\left(p \land q\right) \land r \equiv p \land \left(q \land r\right)$                    &                                    \\
            \hline
            $p \lor \left(q \land r\right) \equiv \left(p \lor q\right) \land \left(p \lor r\right)$  & \multirow{2}{*}{Distributive laws} \\
            $p \land \left(q \lor r\right) \equiv \left(p \land q\right) \lor \left(p \land r\right)$ &                                    \\
            \hline
        \end{tabular}
    \end{center}
}

\parag{Equivalences with implications}{
    The following equivalences using implications are less important, but still it is interesting to see them:
    \begin{center}
        \begin{tabular}{|c|}
            \hline
            $p \to q \equiv \lnot p \lor q$                                                      \\
            $p \to q \equiv \lnot q \to \lnot p$                                                 \\
            \hline
            $p \lor q \equiv \lnot p \to q$                                                      \\
            $p \land q \equiv \lnot\left(p \to \lnot q\right)$                                   \\
            $\lnot\left(p \to q\right) \equiv p \land \lnot q$                                   \\
            \hline
            $\left(p \to q\right) \land\left(p \to r\right) \equiv p \to \left(q \land r\right)$ \\
            $\left(p \to r\right) \land \left(q \to r\right) \equiv \left(p \lor q\right) \to r$ \\
            $\left(p \to q\right) \lor \left(p \to r\right) \equiv p \to \left(q \lor r\right)$  \\
            $\left(p \to r\right) \lor \left(q \to r\right) \equiv \left(p \land q\right) \to r$ \\
            \hline
        \end{tabular}
        \hspace{2em}
        \begin{tabular}{|c|}
            \hline
            $p \leftrightarrow q \equiv \left(p \to q\right) \land \left(q \to p\right)$               \\
            $p \leftrightarrow q \equiv \lnot p \leftrightarrow \lnot q$                               \\
            $p \leftrightarrow q \equiv \left(p \land q\right) \lor \left(\lnot p \land\lnot q\right)$ \\
            $\lnot\left(p \leftrightarrow q\right) \equiv p \leftrightarrow \lnot q$                   \\
            \hline
        \end{tabular}
    \end{center}
}

\parag{Contrapositive, converse and inverse}{
    We say that $\lnot q \to \lnot p$ is the \important{contrapositive} of $p \to q$. We have the following equivalence:
    \[p\to q \equiv \lnot q \to \lnot p\]

    We also define that $q \to p$ is the \important{converse} of $p \to q$, and $\lnot q \to \lnot p$ is the \important{inverse} of $p \to q$ (and the contrapositive of $q \to p$; in other words, the contrapositive of the converse is called the inverse). The converse and the inverse are equivalent, but they are not equivalent to $p \to q$ (this is a common fallacy politicians use).
}

\subsection{Normal forms}

\parag{Definition of the Disjunctive Normal Form}{
    A propositional formula is in \important{Disjunctive Normal Form (DNF)} if it only consists of a disjunction of compound expressions, where each compound expression consists of a conjunction of a set of propositional variables or their negation.
}

\parag{Construction of DNF}{
    Every compound proposition can be put in DNF using the following procedure:
    \begin{enumerate}
        \item Construct the truth table for the proposition.
        \item Select the rows that evaluate to $T$.
        \item For each of the propositional variables in the selected rows, add a conjunction which includes the positive form of the propositional if the variable is assigned $T$ in that row, or the negated form if the variable is assigned $F$ in that row.
    \end{enumerate}

    We can then simplify further the resulting proposition using the following equivalence:
    \[\left(p \land q\right) \lor \left(p \land \lnot q\right) \equiv p\]
}

\parag{Definition of the Conjunctive Normal Form}{
    We say that a compound proposition is in \important{Conjunctive Normal Form (CNF)} if it consists of a conjunction of compound expressions, where each compound expression consists of a disjunction of a set of propositional variables or their negations.
}

\parag{Construction of the CNF}{
    One of the ways to construct a CNF, is to:
    \begin{enumerate}
        \item Eliminate implications.
        \item Move negations inward.
        \item Use distributive and associative laws.
    \end{enumerate}

    Another way is to see that $p \equiv \lnot\left(\lnot p\right)$ and that the negation of a DNF is a CNF (but of the negated expression, they are not equivalent) by De Morgan's Laws. Thus we can find the DNF of $\lnot p$, meaning doing the exact same thing on the truth table, but on the rows where $p$ is false, and then take the negation on it.
}

\end{document}
