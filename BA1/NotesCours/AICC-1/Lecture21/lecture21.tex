\documentclass[a4paper]{article}

% Expanded on 2021-11-30 at 08:39:57.

\usepackage{../../style}

\title{AICC}
\author{Joachim Favre}
\date{Mardi 30 novembre 2021}

\begin{document}
\maketitle

\lecture{21}{2021-11-30}{Pigeons, pigeons, holes and recursive counting}{
}

\subsection{The pigeon-hole principle}

\parag{Theorem}{
    Let $k \in \mathbb{N}_+$. If we have $k+1$ objects we want to put in $k$ boxes, then at least one box contains two or more objects.
}

\parag{Corollary}{
    A function $f$ from a set with $k+1$ elements to a set with $k$ elements is not one-to-one.
}

\parag{Generalised version}{
    If $N$ objects are placed into $k$ boxes, then there is at least one box containing at least $\left\lceil \frac{N}{k} \right\rceil $ objects.
}

\subsection{Counting with recurrence relations}

\parag{Definition}{
    A \important{recurrence relation} for the sequence $\left\{a_n\right\}$ is an equation that expresses $\left\{a_n\right\}$ in terms of a finite number $k$ of the preceding terms of the sequence, i.e:
    \[a_n = f\left(n, a_{n-1}, a_{n-2}, \ldots, a_{n-k}\right)\]

    A sequence $\left\{a_n\right\}$ is called a \important{solution} of a recurrence relation if its terms satisfy the recurrence relation.

    The \important{initial conditions} for a sequence specify the terms $a_0, a_1, \ldots, a_{k-1}$.
}

\parag{Method}{
    \begin{enumerate}[left=0pt]
        \item Define a set $P_n$ depending on a parameter $n$.
        \item Describe $P_n$ in terms of $P_{n-1}, \ldots, P_{n-k}$
        \item Derive a recurrence relation for $\left|P_n\right|$.
        \item Solve the recurrence relation.
    \end{enumerate}
}

\end{document}
