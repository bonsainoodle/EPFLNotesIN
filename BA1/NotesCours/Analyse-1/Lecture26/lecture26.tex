\documentclass[a4paper]{article}

% Expanded on 2021-12-22 at 10:15:24.

\usepackage{../../style}

\title{Analyse}
\author{Joachim Favre}
\date{Mercredi 22 décembre 2021}

\begin{document}
\maketitle

\lecture{26}{2021-12-22}{J'ai un exam cet aprèm, alèd}{
}

\section{Révision}

\subsection[Séries numériques et intégrales généralisées]{Lien entre les séries numériques et les intégrales généralisées}
\parag{Proposition}{
    Soit $f \geq 0$ une fonction continue et strictement décroissante pour tout $x \geq a$ pour un certain $a \geq 1$.

    Alors:
    \[\sum_{n=1}^{\infty} f\left(n\right) \text{ converge} \iff \int_{1}^{\infty} f\left(x\right)dx \text{ converge}\]

    En d'autres mots, les deux convergent, ou les deux divergent.
}

\parag{A faire} {
    Le contenu de cette section est intéréssant à lire (dans les notes de Joachim) une fois le cours appris. Je n'ai gardé qu'une section puisque c'est celle qui se rapportait le plus à une propriété.
}

\end{document}
