\documentclass[a4paper]{article}

% Expanded on 2022-03-01 at 13:04:22.

\usepackage{../../style}

\title{Analyse 2}
\author{Joachim Favre}
\date{Mardi 02 mars 2022}

\begin{document}
\maketitle

\lecture{4}{2022-03-02}{On rajoute un prime}{
}

\subsection[EDL2]{Équations différentielles linéaires du second ordre}
\parag{Définition: EDL2}{
    Soit $I$ un intervalle ouvert. On appelle \important{équation différentielle linéaire du second ordre} (EDL2) une équation de la forme:
    \[y''\left(x\right) + p\left(x\right)y'\left(x\right) + q\left(x\right)y\left(x\right) = f\left(x\right)\]
    où $p, q, f: I \mapsto \mathbb{R}$ sont des fonctions continues.

    Nous appelons \important{EDL2 homogène} une équation de la forme suivante:
    \[y''\left(x\right) + p\left(x\right)y'\left(x\right) + q\left(x\right)y\left(x\right) = 0\]

    Nous cherchons une solution de cette équation de classe $C^2\left(I, \mathbb{R}\right)$.
}

\parag{EDL2 homogène à coefficients constants}{
    \subparag{Résumé}{
        Nous commençons avec une équation de la forme:
        \[y''\left(x\right) + py'\left(x\right) + qy\left(x\right) = 0, \mathspace p, q \in \mathbb{R}\]

        Soient $a, b \in \mathbb{C}$ les racines de l'équation $\lambda^2 + p\lambda + q = 0$. Alors, la solution générale est:
        \begin{functionbypart}{y\left(x\right)}
            C_1 e^{ax} + C_2 e^{bx}, \mathspace \text{ si } a, b \in \mathbb{R}, a \neq b \\
            C_1 e^{ax} + C_2 xe^{ax}, \mathspace \text{ si } a = b \\
            C_1 e^{\alpha x} \cos\left(\beta x\right) + C_2 e^{\alpha x} \sin\left(\beta x\right), \mathspace \text{ si } a = \alpha + i\beta = \bar{b} \not \in \mathbb{R}
        \end{functionbypart}
        pour des constantes arbitraires $C_1, C_2 \in \mathbb{R}$ et pour tout $x \in \mathbb{R}$.
    }

    \subparag{Note personnelle: Intuition}{
        Il peut paraitre très bizarre de cherchez les racines du polynôme au départ. Cela fonctionne, mais il est aussi intéressant de savoir comment est-ce que les mathématiciens l'ont deviné aux premiers abords.

        Nous avons donc l'équation suivante:
        \[y''\left(x\right) + py'\left(x\right) + qy\left(x\right) = 0, \mathspace p, q \in \mathbb{R}\]

        Nous savons que l'exponentielle est très pratique, donc faisons l'\textit{Ansatz} $y\left(x\right) = e^{\lambda x}$. Cela nous donne:
        \[\lambda^2 e^{\lambda x} + p \lambda e^{\lambda x} + q e^{\lambda x} = 0 \implies e^{\lambda x} \left(\lambda^2 + p\lambda + q\right) = 0\]

        Or, puisque l'exponentielle est non-nulle pour tout $x$, nécessairement, $\lambda^2 + p\lambda + q = 0$.
    }

}

\parag{EDL2 homogène}{
    Considérons l'équation suivante:
    \[y''\left(x\right) + p\left(x\right) y'\left(x\right) + q\left(x\right) y\left(x\right) = 0, \mathspace p, q : I \mapsto \mathbb{R}\]

    Nous pouvons faire les observations suivantes:
    \begin{enumerate}
        \item La solution générale d'une EDL2 homogène à coefficients constants contient 2 constantes arbitraires.
        \item Si $y_1\left(x\right)$ et $y_2\left(x\right)$ sont deux solutions d'une EDL2 homogène, alors la fonction suivante est aussi une solution:
              \[y\left(x\right) = A y_1\left(x\right) + By_2\left(x\right), \mathspace \text{où } A, B \in \mathbb{R}\]
    \end{enumerate}
}

\parag{Théorème}{
    Une EDL2 homogène admet \textit{une seule solution} $y\left(x\right) : I \mapsto \mathbb{R}$ de classe $C^2$ telle que $y\left(x_0\right) = t$ et $y'\left(x_0\right) = s$ pour un $x_0 \in I$ et les nombres arbitraires $s, t \in \mathbb{R}$.
}

\parag{Définition: indépendance linéaire}{
    Deux solutions $y_1\left(x\right), y_2\left(x\right) : I \mapsto \mathbb{R}$ sont \important{linéairement indépendantes} s'il n'existe pas de constante $C \in \mathbb{R}$ telle que:
    \[y_2\left(x\right) = Cy_1\left(x\right) \text{ ou } y_1\left(x\right) = Cy_2\left(x\right), \mathspace \forall x \in I\]

    En particulier, cela implique que $y_1\left(x\right)$ et $y_2\left(x\right)$ ne sont pas des fonctions constantes égales à 0 sur $I$.

    \subparag{Remarque}{
        Le théorème que nous avons vu juste avant nous dit que les EDL2 homogènes possèdent exactement deux solutions linéairement indépendantes.
    }
}

\parag{Construction d'une deuxième solution}{
Supposons que $v_1\left(x\right)$ est une solution de $y''\left(x\right) + p\left(x\right)y'\left(x\right) + q\left(x\right) y\left(x\right) = 0$ telle que $v_1\left(x\right) \neq 0$ pour tout $x \in I$. Nous nous demandons comment trouver une autre solution linéairement indépendante.

Prenons l'\textit{Ansatz} $v_2\left(x\right) = c\left(x\right) v_1\left(x\right)$ telle que $c\left(x\right)$ n'est pas constante (sinon la solution serait linéairement dépendante). Alors, on obtient que:
\[v_2'\left(x\right) = c'\left(x\right) v_1\left(x\right) + c\left(x\right) v_1'\left(x\right)\]
\[v_2''\left(x\right) = c''\left(x\right) v_1\left(x\right) + 2c'\left(x\right)v_1'\left(x\right) + c\left(x\right)v_1''\left(x\right)\]

Ainsi, nous pouvons remplacer notre solution dans notre équation:
\[c''\left(x\right) v_1\left(x\right) + 2c'\left(x\right) v_1'\left(x\right) + {\color{red}c\left(x\right)v_1''\left(x\right)} + p\left(x\right)c'\left(x\right) v_1\left(x\right) + {\color{red}p\left(x\right) c\left(x\right) v_1'\left(x\right) + q\left(x\right) c\left(x\right) v_1\left(x\right)} = 0\]

Les termes en rouge sont déjà égaux à 0 puisque $v_1\left(x\right)$ est une solution. Cela nous permet donc de simplifier notre équation en:
\[c''\left(x\right)v_1\left(x\right) + 2c'\left(x\right) v_1'\left(x\right) + p\left(x\right)c'\left(x\right) v_1\left(x\right) = 0\]

On suppose maintenant aussi que $v_1\left(x\right) \neq 0$ sur $I$ et $c'\left(x\right) \neq 0$ sur $I$ (ils ne s'annulent pas en aucun point de l'intervalle). Ceci nous donne donc que:
\[\frac{c''\left(x\right)}{c'\left(x\right)} = -p\left(x\right) - 2 \frac{v_1'\left(x\right)}{v_1\left(x\right)}\]
qui est une EDVS pour $c'\left(x\right)$.

Nous pouvons intégrer des deux côté, en prenant $\log\left(C\right)$ comme constante:
\[\log\left|c'\left(x\right)\right| = -P\left(x\right) - 2\log\left|v_1\left(x\right)\right| + \log\left(C\right) = \log\left(\frac{Ce^{-P\left(x\right)}}{v_1^2\left(x\right)}\right)\]

Or, puisque le logarithme est une fonction bijective:
\[c'\left(x\right) = \pm C \frac{e^{-P\left(x\right)}}{v_1^2\left(x\right)} = C_1 \frac{e^{-P\left(x\right)}}{v_1^2\left(x\right)}, \mathspace C_1 \in \mathbb{R}^*, C_1 = \pm C\]

Nous pouvons maintenant intégrer:
\[c\left(x\right) = \int c_1 \frac{e^{-P\left(x\right)}}{v_1^2\left(x\right)}dx + C_2, \mathspace C_2 \in \mathbb{R}\]

On obtient alors que $v_2\left(x\right) = c\left(x\right) v_1\left(x\right)$ est une solution. Par exemple, nous pouvons prendre $C_1 = 1$ et $C_2 = 0$, ce qui nous donne une solution telle que $v_2\left(x\right)$ et $v_1\left(x\right)$ sont linéairement indépendantes:
\[v_2\left(x\right) = c\left(x\right)v_1\left(x\right) = v_1\left(x\right) \int \frac{e^{-P\left(x\right)}}{v_1^2\left(x\right)}dx\]

Ainsi, à partir du moment où on trouve une solution particulière, nous sommes capable de trouver la solution générale.
}

\end{document}
