\documentclass{article}

% Expanded on 2021-09-29 at 10:24:04.

\usepackage{../../style}

\title{Analyse I}
\author{Joachim Favre}
\date{Mercredi 29 septembre 2021}

\begin{document}
\maketitle

\lecture{3}{2021-09-29}{Développement des infimums et suprémums}{
}

\parag{Théorème}{
    Tout sous-ensemble non-vide majoré $S \subset \mathbb{R}$ possède un suprémum, qui est unique. De la même manière, tout sous-ensemble non-vide minoré $S \subset \mathbb{R}$ possède un infimum, qui est unique.
}

\subsection{Dans le cas des intervalles}
\parag{Notation des intervalles}{
    Soient $a,b \in \mathbb{R}$, tel que $a < b$:
    \begin{itemize}
        \item $\left\{x \in \mathbb{R} : a \leq x \leq b\right\} = \left[a, b\right]$, intervalle fermé borné
        \item $\left\{x \in \mathbb{R} : a < x < b\right\} = \left]a,b\right[$ intervalle ouvert borné
        \item $\left\{x \in \mathbb{R} : a \leq x < b\right\} = \left[a, b\right[$ intervalle semi-ouvert borné
        \item $\left\{x \in \mathbb{R} : a < x \leq b\right\} = \left]a, b\right]$ intervalle semi-ouvert borné
        \item $-\infty$ et $\infty$, $\mathbb{R} \cup \left\{-\infty, \infty\right\} = \bar{\mathbb{R}}$ la droite réelle achevée, $-\infty < x < \infty,\ \forall x \in \mathbb{R}$
    \end{itemize}

    Intervalles non-bornés:
    \begin{itemize}
        \item $\left\{x \in \mathbb{R}: x \geq a\right\} = \left[a, +\infty\right[$ fermé
        \item $\left\{x \in \mathbb{R} : x \leq b\right\} = \left]-\infty, b\right]$ fermé
        \item $\left\{x \in \mathbb{R} : x > a\right\} = \left]a, +\infty\right[$ ouvert
        \item $\left\{x \in \mathbb{R} : x < b\right\} = \left]-\infty, b\right[$ ouvert
    \end{itemize}

    Notations sur les lettres:
    \begin{itemize}
        \item $\mathbb{R}_+ = \left[0, +\infty\right[ = \left\{x \geq 0\right\}$
        \item $\mathbb{R}^*_+ = \left]0, +\infty\right[ = \left\{x > 0\right\}$
        \item $\mathbb{R}_- = \left]-\infty, 0\right] = \left\{x \leq 0\right\}$
        \item $\mathbb{R}^*_- = \left]-\infty, 0\right[ = \left\{x < 0\right\}$
        \item $\mathbb{R}^* = \mathbb{R}^*_+ \cup \mathbb{R}^*_- = \left\{x \in \mathbb{R} : x \neq 0\right\}$
    \end{itemize}


    Il faut connaitre ces notations pour l'examen.
}

\parag{Théorème des suprémums et infimums d'un intervalle borné}{
Nous voulons trouver comment trouver $\sup S$ et $\inf S$ pour un sous-ensembles $S \subset \mathbb{R}$ donné. Ce théorème nous est d'une certaine aide:
\subparag{Proposition}{
\[\sup\left[a, b\right] = \sup\left[a, b\right[ = \sup\left]a, b\right] = \sup\left]a, b\right[ = b\]

Et de la même manière pour les infimums.
}
}

\subsection{Densité d'un ensemble dans un autre}
\parag{Théorème de la propriété d'Archimède}{
    Pour tout couple $\left(x, y\right)$ de nombres réels tel que $x > 0$ et $y \geq 0$, il existe $n \in \mathbb{N}^*$ tel que $nx > y$.

    Puisque ce théorème tient, on dit que $\mathbb{R}$ est un corps \important{archimédien}. On peut aussi démontrer cette propriété pour $\mathbb{Q}$.
}

\parag{Théorème de la densité de $\mathbb{Q}$ dans $\mathbb{R}$}{
    $\mathbb{Q}$ est dense dans $\mathbb{R}$, i.e pour tout couple $x,y \in \mathbb{R}$ avec $x < y$, il existe un nombre rationnel $r \in \mathbb{Q} \telque x < r < y$.
}

\parag{Théorème}{
    Soient $r,q \in \mathbb{Q}$, avec $r < q$. Alors, il existe $x \in \mathbb{R} \setminus \mathbb{Q}$ (irrationnel) tel que $r < x < q$.

    On peut aussi démontrer $\forall x,y \in \mathbb{R}$ avec $x < y$, on a que $\exists z \in \mathbb{R} \setminus \mathbb{Q} \telque x < z < y$.
}


\subsection{Retours aux infimums et supremums}

\parag{Remarque par rapport à la terminologie}{
    Si $\inf S \in S$, on dit que \important{$S$ possède un minimum}, avec $\min S = \inf S$ dans ce cas.

    Si $\sup S \in S$, on dit que \important{$S$ possède un maximum}, avec $\max S = \sup S$ dans ce cas.
}

\end{document}
