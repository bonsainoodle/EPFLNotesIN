\documentclass[a4paper]{article}

% Expanded on 2022-02-21 at 10:07:54.

\usepackage{../../style}

\title{Analyse 2}
\author{Joachim Favre}
\date{Lundi 21 février 2022}

\begin{document}
\maketitle

\lecture{1}{2022-02-21}{Le meilleur sujet}{
}

\section{Équations différentielles ordinaires}

\subsection{Définitions et exemples}

\parag{Définition}{
    Une \important{équation différentielle ordinaire} est une expression:
    \[E\left(x, y, y', \ldots, y^{\left(n\right)}\right) = 0\]
    où $E$ est une expression fonctionnelle, $n \in \mathbb{N}_0$, et $y = y\left(x\right)$ est une fonction inconnue de $x$.

    On cherche un intervalle ouvert $I \subset \mathbb{R}$ et une fonction $y : I \mapsto \mathbb{R}$ de classe $C^n$ telle que l'équation donnée est satisfaite $\forall x \in I$.
}

\parag{Terminologie}{
    On appelle $E\left(X, y, \ldots, y^{\left(n\right)}\right) = 0$ une équation différentielle (ED).

    \subparag{Ordre}{
        Un nombre naturel $n \in \mathbb{N}^*$ est \important{l'ordre} d'une équation différentielle si $n$ est l'ordre maximal de dérivée de $y\left(x\right)$ dans l'équation.
    }

    \subparag{Équation linéaire}{
        Si notre équation différentielle est un polynôme linéaire en $y, y', \ldots, y^{\left(n\right)}$, alors l'équation est dite \important{linéaire}.
    }

    \subparag{Équation autonome}{
        Si l'expression ne contient pas de $x$, l'équation est dite \important{autonome}.
    }

    \subparag{Solution générale}{
        La \important{solution générale} d'une équation différentielle est l'ensemble de toutes les solutions de l'équation.
    }
}


\parag{Définition du problème de Cauchy}{
    Résoudre le \important{problème de Cauchy} (équation différentielle avec des conditions initiales) pour l'équation $E\left(x, y, y', \ldots, y^{\left(n\right)}\right) = 0$ consiste à trouver l'intervalle ouvert $I \subset \mathbb{R}$ et une fonction $y : I \mapsto \mathbb{R}$ de classe $C^n\left(I\right)$ telle que $E\left(x, y, \ldots, y^{\left(n\right)}\right) = 0$ sur $I$ et pour laquelle des conditions initiales sont satisfaites:
    \[y\left(x_0\right) = b_0, y\left(x_1\right) = b_1, y'\left(x_2\right) = b_2, \ldots\]

    Le nombre des conditions initiales dépend du type de l'équation différentielle.
}

\subsection[EDVS]{Équations différentielles à variables séparées}

\parag{Définition}{
    On appelle une \important{équation différentielle à variables séparées} (EDVS) une équation sous la forme:
    \[f\left(y\right) \cdot y' = g\left(x\right)\]
    où $f: I \mapsto \mathbb{R}$ est une fonction continue sur $I \subset \mathbb{R}$ et $g : J \mapsto \mathbb{R}$ est une fonction continue sur $J \in\mathbb{R}$.

    Une fonction $y: J' \subset J \mapsto I$ de classe $C^1$ satisfaisant l'équation $f\left(y\right) y' = g\left(x\right)$ est une solution.
}

\parag{Théorème: Existence et unicité d'une solution des EDVS}{
    Soit $f: I \mapsto \mathbb{R}$ une fonction continue telle que $f\left(y\right) \neq 0$ pour tout $y \in I$, et soit $g : J \mapsto \mathbb{R}$ une fonction continue.

    \important{Existence:} Alors, pour tout couple $\left(x_0, b_0\right)$ où $x_0 \in J$ et $b_0 \in I$, l'équation
    \[f\left(y\right) y' = g\left(x\right)\]
    admet une solution $y : J' \subset J \mapsto I$ vérifiant la condition initiale.

    \important{Unicité:} Si $y_1 : J_1 \mapsto I$ et $y_2 : J_2 \mapsto I$ sont deux solutions telles que $y_1\left(x_0\right) = y_2\left(x_0\right) = b_0$, alors:
    \[y_1\left(x\right) = y_2\left(x\right), \mathspace \forall x \in J_1 \cap J_2\]
}

\end{document}
