\documentclass{article}

% Expanded on 2021-10-05 at 08:06:35.

\usepackage{../../style}

\title{AICC-1}
\author{Joachim Favre}
\date{Mardi 05 octobre 2021}

\begin{document}
\maketitle

\lecture{5}{2021-10-05}{Inference rules and the beginning of proofs}{
}

\parag{Summary}{
    \begin{center}
        % See document preamble to get how I centered the cells
        \begin{tabularx}{\linewidth}{|>{\hsize=0.15\hsize}C|>{\hsize=0.4\hsize}C|>{\hsize=0.35\hsize}C|}
            \hline
            \fullbf{Deduction}                  & \fullbf{Corresponding tautology}                                                                 & \fullbf{Name}          \\
            \hline
            \pdfhere{Conjunction.pdf}           & $p \land q \to p \land q$                                                                        & Conjunction            \\
            \hline
            \pdfhere{ModusPonens.pdf}           & $\left(p \land p \to q\right) \to q$                                                             & Modus Ponens           \\
            \hline
            \pdfhere{ModusTollens.pdf}          & $\left(\lnot q \land p \to q\right) \to \lnot p$                                                 & Modus Tollens          \\
            \hline
            \pdfhere{HypotheticalSyllogism.pdf} & $\left(p\to q \land q\to r\right) \to \left(p \to r\right)$                                      & Hypothetical Syllogism \\
            \hline
            \pdfhere{Resolution.pdf}            & $\left(\left(\lnot p \lor r\right) \land \left(p \lor q\right)\right) \to \left(q \lor r\right)$ & Resolution             \\
            \hline
        \end{tabularx}
    \end{center}
}

\parag{Other inference rule}{
    We can also derive the following inference rule. We give them names, but they are directly implied by the other inference rules:
    \begin{center}
        % See document preamble to get how I centered the cells
        \begin{tabularx}{\linewidth}{|>{\hsize=0.15\hsize}C|>{\hsize=0.4\hsize}C|>{\hsize=0.35\hsize}C|}
            \hline
            \fullbf{Deduction}                 & \fullbf{Corresponding tautology}                        & \fullbf{Name}         \\
            \hline
            \pdfhere{DisjunctiveSyllogism.pdf} & $\left(\left(p \lor q\right) \land\lnot p\right) \to q$ & Disjunctive syllogism \\
            \hline
            \pdfhere{Addition.pdf}             & $p \to \left(p \lor q\right)$                           & Addition              \\
            \hline
            \pdfhere{Simplification.pdf}       & $\left(p \land q\right) \to p$                          & Simplification        \\
            \hline
        \end{tabularx}
    \end{center}

    Note that the disjunctive syllogism is a simpler form of resolution, addition is dual to conjunction and simplification is a simpler form of Modus Ponens.
}

\subsection{Inference rules in predicate logic}

\parag{Summary}{
    \begin{center}
        % See document preamble to get how I centered the cells
        \begin{tabularx}{\linewidth}{|>{\hsize=0.4\hsize}C|>{\hsize=0.35\hsize}C|>{\hsize=0.15\hsize}C|}
            \hline
            \fullbf{Deduction}                      & \fullbf{Name}              & \fullbf{Shorten name} \\
            \hline
            \pdfhere{UniversalInstantiation.pdf}    & Universal Instantiation    & UI                    \\
            \hline
            \pdfhere{UniversalGeneralization.pdf}   & Universal Generalization   & UG                    \\
            \hline
            \pdfhere{UniversalModusPonens.pdf}      & Universal Modus Ponens     &                       \\
            \hline
            \pdfhere{ExistentialInstantiation.pdf}  & Existential Instantiation  & EI                    \\
            \hline
            \pdfhere{ExistentialGeneralization.pdf} & Existential Generalization & EG                    \\
            \hline
        \end{tabularx}
    \end{center}
}

\subsection{Proofs}

\parag{Terminology}{
    A \important{mathematical proof} is a valid argument that establishes the truth of a statement, in particular of a theorem. A \important{theorem} is a statement that can be shown to be true using definitions, axioms, other theorems or rules of inference. An \important{axiom} is a statement which is given as true.

    A \important{lemma} is a ``helping theorem'', it is something we prove in order to prove a theorem right after. A \important{corollary} is a result which follows directly and trivially from a theorem. \important{Propositions} are less important theorems.

    A \important{conjecture} is a statement that is being proposed to be true; we do not have a proof for it, but we think it may be true (but it may end up wrong).

    \important{Informal proofs}, which are generally shorter, often use multiple rules of inferences at each step, skip steps, do not state every inference rule explicitly, and so on. They are much easier to understand and to explain to people, but they make it also easier to introduce errors.
}

\parag{Theorems}{
    Many theorems assert that a property holds for all elements in a domain. Thus, they are often of the form:
    \[\forall x\left(P\left(x\right) \to Q\left(x\right)\right)\]

    To prove a theorem of this form, we show that:
    \[P\left(c\right) \to Q\left(c\right)\]
    where $c$ is an  arbitrary element of the domain. We can then use the universal generalisation to get the theorem.
}

\parag{Types of proof}{
    \begin{center}
        % See document preamble to get how I centered the cells
        \begin{tabularx}{\linewidth}{|>{\hsize=0.25\hsize}X>{\hsize=0.4\hsize}X>{\hsize=0.35\hsize}X|}
            \hline
            \textit{Trivial proof}           & If we know that $q$ is true, then $p \to q$ is true as well.                                                                                    & ``If it is raining, then $1 = 1$''                       \\
            \hline
            \textit{Vacuous proof}           & If we know that $p$ is false, then $p\to q$ is true.                                                                                            & ``If I am both rich and poor, then $2 + 2 = 5$''         \\
            \hline
            \textit{Direct proof}            & Assume that $p$ is true, then use definitions, axioms and theorems together with rules of inference till the statement $q$ results.             & This is when we usually use syllogisms.                  \\
            \hline
            \textit{Proof by contraposition} & Assume that $\lnot q$ is true, then use definitions, axioms and theorems together with rules of inference till the statement $\lnot p$ results. & Uses $p \to q \equiv \lnot q \to \lnot p$                \\
            \hline
            \textit{Proof by contradiction}  & Assume that $p$ and $\lnot q$ are true, then perform a direct proof to construct a contradiction.                                               & Uses $p \to q \equiv \left(p \land \lnot q\right) \to F$ \\
            \hline
        \end{tabularx}
    \end{center}

    The two last proofs are called \important{indirect proofs}.
}

\end{document}
