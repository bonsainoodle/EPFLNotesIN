\documentclass[a4paper]{article}

% Expanded on 2021-10-27 at 10:07:19.

\usepackage{../../style}

\title{Analyse 1}
\author{Joachim Favre}
\date{Mercredi 27 octobre 2021}

\begin{document}
\maketitle

\lecture{11}{2021-10-27}{Cratères de convergence lunaires}{
}

\parag{Proposition}{
    La série
    \[\sum_{n=1}^{\infty} \frac{1}{n^{p}}\]
    est convergente pour tout $p > 1$.
}

\parag{La fonction zêta de Riemann}{
    La fonction zêta de Riemann est définie telle que
    \[\zeta\left(s\right) = \sum_{n=1}^{\infty} \frac{1}{n^{s}}, \mathspace s > 1\]

    \subparag{Valeurs connues}{
        \begin{itemize}[left=0pt]
            \item $\zeta\left(2\right) = \frac{\pi^2}{6}$
            \item $\zeta\left(2k\right) = C_k \pi^{2k}$, où $C_k \in \mathbb{Q}$.

                  On sait que $\zeta\left(2k\right)$ est transcendant pour tout $k \in \mathbb{N}^*$

            \item $\zeta\left(3\right)$ est transcendant.
            \item Au moins une valeur entre $\zeta\left(5\right)$, $\zeta\left(7\right)$, $\zeta\left(9\right)$ et $\zeta\left(11\right)$ est irrationnelle.
        \end{itemize}
    }
}


\parag{Définition (Convergence absolue)}{
    Une série $\sum_{n = 1}^{\infty} a_n$ est dite \important{absolument convergente} si la série
    \[\sum_{n = 1}^{\infty} \left|a_n\right| \]
    est convergente.
}

\parag{Proposition}{
    Une série absolument convergente est convergente.
}

\parag{Proposition (condition nécessaire)}{
    Si la série $\sum_{n=1}^{\infty} a_n$ converge, alors
    \[\lim_{n \to \infty} a_n = 0\]

    \subparag{Remarque}{
        Attention, la contraposée n'est pas vraie ; si la limite est 0, alors la série ne converge pas forcément.

        Par exemple, la série harmonique, i.e:
        \[\sum_{n = 1}^{\infty} \frac{1}{n}\]
    }
}

\subsection{Critères de convergence}

\parag{Proposition (Critère de Leibnitz pour les séries alternées)}{
    Soit $\left(a_n\right)$ une suite telle que:
    \begin{enumerate}
        \item Il existe $p \in \mathbb{N}$ tel que pour tout $n \geq p$ on a:
              \[\left|a_{n+1}\right| \leq \left|a_n\right|\]

              En d'autres mots, la suite est décroissante en valeur absolue.
        \item Il existe $p \in \mathbb{N}$ tel que pour tout $n \geq p$ on a:
              \[a_{n+1}\cdot a_n \leq 0\]

              En d'autres mots, la série est alternée.
        \item On a :
              \[\lim_{n \to \infty} a_n = 0\]

    \end{enumerate}

    Alors, la série $\sum_{n=1}^{\infty} a_n$ est convergente.
}

\parag{Série harmonique alternée}{
    Prenons la série
    \[\sum_{n = 1}^{\infty} \left(-1\right)^{n} \frac{1}{n}\]

    Elles est convergente par le critère de Leibninz.
}

\parag{Proposition (critère de comparaison pour les séries à termes non-négatifs)}{
    Soit $\left(a_n\right)$ et $\left(b_n\right)$ deux suites telles que $\exists k \in \mathbb{N}$ tel que $0 \leq a_n \leq b_n$ pour tout $n \geq k$.

    Si $\sum_{n=0}^{\infty} b_n$ converge, alors $\sum_{n=0}^{\infty} a_n$ converge.

    Si $\sum_{n=0}^{\infty} a_n$ diverge, alors $\sum_{n=0}^{\infty} b_n$ diverge.
}


\parag{Remarque}{
    Si $\sum_{n=0}^{\infty} a_n$ ne possède que des termes positifs, et la suite des sommes partielles est majorée, alors la série est convergente par définition.

    De la même manière, si $\sum_{n=0}^{\infty} a_n$ ne possède que des termes négatifs, et la suite des sommes partielles est minorée, alors la série est convergente par définition.
}

\parag{Proposition (Critère de d'Alembert)}{
    Soit $\left(a_n\right)$ une suite telle que $a_n \neq 0$ pour tout $n \in \mathbb{N}$ et telle que
    \[\lim_{n \to \infty} \left|\frac{a_{n+1}}{a_n}\right| = \rho \in \mathbb{R}\]

    Alors, si $\rho < 1$, la série $\sum_{n=0}^{\infty} a_n$ est absolument convergente.

    Si $\rho > 1$, alors la série $\sum_{n=0}^{\infty} \left|a_n\right|$ diverge.
}

\parag{Proposition (Critère de Cauchy (de la racine))}{
Soit $\left(a_n\right)$ une suite telle que la limite existe, et:
\[\lim_{n \to \infty} \left|a_n\right|^{\frac{1}{n}} = \rho \in \mathbb{R}\]

Alors, si $\rho < 1$, la série $\sum_{n=0}^{\infty} a_n$ est absolument convergente.

Si $\rho > 1$, alors la série $\sum_{n=0}^{\infty} a_n$ diverge.
}

\parag{Remarques}{
    \begin{enumerate}[left=0pt]
        \item Si $\lim_{n \to \infty} \left|\frac{a_{n+1}}{a_n}\right| = r$ et $\lim_{n \to \infty} \left|a_n\right|^{\frac{1}{n}} = \ell$. Alors, nécessairement $ r = \ell$.
        \item Parfois, $\lim_{n \to \infty} \sqrt[n]{\left|a_n\right|}$ existe, mais $\lim_{n \to \infty} \left|\frac{a_{n+1}}{a_n}\right|$ n'existe pas.

              Donc, le critère de Cauchy est plus fort que le critère de d'Alembert.
        \item Si $\lim_{n \to \infty} \left|\frac{a_{n+1}}{a_n}\right| = 1$ ou $\lim_{n \to \infty} \sqrt[n]{\left|a_n\right|} = 1$, alors on ne peut pas faire de conclusion sur la convergence de la série.
    \end{enumerate}

    \subparag{Exemple du point (3)}{
        La série suivante diverge:
        \[\sum_{n = 1}^{\infty} \frac{1}{k} \implies \lim_{n \to \infty} \left|\frac{\frac{1}{n+1}}{\frac{1}{n}}\right| = \lim_{n \to \infty} \frac{n}{n+1} = 1\]

        Alors que la série suivante converge:
        \[\sum_{n = 1}^{\infty} \frac{1}{n^2} \implies \lim_{n \to \infty} \left|\frac{\frac{1}{\left(n+1\right)^2}}{\frac{1}{n^2}}\right| = \lim_{n \to \infty} \frac{n^2}{\left(n+1\right)^2} = 1\]
    }
}

\end{document}
