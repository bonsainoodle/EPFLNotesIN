\documentclass[a4paper]{article}

% Expanded on 2021-10-18 at 10:11:39.

\usepackage{../../style}

\title{Analyse I}
\author{Joachim Favre}
\date{Lundi 18 octobre 2021}

\begin{document}
\maketitle

\lecture{8}{2021-10-18}{Suite de la suite des limites de suites}{
}

\parag{Limite de racine}{
    Soit $a_0 = 1$, et $a_n = \sqrt[n]{a}$ quand $n \geq 1$, pour $a > 0$. Alors:
    \[\lim_{n \to \infty} a_n = 1\]
}

\parag{Suite géométrique}{
    Soit $a_n = a_0 r^{n}$, avec $a_0 \in \mathbb{R}$, $a_0 \neq 0$, $r \in \mathbb{R}$. Alors:
    \begin{itemize}
        \item $\lim\limits_{n \to \infty} a_0 r^{n} = 0$, $\left|r\right| < 1$
        \item $\lim\limits_{n \to \infty} a_0 r^{n} = a_0$, $r = 1$
        \item $\left(a_n\right)$ diverge quand $\left|r\right| > 1$ ou $r = -1$.
    \end{itemize}
}

\parag{Remarques}{
    \begin{enumerate}[left=0pt]
        \item Si $\lim\limits_{n \to \infty} x_n = \ell \in \mathbb{R}$, alors:
              \[\lim\limits_{n \to \infty} \left|x_n\right| = \left|\ell\right|\]

        \item Si $\lim\limits_{n \to \infty} \left|x_n\right| = 0$. Alors, on sait que
              \[\lim\limits_{n \to \infty} x_n = 0\]

        \item En général $\lim\limits_{n \to \infty} \left|x_n\right| = \ell \neq 0$ n'implique pas que $\left(x_n\right)$ converge.
        \item Si $\left(a_n\right)$ est bornée et $\lim\limits_{n \to \infty} b_n = 0$, alors:
              \[\lim_{n \to \infty} a_n b_n = 0\]

    \end{enumerate}
}

\parag{Théorème (Critère de d'Alembert)}{
    Soit $\left(a_n\right)$ une suite telle que $a_n \neq 0$ pour tout $n \in \mathbb{N}$ et
    \[\lim_{n \to \infty} \left|\frac{a_{n+1}}{a_n}\right| = \rho \geq 0\]

    Alors, si $\rho < 1$, $\lim_{n \to \infty} a_n = 0$. Sinon, si $\rho > 1$, alors $\left(a_n\right)$ diverge. Si $\rho = 1$, on ne sait rien.
}

\subsection{Limites infinies}
\parag{Définition}{
    On dit que $\left(a_n\right)$ tend vers $+\infty$ si $\forall A > 0$, $\exists n_0 \in \mathbb{N}$ tel que $\forall n \geq n_0$, alors $a_n \geq A$.

    On note:
    \[\lim_{n \to \infty} a_n = \infty\]
}

\parag{Définition}{
    On dit que $\left(a_n\right)$ tend vers $-\infty$ si $\forall A > 0$, $\exists n_0 \in \mathbb{N}$ tel que $\forall n \geq n_0$, alors $a_n \leq -A$.

    On note:
    \[\lim_{n \to \infty} a_n = -\infty\]
}

\parag{Propriétés}{
    \begin{enumerate}[left=0pt]
        \item Si $\lim\limits_{n \to \infty} a_n = \infty = \lim\limits_{n \to \infty} b_n$, alors:
              \[\lim\limits_{n \to \infty} \left(a_n + b_n\right) = \infty\]

        \item $\lim\limits_{n \to \infty} a_n = \pm \infty$ et $\left(b_n\right)$ est bornée, alors
              \[\lim_{n \to \infty} \left(a_n \pm b_n\right) = \pm \infty\]

        \item (Théorème de minoration) Si $\lim\limits_{n \to \infty} b_n = \infty$ et $a_n \geq b_n$, alors
              \[\lim_{n \to \infty} a_n = \infty\]

        \item (Théorème de majoration) Si $\lim\limits_{n \to \infty} b_n = -\infty$ et $a_n \leq b_n$, alors
              \[\lim_{n \to \infty} a_n = -\infty\]

        \item Si $\left(a_n\right)$ est bornée et $\lim\limits_{n \to \infty} b_n = \pm \infty$, alors
              \[\lim_{n \to \infty} \frac{a_n}{b_n} = 0\]

        \item Si $\lim\limits_{n \to \infty} \left|\frac{a_{n+1}}{a_n}\right| = \infty$ et $a_n \neq 0$ pour tout $n$. Alors, $\left(a_n\right)$ diverge.
    \end{enumerate}
}

\parag{Formes indéterminées}{
    \begin{enumerate}[left=0pt]
        \item $\infty - \infty$
        \item $0\cdot \infty$
        \item $\frac{\infty}{\infty}$
        \item $\frac{0}{0}$
        \item $1^{\infty}$
        \item $0^{0}$
        \item $\infty^{0}$
    \end{enumerate}

}

\parag{Suites trigonométriques}{
    $\left(\cos n\right)$ et $\left(\sin n\right)$ sont divergentes (mais bornées).
}

\parag{Théorème (convergence des suites monotones)}{
    Toute suite croissante qui est majorée converge vers le supremum de son ensemble. De la même manière, toute suite décroissante qui est minorée converge vers l'infimum de son ensemble.

    De plus, toute suite croissante qui n'est pas majorée tend vers $+\infty$ (diverge). De manière similaire, toute suite décroissante qui n'est pas minorée tend vers $-\infty$ (diverge).
}

\end{document}
