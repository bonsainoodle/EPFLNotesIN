\documentclass[a4paper]{article}

% Expanded on 2021-11-08 at 10:12:45.

\usepackage{../../style}

\title{Analyse 1}
\author{Joachim Favre}
\date{Lundi 08 novembre 2021}

\begin{document}
\maketitle

\lecture{14}{2021-11-08}{Limites vers l'infini, de tous genres}{
}

\parag{Proposition}{
    La limite suivante est remarquable:
    \[\lim_{x \to 0} \frac{\sin x}{x} = 1\]

    \subparag{Remarque}{
        On utilise cette limite pour démontrer la dérivée du sinus et du cosinus.
    }

}

\parag{Théorème (limite de la composée de deux fonctions)}{
    Soit $f: E \mapsto F$ et $g : G \mapsto H$ telles que:
    \[\lim_{x \to x_0}  f\left(x\right) = y_0 \mathspace \text{ et } \mathspace \lim_{y \to y_0} g\left(y\right) = \ell\]

    De plus, supposons que $f\left(E\right) \subset G$ et $\exists\alpha > 0$ tel que
    \[0 < \left|x - x_0\right| < \alpha \implies f\left(x\right) \neq y_0\]

    Alors, on sait que:
    \[\lim_{x \to x_0} \left(g\circ f\right)\left(x\right) = \lim_{x \to x_0} g\left(f\left(x\right)\right) = \ell\]

    \subparag{Remarque}{
        On a besoin de la deuxième supposition, parce que potentiellement $g\left(y\right)$ n'est pas définie en $y_0$.
    }
}

\subsection{Limites lorsque $x$ tend vers $\pm \infty$}
\parag{Définition du voisinage}{
Soit $f: E \mapsto F$. On dit qu'elle est définie au voisinage de $+\infty$ si $\exists \alpha \in \mathbb{R}$ tel que $\left]\alpha, +\infty\right[ \subset E$.

De la même manière, on dit qu'elle est définie au voisinage de ${\color{red}-\infty}$ si $\exists \alpha \in \mathbb{R}$ tel que ${\color{red}\left]-\infty, \alpha\right[} \subset E$.
}

\parag{Définition de la limite}{
    On dit que $f : E\mapsto F$ définie au voisinage de $+\infty$ admet pour limite le nombre réel $\ell$ lorsque $x$ tend vers $+\infty$ si pour tout $\epsilon > 0$, il existe $\alpha \in \mathbb{R}$ tel que:
    \[\forall x \in E, x \geq \alpha \implies \left|f\left(x\right) - \ell\right| \leq \epsilon\]

    On note:
    \[\lim_{x \to \infty} f\left(x\right) = \ell\]

    Dans ce cas, on dit que la fonction $f$ admet une asymptote horizontale $y = \ell$ lorsque $x \to \infty$.

    De la même manière, on dit que $f : E\mapsto F$ définie au voisinage de ${\color{red}-\infty}$ admet pour limite le nombre réel $\ell$ lorsque $x$ tend vers ${\color{red}-\infty}$ si pour tout $\epsilon > 0$, il existe $\alpha \in \mathbb{R}$ tel que:
    \[\forall x \in E, {\color{red}x \leq \alpha} \implies \left|f\left(x\right) - \ell\right| \leq \epsilon\]

    On note:
    \[\lim_{x \to {\color{red}-\infty}} f\left(x\right) = \ell\]

    Dans ce cas, on dit que la fonction $f$ admet une asymptote horizontale $y = \ell$ lorsque ${\color{red}x \to -\infty}$.
}

\subsection{Limites infinies}
\parag{Définition}{
$f : E \mapsto F$ définie au voisinage de $x_0 \in \mathbb{R}$ tend vers $+\infty$ lorsque $x \to x_0$ si pour tout $A > 0$, il existe un $\delta > 0$ tel que:
\[0 < \left|x - x_0\right| \leq \delta \implies f\left(x\right) \geq A\]

On note:
\[\lim_{x \to x_0} f\left(x\right) = +\infty\]

De la même manière, on dit que $f : E \mapsto F$ définie au voisinage de $x_0 \in \mathbb{R}$ tend vers ${\color{red}-\infty}$ lorsque $x \to x_0$ si pour tout $A > 0$, il existe un $\delta > 0$ tel que:
\[0 < \left|x - x_0\right| \leq \delta \implies f\left(x\right)\ {\color{red}\leq -A}\]

On note:
\[\lim_{x \to x_0} f\left(x\right) = {\color{red} -\infty}\]
}

\subsection{Limites infinies lorsque $x$ tend vers $\pm\infty$}
\parag{Définition}{
On dit que $f : E \mapsto F$ définie au voisinage de $+\infty$ tend vers $+\infty$ lorsque $x \to +\infty$ si:
\[\forall A > 0\ \exists \alpha \in \mathbb{R} \telque x \geq \alpha \implies f\left(x\right) \geq A\]

On note:
\[\lim_{x \to +\infty} f\left(x\right) = +\infty\]

De la même manière, on dit que $f : E \mapsto F$ définie au voisinage de $+\infty$ tend vers ${\color{red}-\infty}$ lorsque $x \to +\infty$ si:
\[\forall A > 0\ \exists \alpha \in \mathbb{R} \telque x \geq \alpha \implies {\color{red}f\left(x\right) \leq -A}\]

On note:
\[\lim_{x \to +\infty} f\left(x\right) = {\color{red}-\infty}\]

Semblablement, on dit que $f : E \mapsto F$ définie au voisinage de ${\color{blue}-\infty}$ tend vers $+\infty$ lorsque ${\color{blue}x \to -\infty}$ si:
\[\forall A > 0\ \exists \alpha \in \mathbb{R} \telque {\color{blue}x \leq \alpha} \implies f\left(x\right) \geq A\]

On note:
\[\lim_{{\color{blue}x \to -\infty}} f\left(x\right) = +\infty\]

Mêmement (oui c'est un mot qui existe), on dit que $f : E \mapsto F$ définie au voisinage de ${\color{blue}-\infty}$ tend vers ${\color{red}-\infty}$ lorsque ${\color{blue}x \to -\infty}$ si:
\[\forall A > 0\ \exists \alpha \in \mathbb{R} \telque {\color{blue}x \leq \alpha} \implies {\color{red}f\left(x\right) \leq -A}\]

On note:
\[\lim_{{\color{blue}x \to -\infty}} f\left(x\right) = {\color{red}-\infty}\]
}

\subsection{Toutes les limites}
\parag{Récapitulatif}{
    On a défini quatre types de limites:
    \begin{enumerate}
        \item $\lim\limits_{x \to x_0} f\left(x\right) = \ell \in \mathbb{R}$
        \item $\lim\limits_{x \to \pm\infty} f\left(x\right) = \ell \in \mathbb{R}$
        \item $\lim\limits_{x \to x_0} f\left(x\right) = +\infty$ ou $-\infty$
        \item $\lim\limits_{x \to \pm\infty} f\left(x\right) = +\infty$ ou $-\infty$
    \end{enumerate}

    Tous les résultats obtenus pour $\lim_{x \to x_0} $ restent valable pour $\lim_{x \to \pm\infty} $.
}

\parag{Formes indéterminées}{
    \begin{itemize}[left=0pt]
        \item $\infty - \infty$
        \item $\frac{\infty}{\infty}$
        \item $\frac{0}{0}$
        \item $0\cdot \infty$
        \item $0^0$
        \item $1^{\infty}$
        \item $\infty^0$
    \end{itemize}

    Nous verrons les trois dernières plus tard, quand nous aurons défini les puissances réelles.
}

\end{document}
