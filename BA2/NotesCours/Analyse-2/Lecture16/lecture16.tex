\documentclass[a4paper]{article}

% Expanded on 2022-04-11 at 17:22:48.

\usepackage{../../style}

\title{Analyse 2}
\author{Joachim Favre}
\date{Mercredi 13 avril 2022}

\begin{document}
\maketitle

\lecture{16}{2022-04-13}{Retour aux intégrales}{
}

\parag{Application: Changement de variable}{
    Supposons que nous avons le schéma de fonction suivant:
    \[\mathbb{R}^n \over{\mapsto}{$\bvec{h}$} \mathbb{R}^n \over{\mapsto}{$\bvec{g}$} \mathbb{R}^n\]
    tels que $\bvec{h}$ est un changement de variable et $\bvec{g}$ est sa fonction réciproque, i.e.:
    \[\bvec{g} \circ \bvec{h}\left(x_1, \ldots, x_n\right) = \left(x_1, \ldots, x_n\right)\]

    Nous savons donc que le Jacobien de notre composée est:
    \[J_{\bvec{g} \circ \bvec{h}} = \begin{pmatrix} 1 & 0 & \cdots & 0 \\ 0 & 1 & \cdots & 0 \\ \vdots & \vdots & \ddots & \vdots \\ 0 & 0 & \cdots & 1 \end{pmatrix} = I_{n\times n}\]

    Supposons maintenant aussi que $\bvec{h}$ et $\bvec{g}$ sont dérivables sur leurs domaines. Par le théorème de la fonction composée, nous obtenons:
    \begin{multiequation}
        & J_{\bvec{g}}\left(\bvec{h}\left(\bvec{a}\right)\right) \cdot J_{\bvec{h}}\left(\bvec{a}\right) = I_{n \times n} \\
        \implies & J_{\bvec{g}}\left(\bvec{h}\left(\bvec{a}\right)\right) = \left(J_{\bvec{h}}\left(\bvec{a}\right)\right)^{-1} \text{ et } \det\left(J_{\bvec{g}}\right) \det\left(J_{\bvec{h}}\right) = 1
    \end{multiequation}

    Puisqu'une matrice est bijective si et seulement si elle est inversible, nous en déduisons donc la proposition suivante.

    \subparag{Proposition}{
        Soit $\bvec{g} : \mathbb{R}^n \mapsto \mathbb{R}^n$ une fonction dérivable en $\bvec{a}$. $\bvec{g}$ est bijective dans un voisinage de $\bvec{a}$ si et seulement si $\det\left(J_{\bvec{g}}\left(\bvec{a}\right)\right)\neq 0$.
    }
}

\subsection{Dérivée d'une intégrale qui dépend d'un paramètre}

\parag{Théorème}{
    Soit $I \subset \mathbb{R}$ un ensemble ouvert, soit $f: \left[a, b\right] \times I \mapsto \mathbb{R}$ telle que $\frac{\partial f}{\partial y}$ est continue sur $\left[a, b\right] \times I$, et soit:
    \[g\left(y\right) = \int_{a}^{b} f\left(x, y\right)dx\]

    Alors, $g\left(y\right)$ est de classe $C^1$ sur $I$, et nous avons:
    \[g'\left(y\right) = \int_{a}^{b} \frac{\partial f}{\partial y}\left(x, y\right)dx, \mathspace \forall y \in I\]
}

\parag{Rappel: Théorème Fondamental du calcul intégral}{
    Soit $f$ une fonction continue. Alors, nous avons:
    \[\frac{d}{dt}\left(\int_{a}^{t} f\left(y\right)dy\right) = f\left(t\right), \mathspace \frac{d}{dt}\left(\int_{t}^{b} f\left(y\right)dx\right) = \frac{d}{dt}\left(- \int_{b}^{t} f\left(y\right) dx\right) = -f\left(t\right)\]
}

\parag{Théorème}{
    Soient $I, J \subset \mathbb{R}$ deux ensembles ouverts, soient $g, h : I \mapsto \mathbb{R}$ des fonctions continûment dérivables, et soit $f : J \times I \mapsto \mathbb{R}$ une fonction telle que $\frac{\partial f}{\partial t}\left(x, t\right)$ est continue sur $I$. Finalement, soit:
    \[A\left(t\right) = \int_{h\left(t\right)}^{g\left(t\right)} f\left(x, t\right)dx\]

    Alors, $A\left(t\right)$ est continûment dérivable sur $I$, et on a:
    \[A'\left(t\right) = f\left(g\left(t\right), t\right) g'\left(t\right) - f\left(h\left(t\right), t\right)h'\left(t\right) + \int_{h\left(t\right)}^{g\left(t\right)} \frac{\partial f}{\partial t}\left(x, t\right)dx \]

    \subparag{Remarque}{
        Ce théorème doit être connu, car il y a souvent un exercise où nous devons l'utiliser en examen.
    }
}

\end{document}
