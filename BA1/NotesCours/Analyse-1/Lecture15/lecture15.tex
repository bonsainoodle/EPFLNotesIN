\documentclass[a4paper]{article}

% Expanded on 2021-11-10 at 10:09:24.

\usepackage{../../style}

\title{Analyse 1}
\author{Joachim Favre}
\date{Mercredi 10 novembre 2021}

\begin{document}
\maketitle

\lecture{15}{2021-11-10}{Exponentielle et logarithme cupide}{
}

\parag{Propriétés des limites infinies}{
    Ces propriétés fonctionnent de la même manière quand on a $x \to \pm\infty$.

    \begin{enumerate}[left=0pt]
        \item Si $\lim\limits_{x \to x_0} f\left(x\right) = +\infty$ et $\lim\limits_{x \to x_0} g\left(x\right) = +\infty$, alors:
              \[\lim_{x \to x_0} \left(f\left(x\right) + g\left(x\right)\right) = + \infty\]
        \item Si $\lim\limits_{x \to x_0} f\left(x\right) = -\infty$ et $\lim\limits_{x \to x_0} g\left(x\right) = -\infty$, alors:
              \[\lim_{x \to x_0} \left(f\left(x\right) + g\left(x\right)\right) = - \infty\]
        \item Si $\lim\limits_{x \to x_0} f\left(x\right) = \pm \infty$ et que $g\left(x\right)$ est bornée au voisinage de $x_0$:
              \[\lim_{x \to x_0} \left(f\left(x\right) \pm g\left(x\right)\right) = \pm\infty\]
        \item Si $\lim\limits_{x \to x_0} f\left(x\right) = +\infty$ et $\lim\limits_{x \to x_0} g\left(x\right) = \ell \neq 0$, alors:
              \begin{functionbypart}{\lim_{x \to x_0} \left(f\left(x\right)g\left(x\right)\right)}
                  +\infty, \mathspace \text{si } \ell > 0 \\
                  -\infty, \mathspace \text{si } \ell < 0
              \end{functionbypart}
        \item Si $\lim\limits_{x \to x_0} f\left(x\right) = \pm \infty$, alors:
              \[\lim_{x \to x_0} \frac{1}{f\left(x\right)} = 0\]


        \item Si $\lim\limits_{x \to x_0} f\left(x\right) = 0$ et $f\left(x\right) \neq 0$ au voisinage de $x_0$, alors:
              \begin{functionbypart}{\lim_{x \to x_0} \frac{1}{f\left(x\right)}}
                  +\infty, \mathspace \text{si } f\left(x\right) > 0\ \forall x \text{ au voisinage de } x_0 \\
                  -\infty, \mathspace \text{si } f\left(x\right) < 0\ \forall x \text{ au voisinage de } x_0 \\
                  \text{n'existe pas autrement}
              \end{functionbypart}
    \end{enumerate}
}

\parag{Théorème du gendarme pour les limites infinie}{
Si $\lim\limits_{x \to x_0} f\left(x\right) = +\infty$ et que $g\left(x\right) \geq f\left(x\right)$ pour tout $x$ au voisinage de $x_0$, alors:
\[\lim_{x \to x_0} g\left(x\right) = +\infty\]

De la même manière, si $\lim\limits_{x \to x_0} f\left(x\right) = {\color{red}-\infty}$ et que $g\left(x\right)\ {\color{red}\leq}\ f\left(x\right)$ pour tout $x$ au voisinage de $x_0$, alors:
\[\lim_{x \to x_0} g\left(x\right) = {\color{red}-\infty}\]

\subparag{Remarque}{
    Ce théorème fonctionne aussi quand $x \to \pm \infty$.
}
}

\subsection{Limites à droite et à gauche}
\parag{Définition}{
$f: E \mapsto F$ est définie à droite de $x_0$ s'il existe $\alpha > 0$ tel que $\left]x_0, x_0 + \alpha\right[ \subset E$.

Mêmement, $f: E \mapsto F$ est définie à \textcolor{red}{gauche} de $x_0$ s'il existe $\alpha > 0$ tel que ${\color{red}\left]x_0 - \alpha, x_0\right[ } \subset E$.
}

\parag{Définition des limites à droite et à gauche}{
    $f: E \mapsto F$ définie à droite de $x_0$ admet pour limite à droite de $x_0$ le nombre réel $\ell$ si $\forall \epsilon > 0$ $\exists \delta > 0$ tel que, pour tout $x \in E$:
    \[0 < x - x_0 \leq \delta \implies \left|f\left(x\right) - \ell\right| \leq \epsilon\]

    On note:
    \[\lim_{x \to x_0^+} f\left(x\right) = \ell\]

    De manière similaire, $f: E \mapsto F$ définie à \textcolor{red}{gauche} de $x_0$ admet pour limite à \textcolor{red}{gauche} de $x_0$ le nombre réel $\ell$ si $\forall \epsilon > 0$ $\exists \delta > 0$ tel que, pour tout $x \in E$:
    \[\forall x \in E: 0 < {\color{red}x_0 - x} \leq \delta \implies \left|f\left(x\right) - \ell \leq \epsilon\right|\]

    On note:
    \[\lim_{{\color{red}x \to x_0^-}} f\left(x\right) = \ell\]

    \subparag{Remarque}{
        C'est la même définition que ce que les limites ``normales'', sauf que nous enlevons les valeurs absolues.
    }

    \subparag{Observation}{
        On remarque que:
        \[\lim_{x \to x_0} f\left(x\right) = \ell \iff \lim_{x \to x_0^+} f\left(x\right) = \ell \text{ et } \lim_{x \to x_0^-} f\left(x\right) = \ell\]

        On peut démontrer les deux directions de l'implication à l'aide de la définition de ces limites.
    }
}

\subsection{Fonction exponentielle et logarithmique}
\parag{Définition fonction exponentielle}{
    Nous définissons:
    \[e^x \over{=}{déf} \sum_{n=0}^{\infty} \frac{x^n}{n!}, \mathspace x \in \mathbb{R}\]

    \subparag{Rappel}{
        On sait que cette série converge absolument pour tout $x \in \mathbb{R}$ par le critère de d'Alembert.
    }
}

\parag{Convention}{
    On prend $0^0 = 1$ et $0! = 1$.
}

\parag{Proposition}{
    Pour tout $x, y \in \mathbb{R}$:
    \begin{enumerate}
        \item $e^{x + y} = e^x e^y$
        \item $e^{-x} = \frac{1}{e^x}$
        \item $e^x > 0$
    \end{enumerate}
}

\parag{Propriétés}{
    \begin{enumerate}[left=0pt]
        \item $\lim\limits_{x \to \infty} e^x = +\infty$
        \item $\lim\limits_{x \to -\infty} e^x = 0$
        \item $\left(e^x\right)\uparrow$ pour tout $x \in \mathbb{R}$
    \end{enumerate}

    On en déduit que $e^{x} : \mathbb{R} \mapsto \mathbb{R}^*_+$ est injective, puisqu'elle est strictement croissante. On verra aussi plus tard qu'elle est surjective, et nous n'utiliserons pas ce que nous allons faire ci-dessous pour le démontrer, donc nous allons, pour l'instant, simplement le prendre comme tel.

    Ainsi, on sait que $e^x : \mathbb{R} \mapsto \mathbb{R}^*_+$ est bijective.
}

\parag{Fonction réciproque}{
    On sait que $e^{x} : \mathbb{R} \mapsto \mathbb{R}^*_+$ est bijective, donc on peut définir la fonction réciproque. On appelle cette fonction le logarithme naturel, ou logarithme Népérien.

    On sait dont qu'il existe $\log\left(x\right) : \mathbb{R}^*_+ \mapsto \mathbb{R}$ telle que:
    \[e^x = y \iff x = \log\left(y\right), \mathspace \forall x \in \mathbb{R}, \forall y \in \mathbb{R}^*_+\]
}

\parag{Proposition}{
Les limites suivantes sont remarquables:
\[\lim_{x \to 0} \frac{e^x - 1}{x} = 1\]

\subparag{Remarque}{
    On utilise normalement cette limite pour démontrer la dérivée de l'exponentielle.
}

\[\lim_{x \to 0} \frac{\log\left(1 + x\right)}{x} = 1\]

\subparag{Remarque}{
    Cette limite se démontre avec le changement de variable $x = \log\left(y\right)$ puis $y = 1 + z$.
}

\[\lim_{x \to 0} \left(1+x\right)^{\frac{1}{x}} = e\]

\subparag{Remarque}{
    Cette limite se démontre grâce à la continuité de la fonction exponentielle (on passe à l'exponentielle des deux côtés).
}
}

\parag{Propriétés}{
    \begin{enumerate}[left=0pt]
        \item $e^{\log\left(x\right)} = x, \mathspace \forall x \in \mathbb{R}^*_+$
        \item $\log\left(e^x\right) = x, \mathspace \forall x \in \mathbb{R}$
        \item $\log\left(xy\right) = \log\left(x\right) + \log\left(y\right)$
        \item $\log\left(\frac{x}{y}\right) = \log\left(x\right) - \log\left(y\right)$
        \item $\log\left(x^{r}\right) = r\log\left(x\right), \mathspace \forall r \in \mathbb{N}^*$
        \item $\log\left(1\right) = 0$
        \item $\log\left(e\right) = 1$
    \end{enumerate}
}

\end{document}
