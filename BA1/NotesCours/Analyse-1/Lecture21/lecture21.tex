\documentclass[a4paper]{article}

% Expanded on 2021-12-06 at 10:18:50.

\usepackage{../../style}

\title{Analyse 1}
\author{Joachim Favre}
\date{Lundi 06 décembre 2021}

\begin{document}
\maketitle

\lecture{21}{2021-12-06}{Propriétés des DL, et rayons de convergence}{
}

\subsection{Retour aux développements limités}
\parag{Proposition: opérations algébriques sur les DL}{
    Soient $f,g : E \mapsto \mathbb{R}$ deux fonction admettant un DL autour de $x = a$:
    \[f\left(x\right) = a_0 + a_1\left(x - a\right) + \ldots + a_n\left(x - a\right)^n + \left(x - a\right)^n \epsilon_1\left(x\right) = P_f^n\left(x\right) + \left(x - a\right)^n \epsilon_1\left(x\right)\]
    \[g\left(x\right) = b_0 + b_1\left(x - a\right) + \ldots + b_n\left(x - a\right)^n + \left(x - a\right)^n \epsilon_2\left(x\right) = P_g^n\left(x\right) + \left(x - a\right)^n \epsilon_2\left(x\right)\]
    \[\lim_{x \to a} \epsilon_1\left(x\right) = 0, \mathspace \lim_{x \to a} \epsilon_2\left(x\right) = 0\]

    Alors:
    \begin{enumerate}
        \item $\alpha f\left(x\right) + \beta g\left(x\right)$ admet un DL d'ordre $n$ autour de $x = a$ et:
              \[P^n_{\alpha f + \beta g}\left(x\right) = \alpha P_{f}^n\left(x\right) + \beta P_g^n\left(x\right)\]
        \item $f\left(x\right)g\left(x\right)$ admet aussi un DL d'ordre $n$:
              \[P_{f \cdot g}^n\left(x\right) = P_f^n\left(x\right) P_g^n\left(x\right)\]
              où on ne conserve que les termes d'ordre $\leq n$.

        \item Si $b_0 \neq 0$ et $g\left(x\right) \neq 0$, alors:
              \[P^n_{\frac{f}{g}}\left(x\right) = \frac{P_f^n\left(x\right)}{P_{g}^n\left(x\right)}\]

    \end{enumerate}
}

\parag{Proposition: DL d'une fonction composée}{
    Soient $f$ et $g$, deux fonctions, où $f$ admet un DL autour de $x = a$, et $g$ autour de $y = 0$:
    \[f\left(x\right) = a_1\left(x - a\right) + \ldots + a_n\left(x - a\right)^n + \left(x - a\right)^n \epsilon_1\left(x\right)\]
    \[g\left(y\right) = g\left(0\right) + b_1 y + \ldots + b_n y^n + y^n \epsilon_2\left(y\right)\]

    Alors, $g \circ f$ admet un DL d'ordre $n$ autour de $x = a$, donné par:
    \[P_{g \circ f}^n\left(x\right) = g\left(0\right) + b_1\left(P_f^n\left(x - a\right)\right) + b_2\left(P_f^n\left(x - a\right)\right)^2 + \ldots + b_n\left(P_f^n\left(x - a\right)^n\right)\]
    où on ne conserve que les termes d'ordre $\leq n$.
}

\subsection{Développements limités pour le calcul des limites}
\parag{Note 1}{
    Si on n'avait pas calculé assez de terme pour un développement limité, on se serait retrouvé avec une forme indéterminée de la forme $\frac{\epsilon_1\left(x\right)}{\epsilon_2\left(x\right)}$. Si on avait calculé trop de termes, alors ils tendraient juste vers 0, donc ce ne serait pas un problème.
}

\parag{Note 2}{
    Cette méthode est plus puissante que Bernoulli-L'Hospital. Si BL fonctionne, alors cette méthode fonctionne.
}

\section{Séries entières}
\subsection{Rayon de convergence}
\parag{Définition}{
    On appelle \important{série entière} les expressions sous la forme:
    \[\sum_{k=0}^{\infty} a_k\left(x - x_0\right)^k, \mathspace a_k \in \mathbb{R}\ \forall k \in \mathbb{N}\]

    Le \important{domaine de convergence} est défini par:
    \[D = \left\{x \in \mathbb{R} \telque \sum_{k=0}^{\infty} a_k\left(x - x_0^k\right) \text{ converge}\right\}\]

    La fonction $f\left(x\right) = \sum_{k=0}^{\infty} a_k\left(x - x_0\right)^k$ avec $x\in D$ est définie par la série entière.
}

\parag{Théorème: rayon de convergence}{
    Soit la série entière $\sum_{k=0}^{\infty} a_k\left(x - x_0\right)^{k}$.

    Il existe $r$, son \important{rayon de convergence}, où $0 \leq r \leq +\infty$, tel que:
    \begin{enumerate}
        \item La série converge absolument pour tout $x$ tels que $\left|x - x_0\right| < r$.
        \item La série diverge pour tout $x$ tels que $\left|x - x_0\right| > r$
    \end{enumerate}

    \subparag{Remarque 1}{
        La convergence de la série entière en $x = x_0 \pm r$ doit être étudié séparément.
    }

    \subparag{Remarque 2}{
        On voit donc que $D$ est un intervalle qui contient $x_0$, et centré en $x_0$.
    }
}

\parag{Proposition}{
    Soit $\sum_{k=0}^{\infty} a_k\left(x - x_0\right)^k$ une série entière de rayon de convergence $r$.
    \begin{enumerate}
        \item Supposons que $a_k \neq 0$ pour tout $k \in \mathbb{N}$. Si $\lim_{k \to \infty} \left|\frac{a_{k+1}}{a_k}\right| = \ell$, où $0 \leq \ell \leq +\infty$, alors $r = \frac{1}{\ell}$.

              On a donc:
              \[r = \lim_{k \to \infty} \left|\frac{a_k}{a_{k+1}}\right|\]

        \item Si $\lim_{k \to \infty} \left|a_k\right|^{\frac{1}{k}} = \ell$ avec $0 \leq \ell \leq +\infty$, alors $r = \frac{1}{\ell}$.
    \end{enumerate}
}

\end{document}
