\documentclass[a4paper]{article}

% Expanded on 2021-11-25 at 08:15:36.

\usepackage{../../style}

\title{Algèbre linéaire}
\author{Joachim Favre}
\date{Jeudi 25 novembre 2021}

\begin{document}
\maketitle

\lecture{19}{2021-11-25}{Norme et orthogonalité}{
}

\parag{Propriétés}{
    Soient $\bvec{u}, \bvec{v}, \bvec{w} \in \mathbb{R}^n$ et $c \in \mathbb{R}$. On a:
    \begin{enumerate}
        \item $\bvec{u} \dotprod\bvec{v} = \bvec{v}\dotprod \bvec{u}$
        \item $\left(\bvec{u} + \bvec{v}\right)\dotprod\bvec{w} = \bvec{u}\dotprod \bvec{w} + \bvec{v} \dotprod \bvec{w}$
        \item $c\left(\bvec{u}\right)\dotprod \bvec{v} = c\left(\bvec{u} \dotprod \bvec{v}\right) = \bvec{u}\dotprod\left(c \bvec{v}\right)$
        \item $\bvec{u} \dotprod \bvec{u} \geq 0$
        \item $\bvec{u} \dotprod \bvec{u} = 0 \iff\bvec{u} = \bvec{0}$.
    \end{enumerate}
}

\parag{Définition de norme}{
    Soit $\bvec{v}$ un vecteur de $\mathbb{R}^{n}$.

    La \important{norme} (ou longueur) de $\bvec{v}$ est le scalaire positif ou nul noté $\left\|\bvec{v}\right\|$ défini par:
    \[\left\|\bvec{v}\right\| = \sqrt{\bvec{v} \dotprod \bvec{v}} = \sqrt{v_1^2 + v_2^2 + \ldots + v_n^2}\]
}

\parag{Normalisation}{
    Si $\left\|\bvec{u}\right\| = 1$, on dit que $\bvec{u}$ est un \important{vecteur unitaire}.

    Pour $\bvec{v}$ non-nul, on peut toujours prend le vecteur $\bvec{u}$ suivant, de sorte que $\bvec{u}$ est unitaire et pointe dans la même direction que $\bvec{v}$:
    \[\bvec{u} = \frac{1}{\left\|\bvec{v}\right\|}\bvec{v}\]

    On appelle cela la \important{normalisation} de $\bvec{v}$.
}

\parag{Définition de distance}{
    Soient $\bvec{u}$ et $\bvec{v}$ deux vecteurs de $\mathbb{R}^n$.

    On définit la \important{distance} entre $\bvec{u}$ et $\bvec{v}$ comme:
    \[\dist\left(\bvec{u}, \bvec{v}\right) = \left\|\bvec{v} - \bvec{u}\right\|\]

    \subparag{Observation}{
        On peut voir que:
        \[\dist\left(\bvec{u}, \bvec{v}\right) = \dist\left(\bvec{v}, \bvec{u}\right) = \left\|\bvec{u} - \bvec{v}\right\| = \left\|\bvec{v} - \bvec{u}\right\|\]
    }
}

\parag{Définition d'orthogonalité}{
    On dit que deux vecteurs $\bvec{u}, \bvec{v} \in \mathbb{R}^n$ sont \important{orthogonaux} (l'un à l'autre) si:
    \[\bvec{u} \dotprod \bvec{v} = \bvec{0}\]

    \subparag{Observation}{
        On remarque que $\bvec{0}$ est orthogonal avec n'importe quel vecteur puisque:
        \[\bvec{0} \dotprod \bvec{v} = 0\cdot v_1 + \ldots + 0\cdot v_n = 0\]
    }
}

\parag{Théorème de Pythagore}{
    $\bvec{u}, \bvec{v} \in \mathbb{R}^{n}$ sont orthogonaux si et seulement si:
    \[\left\|\bvec{u} + \bvec{v}\right\|^2 = \left\|\bvec{u}\right\|^2 + \left\|\bvec{v}\right\|^2\]
}

\parag{Théorème de Pythagore généralisé}{
    \imagehere[0.7]{Pythagore.jpg}
}

\subsection{Orthogonalité à tout un sous-espace vectoriel}

\parag{Définition}{
    Soit $W$ un sous-espace vectoriel de $\mathbb{R}^{n}$.

    On note $W^{\perp}$ l'ensemble des vecteurs de $\mathbb{R}^{n}$ qui sont orthogonaux à \textit{tous} les vecteurs de $W$. On appelle $W^{\perp}$ le \important{complément orthogonal de $W$} (ou ``l'orthogonal de $W$'')
}

\parag{Théorème}{
    $W^{\perp}$ est un sous-espace vectoriel de $\mathbb{R}^n$.
}

\parag{Théorème}{
    \[\left(W^{\perp}\right)^{\perp} = W\]
}

\parag{Théorème}{
    Si $W = \vect\left\{\bvec{w}_1, \ldots, \bvec{w}_p\right\}$, alors $\bvec{u} \in W^{\perp}$ si et seulement si:
    \[\bvec{u} \dotprod \bvec{w}_i = 0, \mathspace \forall i = 1, \ldots, p\]
}

\end{document}
