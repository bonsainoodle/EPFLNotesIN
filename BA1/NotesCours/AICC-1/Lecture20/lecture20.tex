\documentclass[a4paper]{article}

% Expanded on 2021-11-24 at 15:14:17.

\usepackage{../../style}

\title{AICC}
\author{Joachim Favre}
\date{Mercredi 24 novembre 2021}

\begin{document}
\maketitle

\lecture{20}{2021-11-24}{Repetitions and Pascal's triangle}{
}

\parag{Theorem}{
    The number of $r$-combinations of a set with $n$ elements, where $n \geq r \geq 0$, equals:
    \[C\left(n,r\right) = \frac{n!}{\left(n - r\right)!r!}\]
}

\parag{Corollary}{
    Let $n$ and $r$ be nonnegative integers where $n \geq r \geq 0$. Then:
    \[C\left(n, n - r\right) = C\left(n, r\right)\]
}

\parag{Vocabulary}{
    The kind/face of a card is its value, its colour/suit is its ``type'' (there are four of them: club, diamond, heart and spade).
}

\parag{Definition}{
    An \important{$r$-permutation with repetitions} of a set of distinct objects is an ordered arrangement of $r$ elements from the set, where elements can occur multiple times.
}

\parag{Theorem}{
    The number of $r$-permutations of a set of $n$ objects with repetitions allowed is;
    \[n^r\]
}

\parag{Definition}{
    An \important{$r$-combination} with repetition of elements of a set is an unordered selection of $r$ elements from the set, where elements can occur multiple times.
}

\parag{Example}{
    Let's say we want to count the number of ways there are to select four pieces of apples, oranges, and pears if the order does not matter and the fruits are indistinguishable.

    The size of the set is $n = 3$, and we want to select $r = 4$ elements with repetitions. We can represent this as 3 compartiments: one for each fruit. We can do the following drawing where the points on the left of the two bars are apples, the ones that are between two bars are oranges, and the ones on the rightmost end are pears. On every line, there are fours dots, representing the fact that there are 4 fruits:

    \begin{center}
        \begin{tabular}{cccccc}
            |         & |         & $\bullet$ & $\bullet$ & $\bullet$ & $\bullet$ \\
            |         & $\bullet$ & |         & $\bullet$ & $\bullet$ & $\bullet$ \\
            |         & $\bullet$ & $\bullet$ & |         & $\bullet$ & $\bullet$ \\
            |         & $\bullet$ & $\bullet$ & $\bullet$ & |         & $\bullet$ \\
            |         & $\bullet$ & $\bullet$ & $\bullet$ & $\bullet$ & |         \\
            $\bullet$ & |         & |         & $\bullet$ & $\bullet$ & $\bullet$ \\
            $\bullet$ & |         & $\bullet$ & |         & $\bullet$ & $\bullet$ \\
            $\bullet$ & |         & $\bullet$ & $\bullet$ & |         & $\bullet$ \\
            $\bullet$ & |         & $\bullet$ & $\bullet$ & $\bullet$ & |         \\
            $\bullet$ & $\bullet$ & |         & |         & $\bullet$ & $\bullet$ \\
            $\bullet$ & $\bullet$ & |         & $\bullet$ & |         & $\bullet$ \\
            $\bullet$ & $\bullet$ & |         & $\bullet$ & $\bullet$ & |         \\
            $\bullet$ & $\bullet$ & $\bullet$ & |         & |         & $\bullet$ \\
            $\bullet$ & $\bullet$ & $\bullet$ & |         & $\bullet$ & |         \\
            $\bullet$ & $\bullet$ & $\bullet$ & $\bullet$ & |         & |         \\
        \end{tabular}
    \end{center}


    We can see that we, in fact, we want to choose the $n - 1 = 2$ positions for the bar amongst the $r + n - 1 = 6$ possible. Another way of seeing that, is that we want to choose $r$ positions for the fruits amongst the $r + n - 1 = 6$ possible. Thus, we have:
    \[C\left(n + r - 1, n - 1\right) = C\left(n + r - 1, r\right) = C\left(6, 2\right) = \frac{6!}{4!2!} = 15\]
}

\parag{Theorem}{
    The number of $r$-combinations from a set with $n$ elements when repetition of elements is allowed is:
    \[C\left(n + r - 1, r\right) = C\left(n + r - 1, n - 1\right)\]
}

\parag{Theorem}{
    The number of different permutations of $n$ objects, where there are $n_1$ indistinguishable objects type 1, $n_2$ of type 2, and so on until $n_k$ indistinguishable objects of type $k$, is:
    \[\frac{n!}{n_1! n_2! \cdot \ldots \cdot n_k!}\]
}

\parag{Summary}{
    We can make the following summary:
    \imagehere[0.7]{PermutationsCombinationsSummary.png}

    Moreover, if we take $n = 4$, $r = 2$ and $S = \left\{1, 2, 3, 4\right\}$, we have:
    \imagehere{PermutationsCombinationsSummaryExample.png}
}

\subsection{Combinatorial Proofs}

\parag{Bijection principle}{
    Let $A$ and $B$ be two finite sets. If there exists a bijection $f: A \to B$, then $\left|A\right| = \left|B\right|$.
}

\parag{Double counting principle}{
    Let $A$ be a finite set. If there exists two different ways of counting the elements of $A$, then the two ways of counting must yield the same result.
}

\subsection{Binomial theorem}

\parag{Binomial theorem}{
    Let $x$ and $y$ be variables, and $n$ a nonnegative integer. Then:
    \[\left(x + y\right)^n = \sum_{j=0}^{n} \binom{n}{j} x^{n - j} y^{j}\]

    Thus, the coefficients of the expansion of the powers of $x + y$ are related to the number of combinations.
}

\parag{Corollary}{
    Let $n \geq 0$. We have:
    \[\sum_{k = 0}^{n} \binom{n}{k} = 2^{n}\]
}

\parag{Pascal's identity}{
    If $n$ and $k$ are integers where $n \geq k \geq 0$, then:
    \[\binom{n+1}{k} = \binom{n}{k - 1} + \binom{n}{k}\]

    \subparag{Usefulness}{
        This is a recursive multiplication-free definition of the binomial coefficient.
    }
}

\parag{Property}{
    The following identity holds:
    \[\sum_{k=1}^{n} k \binom{n}{k} = n2^{n-1}\]
}

\end{document}
