\documentclass{article}

% Expanded on 2021-09-21 at 23:02:17.

\usepackage{../../style}

\title{Analyse 1}
\author{Joachim Favre}
\date{Mercredi 22 septembre 2021}

\begin{document}
\maketitle

\lecture{1}{2021-09-22}{Organisation et prérequis}{
}


\section{Prérequis}
\subsection{Identités algébriques}
\parag{Polynomiales}{
    \begin{itemize}[left=0pt]
        \item $\left(x + y\right)^2 = x^2 + 2xy + y^2$
        \item $x^2 - y^2 = (x + y)(x - y)$
        \item $x^3 - y^3 = (x - y)(x^2 + xy + y^2)$
        \item $x^3 + y^3 = \left(x + y\right)\left(x^2 -xy + y^2\right)$
    \end{itemize}

}

\parag{Exponentielles}{
    Soit $a, b$ nombre réels positifs; $x, y$ nombres réels, $n$ naturel positif:
    \begin{itemize}
        \item $a^x a^y = a^{x+y}$
        \item $\frac{a^x}{a^y} = a^{x-y}$
        \item $\left(ab\right)^x = a^x b^x$
        \item $a^0$ = 1
        \item $\left(a^x\right)^y = a^{x\cdot y}$
        \item $\sqrt[n]{a} = a^{\frac{1}{n}}$
        \item $\left(\frac{a}{b}\right)^x = \frac{a^x}{b^x}$
        \item $a^1 = a$
    \end{itemize}

}

\parag{Logarithmes}{
    Dans ce cours, $\log\left(x\right) = \log_e\left(x\right)$. Avec $x, y$ réels positifs:
    \begin{itemize}
        \item $\log\left(xy\right) = \log\left(x\right) + \log\left(y\right)$
        \item $\log\left(\frac{x}{y}\right) = \log\left(x\right) - \log\left(y\right)$
        \item $\log\left(x^c\right) = c\log\left(x\right) \mathspace c\in\mathbb{R}$
        \item $\log_a\left(1\right) = 0 \mathspace a \in \mathbb{R} \setminus \left\{1\right\}$
        \item $\log_a\left(a\right)=1$
    \end{itemize}

}

\parag{Trigonométrie}{
    On définit $\sin\left(x\right)$ et $\cos\left(x\right)$ comme projection pour tout $x$ réel. On définit
    \[\tg\left(x\right) = \frac{\sin\left(x\right)}{\cos\left(x\right)}, \cos\left(x\right) \neq 0 \mathspace \text{ et } \mathspace \ctg\left(x\right) = \frac{\cos\left(x\right)}{\sin\left(x\right)}, \sin\left(x\right) \neq 0\]

    Il est important de connaître la propriété suivante:
    \[\sin\left(x \pm y\right) = \sin\left(x\right)\cos\left(y\right) \pm \cos\left(x\right)\sin\left(y\right) \mathspace \text{ et } \mathspace \cos\left(x \pm y\right) = \cos\left(x\right)\cos\left(y\right) \mp \sin\left(x\right)\sin\left(y\right)\]
}

\subsection{Transformation des graphiques}
\parag{Remarque}{
    On fait monter le graphique de $f\left(x\right)$ en faisant $f\left(x\right) + c$, ou on déplace le graphique \textbf{vers la gauche} en faisant $f\left(x + c\right)$.

    En prenant $cf\left(x\right)$, il faut étendre le graphique de $f\left(x\right)$ en direction verticale (si $\left|c\right| > 1$).

    En prenant $f\left(cx\right)$, alors il faut serrer le graphique dans la direction horizontale (si $\left|c\right| > 1$, sinon (si on divise par une valeur, en gros) il faut l'étendre).
}

\end{document}
