\documentclass[a4paper]{article}

% Expanded on 2021-11-22 at 10:16:16.

\usepackage{../../style}

\title{Analyse 1}
\author{Joachim Favre}
\date{Lundi 22 novembre 2021}

\begin{document}
\maketitle

\lecture{17}{2021-11-22}{Des calculs de différences, c'est facile non?}{
}


\parag{Corollaire 2}{
    Soient $I$ un intervalle ouvert et $f : I \mapsto \mathbb{R}$ une fonction continue strictement monotone. Alors, $f\left(I\right)$ est un intervalle ouvert
}

\parag{Corollaire 3}{
    Toute fonction injective continue sur un intervalle est strictement monotone.
}

\parag{Corollaire 4}{
    Toute fonction bijective continue sur un intervalle admet une fonction réciproque qui est continue (et strictement monotone par le corollaire 3 - étant donné qu'elle est bijective, elle est injective).
}

\section{Calcul différentiel}
\subsection{Fonctions dérivables}
\parag{Définition de fonction dérivable}{
    Une fonction $f: E \mapsto F$ est dite \important{dérivable} en $x_0 \in E$ si la limite suivante existe (et est finie):
    \[\lim_{x \to x_0} \frac{f\left(x\right) - f\left(x_0\right)}{x - x_0} \over{=}{déf} f'\left(x_0\right) \in \mathbb{R}\]

    Cette limite est appelée la \important{dérivée} de $f$ en $x_0$, notée $f'\left(x_0\right)$.

    \subparag{Note personnelle}{
        En faisant le changement de variable $h = x - x_0$, on a:
        \[f'\left(x_0\right) = \lim_{x \to x_0} \frac{f\left(x\right) - f\left(x_0\right)}{x - x_0} = \lim_{h \to 0} \frac{f\left(x_0 + h\right) - f\left(x_0\right)}{h}\]

        Cela peut être plus pratique pour les calculs.
    }

}

\parag{Fonction différentiable}{
    Si $f$ est dérivable en $x = x_0$, on peut écrire:
    \[f\left(x\right) = f\left(x_0\right) + f'\left(x_0\right)\left(x - x_0\right) + r\left(x\right), \mathspace \text{où } r\left(x\right) \over{=}{déf} f\left(x\right) - f\left(x_0\right) - f'\left(x_0\right)\left(x - x_0\right)\]

    On remarque que:
    \[\lim_{x \to x_0} \frac{r\left(x\right)}{x - x_0} = \lim_{x \to x_0}  \left(\underbrace{\frac{f\left(x\right) - f\left(x_0\right)}{x - x_0}}_{\to f'\left(x_0\right)} - f'\left(x_0\right)\right) = 0\]

    Ainsi, on en déduit que toute fonction dérivable en $x = x_0$ admet une présentation:
    \[f\left(x\right) = f\left(x_0\right) + a\left(x - x_0\right) + r\left(x\right), \mathspace \text{où } \lim_{x \to x_0} \frac{r\left(x\right)}{x - x_0} = 0\]

    Dans ce cas, on dit que $f$ est \important{différentiable} en $x_0$.

    Réciproquement, si:
    \[f\left(x\right) = f\left(x_0\right) + a\left(x - x_0\right) + r\left(x\right), \mathspace \telque \lim_{x \to x_0} \frac{r\left(x\right)}{x - x_0} = 0\]

    Alors:
    \[\lim_{x \to x_0} \frac{f\left(x\right) - f\left(x_0\right)}{x - x_0} = \lim_{x \to x_0} \underbrace{\frac{a\left(x - x_0\right)}{x - x_0}}_{= a} + \underbrace{\frac{r\left(x\right)}{x - x_0}}_{\to 0} = a\]

    Ainsi, on en déduit que $f$ est dérivable en $x_0$ si et seulement si elle est différentiable en $x = x_0$, et $f'\left(x\right) = a$.
}

\parag{Définition de fonction dérivée}{
    Soit $f : E \mapsto F$ une fonction dérivable sur un ensemble $D\left(f'\right) \subset E$.

    On définit la \important{fonction dérivée}:
    \[\begin{split}
            f': D\left(f'\right) &\longmapsto \mathbb{R} \\
            x &\longmapsto f'\left(x\right)
        \end{split}\]
}

\parag{Equation de la tangente}{
    Soit $f : E \mapsto F$ une fonction dérivable en $x = x_0$. L'équation de la tangente au graphe de $f$ en $x = x_0$ est:
    \[y = f\left(x_0\right) + f'\left(x_0\right)\left(x - x_0\right)\]
}

\parag{Définition des dérivées sur le côté}{
La \important{dérivée à droite} est définie par:
\[f'_d\left(x_0\right) \over{=}{déf}  \lim_{x \to x_0^+} \frac{f\left(x\right) - f\left(x_0\right)}{x - x_0}\]

De la même manière, la \important{dérivée à gauche} est définie par:
\[f'_{{\color{red}g}}\left(x_0\right) \over{=}{déf} \lim_{{\color{red}x \to x_0^-}} \frac{f\left(x\right) - f\left(x_0\right)}{x - x_0} \]

On remarque que $f'\left(x_0\right)$ existe si et seulement s'il existe $f'_d\left(x_0\right)$ et $f'_g\left(x_0\right)$, et que $f'_d\left(x_0\right) = f'_g\left(x_0\right)$
}

\parag{Proposition}{
    Une fonction dérivable en $x = x_0$ est continue en $x = x_0$.

    \subparag{Réciproque}{
        La réciproque de cette proposition est fausse; une fonction continue n'est pas nécessairement dérivable.
    }
}

\parag{Remarque}{
    On peut introduire la limite infinie si:
    \[\lim_{x \to x_0} \frac{f\left(x\right) - f\left(x_0\right)}{x - x_0} = \pm \infty\]

    Puisque la limite n'existe pas, $f$ n'est pas dérivable en $x = x_0$.

    Dans ce cas, le graphique de $f$ admet une tangente verticale en $x_0$.
}

\parag{Opérations algébriques sur les dérivées}{
    Soient $f, g : E \mapsto F$ deux fonctions dérivables en $x = x_0$. Alors:
    \begin{enumerate}
        \item $\left(\alpha f + \beta g\right)'\left(x_0\right) = \alpha f'\left(x_0\right) + \beta g'\left(x_0\right), \mathspace \forall \alpha, \beta \in \mathbb{R}$
        \item $\left(f\cdot g\right)'\left(x_0\right) = f'\left(x_0\right)g\left(x_0\right) + f\left(x_0\right)g'\left(x_0\right)$
        \item $\left(\dfrac{f}{g}\right)'\left(x_0\right) = \dfrac{f'\left(x_0\right)g\left(x_0\right) - g'\left(x_0\right)f\left(x_0\right)}{g^2\left(x\right)}$, si $g\left(x\right) \neq 0$ au voisinage de $x_0$
    \end{enumerate}
}

\parag{Proposition}{
    Soit $f\left(x\right) = x^{n}$, où $n \in \mathbb{N}^*$. On a:
    \[\left(x^{n}\right)' = nx^{n - 1}, \mathspace \forall x \in \mathbb{R}\]
}

\parag{Proposition (dérivée de la fonction composée de deux fonctions dérivables)}{
    Soient $f : E\mapsto F$ une fonction dérivable en $x = x_0 \in E$ et $g : G \mapsto H$ dérivable en $f\left(x_0\right)$, où $f\left(E\right) \subset G$.

    Nous avons:
    \[\left(g \circ f\right)'\left(x_0\right) = g'\left(f\left(x_0\right)\right)f'\left(x_0\right)\]
}

\end{document}
