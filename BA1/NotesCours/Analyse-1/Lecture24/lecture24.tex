\documentclass[a4paper]{article}

% Expanded on 2021-12-15 at 10:21:32.

\usepackage{../../style}

\title{Analyse}
\author{Joachim Favre}
\date{Mercredi 15 décembre 2021}

\begin{document}
\maketitle

\lecture{24}{2021-12-15}{''Les intégrales c'est un art'' (prof. Lachowska)}{
}

\parag{Proposition: intégration par partie}{
Soient $g, f : I \mapsto \mathbb{R}$ des fonctions continûment dérivables sur $I$, où $\left[a, b\right] \subset I$. Alors:
\[\int_{a}^{b} f\left(x\right) g'\left(x\right)dx = f\left(x\right)g\left(x\right) \eval_{a}^{b} - \int_{a}^{b} g\left(x\right) f'\left(x\right)dx\]

\subparag{Utilité}{
    Cette méthode marche très bien pour:
    \begin{itemize}
        \item polynômes$\displaystyle \cdot \left(\log x\right)^k$
        \item polynômes$\displaystyle \cdot \sin\left(x\right)$
        \item polynômes$\displaystyle \cdot \cos\left(x\right)$
        \item polynômes$\displaystyle \cdot e^x$
    \end{itemize}
}


\subparag{Technique}{
    Méthode DI, de BlackPenRedPen.
}

\parag{Intégration des fonctions rationnelles}{
    L'intégrale de fonctions rationnelles, $\int \frac{P\left(x\right)}{Q\left(x\right)}dx$, s'exprime toujours en termes de fonctions élémentaires.

    Pour commencer, on sait qu'on peut toujours calculer la division de nos deux polynômes, de manière à avoir un polynôme (dont l'intégrale est très facile à calculer) auquel on ajoute un reste. Ce reste est toujours de telle forme que son numérateur est de degré est strictement inférieur à celui de son dénominateur. De plus, on sait qu'on peut toujours factoriser un polynôme à coefficients réels à l'aide de polynômes à coefficients réels de degré au plus 2. Ainsi, on peut toujours appliquer les fonctions partielles pour séparer notre reste en termes de la forme:
    \begin{enumerate}
        \item $\displaystyle \frac{1}{ax + b}, \mathspace a \neq 0$
        \item $\displaystyle \frac{\left(cx + d\right)}{\left(x - a\right)\left(x - b\right)}, \mathspace a \neq b$
        \item $\displaystyle \frac{1}{\left(ax + b\right)^k}, \mathspace k \geq 2$
        \item $\displaystyle \frac{1}{x^2 + px + q}, \mathspace p^2 - 4q < 0$
        \item $\displaystyle \frac{x}{x^2 + px + q}, \mathspace p^2 - 4q < 0$
        \item $\displaystyle \frac{1}{\left(1 +  x^2\right)^n}, \mathspace n \geq 2$
        \item $\displaystyle \frac{x}{\left(1 + x^2\right)^n}, \mathspace n \geq 2$
    \end{enumerate}

    Si les résultats de ces intégrales se sont pas connus, se référer à au notes complètes de Joachim.

    \subparag{Technique}{
        ``Cover-up method'' de BlackPenRedPen.
    }

}

\parag{Aire d'une ellipse}{
    L'équation des ellipses est donné par:
    \[\frac{y^2}{b^2} + \frac{x^2}{a^2} = 1 \implies y^2 = b^2 - \frac{b^2}{a^2} x^2 \implies y = \pm b\sqrt{1 - \frac{x^2}{a^2}}\]

    Ainsi, on peut prendre la partie positive, et l'intégrer entre $0$ et $a$ de manière à uniquement avoir $\frac{1}{4}$ de l'aire de l'ellipse.
}

\end{document}
