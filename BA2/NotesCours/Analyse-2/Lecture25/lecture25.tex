\documentclass[a4paper]{article}

% Expanded on 2022-05-23 at 12:59:59.

\usepackage{../../style}

\title{Analyse 2}
\author{Joachim Favre}
\date{Mercredi 25 mai 2022}

\begin{document}
\maketitle

\lecture{25}{2022-05-25}{$\pi$ apparaît de nulle part}{
}

\parag{Application: Changement de variables en coordonnées sphériques}{
Le changement de variable vers les coordonnées sphériques est défini par:
\begin{functionbypart}{G\left(r, \theta, \phi\right)}
    x = r\sin\left(\theta\right)\cos\left(\phi\right) \\
    y = r\sin\left(\theta\right)\sin\left(\phi\right) \\
    z = r\cos\left(\theta\right)
\end{functionbypart}
où $G : \left]0, +\infty\right[ \times \left[0, \pi\right] \times \left[0, 2\pi\right[  \mapsto \mathbb{R}^3 \setminus \left\{0\right\}$.

Nous pouvons illustrer ceci de la manière suivante:
\imagehere[0.4]{ChangementDeVariablesCoordonneesSpheriques.png}

Si $r > 0$, nous avons:
\[r = \sqrt{x^2 + y^2 + z^2}, \mathspace \theta = \arccos\left(\frac{z}{r}\right)\]

De plus, si $\sin\left(\theta\right) \neq 0$, nous pouvons trouver $\phi$ en voyant que:
\[\cos\left(\phi\right) = \frac{x}{r \sin\left(\theta\right)}, \mathspace \sin\left(\phi\right) = \frac{y}{r\sin\left(\theta\right)}\]

Calculons la matrice Jacobienne de notre fonction $G$:
\[J_{G}\left(r, \theta, \phi\right) = \begin{pmatrix} \sin\left(\theta\right)\cos\left(\phi\right) & r\cos\left(\theta\right)\cos\left(\phi\right) & -r\sin\left(\theta\right)\sin\left(\phi\right) \\ \sin\left(\theta\right)\sin\left(\phi\right) & r\cos\left(\theta\right)\sin\left(\phi\right) & r\sin\left(\theta\right)\cos\left(\phi\right) \\ \cos\left(\theta\right) & -r\sin\left(\theta\right) & 0 \end{pmatrix} \]

Nous pouvons maintenant calculer la valeur absolue de son déterminant:
\begin{multiequality}
    & \left|\det J_{G}\left(r, \theta, \phi\right)\right| \\
    =\ & \left|r^2 \cos^2\left(\theta\right) \sin\left(\theta\right)\cos^2\left(\phi\right) + r^2 \sin^3\left(\theta\right)\sin^2\left(\phi\right) + r^2 \cos^2\left(\theta\right) \sin\left(\theta\right)\sin^2\left(\phi\right) + r^2 \sin^3\left(\theta\right) \cos^2\left(\phi\right)\right| \\
    =\ & \left|r^2 \cos^2\left(\theta\right)\sin\left(\theta\right) + r^2 \sin^3\left(\theta\right)\right| \\
    =\ & \left|r^2 \sin\left(\theta\right)\right|
\end{multiequality}

Puisque $0 \leq\theta \leq\pi$, nous savons que $\sin\left(\theta\right) > 0$, ainsi:
\[\left|\det J_{G}\left(r, \theta, \phi\right)\right| = r^2 \sin\left(\theta\right)\]

Nous en déduisons que quand $r > 0$ et $\sin\left(\theta\right) > 0$ (c'est-à-dire $\theta \neq 0$ et $\theta \neq \pi$), alors $G$ est bijective.
}

\parag{Application: Masse totale d'un objet solide de densité donnée}{
    Pour calculer la masse totale $M$ d'un objet solide de volume $V$ et de densité $\rho\left(x, y, z\right)$, nous pouvons calculer:
    \[M = \iiint_V \rho\left(x, y, z\right) dxdydz\]
}

\end{document}
