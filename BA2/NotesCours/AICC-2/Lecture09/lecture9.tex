\documentclass[a4paper]{article}

% Expanded on 2022-03-22 at 15:17:58.

\usepackage{../../style}

\title{AICC-2}
\author{Joachim Favre}
\date{Mardi 22 mars 2022}

\begin{document}
\maketitle

\lecture{9}{2022-03-22}{Alice and Bob}{
}

\section{Cryptography}
\subsection{Cryptosystems from older times}
\parag{Introduction}{
    Cryptography serves multiple purposes.

    \begin{itemize}[left=0pt]
        \item Privacy.
        \item Authenticity.
        \item Integrity.
    \end{itemize}
}

\parag{Terminology}{
    \begin{itemize}[left=0pt]
        \item \important{Cryptography} is the art of composing cryptograms.
        \item \important{Cryptanalysis} is the art of breaking cryptograms.
        \item A \important{cryptanalyst} has \important{broken} the system when she can \textit{quickly} determine the plain text.
        \item An \important{attacker} is a \important{cryptanalyst}.
    \end{itemize}
}

\parag{Caesar's cipher}{
    We are using the concept of permutation, doing a cyclic shift. Ceasar liked doing a shift of 3, which gives:
    A way of seeing this is considering that our alphabet is linked to numbers:
    \begin{center}
        \begin{tabular}{cccccccccccc}
            A & B & C & \ldots & X  & Y  & Z  &    & ,  & .  \\
            0 & 1 & 2 & \ldots & 23 & 24 & 25 & 26 & 27 & 28
        \end{tabular}
    \end{center}
    where the \nth{26} character is a space.

    Then our algorithm becomes:
    \[E_{k} = t + k \Mod 29, \mathspace D_k = t - k \Mod 29, \mathspace k \in \left\{0, \ldots, 28\right\}\]
}

\parag{Monoalphabetic cipher}{
    More generally than a Caesar cipher, a monoalphabetic cipher is an arbitrary permutation of the alphabet.

    We have to make sure that this is a permutation, so we can go in both directions. Note that instead we can do a permutation of all possible words, or all possible sentences.

    The ``monoalphabetic'' is here since we are using the same permutation table for every letter.
}

\parag{Vigenère Cipher}{
    The idea of Vigenère Cipher is to use Caesar cipher, but with a different key for every letter. For example, having $k = 3$, $k = 1$, $k = 10$, $k = 4$, makes ALEA become DMOE. If we have more letters to encode than keys, then we begin the sequence of keys again. Remembering the sequence of key can be complicated, so we sometimes use a word instead (abc to mean $k_1 = 0$, $k_2 = 1$ and $k_3 = 2$, for example). If we have a word of 7 letters and a key of 4 letters, our algorithm can be summed up as:
    \begin{multiequality}
        V_K\left(t\right) =\ & V_{k_1, k_2, k_3, k_4}\left(t_1, t_2, t_3, t_4, t_5, t_6, t_7\right)  \\
        =\ & \left(t_1 \oplus k_1, t_2 \oplus k_2, t_3 \oplus k_3, t_4 \oplus k_4, t_5 \oplus k_1, t_6 \oplus k_2, t_7 \oplus k_3\right)
    \end{multiequality}
    where $\oplus$ is an addition modulo.
}

\parag{Key assumption (lol)}{
    In modern cryptography, we consider that the algorithm is public, but the key is private. We must not base our cryptographic secrecy on the algorithm.
}

\parag{Attacks}{
    We can differentiate three categories of attacks a cryptanalyst can do.
    \begin{enumerate}[left=0pt]
        \item The first one is the most complicated one: we only have access to some cryptograms, which have been encoded with the same key. This is named the \important{ciphertext-only} attack.
        \item The second one is named the \important{known plaintext} attack: we have access to some plaintexts and their resulting cryptograms, which have all been encoded using the same key.
        \item The third one is named the \important{chosen plaintext} attack. We can obtain the ciphertext for arbitrary ciphertexts, with the same key every time.
    \end{enumerate}

    \subparag{Caesar system}{
        This is really a bad code.
    }

    \subparag{Generic monoalphabetic}{
        A generic monoalphabetic cipher is already more secure. Indeed, a ciphertext-only attack cannot be reached through brute-force since $29! \approx 10^{31}$ is too much. However, we can be more intelligent than that: for a given language, we know the frequency of each letter, the probability to have a letter following another, and so on. Using those, this attack becomes doable.

        As for the Ceasar system, the generic monoalphabetic cipher is very bad against the two other attacks.
    }

    \subparag{Vigenère}{
        This code is already much more robust.
        As for the other codes we have seen, this code is not robust against the two other attacks.
    }
}

\parag{One-time pad}{
Let's assume that we are using binary, for simplicity. We take our key to be as long as our ciphertext, thus:
\[t, k, c \in \left\{0, 1\right\}^{n}\]

Moreover, we also take $k$ so that it is a \textit{random} (uniform) IID sequence. The encryption is $c = k + t$, and the decryption is $t = c - k = k + c$, since we are using $n$-bits encryption ($\congruent{b + b}{0}{2}$ for $b \in \left\{0, 1\right\}$).

We notice that this is exactly Vigenère's code, where the key and the plain text are of the same length, and $k$ is random.
}

\parag{Definition: perfect secrecy}{
    \important{Perfect secrecy} is reached when the plaintext $t$ and $c$ are statistically independent.
}

\parag{Lemma: entropy}{
    We will need for the next theorem to see that if we have $n$ IID random variables uniformly distributed over $\mathcal{A}$, $K_1, \ldots, K_n$, then:
    \[H\left(K_1, \ldots, K_n\right) = n\log_2\left|\mathcal{A}\right|\]

    More precisely, if we are working over binary:
    \[H\left(K_1, \ldots, K_n\right) = n\]
}

\parag{Theorem}{
    The one-time pad is perfectly secret.


    \subparag{Remark}{
        The problem for this code is that we need a huge amount of huge keys.
    }
}

\parag{Theorem}{
    If we have perfect secrecy, then:
    \[H\left(T\right) \leq H\left(K\right)\]

    \subparag{Implication}{
        This implies that if we have a message of 100 bits, then we need a key of at least 100 bits to reach perfect secrecy.
    }
}

\end{document}
