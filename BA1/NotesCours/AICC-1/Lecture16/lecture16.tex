\documentclass[a4paper]{article}

% Expanded on 2021-11-10 at 15:10:29.

\usepackage{../../style}

\title{AICC 1}
\author{Joachim Favre}
\date{Mercredi 10 novembre 2021}

\begin{document}
\maketitle

\lecture{16}{2021-11-10}{Generalisation of induction and recursion}{
}

\subsection{Recursively defined sets and structure}

\parag{Recursively defined sets and structure}{
    Recursion can be also used to define sets. \important{Recursive definitions of sets} have two parts. First, the basis step specifies an initial collection of elements. The recursive step gives the rules for forming new elements in the set from those already known to be in the set.

    The \important{exclusion rule} states that only elements generated using the basis and recursive steps belong to this set.
}

\parag{Strings}{
    Let's define the set $\Sigma^*$ of strings over the alphabet $\Sigma$:
    \begin{itemize}
        \item Basis step: $\lambda \in \Sigma^*$ ($\lambda$ is the empty string)
        \item Recursive step: if $w$ is in $\Sigma^*$ and $x$ in $\Sigma$, then $wx \in \Sigma^*$ (where $w\cdot x = wx$ is the concatenation of the string $w$ with the character $x$; it is the primary operation for strings).
    \end{itemize}

}

\parag{Definition of well-formed formulae}{
    The set of \important{well-formed formulae} in propositional logic involving $T, F$, propositional variables, and operators from the set $\left\{\lnot, \land, \lor, \to, \leftrightarrow\right\}$ is recursively defined as:
    \begin{itemize}
        \item Basis step: $T, F$ and $s$, where $s$ is a propositional variable, are well-formed formulae.
        \item Recursive step: If $E$ and $F$ are well formed formule, then:
              \[\left(\lnot E\right), \mathspace \left(E \land F\right), \mathspace\left(E \lor F\right), \mathspace \left(E \to F\right), \mathspace\left(E \leftrightarrow F\right)\]
              are well formed formulae. The parenthesis are important, so that there is no problem of precedence after multiple generations.
    \end{itemize}
}

\subsection{Recursively defied functions on recursively defined sets}
\parag{Definition}{
    We can define functions by recursion on recursively defined sets. We follow the definition of the set.

    This is basically the same idea as recursive sequences, but instead we can have any basis step (which is a natural number for $\mathbb{N}$) and any recursive step (which is to explain how it goes from $n$ to $n+1$ for $\mathbb{N}$).
}

\subsection{Structural induction}
\parag{Structural induction}{
    To prove a property of the elements of a recursively defined set, we use \important{structural induction}:
    \begin{itemize}
        \item \important{Basis step:} Show that the result holds for all elements specified in the basis step of the recursive definition.
        \item \important{Recursive step:} Show that if the statement is true for each of the elements used to construct new elements in the recursive step of the definition, the result hold for these new elements.
    \end{itemize}

}

\parag{Summary}{
    We generalised the idea of recursion and induction to any kind of recursively defined sets.

    \imagehere{DiagramSummaryInductionRecursion.png}
}

\subsection{Recursive algorithms}

\parag{Recursive algorithms definition}{
    An algorithm is called \important{recursive} if it solves a problem by reducing it to an instance of the same problem with smaller input.

    For the algorithm to terminate, the instance of the problem must eventually be reduced to some initial case for which the solution is known. (Don't forget the exit condition! \wink)
}

\begin{filecontents*}[overwrite]{factorial.code}
    procedure factorial(n):
    if n = 0 then return 1
    else return n*factorial(n-1)
\end{filecontents*}

\parag{Recursive factorial algorithm}{
    Here is an example of a recursive algorithm for computing $n!$ where $n$ is a nonnegative integer:

    \importcode{factorial.code}{pseudo}
}

\begin{filecontents*}[overwrite]{power.code}
    procedure power(a, n):
    if n <= 0 then return 1
    else return a*power(a, n-1)
\end{filecontents*}

\begin{filecontents*}[overwrite]{fastpower.code}
    procedure fast_power(a, n):
    if n <= 0 then return 1
    else a^(n & 1) * (fast_power(a, floor(n/2)))^2
\end{filecontents*}


\parag{Recursive exponential algorithm}{
    The following algorithm is an example of a recursive algorithm for computing $a^n$, where $a$ is a nonzero real number and $n$ is a nonegative integer.

    \importcode{power.code}{pseudo}

    It is a really bad algorithm since $\Theta\left(n\right)$. The algorithm below is much better since it is $\Theta\left(\log\left(n\right)\right)$.
    \importcode{fastpower.code}{pseudo}

    We use \texttt{n \& 1} to make a bitwise and between \texttt{n} and \texttt{1}, which is equivalent to take a modulo 2.
}

\begin{filecontents*}[overwrite]{recursivelinearsearch.code}
    procedure recursive_linear_search(a: array of n real numbers, x: real number, i <= n: integer):
    if a[i] = x then return i
    else if i = n then return -1
    else return recursive_linear_search(a, x, i+1)
\end{filecontents*}


\parag{Recursive linear search}{
    We can use the following recursive algorithm to do some linear search:
    \importcode{recursivelinearsearch.code}{pseudo}

    In the end, it does the same thing as linear search, but while executing it, we keep track of everything having happened before, so it is not very efficient.
}

\begin{filecontents*}[overwrite]{recursivebinarysearch.code}
    procedure recursive_binary_search(a: array of n integers, x, i, j):
    m := floor((i + j)/2)
    if x = a[m] then return m
    else if(x < a[m] and i < m) then return recursive_binary_search(a, x, i, m-1)
    else if (x > a[m] and j > m) then return recursive_Binary_sarch(a, x, m+1, j)
    else return -1
\end{filecontents*}

\parag{Recursive binary search}{
    Here is the recursive version of binary search:
    \importcode{recursivebinarysearch.code}{pseudo}

    The recursive version of this algorithm is not more efficient, however it is easier to ready.
}

\parag{Link between recursion and induction}{
    Induction and recursion are different approaches to proving result and solving problems. They both rely on the ability of achieving the desired result for the smallest possible version of the problem. However, induction extends this ability to problems of any size, whereas recursion reduces a problem of any size to the smallest possible ones.
}

\parag{Recursion and iteration}{
    A recursively defined function can be evaluated in two different ways:
    \begin{itemize}
        \item \important{Recursively:} For a value, apply directly the recursive definition, until a base case is reached.
        \item \important{Iterative:} Start with a base case, and apply the recursive definition to compute the function for larger values.
    \end{itemize}
}

\end{document}
