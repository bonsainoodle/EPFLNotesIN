\documentclass[a4paper]{article}

% Expanded on 2022-04-15 at 16:02:48.

\usepackage{../../style}

\title{AICC-2}
\author{Joachim Favre}
\date{Jeudi 14 avril 2022}

\begin{document}
\maketitle

\lecture{16}{2022-04-14}{Rivest-Shamir-Aldeman's cryptosystem}{
}

\parag{Theorem}{
    Let us consider a group $\left(G, \star\right)$, with identity element $e$.

    $a^k = e$ if and only if $k$ is a multiple of the order of $a$.
}

\parag{Lagrange's theorem}{
    Let $\left(G, \star\right)$ be a finite commutative group of cardinality $n$.

    The order of each of its element divides $n$.
}

\parag{Corollary}{
    Let $\left(G, \star\right)$ be a finite commutative group of cardinality $n$ and identity $e$. Then, for all $a \in G$:
    \[a^n = e\]
}


\parag{Corollary: Euler's theorem}{
    Let $m \leq 2$ be an integer. Then, for all $a \in \left(\mathbb{Z} / m\mathbb{Z}^*, \cdot\right)$, we have:
    \[a^{\phi\left(m\right)} = \left[1\right]_m\]

    Equivalently, for all integers $a$ relatively prime to $m$, we have:
    \[\congruent{a^{\phi\left(m\right)}}{1}{m}\]
}

\parag{Corollary: Fermat's little theorem}{
    Let $p$ be a prime number. Then, for all $a \in \mathbb{Z} / p\mathbb{Z}$, we have:
    \[a^p = a\]

    Equivalently, for all integers $a$:
    \[\congruent{a^p}{a}{p}\]
}

\parag{Theorem: Chinese remainders}{
    If $m_1$ and $m_2$ are relatively prime, then the map defined by:
    \[\begin{split}
            \psi: \mathbb{Z} / m_1 m_2 \mathbb{Z} &\longmapsto \left(\mathbb{Z} / m_1 \mathbb{Z}\right) \times \left(\mathbb{Z} / m_2\mathbb{Z}\right) \\
            \left[k\right]_{m_1 \cdot m_2} &\longmapsto \left(\left[k\right]_{m_1}, \left[k\right]_{m_2}\right)
        \end{split}\]
    is bijective and an isomorphism with respect to $+$ and $\cdot$.

    Note that, if $m_1$ and $m_2$ are not relatively prime, then $\psi$ is neither onto nor one-to-one.
}

\parag{Backward mapping}{
    We want to know how to get $\left[x\right]_{pq}$ when having $\left(\left[x\right]_p, \left[x\right]_q\right)$.

    First, we use extended Euclid's algorithm to find integers $u$ and $v$ such that:
    \[1 = \gcd\left(p, q\right) = pu + qv\]

    We let $a = qv$ and $b = pu$, and we see that:
    \[\left[a\right]_q = \left[qv\right]_q = 0, \mathspace \left[a\right]_p = \left[qv\right]_p = \left[1 - pu\right]_p = \left[1\right]_p\]

    The same thing holds for $b$:
    \[\left[b\right]_p = \left[pv\right]_p = 0, \mathspace \left[b\right]_q = \left[pv\right]_q = \left[1 - qu\right]_q = \left[1\right]_q\]

    Thus, for any integers $k_1$ and $k_2$, the Chinese remainder theorem gives us the mapping:
    \[\psi\left(\left[a k_1 + b k_2\right]_{pq}\right) = \left(\left[ak_1 + bk_2\right]_p, \left[ak_1 + bk_2\right]_q\right) = \left(\left[k_1\right]_{p}, \left[k_2\right]_{q}\right)\]

    But since our function is bijective, we find that its inverse is:
    \[\psi^{-1}\left(\left[k_1\right]_p, \left[k_2\right]_q\right) = \left[a k_1 + b k_2\right]_{pq}\]
}


\subsection{RSA cryptosystem}

\parag{Theorem: Mix of Chinese remainders and Fermat}{
    Let $p$ and $q$ be distinct prime numbers, and let $k$ be a multiple of both $\left(p - 1\right)$ and $\left(q - 1\right)$.

    Then, for every integer $a$ and every non-negative integer $\ell$, we have:
    \[\left(\left[a\right]_{pq}\right)^{\ell k + 1} = \left[a\right]_{pq}\]
}

\parag{RSA encryption}{
    Let's suppose that we can find three integers $m$ (modulus), $e$ (encoding exponent) and $d$ (decoding exponent) such that, for all integers $t \in \mathbb{Z} / m\mathbb{Z}$ (our plaintext), we have:
    \[\left[\left(t^e\right)^d\right]_m = \left[t\right]_m\]

    Then, the receiver can generate those $m, e, d$. The $\left(m, e\right)$ is the public encoding key, announced in a public directory, and $\left(m, d\right)$ is the private decoding key ($d$ never leaves the receiver). To send the plaintext $t \in \mathbb{Z} / m\mathbb{Z}$, the encoder forms the cryptogram by doing:
    \[c = t^e \Mod m\]

    This is not a problem since doing exponents with big numbers is easy thanks to the fast exponentiation algorithm. Then, the intended decoder performs $c^d \Mod m$ and obtains the plaintext $t$.
}

\parag{Key generation}{
We use the following algorithm to generate our keys:
\begin{itemize}
    \item We generate large primes $p$ and $q$ at random.
    \item We compute the modulus as $m = pq$.
    \item We compute $k$, a number we keep secret, to be a multiple of $\left(p-1\right)$ and $\left(q-1\right)$. For instance, we can take $k = \phi\left(pq\right)$ or $k = \lcm\left(p-1, q-1\right)$.
    \item We produce the public encoding exponent by taking a number $e$ such that $\gcd\left(e, k\right) = 1$. A common choice is the prime number $e = 2^{16} + 1$. Note that there is no need for $e$ to be distinct for each recipient.
    \item We use Bézout's construction (the extended Euclid's algorithm) to produce the positive decoding exponent $d$ such that $de + k \ell = 1 \iff de = \hat{k} \ell + 1$.
\end{itemize}

As explained before, the public key is $\left(m, e\right)$ and the private key is $\left(m, d\right)$.

\subparag{Decoding}{
For the decoding, we have that:
\[\left(\left[t\right]^e\right)^d = \left[t\right]_m^{ed} = \left[t\right]_{pq}^{1 + \hat{k} \ell } = \left[t\right]_{pq} = \left[t\right]_m\]
as expected.
}

}

\end{document}
