\documentclass[a4paper]{article}

% Expanded on 2022-03-21 at 10:12:40.

\usepackage{../../style}

\title{Analyse 2}
\author{Joachim Favre}
\date{Lundi 21 mars 2022}

\begin{document}
\maketitle

\lecture{9}{2022-03-21}{Ça devient limite là}{
}

\section{Fonctions réelles de plusieurs variables réelles}
\subsection{Définitions et exemples}
\parag{Définition}{
    Soit $E \subset \mathbb{R}^n$ un sous-ensemble non-vide où $n \geq 1$.

    Une \important{fonction} $E \mapsto \mathbb{R}^n$ est une application qui envoie chaque point $\bvec{x} = \begin{pmatrix} x_1 & \ldots & x_n \end{pmatrix} \in E$ dans $\mathbb{R}$. $E$ est \important{le domaine de définition} de $f$, et $f\left(E\right) \subset \mathbb{R}$ est \important{l'ensemble image}.
}

\parag{Définition}{
    Soit $f : E \mapsto \mathbb{R}^n$ et $c \in f\left(E\right) \subset \mathbb{R}^n$. Alors, l'ensemble suivant est appelé \important{l'ensemble de niveau} de $f$:
    \[N_f\left(c\right) = \left\{\bvec{x} \in E \telque f\left(\bvec{x}\right) = c\right\} \subset E\]
}

\subsection{Limites et continuité}
\parag{Définition: Définition au voisinage}{
    Une fonction est \important{définie au voisinage} de $\bvec{x_0}$ si:
    \[\exists \delta >0 \telque B\left(\bvec{x_0}, \delta\right) \subset E \cup \left\{x_0\right\}\]

    \subparag{Remarque}{
        La fonction n'a pas besoin d'être définie en $\bvec{x_0}$ pour être définie au voisinage de ce point.
    }
}

\parag{Définition: Limite}{
    Une fonction définie au voisinage de $\bvec{x_0}$ admet pour \important{limite} le nombre réel $\ell$ lorsque $\bvec{x}$ tend vers $\bvec{x_0}$, si pour tout $\epsilon > 0$, il existe $\delta > 0$ tel que pour tout $\bvec{x} \in E$:
    \[0 < \left\|\bvec{x} - \bvec{x_0}\right\| \leq \delta \implies \left|f\left(\bvec{x}\right) - \ell\right| \leq \epsilon\]

    Dans ce cas, nous notons:
    \[\lim_{\bvec{x} \to \bvec{x_0}} f\left(\bvec{x}\right) = \ell\]
}

\parag{Définition: Continuité}{
    Soit $\bvec{x_0} \in E$ un point intérieur de $E$.

    $f : E \mapsto \mathbb{R}$ est continue en $\bvec{x} = \bvec{x_0}$ si et seulement si:
    \[\lim_{\bvec{x} \to \bvec{x_0}} f\left(\bvec{x}\right) = f\left(\bvec{x_0}\right)\]

    Il faut donc à la fois que la limite existe, et qu'elle soit égale à la valeur de la fonction.
}

\parag{Théorème: Caractérisation des limites à partir des suites convergentes}{
    Une fonction $f: E \mapsto \mathbb{R}$ définie au voisinage de $\bvec{x_0}$ admet pour limite $\ell \in \mathbb{R}$ lorsque $\bvec{x} \to \bvec{x_0}$ si et seulement si \textit{pour toute} suite d'éléments $\left\{\bvec{a_k}\right\}$ de $\left\{\bvec{x} \in E \telque \bvec{x} \neq \bvec{x_0}\right\}$, qui converge vers $\bvec{x_0}$, la suite $\left\{f\left(\bvec{a_k}\right)\right\}$ converge vers $\ell$.

    En d'autres mots:
    \[\left(\lim_{\bvec{x} \to \bvec{x_0}} f\left(\bvec{x}\right) = \ell\right) \iff \left(\lim_{k \to \infty} f\left(\bvec{a_k}\right) = \ell,\ \forall \left\{\bvec{a_k}\right\} \subset E \setminus \left\{\bvec{x_0}\right\} \text{ telle que } \lim_{k \to \infty} \bvec{a_k} = \bvec{x_0}\right)\]
}

\parag{Opérations algébriques}{
    Soient $f,g$ deux fonctions $E_{\mathbb{R}^n} \mapsto \mathbb{R}$ telles que:
    \[\lim_{\bvec{x} \to \bvec{x_0}} f\left(\bvec{x}\right) = \ell_1 \mathspace \text{et} \mathspace \lim_{\bvec{x} \to \bvec{x_0}} g\left(\bvec{x}\right) = \ell_2\]

    Alors, nous avons:
    \begin{enumerate}
        \item $\displaystyle \lim_{\bvec{x} \to \bvec{x_0}} \left(\alpha f + \beta g\right)\left(\bvec{x}\right) = \alpha \ell_1 + \beta \ell_2$ pour $\alpha, \beta \in \mathbb{R}$
        \item $\displaystyle \lim_{\bvec{x} \to \bvec{x_0}} \left(f\cdot g\right)\left(\bvec{x}\right) = \ell_1 \ell_2$
        \item Si $\ell_2 \neq 0$, alors $\displaystyle \lim_{\bvec{x} \to \bvec{x_0}} \left(\frac{f}{g}\right)\left(\bvec{x}\right) = \frac{\ell_1}{\ell_2}$
    \end{enumerate}

    \subparag{Implication}{
        On peut en déduire que tous les polynômes de plusieurs variables et toutes les fonctions rationnelles sont continues sur leur domaines de définition.
    }
}

\parag{Remarque}{
    La caractérisation de la limite à partir des suites convergentes est pratique pour montrer qu'une fonction n'admet pas de limite en $\bvec{x_0} \in \mathbb{R}^n$.
}

\end{document}
