\documentclass[a4paper]{article}

% Expanded on 2022-05-17 at 15:17:49.

\usepackage{../../style}

\title{AICC2}
\author{Joachim Favre}
\date{Mardi 17 mai 2022}

\begin{document}
\maketitle

\lecture{23}{2022-05-17}{Linear codes}{
}

\parag{Summary}{
    From now on, we will work on $\mathbb{F}_p^n$, where $p$ is a prime number. This contains $p^n$ vectors.

    We can do everything we did in Linear Algebra: linear independence, subspaces, matrices rank, and so on. The only thing we do not have is dot product
}


\subsection{Linear codes}

\parag{Definition: Linear code}{
    A code $\mathcal{C} \subseteq \mathbb{F}^n$ is \important{linear} if $\mathcal{C}$ is a subspace of $\mathbb{F}^n$.

    \subparag{Observations}{
        \begin{enumerate}[left=0pt]
            \item A linear code must contain the all-zero sequence $\bvec{c} = \left(0, 0, \ldots, 0\right)$.
            \item The sum of two codewords is again a codeword.
            \item Scaled versions of codewords are again codewords.
        \end{enumerate}
    }

    \subparag{Terminology}{
        We call a $\left(n, k\right)$ linear block code over $\mathbb{F}_p$ to be a $k$-dimensional subspace of $\mathbb{F}_p^n$.
    }

}

\parag{Definition: Dimension}{
    The \important{dimension} of a linear code $\mathcal{C} \subseteq \mathbb{F}^n$ is the dimension of the subspace $\mathcal{C}.$
}

\parag{Lemma}{
    The number of codewords in a linear code $\mathcal{C} \subseteq \mathbb{F}_q^n$ must be of the form $q^k$ for some $k \in \left\{0, 1, 2, \ldots, n\right\}$ (where $q$ must be a prime power, meaning it is of the form $q = p^m$ for $p$ being a prime number and $m \in \mathbb{Z}_+$).
}

\parag{Definition: Hamming weight}{
    Let $\bvec{x} \in \mathbb{F}^n$. The \important{Hamming weight} of $\bvec{x}$ is:
    \[w\left(\bvec{x}\right) = d\left(\bvec{x}, \bvec{0}\right)\]

    In other words, $w\left(\bvec{x}\right)$ is the number of non-zero positions in $\bvec{x}$.

    \subparag{Property}{
        We can see that $w\left(\bvec{x} - \bvec{u}\right)= d\left(\bvec{x}, \bvec{u}\right)$.
    }
}

\parag{Theorem}{
    Let $\mathcal{C} \subseteq \mathbb{F}^n$ be a linear code. Then:
    \[d_{min}\left(\mathcal{C}\right) = \min_{\bvec{x} \in \mathcal{C} \setminus \left\{\bvec{0}\right\}} w\left(\bvec{x}\right)\]
}

\parag{Observation}{
    The three last examples we saw are the only binary linear codes that are MDS.

    \subparag{Remark}{
        There are other non-binary linear codes which are MDS (such as Reed-Solomon Codes), and some binary non-linear codes that are MDS (as we will see in the following example).
    }
}

\parag{Example}{
    Let's consider the following code:
    \[\mathcal{C} = \left\{\left(1, 0, 0, \ldots, 0\right), \left(0, 1, 1, \ldots, 1\right)\right\}\]

    This is a MDS code. However, since it is not linear, this is fine with our observation.
}

\parag{Definition: Generator matrix}{
    We know a linear code can be described by a basis $\left\{\bvec{c_1}, \ldots, \bvec{c_k}\right\}$.

    We define the following matrix to be the \important{generator matrix} of the code:
    \[G = \begin{pmatrix} & & \bvec{c_1} & & \\ & & \vdots & & \\  & & \bvec{c_k} &  & \end{pmatrix} \in \mathbb{F}^{k \times n} \]
}


\parag{Number of generator matrices}{
    We wonder how many different generator matrices there are for a certain code $\mathcal{C} \in \mathbb{F}_q^n$ of dimension $k$.

    This number is given by:
    \[\left(q^k - 1\right)\left(q^k - q\right)\left(q^k - q^2\right)\cdots\left(q^k - q^{k-1}\right)\]

    \subparag{Example}{
        For instance, picking a subspace of dimension $k = 3$ of $\mathbb{F}_2^n$, we have:
        \[\left(2^3 - 1\right)\left(2^3 - 2\right)\left(2^3 - 2^2\right) = 7\cdot 6\cdot 4 = 168 \text{ choices}\]
    }

}

\parag{Definition: Echelon form}{
    A matrix is said to be of \important{echelon form} if it is of the form:
    \[\begin{pmatrix} \star & \times & \times & \times & \times & \times \\ 0 & \star & \times & \times & \times & \times \\ 0 & 0 & 0 & \star & \times & \times \\ 0 & 0 & 0 & 0 & \star & \times \end{pmatrix} \]
    where the $\times$ represent any number and the $\star$ represent non-zero numbers, named pivots.

    A matrix is said to be of \important{reduced echelon form} if it is of the form:
    \[\begin{pmatrix} 1 & 0 & \times & 0 & 0 & \times \\ 0 & 1 & \times & 0 & 0 & \times \\ 0 & 0 & 0 & 1 & 0 & \times \\ 0 & 0 & 0 & 0 & 1 & \times \end{pmatrix}\]
}

\parag{Theorem}{
    Any matrix can be transformed into a unique reduced echelon form through the Gaussian elimination algorithm.
}

\parag{Remark}{
    We can use the Gaussian elimination algorithm to turn our generator matrix into another matrix generating the same space but which may be easier to use.
}

\parag{Definition: Systematic generator matrices}{
    A \important{systematic generator matrix} is a generator matrix such that its first $k$ columns represent the identity matrix.
    \[G = \begin{pmatrix}  &  &  &  &  &  \\  & I_k &  &  & P &  \\  &  &  &  &  &  \end{pmatrix} \]

    The identity matrix takes $k$ columns, and the matrix $P$, sometimes called the parity check, takes $n-k$ columns. We can notice that such a matrix is in its reduced echelon form.

    \subparag{Observation}{
        If we are allowed to swap columns, then a systematic generator matrix always exists.
    }
}

\end{document}
