\documentclass{article}

% Expanded on 2021-10-06 at 15:14:02.

\usepackage{../../style}

\title{AICC-1}
\author{Joachim Favre}
\date{Mercredi 06 octobre 2021}

\begin{document}
\maketitle

\lecture{6}{2021-10-06}{The end of proofs and the beginning of sets}{
}

\parag{Contraposition versus contradiction}{
    We can realise that the last proof could also have been done with a proof by contraposition.

    More generally, any proof by contraposition can be formulated as a proof by contradiction, but the reciprocal is not always true. Indeed, for a proof by contraposition, we show that:
    \[\lnot q \to \lnot p\]

    Thus, if we have a direct proof for $\lnot q \to \lnot p$, we can also do a proof by contradiction. Indeed, since we assumed $\lnot q$, we get $\lnot p$, which is a contradiction when put together with out other assumption, $p$:
    \[\left(p \land \lnot q\right) \to \left(p \land\lnot p\right)\]

    Generally, proofs by contradiction are stronger, since they allow us to show that:
    \[\left(p \land\lnot q\right) \to \left(r \land\lnot r\right)\]
}

\subsection{Other proof methods}
\parag{Proof for Biconditional Statements}{
    To prove a theorem which is a biconditional statement, i.e of the form $p \leftrightarrow q$, we show that $p \to q$ and $q \to p$ are both true.

    This is based on:
    \[p \leftrightarrow q \equiv p\to q \land q \to p\]
}

\parag{Proof by cases}{
    We can see the following equivalence:
    \[\left(p_1 \lor \ldots \lor p_n\right) \to q \equiv \left(p_1 \to q\right) \land\ldots \land \left(p_n \to q\right)\]

    Thus, to prove a theorem of the form $\left(p_1 \lor \ldots \lor p_n\right) \to q$, we only have to prove every $p_i \to q$ --- called a case --- separately.

    \subparag{WLOG}{
        In the context of a proof by case, there may be a case that follows trivially from another (for example, by swapping roles of variables). In such a scenario, we can use the word ``WLOG'' --- ``Without Loss Of Generality''.
    }
}

\parag{Proof by counterexample}{
    To establish that $\forall x P\left(x\right)$ is false, i.e. that $\lnot \forall x P\left(x\right)$ is true, we only need to find a $c$ such that $\lnot P\left(c\right)$ is true. Indeed:
    \[\lnot \forall x P\left(x\right) = \exists x \lnot P\left(x\right)\]

    In this case, $c$ is called a \important{counterexample} to the assertion $\forall x P\left(x\right)$.
}

\parag{Existence proof}{
    To show that $\exists x P\left(x\right)$, we have two possibilities:
    \begin{itemize}[left=0pt]
        \item \important{Constructive Proof:} Find a $c$ such that $P\left(c\right)$ is true.
        \item \important{Non-constructive proof:} Find $c_1, c_2$ such that $P\left(c_1\right) \lor P\left(c_2\right)$ is true. Indeed:
              \[\left(P\left(c_1\right) \to \exists x P\left(x\right)\right) \land \left(P\left(c_2\right) \to \exists x P\left(x\right)\right) \equiv \left(P\left(c_1\right) \lor P\left(c_2\right)\right) \to \exists x P\left(x\right)\]

              This is called ``non-constructive'', because we do not know if it is $c_1$ or $c_2$ which satisfies this property.
    \end{itemize}
}

\parag{Uniqueness proof}{
    To show that $\exists! x P\left(x\right)$, we take $x$ and $y$ such that $P\left(x\right)$ and $P\left(y\right)$ are true, and show that $x = y$.
}

\section{Sets and functions}

\parag{Definition}{
    A \important{set} is an unordered collection of objects. The objects in a set are called the \important{elements} of the set. A set is said to \important{contain} its elements.

    The notation $a \in A$ denotes that a  is an element of the set $A$. If $a$ is not an element of $A$, we write $a \not\in A \equiv \lnot\left(a \in A\right)$
}

\parag{Roster method}{
    To describe a set, we can use what is called the \important{Roster method}. Basically, we list all elements:
    \[S = \left\{a, b, c, d\right\}\]

    We are basically saying
    \[\forall x \left(x \in S \leftrightarrow \left(x = a \lor x = b \lor x = c \lor x = d\right)\right)\]

    If we can continue the set in a logical way, we can use ellipses:
    \[T = \left\{0, 1, 2, 3, 4, \ldots\right\}\]

    \subparag{Order}{
        Note that sets are unordered, so the order is not important:
        \[S = \left\{a, b, c, d\right\} \equiv \left\{b, c, a, d\right\}\]
    }
}

\parag{Sets of number}{
    The following sets of numbers are really important:
    \begin{center}
        \begin{tabular}{lll}
            $\mathbb{N}$   & Natural numbers       & $\left\{0, 1, 2, 3, \ldots\right\}$              \\
            $\mathbb{Z}$   & Integers              & $\left\{\ldots, -2, -1, 0, 1, 2, \ldots\right\}$ \\
            $\mathbb{Z}_+$ & Positive integers     & $\left\{1, 2, 3, \ldots\right\}$                 \\
            $\mathbb{Q}$   & Rational numbers      &                                                  \\
            $\mathbb{R}$   & Real numbers          &                                                  \\
            $\mathbb{R}_+$ & Positive real numbers &                                                  \\
            $\mathbb{C}$   & Complex numbers       &
        \end{tabular}
    \end{center}

    Note that \textit{\important{0 is NOT positive, this is really important for tests!}}
}

\parag{Set-builder notation}{
    We can specify the property or properties that all members satisfy:
    \[S = \left\{x | P\left(x\right)\right\}\]

    $P\left(x\right)$ might be expressed in natural language or predicate logic.
}

\parag{Interval notation}{
For sets of number we can use the following notations:
\[\left[a, b\right]  = \left\{x | a \leq x \leq b\right\}, \mathspace \left[a, b\right) = \left\{x | a \leq x < b\right\}\]
        \[\left(a, b\right]  = \left\{x | a < x \leq b\right\}, \mathspace \left(a, b\right) = \left\{x | a < x < b\right\}\]

Note that we call $\left[a, b\right]$ a \important{closed interval}, and $\left(a, b\right)$ an \important{open interval}. Sometimes, we also note:
\[\left(a, b\right) = \left]a, b\right[ \]
}

\parag{Important sets}{
    The \important{universal set}, $U$, is the set containing everything currently under consideration. It depends on the context. Note that it is very dangerous, because it is often implicit.

    The \important{empty set} is the set with no element. It can be noted as $\o$ or $\left\{\right\}$.
}

\end{document}
