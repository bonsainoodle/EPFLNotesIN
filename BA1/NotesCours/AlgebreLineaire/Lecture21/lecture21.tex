\documentclass[a4paper]{article}

% Expanded on 2021-12-02 at 08:19:11.

\usepackage{../../style}

\title{Algèbre linéaire}
\author{Joachim Favre}
\date{Jeudi 02 décembre 2021}

\begin{document}
\maketitle

\lecture{21}{2021-12-02}{Projections et matrices orthogonales}{
}

\parag{Théorème de projection orthogonale}{
    Soit $W$, un sous-espace vectoriel de $\mathbb{R}^n$.

    Tout vecteur $\bvec{y}$ de $\mathbb{R}^n$ s'écrit de façon \textit{unique} sous la forme:
    \[\bvec{y} = \bvec{\hat{y}} + \bvec{z}, \mathspace \bvec{\hat{y}} \in W, \bvec{z} \in W^\perp\]

    De façon plus précise, si $\left(\bvec{u}_1 \ldots, \bvec{u}_p\right)$ est une base orthogonale quelconque de $W$, alors:
    \[\bvec{\hat{y}} = \frac{\bvec{y} \dotprod \bvec{u}_1}{\bvec{u}_1 \dotprod \bvec{u}_1} \bvec{u}_1 + \ldots + \frac{\bvec{y} \dotprod \bvec{u}_p}{\bvec{u}_p \dotprod \bvec{u}_p} \bvec{u}_p, \mathspace \bvec{z} = \bvec{y} - \bvec{\hat{y}}\]

    \subparag{Terminologie}{
        On appelle $\bvec{\hat{y}}$ la \important{projection orthogonale} de $\bvec{y}$ sur $W$, notée $\proj_W \bvec{y}$.
    }
}

\parag{Théorème de meilleure approximation (quadratique)}{
    Soit $W$ un sous-espace vectoriel de $\mathbb{R}^n$, $\bvec{y} \in \mathbb{R}^n$ et $\bvec{\hat{y}} = \proj_W \bvec{y}$.

    Alors, $\bvec{\hat{y}}$ est le point de $W$ le plus proche de $\bvec{y}$, c'est-à-dire:
    \[\left\|\bvec{y} - \bvec{\hat{y}}\right\| < \left\|\bvec{y} - \bvec{v}\right\|, \mathspace \forall \bvec{v} \in W, \bvec{v} \neq \bvec{\hat{y}}\]
}

\parag{Observations}{
    On voit que les cas suivants sont un peu spéciaux:
    \begin{itemize}
        \item Si $\bvec{y} \in W$, alors $\proj_W \bvec{y} = \bvec{y}$.
        \item Si $\bvec{y} \in W^\perp$, alors $\proj_W \bvec{y} = \bvec{0}$.
    \end{itemize}

    De plus, on peut voir que, si on connait la projection dans $W$, alors on peut facilement calculer la projection dans $W^\perp$:
    \[\proj_{W^\perp}\bvec{y} = \bvec{y} - \proj_W \bvec{y}\]
}

\parag{Définition}{
    On dit qu'une base $\mathcal{B} = \left(\bvec{u}_1, \ldots, \bvec{u}_p\right)$ d'un sous-espace $W$ est \important{orthonormé} si:
    \begin{functionbypart}{\bvec{u}_i \dotprod \bvec{u}_j}
        1 \mathspace \text{si } i = j  \\
        0 \mathspace \text{si } i \neq j
    \end{functionbypart}
}

\parag{Théorème}{
    Les colonnes d'une matrice $U \in \mathbb{R}^{m \times n}$ sont orthonormées \textit{si et seulement si}
    \[U^T U = I\]
}

\parag{Théorème}{
    Soit $U$ une matrice $m \times n$ dont les colonnes sont orthonormées, et $\bvec{x}$ et $\bvec{y}$ deux vecteurs de $\mathbb{R}^n$. Alors:
    \begin{enumerate}
        \item $\left\|U \bvec{x}\right\| = \left\|\bvec{x}\right\|$
        \item $\left(U \bvec{x}\right) \dotprod \left(U \bvec{y}\right) = \bvec{x} \dotprod \bvec{y}$
        \item $\left(U \bvec{x}\right) \dotprod \left(U \bvec{y}\right) = 0$ si et seulement si $\bvec{x} \dotprod \bvec{y} = 0$
    \end{enumerate}
}

\parag{Théorème}{
    Soient $\left(\bvec{u}_1, \ldots, \bvec{u}_p\right)$ une base orthonormée d'un sous espace vectoriel $W$ de $\mathbb{R}^n$, et $\bvec{y} \in \mathbb{R}^n$.

    La projection de $\bvec{y}$ sur $W$ est donnée par:
    \[\proj_W \bvec{y} = \left(y \dotprod \bvec{u}_1\right) \bvec{u}_1 + \ldots + \left(\bvec{y} \dotprod \bvec{u}_p\right)\bvec{u}_p\]

    De plus:
    \[\proj_W \bvec{y} = U U^T \bvec{y}, \mathspace \text{où } U = \begin{bmatrix}  &  &  \\ \bvec{u}_1 & \ldots & \bvec{u}_p \\  &  &  \end{bmatrix} \]

    \subparag{Observation}{
        On peut voir que, de manière générale, $U^T U = I$, mais $U U^T \neq I$.
    }

    \subparag{Remarque}{
        Puisque $\proj_W \bvec{y} = U U^T \bvec{y}$ est sous la forme $T\left(\bvec{x}\right) = A \bvec{x}$, on en déduit que $\proj_W$ est une application linéaire. On peut aussi démontrer ce résultat pour n'importe quel base orthogonale.

        De plus, puisque la projection ne dépend pas de la base, $U U^T$ sera toujours le même pour un sous-ensemble, peu importe la base.
    }
}

\parag{Définition}{
    Si $U \in \mathbb{R}^{n\times n}$ est \textit{carrée} et $U^T U = I_n$ (c'est-à-dire que ses colonnes sont orthonormées), alors on dit que la matrice $U$ est \important{orthogonale}.

    \subparag{Remarque}{
        Dans ce cas là, $U^{-1} = U^T$, donc:
        \[U^T U = UU^T = I_n\]
    }

    \subparag{Terminologie}{
        Il faut faire attention au fait que les colonnes sont orthonormées, mais qu'on dit que la matrice est orthogonale.
    }
}

\end{document}
