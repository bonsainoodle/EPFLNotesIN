\documentclass{article}

% Expanded on 2021-09-29 at 15:13:32.

\usepackage{../../style}

\title{AICC-1}
\author{Joachim Favre}
\date{Mercredi 29 septembre 2021}

\begin{document}
\maketitle

\lecture{4}{2021-09-29}{Nested quantifiers, and negating them}{
}

\parag{De Morgan's Laws for Quantifiers}{
    We have the following equivalences:
    \[\lnot \forall P\left(x\right) = \exists x \lnot P\left(x\right)\]
    \[\lnot \exists P\left(x\right) = \forall x \lnot P\left(x\right)\]
}

\subsection{Nested quantifiers}

\parag{Order of quantifiers}{
    The order of quantifiers is really important.

    We can always switch the order of the quantifiers whenever we only have universal quantifiers or only existential quantifiers:
    \[\forall x \forall y P\left(x, y\right) \equiv \forall y \forall x P\left(x, y\right)\]
    \[\exists x \exists y P\left(x, y\right) \equiv \exists y \exists x P\left(x, y\right)\]
}


\parag{Translation from maths}{
    \begin{enumerate}
        \item Rewrite this statement to make the implied quantifiers and domains and domains explicit.
        \item Introduce the variables.
        \item Tanslate it to predicate logic.
    \end{enumerate}
}

\parag{Prenex Normal Form (not in exams I think)}{
    We say that a statement is in Prenex Normal Form (PNF) if and only if it is of the form:
    \[Q_1 x_1 Q_2 x_2 \ldots Q_k x_k P\left(x_1, x_2, \ldots, x_k\right)\]
    where each $Q_i$ is either the existential quantification or the universal quantifier, and $P\left(x_1, \ldots, x_k\right)$ is a predicate involving no quantifier.
}

\section{Proofs (deriving knowledge)}
\subsection{Valid arguments}

\parag{Definitions}{
    An \important{argument} in propositional logic is a sequence of propositions. All but the final proposition are called \important{premises}, and the last statement is the \important{conclusion}. The argument is valid if the premises imply the conclusion.

    An \important{argument form} is an argument that is valid no matter what propositions are substituted into its propositional variables. \important{Inference rules} are simple arguments forms that will be used to construct more complex argument forms.
}

\end{document}
