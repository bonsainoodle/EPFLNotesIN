\documentclass[a4paper]{article}

% Expanded on 2021-11-11 at 08:32:42.

\usepackage{../../style}

\title{Algèbre linéaire}
\author{Joachim Favre}
\date{Jeudi 11 novembre 2021}

\begin{document}
\maketitle

\lecture{15}{2021-11-11}{Monsieur propre}{
}

\section{Valeurs propres et vecteurs propres}

\parag{Définition des vecteurs et valeurs propres}{
    Soit $A$ une matrice $n \times n$ (on travaille uniquement avec des matrices carrées). On appelle \important{vecteur propre} de $A$ tout vecteur non nul $\bvec{x}$ tel que $A \bvec{x} = \lambda \bvec{x}$ pour un certain scalaire $\lambda$.

    Un scalaire $\lambda$ est appelé \important{valeur propre} de $A$ si l'équation $A \bvec{x} = \lambda \bvec{x}$ admet au moins une solution non triviale $\bvec{x}$ ; un tel vecteur $\bvec{x}$ est appelé \important{vecteur propre associé à $\lambda$}.

}

\parag{Cas général}{
    Pour une matrice $A \in \mathbb{R}^{n \times n}$ et un scalaire $\lambda$, les affirmations suivantes sont équivalentes:
    \begin{itemize}
        \item $\lambda$ est une valeur propre de $A$
        \item $A \bvec{x} = \lambda \bvec{x}$ a une solution non-nulle, par définition
        \item $A \bvec{x} - \lambda \bvec{x} = \bvec{0}$ a une solution non-nulle
        \item $A \bvec{x} - \lambda I_n\bvec{x} = \bvec{0}$ a une solution non-nulle
        \item $\left(A - \lambda I_n\right)\bvec{x} = \bvec{0}$ a une solution non-nulle
        \item $A - \lambda I_n$ n'est pas inversible puisque son noyau n'est pas de dimension $0$
        \item $\det\left(A - \lambda I_n\right) = 0$
    \end{itemize}
}

\parag{Observation 1}{
    Les valeurs propres de $A$ sont exactement les solutions de l'équation scalaire (non linéaire) suivante:
    \[\det\left(A - \lambda I_n\right) = 0\]

    On appelle cette équation l'équation caractéristique, et le polynôme qui en découle (ce sera toujours un polynôme), le polynôme caractéristique.
}

\parag{Observation 2}{
    Si $\lambda$ est valeur propre de $A$, alors tous les vecteurs propres $\bvec{x}$ associés à $\lambda$ sont solution de $\left(A - \lambda I_n\right) \bvec{x} = \bvec{0}$. L'ensemble solution de cette équation est un sous-espace de $\mathbb{R}^{n}$ appelé \important{sous-espace propre} de $A$ associé à la valeur propre $\lambda$.
}

\parag{Théorème}{
    Si $A \in \mathbb{R}^{n \times n}$ est triangulaire (inférieure, supérieure ou diagonale), alors ses valeurs propres sont ses entités diagonales: $a_{11}, a_{22}, \ldots, a_{nn}$.

    \subparag{Remarque}{
        Attention, il est important de noter qu'échelonner une matrice ne permet pas de trouver ses valeurs propres, puisqu'un échelonnement modifie les valeurs propres.
    }
}

\parag{Exemple : Fibonacci}{
    Exemple très pertinant qui regroupe toutes les notions de ce chapitre jusqu'ici.
    Je vous invite à consulter les notes complètes de Joachim pour le lire.
}

\end{document}
