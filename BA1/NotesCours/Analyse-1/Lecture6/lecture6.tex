\documentclass[a4paper]{article}

% Expanded on 2021-10-11 at 10:04:35.

\usepackage{../../style}

\title{Analyse I}
\author{Joachim Favre}
\date{Lundi 11 octobre 2021}

\begin{document}
\maketitle

\lecture{6}{2021-10-11}{Débuts des suites et de leur limite}{

}

\section{Suites de nombres réels}
\subsection{Exemples de suites, raisonnement par récurrence}

\parag{Définition de suite}{
Une \important{suite de nombres réels} est une application $f: \mathbb{N} \mapsto \mathbb{R}$ définie pour tout nombre naturel (ou pour tout $n \geq n_0$, avec $n_0 \in \mathbb{N}$).

On utilise la notation suivante: $\left(a_n\right)$ pour une suite où $a_n = f\left(n\right)$. Ou alors:
\[\left(a_n\right)_{n\geq0} = \overbrace{\left\{a_0, a_1, \ldots\right\}}^{\text{ensemble ordonné}} = \left\{a_n, n \in \mathbb{N}\right\} \subset \mathbb{R}\]
}

\parag{Exemples}{
    \begin{enumerate}[left=0pt]
        \item \important{Suite arithmétique}: $a_n = an + b$, avec $a, b \in \mathbb{R}$ et $a \neq 0$
        \item \important{Suite géométrique}: $a_n = ar^n$, avec $a, r \in \mathbb{R}$, $a \neq 0, r\neq 0$ et $r\neq \pm 1$
    \end{enumerate}
}

\parag{Définition de minorant}{
    Une suite est \important{minorée} s'il existe un nombre $m$ réel tel que que $a_n \geq m \ \forall n \in \mathbb{N}$. On appelle $m$ un minorant.
}

\parag{Définition de majorant}{
    Une suite est \important{majorée} s'il existe un nombre $M$ réel tel que que $a_n \leq M \ \forall n \in \mathbb{N}$. On appelle $M$ un majorant.
}

\parag{Définition de la valeur absolue}{
    On définit
    \begin{functionbypart}{\left|x\right|}
        x\text{, si } x \geq 0 \\
        -x\text{, si } x \leq 0
    \end{functionbypart}
}


\parag{Définition de suite bornée}{
    On dit qu'une suite est \important{bornée} si elle est majorée et minorée.
}

\parag{Définition de suite monotone}{
    Une suite $\left(a_n\right)$ est \important{croissante} si pour tout $n \in \mathbb{N}$, on a $a_{n + 1} \geq a_n$.

    Une suite $\left(a_n\right)$ est \important{décroissante} si pour tout $n \in \mathbb{N}$, on a $a_{n + 1} \leq a_n$.

    Une suite est dite \important{monotone} si elle est croissante ou décroissante.
}

\parag{Définition de suite strictement monotone}{
    Une suite dite \important{strictement croissante} ou \important{strictement décroissante} est définie de la même manière, mais en prenant $>$ à la place de $\geq$ et $<$ à la place de $\leq$.

    Une suite dite \important{strictement monotone} est donc strictement croissante ou strictement décroissante.
}

\parag{Raisonnement par récurrence}{
    Soit $P\left(n\right)$ une proposition dépendant d'un entier naturel $n$ tel que:
    \begin{enumerate}
        \item \important{L'initialisation}: $P\left(n_0\right)$ est vraie
        \item \important{L'hérédité}: $\forall n \geq n_0$, on a $P\left(n\right) \implies P\left(n +  1\right)$
    \end{enumerate}

    Alors $P\left(n\right)$ est vraie pour tout $n \geq n_0$.
}

\parag{Généralisation de la méthode de récurrence}{
    Il y a plusieurs manières de généraliser cette méthode, en voici une. Soit $P\left(n\right)$ une proposition pour $n \in \mathbb{N}$ telle que:
    \begin{enumerate}
        \item $P\left(n_0\right), P\left(n_0 + 1\right), \ldots, P\left(n_0 + k\right)$ avec k fixé, sont vraies
        \item $\left\{P\left(n\right), P\left(n+1\right), \ldots, P\left(n+k\right)\right\} \implies P\left(n + k + 1\right)$, $\forall n \geq 0$
    \end{enumerate}

    Alors $P\left(n\right)$ est vraie $\forall n \geq n_0$.
}

\subsection{Limites de suites}
\parag{Définition de suite convergente et de limite}{
    On dit que la suite $\left(x_n\right)$ est \important{convergente} et admet pour \important{limite} le nombre réel $\ell \in \mathbb{R}$ si pour tout $\epsilon > 0$, il existe $n_0 \in \mathbb{N}$ tel que pour tout $n \geq n_0$, on a $\left|x_n - \ell\right| \leq \epsilon$.

    On note la limite:
    \[\lim_{n \to \infty} x_n = \ell\]
}

\parag{Définition de suite divergente}{
Une suite qui n'est pas convergente est dite \important{divergente} (la suite peut partir vers l'infini positif ou négatif, ou osciller (enfin il faudrait démontrer ça, c'est juste une conjecture que je pose là)).
}

\end{document}
