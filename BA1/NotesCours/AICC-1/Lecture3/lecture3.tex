\documentclass{article}

% Expanded on 2021-09-28 at 08:13:59.

\usepackage{../../style}

\title{Lecture 3}
\author{Joachim Favre}
\date{Mardi 28 septembre 2021}

\begin{document}
\maketitle

\lecture{3}{2021-09-28}{Introduction to predicate logic}{
}

\section{Predicate logic}
\subsection{Introduction}

\parag{Variables}{
    As explained above, we want to characterise an object by its properties. Let's call that object $x$; this is a \important{variable}.
}

\parag{Definitions}{
    We define \important{predicates} to be statements that contains a variable.
}

\parag{Propositional functions}{
    We call \important{propositional functions} expressions constructed from predicates and logical connectives containing variables.
}

\subsection{Quantifiers}

\parag{Universal quantifier}{
    The universal quantification of a propositional function $P\left(x\right)$ is the statement ``$P\left(x\right)$ is true for all values $x$ from its domain $U$''. We write this $\forall x\ P\left(x\right)$, and read ``for all $x$, $P\left(x\right)$''. We call $\forall$ the \important{universal quantifier}.

    In other words, $\forall x\ P\left(x\right)$ means that for any $x$ in the universe, the proposition evaluates to true.
}

\parag{Remark}{
    The result of a statement can depend on the universe we choose.
}

\parag{Existential quantifier}{
    The existential quantification of a propositional function $P\left(x\right)$ is the statement ``there exists an element $x$ from the domain $U$ such that $P\left(x\right)$ is true''. We write $\exists x\ P\left(x\right)$, and it reads ``for some $x$, $P\left(x\right)$'', or ``for at least one $x$, $P\left(x\right)$''. We call $\exists$ the \important{existential quantifier}.

    \subparag{Link with universal quantifier}{
        We see that if $\forall x\ P\left(x\right)$ is true, then $\exists x\ P\left(x\right)$ must also be true (if the domain is not empty).
    }
}

\parag{Empty domains}{
    By convention, we never consider empty domains.
}

\parag{Summary}{
    We can draw the following table:
    \begin{center}
        \begin{tabular}{|c|c|c|}
            \hline
            \textbf{Statement}           & \textbf{Is true when}                                & \textbf{Is false when}                                \\
            \hline
            $\forall x\ P\left(x\right)$ & $P\left(x\right)$ is true for every $x$.             & There is an $x$ for which $P\left(x\right)$ is false. \\
            $\exists x\ P\left(x\right)$ & There is an $x$ for which $P\left(x\right)$ is true. & $P\left(x\right)$ is false for every $x$.             \\
            \hline
        \end{tabular}
    \end{center}

    We call a \important{counterexample} for $\forall x\ P\left(x\right)$, a value $x$ for which $P\left(x\right)$ is false; and a \important{witness} for $\exists x\ P\left(x\right)$, a value $x$ for which $P\left(x\right)$ is true.
}

\parag{Uniqueness quantifier}{
    We use $\exists!x\ P\left(x\right)$ to express the fact that $P\left(x\right)$ is true for one and only one $x$ in the domain $U$. We usually say ``there is a unique $x$ such that $P(x)$'', or ``there is one and only one $x$ such that $P\left(x\right)$''.

    \subparag{Expression using other quantifiers}{
        This is skipping a bit ahead, but we can express the uniqueness quantifier using other quantifiers:
        \[\exists!x\ P\left(x\right) \equiv \exists x \left(P\left(x\right) \land \forall y \left(y \neq x \to \lnot P\left(y\right)\right)\right) \]
    }
}

\parag{Composite statements involving quantifiers}{
    The quantifiers have higher precedence than all the logical connectives from propositional logic.
}

\parag{Variable binding}{
    We say that a quantifier binds the variable of a propositional function.

    $P\left(x\right)$ is a propositional function with a \important{free variable} $x$, and $\forall x\ P\left(x\right)$ is a proposition with \important{bound variable} $x$.
}

\parag{Validity and satisfiability}{
    Valid means true whatever the variable we choose, satisfiable means that there exists some case in which it is true, unsatisfiable means that it is always false.
}


\parag{Shorthand notation}{
    We can define the following shorthand notations:
    \[\forall x \in S\ P\left(x\right) \equiv \forall x \left(x \in S \to P\left(x\right)\right)\]
    \[\exists x \in S\ P\left(x\right) \equiv \exists x\left(x \in S \land P\left(x\right)\right)\]

    \subparag{Negations}{
        We can notice that this notation is coherent with negations. Indeed, we do have that:
        \[\lnot \forall x \in S\ P\left(x\right) \equiv \lnot \forall x\left(x \in S \to P\left(x\right)\right) \equiv \lnot \forall x \left(\lnot x \in S \lor P\left(x\right)\right)\]

        Which is equal to:
        \[\exists x \left(x \in S \land \lnot P\left(x\right)\right) \equiv \exists x \in S\ \lnot P\left(x\right)\]


        We also find that:
        \[\lnot \exists x \in S\ P\left(x\right) \equiv \lnot \left(\lnot \forall x \in S\ \lnot P\left(x\right)\right) \equiv \forall x \in S\ \lnot P\left(x\right)\]

        This makes sense, since we have to keep the same hypothesis when negating a statement.
    }
}

\subsection{Logical equivalences in predicate logic}

\parag{Equivalences}{
    Two statements $S$ and $T$ involving predicates and quantifiers are logically equivalent if and only if they have the same truth values no matter which predicates are substituted and which domain of discourse is used.
}

\parag{Distribution of quantifiers of connectives}{
    We have the following equivalences:
    \[\forall x\left(P\left(x\right) \land Q\left(x\right)\right) \equiv \forall x P\left(x\right) \land \forall x Q\left(x\right)\]
    \[\exists x \left(P\left(x\right) \lor Q\left(x\right)\right) \equiv \exists x P\left(x\right) \lor \exists x Q\left(x\right)\]

    However:
    \[\exists x\left(P\left(x\right) \land Q\left(x\right)\right) \not\equiv \exists x P\left(x\right) \land \exists x Q\left(x\right)\]
    \[\forall x \left(P\left(x\right) \lor Q\left(x\right)\right) \not\equiv \forall x P\left(x\right) \lor \forall x Q\left(x\right)\]

    Indeed, for both we can use the following counter example: $P\left(x\right) := $ ``$x$ is even'', $Q\left(x\right) := $ ``$x$ is odd'' and $\mathbb{Z}$ as the domain.
}

\end{document}
