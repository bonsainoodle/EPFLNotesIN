\documentclass[a4paper]{article}

% Expanded on 2022-05-02 at 10:19:02.

\usepackage{../../style}

\title{Analyse 2}
\author{Joachim Favre}
\date{Lundi 02 mai 2022}

\begin{document}
\maketitle

\lecture{19}{2022-05-02}{Fin des études d'extremums}{
}

\parag{Proposition: Hypothèses équivalentes pour le théorème de la condition suffisante pour un extremum local quand $n = 2$}{
    Dans le cas où $n = 2$, nous pouvons réécrire les conditions de notre théorème.

    Notre matrice Hessienne est donnée par:
    \[\Hess_f\left(\bvec{a}\right) = \begin{pmatrix} \frac{\partial^{2} f}{\partial x^{2}} & \frac{\partial^2 f}{\partial y \partial x} \\ \frac{\partial^2 f}{\partial x \partial y} & \frac{\partial^{2} f}{\partial y^{2}}  \end{pmatrix} = \begin{pmatrix} r & s \\ s & t \end{pmatrix} \]

    Nous avons les équivalences suivantes:
    \begin{enumerate}
        \item $\lambda_1 > 0, \lambda_2 > 0 \iff \det \Hess_f\left(\bvec{a}\right) > 0 \text{ et } r > 0$
        \item $\lambda_1 < 0, \lambda_2 < 0 \iff \det \Hess_f\left(\bvec{a}\right) > 0 \text{ et } r < 0$
        \item $\lambda_1 > 0, \lambda_2 < 0 \text{ ou } \lambda_1 < 0, \lambda_2 > 0 \iff \det \Hess_f\left(\bvec{a}\right) < 0$
    \end{enumerate}
}

\parag{Conditions équivalentes aux conditions suffisantes pour $n = 3$}{
On pose:
\[\Delta_1 = \frac{\partial^{2} f}{\partial x_1^{2}}, \mathspace \Delta_2 = \frac{\partial^{2} f}{\partial x_1^{2}} \cdot \frac{\partial^{2} f}{\partial x_2^{2}} - \frac{\partial^2 f}{\partial x_2 \partial x_1} \cdot \frac{\partial^2 f}{\partial x_1 \partial x_2}, \mathspace \Delta_3 = \det\Hess_f\]

\begin{enumerate}[left=0pt]
    \item (+++) Si $\Delta_1 > 0, \Delta_2 > 0, \Delta_3 > 0$, alors $\bvec{a}$ est un point de minimum local (et $\Hess_f\left(\bvec{a}\right)$ est dite définie positive).
    \item (-+-) Si $\Delta_1 < 0, \Delta_2 > 0, \Delta_3 < 0$, alors $\bvec{a}$ est un point de maximum local.
    \item Autrement, si $\Delta_3 \neq 0$, alors il n'y a pas d'extremum local en $\bvec{a}$.
    \item Si $\Delta_3 = 0$, alors nous ne pouvons rien conclure.
\end{enumerate}
}

\subsection[Min et max sur un compact]{Minimum et maximum d'une fonction continue sur un compact}

\parag{Méthode}{
    Nous voulons une méthode pour trouver ces $\bvec{c_1}, \bvec{c_2}$. Pour faire cela, il faut:
    \begin{enumerate}
        \item Trouver les points critiques $\left\{\bvec{c_i}\right\}$ de $f$ sur $\mathring{D}$ (l'intérieur de $D$). Calculer les valeurs $f\left(\bvec{c_i}\right)$.
        \item Trouver les points $\left\{\bvec{d_j}\right\}$ de minimum et maximum de $f\left(\partial D\right)$ ($\partial D$ est la frontière de $D$). Calculer les valeurs $f\left(\bvec{d_j}\right)$.
        \item Choisir le minimum et le maximum parmi les valeurs qu'on a trouvées.
    \end{enumerate}

    Notez que le deuxième point peut être très dur à calculer. La frontière peut par exemple être donnée par morceaux, auquel cas il ne faut pas oublier les coins. Ensuite, nous évaluons $f$ sur la frontière à l'aide de cette dépendance entre $x$ et $y$.
}

\subsection{Théorème des fonctions implicites}
\parag{Définition: Fonction implicite}{
    Une \important{fonction implicite} est une dépendance $f = f\left(\bvec{x}\right)$ qui est définie par une équation.
}

\end{document}
