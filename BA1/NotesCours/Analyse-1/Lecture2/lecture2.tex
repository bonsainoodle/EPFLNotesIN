\documentclass{article}

% Expanded on 2021-09-26 at 22:53:53.

\usepackage{../../style}

\title{Analyse I}
\author{Joachim Favre}
\date{Lundi 27 septembre 2021}

\begin{document}
\maketitle

\lecture{2}{2021-09-27}{Axiomatique des nombres réels}{
}

\section{Nombres réels}
\subsection{Ensembles}

\parag{Définition d'ensemble}{
    De manière naïve, un \important{ensemble} est une ``collection d'objets définis et distincts'' (G. Cantor).
}

\parag{Notations}{
    \begin{itemize}[left=0pt]
        \item $b \in Y$ \textleftrightarrow $b$ appartient à $Y$ (ou, de manière équivalent, $b$ est un élément de $Y$).
        \item $a \not\in Y$ \textleftrightarrow $a$ n'appartient pas à $Y$
        \item $\forall$ \textleftrightarrow pour tout
        \item $\exists$ \textleftrightarrow il existe
        \item $Y \subset X$ \textleftrightarrow $Y$ est un sous-ensemble de $X$. Plus précisément, $Y \subset X \over{\iff}{\text{déf}} \forall b \in Y \implies b \in X$
        \item $Y \not \subset X \iff \exists b \in Y \telque b \not \in X$
        \item $Y = X \over{\implies}{\text{déf}} Y \subset X \text{ et } X \subset Y$
        \item $Y \neq X \implies Y \not \subset X \text{ ou } X \not \subset Y$
        \item $\o$, l'ensemble vide, est défini comme $\o = \left\{\right\}$. On trouve donc que $\o \subset X\ \forall X$ et $X \subset X \forall X$
    \end{itemize}
}

\parag{Ensemble universel}{
    Supposons que les ensembles considérés sont sous-ensemble d'un d'un ensemble universel $U$.

    \subparag{Opérations ensemblistes}{
        Soient $X, Y, Z \subset U$.
        \begin{enumerate}
            \item Réunion: $X \cup Y \over{=}{\text{déf}} \left\{a \in U \telque a \in X \text{ ou } a \in Y\right\}$

                  De plus, l'associativité tient:
                  \[\left(X \cup Y\right) \cup Z = X \cup \left(Y \cup Z\right)\]

            \item Intersection: $X \cap Y \over{=}{\text{déf}} \left\{a \in U \telque a \in X \text{ et } a \in Y\right\}$

                  L'associativité tient aussi:
                  \[\left(X \cap Y\right) \cap Z = X \cap \left(Y \cap Z\right)\]
            \item Différence: $X \setminus Y = \left\{a \in U \telque a \in X \text{ et } a \not \in Y\right\}$

                  Attention, en général, elle n'est pas symétrique: $X \setminus Y \neq Y \setminus X$.
        \end{enumerate}
    }
}

\parag{Proposition}{
    Soient $X, Y, Z \subset U$. Alors $X \setminus \left(Y \cap Z\right) = \left(X \setminus Y\right) \cup \left(X \setminus Z\right)$
}

\subsection{Nombres naturels, rationnels et réels}
\parag{Nombres naturels}{
    Les nombres naturels $\mathbb{N} = \left\{0, 1, 2, \ldots\right\}$ sont pris avec ``$+$'', ``$\cdot$'' et la relation d'ordre suivante:
    \[a \leq b, a, b \in \mathbb{N} \iff \exists c \in \mathbb{N} \telque a + c = b\]

    Nous avons besoin de l'un des deux axiomes suivants (ils sont purement équivalents, l'un découle de l'autre):

    \subparag{Propriété de bon ordre}{
        Tout sous-ensemble non-vide de $\mathbb{N}$ contient un plus petit élément.
    }

    \subparag{Propriété de récurrence}{
        Soit $S \subset \mathbb{N}$ tel que (a) $0 \in S$ et (b) si $n \in S \implies n + 1 \in S$. Alors $S = \mathbb{N}$.
    }
}

\parag{Entiers relatifs}{
    Les entiers relatifs $\mathbb{Z} = \left\{0, \pm 1, \pm 2, \ldots \right\}$.

    Cet ensemble a comme propriété que tout $x \in \mathbb{Z}$ possède un élément \important{opposé} par rapport à l'addition, i.e:
    \[\forall x \in \mathbb{Z}, \exists y \in \mathbb{Z} \telque x + y = 0\]

    On note $y = -x$.
}

\parag{Nombres rationnels}{
    Les nombres rationnels $\mathbb{Q} = \left\{\frac{p}{q},\ p, q \in \mathbb{Z}, q \neq 0\right\}$. On a que $\frac{p}{q} = \frac{t}{s}$ si $p\cdot s = t\cdot q$.

    Cet ensemble a la propriété suivante:
    \[\forall x \in \mathbb{Q} \telque x \neq 0 \implies \exists y \in \mathbb{Q} \telque x\cdot y = 1\]

    On dit que $y$ est réciproque à $x$. On le note $y = \frac{1}{x}$.
}

\parag{Définition de l'axiomatique de $\mathbb{R}$}{
    \begin{enumerate}[left=0pt]
        \item \important{$\mathbb{R}$ est un corps}: c'est un ensemble avec $+$, $\cdot$ satisfaisant les axiomes suivants pour tout $x, y, z \in \mathbb{R}$. Pour l'addition :
              \begin{itemize}
                  \item $\left(x + y\right) + z = x + \left(y + z\right)$ (associativité)
                  \item $x + y = y + z$ (commutativité)
                  \item $\exists 0 \in \mathbb{R} \telque x + 0 = x$ (existence de l'élément nul par rapport à l'addition)
                  \item $\forall x \in \mathbb{R}\ \exists y \in \mathbb{R} \telque x + y = 0$ (existence de l'opposé additif)
              \end{itemize}

              Pour la multiplication:
              \begin{itemize}
                  \item $\left(x \cdot y\right)\cdot z = x\cdot\left(y\cdot z\right)$ (associativité)
                  \item $x \cdot y = y\cdot x$ (commutativité)
                  \item $\exists 1 \in \mathbb{R}, 1 \neq 0, \telque x\cdot 1 = x \mathspace \forall x \in \mathbb{R}$  (existence de l'élément nul par rapport à la multiplication)
                  \item $\forall x \in \mathbb{R}, x \neq 0 \implies \exists y \in \mathbb{R} \telque x\cdot y = 1$  (existence de l'inverse mulitplicatif pour les éléments non-nuls)
              \end{itemize}

              Et pour faire un lien entre les opérations, nous avons aussi besoin de l'axiome suivant, appelé distributivité:
              \[x \cdot \left(y + z\right) = x\cdot y + x\cdot z\]

        \item \important{$\mathbb{R}$ est un corps ordonné :} Il existe une relation d'ordre $\leq$ tel que, pour tout couple d'élément $x, y \in \mathbb{R}$, on a: $x \leq y$ ou $y \leq x$. Si $x \leq y$ et $y \leq x \implies x = y$

              Pour tout triple d'élément $x, y, z \in \mathbb{R}$ on a:
              \begin{itemize}
                  \item $x \leq y$ et $y \leq z \implies x \leq z$
                  \item Si $x \leq y \implies x + z \leq y + z$
                  \item Si $x \geq 0$ et $y \geq 0 \implies x\cdot y \geq 0$
              \end{itemize}

              \important{Notation : } Si $x \leq y$ et $x \neq y$, alors $x < y$. De la même manière, si $x \geq y$ et $x \neq y$, alors $x > y$.

              Cet axiome nous permet d'éliminer $\mathbb{C}$ de la définition. Cependant, $\mathbb{Q}$  est toujours là.

        \item \important{Axiome de la borne inférieure : } Pour tout sous-ensemble $S$ non-vide de $\mathbb{R}^*_+$, il existe un nombre $a \in \mathbb{R}_+$ tel que
              \begin{enumerate}
                  \item $a \leq x \mathspace \forall x \in S$
                  \item Quel que soit $\epsilon > 0$, il existe un élément $x_{\epsilon} \in S$ tel que $x_{\epsilon} - a \leq \epsilon$ ($\iff a + \epsilon \geq x_{\epsilon}$). En d'autres mots, peu importe le $\epsilon$, il existe toujours un nombre $x_{\epsilon}$ entre $a$ et $a + \epsilon$ appartenant à $S$.
              \end{enumerate}

              On dit que $a \in \mathbb{R}$ est \important{l'infimum} (= la borne inférieure) de l'ensemble $S$.

              $\mathbb{Q}$ ne suit pas cet axiome.
    \end{enumerate}

    Puisque $\mathbb{R}$ suit ces trois axiomes, on dit qu'il est un corps commutatif, ordonné et complet.
}

\subsection{Infimums et suprémums}

\parag{Borne inférieure et supérieure}{
    \subparag{Définition de minorant}{
        Soit $S \subset \mathbb{R}$, $S \neq \o$. On dit que $a \in \mathbb{R}$ est un \important{minorant} de $S$ si $\forall x \in S$, on a $x \geq a$.


        Si $S$ admet un minorant, alors $S$ est minoré.
    }

    \subparag{Définition de majorant}{
        Soit $S \subset \mathbb{R}$, $S \neq \o$. On dit que $b \in \mathbb{R}$ est un \important{majorant} de $S$ si $\forall x \in S$, on a $x \leq b$.

        Si $S$ admet un majorant, alors $S$ est majoré.
    }

    \subparag{Définition d'ensemble borné}{
        Si $S$ est minoré et majoré, alors $S$ est \important{borné}.
    }

    \subparag{Définition d'infimum}{
        Soit $S$ un sous-ensemble non-vide de $\mathbb{R}$. Un nombre réel $a$ vérifiant les propriétés suivantes:
        \begin{itemize}
            \item $\forall x \in S, x \geq a$
            \item $\forall \epsilon > 0$, il existe un élément $x_{\epsilon} \in S$ tel que $x_{\epsilon} - a \leq \epsilon$
        \end{itemize}
        est \important{l'infimum} de $S$, $a = \inf S$. Il est unique.
    }


    \subparag{Définition de suprémum}{
        Soit $S$ un sous-ensemble non-vide de $\mathbb{R}$. Un nombre réel $b$ vérifiant les propriétés suivantes:
        \begin{itemize}
            \item $\forall x \in S, x \leq b$
            \item $\forall \epsilon > 0$, il existe un élément $x_{\epsilon} \in S$ tel que $b - x_{\epsilon} \leq \epsilon$
        \end{itemize}
        est le \important{suprémum} de $S$, $b = \sup S$. Il est unique.
    }

}

\end{document}
